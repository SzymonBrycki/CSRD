\documentclass[10pt, a4paper, twocolumn, openright]{book}
\usepackage[pdftex, breaklinks=true]{hyperref}
\usepackage{polski}
\usepackage[utf8]{inputenc}
\usepackage{hyperref}
\usepackage{makeidx}
\usepackage{tabularx}
\usepackage{xcolor }
\usepackage{soul}
\usepackage{afterpage}

% TABLES WIDTH !!!
% 0.10 and 0.45

\definecolor{purple}{HTML}{92268F}
% \definecolor{gray}{HTML}{D3D3D3}

% \sethlcolor{gray} 

\makeindex

\newcommand{\mytext }[1] {{\color{purple}  \texttt {#1}}}

% \title{Cypher System Reference Document 2024-07-02 (Edycja Polska)}
% \author{Zespół Monte Cook Games\thanks{Strona projektu: \url{https://www.montecookgames.com/cypher-system-open-license/}} \and Szymon ``Kaworu'' Brycki\thanks{\href{mailto:szymon.brycki@gmail.com}{\tt szymon.brycki@gmail.com}}}

\begin{document}

\begin{titlepage}
	\centering
	{\Huge\bfseries\title  CCypher System Reference Document \par}
	\vspace{1cm}
	{\large\itshape 2024-07-02 \par}
	{\large\itshape Edycja polska \par}
	\vspace{1cm}
	{\normalsize Zespół Monte Cook Games,   Szymon ``Kaworu'' Brycki \par}
	\vspace{1cm}
	{\normalsize Licencja: \bfseries Cypher System Open License\par}
	\vspace{1cm}
	{\normalsize Stworzono w technologii \LaTeX \par}
	\vspace{1cm}
	{\large \today \par}
\end{titlepage}

% \maketitle

\tableofcontents

% here go all the chapters

\chapter {Jak grać w Cypher System}

Zasady Cypher System są całkiem proste i cała rozgrywka bazuje na ledwie kilku podstawowych konceptach.

Ten rozdział zapewnia krótkie wyjaśnienie jak grać w tę grę, i jest przydatny dla dopiero uczących się rozgrywki. Kiedy zrozumiesz już podstawowe koncepty, będziesz pewnie chcieć przeczytać \mytext{Zasady Gry} po więcej szczegółów. 
Cypher System korzysta z kości dwudziestościennej (k20) by określić wynik większości akcji. Za każdym razem, gdy wymagany jest rzut, a nie podano kości, rzuć k20.

Mistrz Gry określa stopień trudności danego zadania. Istnieje 10 stopni trudności. Tak więc, trudność można określić na skali od 1 do 10.

Każda trudność ma minimalny wynik powiązany z sobą. Minimalny wynik (inaczej zwany stopniem trudności) to zawsze 3x poziom trudności, więc stopień trudności 1 ma minimalny wynik 3, a stopień trudności 4 ma minimalny wynik 12. By odnieść sukces, należy wyrzucić minimalny wynik lub więcej danego ST. Patrz Tabela Stopnie Trudności po więcej danych.
Umiejętności postaci, przydatne okoliczności lub doskonały ekwipunek mogą zmniejszyć trudność zadania. Dla przykładu, postać wytrenowana we wspinaczce może zamienić trudność 6 testu wspinaczki na trudność 5. Nazywa się to  Ułatwianiem albo Obniżaniem trudności o jeden stopień (albo po prostu Obniżaniem trudności, gdzie przyjmuje się domyślnie, że dotyczy ona jednego stopnia). Jeśli postać jest wyspecjalizowana we wspinaczce, zamienia ona trudność 6 na trudność 4. Nazywa się to Obniżaniem trudności o dwa stopnie. Obniżanie poziomu trudności może także być nazywane ułatwieniem zadania. Niektóre sytuacje zwiększają, lub Utrudniają, trudność zadania. Jeśli zadanie jest utrudnione, należy zwiększyć jego trudność o jeden poziom.

Umiejętność to kategoria wiedzy, zdolności lub aktywności powiązania z zadaniem, np.: wspinaczka, geografia lub perswazja. Postać, która posiada umiejętność, jest lepsza w powiązanych z nią zadaniach niż postać, która nie posiada danej umiejętności. Posta posiada albo wytrenowaną (do pewnego stopnia) umiejętność, albo wyspecjalizowaną (bardzo dużą).
Jeśli jesteś wytrenowany w umiejętności powiązanej z danym zadaniem, ułatwiasz rzut o stopień. Jeśli jesteś wyspecjalizowany, obniżasz poziom trudności o dwa stopnie. Umiejętność nigdy nie może obniżyć trudności testu o więcej niż dwa stopnie. 

Wszystko inne co obniża trudność danego zadania nazywa się Wysiłkiem. (Wysiłek opisano szczegółowo w rozdziale Zasady Gry).

Podsumowując, trzy rzeczy mogą obniżyć trudność zadania: umiejętności, atuty i Wysiłek. 

Jeśli ułatwisz rzut tak mocno, że jego trudność wynosi 0, wtedy automatycznie uzyskujesz sukces i nie musisz rzucać kośćmi. 

\section {Kiedy rzucać kośćmi?}

Za każdym razem, gdy Twoja postać chce wykonać jakieś zadanie, MG daje mu Poziom Trudności i rzucasz k20 przeciwko Stopniowi Trudności powiązanemu z danym Poziomem Trudności.

Kiedy wyskakujesz z płonącego pojazdu, zamachujesz się toporem na zmutowanąbestię, płyniesz poprzez rwącą rzekę, identyfikujesz dziwne urządzenie, przekonujesz handlarza, by dał Ci niższą cenę, tworzysz obiekt, korzystasz z mocy, by kontrolować umysł przeciwnika lub korzystasz z laserowego działka, by zrobić dziurę w ścianie, wykonujesz rzut k20.
Jednakże, jeśli twój Poziom Trudności ma wartość 0, rzut nie jest konieczny – automatycznie uzyskujesz sukces. Wiele akcji ma trudność 0. Przykłady to przejście przez pokój i otworzenie drzwi, skorzystanie ze specjalnej zdolności lotu, korzystanie z mocy, by ochronić swojego przyjaciela przed promieniowaniem, lub aktywowanie urządzenia (które się rozumie) by stworzyć pole siłowe. To wszystko to są rutynowe akcje i nie wymagają one rzutów.

Korzystając z umiejętności, atutów i Wysiłku, można teoretycznie obniżyć trudność dowolnej akcji do 0 i zlikwidować konieczność rzucania kostką. Przejście po wąskim drewnie jest trudne dla większości ludzi, ale nie dla doświadczonego gimnastyka. Możesz nawet obniżyć trudność ataku na swojego wroga do 0 i odnieść sukces bez rzucania.

Jeśli nie ma rzutu, nie ma szansy, by odnieść porażkę. Jednakże, nie ma także szansy na wyjątkowy sukces (w Cypher System zazwyczaj oznacza to wyrzucenie 19 lub 20, co jest znane jako specjalne rzuty; rozdział Zasady Gry omawia je w szczegółach).

\begin{table*}[t]
 \centering
 \begin{tabularx}{\textwidth}{ | X | X | X | X |}
  \hline
   Poziom trudności & Opis & Stopień trudności & Szczegóły  \\ \hline
    0 & Rutyna & 0 & Każdy może to zrobić zawsze \\ \hline
    1 & Proste & 3 & Większość ludzi może to zrobić przez większość czasu  \\ \hline
 \end{tabularx}
  \caption {Tabela: Trudność zadań}
  \label {Tabela: Trudność zadań}
 \end{table*}
 
\section {Walka}\index{Walka!Wstęp}
Wykonywanie ataków w walce działa tak samo jak inne rzuty – MG określa trudność zadania, a następnie należy rzucić k20 przeciwko Stopniowi Trudności.

Trudność Twojego testu ataku zależy od tego, jak bardzo potężny jest przeciwnik. Istoty mają poziomy od 0 do 10, tak jak i zadania, które może wykonać postać. Zazwyczaj trudność rzutu to ST powiązanie z poziomem istoty. Dla przykładu, atak na bandytę 2 poziomu to zadanie o Poziomie Trudności 2, więc Stopień Trudności wynosi 6. 

Trzeba zaznaczyć, że gracze wykonują wszystkie rzuty w Cypher System. Jeśli gracz atakuje istotę, ten gracz wykonuje rzut na atak. Jeśli istota atakuje gracza, to on wykonuje rzut obronny. 

Obrażenia zadawane przez atak nie są definiowane przez rzut kością – jest to stała wartość bazująca na broni lub ataku. Dla przykładu, włócznia zawsze zadaje 4 punkty obrażeń.

Twój Pancerz redukuje obrażenia które otrzymujesz. Otrzymujesz Pancerz za noszenie fizycznej zbroi (takiej jak skórzana kurtka e współczesnym świecie lub pancerz w świecie fantasy) lub ze specjalnych zdolności. Tak jak wartość obrażeń, Pancerz to stała wartość, nie wynik rzutu. Jeśli jesteś zaatakowany, odejmij swój Pancerz od otrzymanych obrażeń. Dla przykładu, skórzana kurtka daje Ci +1 do Pancerza, co oznacza, że otrzymujesz o 1 mniejsze obrażenia z ataków. Jeśli ktoś trafi Cię atakiem nożem za 2 punkty obrażeń, kiedy ją nosisz, otrzymasz tylko 1 punkt obrażeń. Jeśli Pancerz redukuje obrażenia do 0, wtedy nie otrzymujesz w ogóle żadnych obrażeń. 

Kiedy widzisz w zasadach gry słowo „Pancerz” pisane wielką literą, odnosi się do to statystyki Pancerz – do liczby, o którą obniżasz obrażenia. Kiedy widzisz „pancerz” pisany małą literą, dotyczy to dowolnego fizycznego pancerza, który postać może nosić. 

Fizyczne bronie posiadają 3 kategorie: lekkie, średnie i ciężkie.  

Lekkie bronie zadają tylko 2 punkty obrażeń, ale ułatwiają rzuty na atak, ponieważ są szybkie i łatwe w użyciu. Lekkie bronie to ciosy pięścią, kopnięcia, maczugi, noży, toporki ręczne, rapiery, małe pistolety itp. Bronie, które są małe, są broniami lekkimi.
Średnie bronie zadają 4 punkty obrażeń. Średnie bronie to między innymi miecze, topory bojowe, większe maczugi, kusze, włócznie, pistolety, blastery itp. Większość broni to bronie średnie. Wszystko, co może być użyte w jednej dłoni (nawet, jeśli często korzysta się z dwóch, jak w przypadku kostura i włóczni) jest średnią bronią. 

Ciężkie bronie zadają 6 punktów obrażeń, i trzeba korzystać z obydwu dłoni, by z nich korzystać. Ciężkie bronie to wielkie miecze, młoty bojowe, potężne topory, halabardy, ciężkie kusze, karabinki laserowe itp. Wszystko, z czego trzeba korzystać z obydwu dłoni, to ciężkie bronie.

\section {Specjalne wyniki rzutów}\index{Specjalne wyniki rzutów!Wstęp}

Kiedy wyrzucasz naturalne 19 (k20 pokazuje „19”) i test jest sukcesem, uzyskujesz mniejszy efekt. W walce, mniejszy efekt zadaje dodatkowe 3 do obrażeń, lub, jeśli wolisz efekt specjalny, możesz odrzucić wroga do tyłu, rozproszyć jego uwagę lub coś podobnego. Kiedy nie walczysz, mniejszy efekt może oznaczać, że wykonałeś akcję ze stylem. Przykładowo, gdy przeskakujesz przez płot, lądujesz z gracją na własnych stopach, lub gdy przekonujesz kogoś, wierzy on, że jesteś mądrzejszy, niż jesteś naprawdę. W innych słowach, nie tylko osiągasz zwykły sukces, ale także uzyskujesz pomniejszy bonus. 

Kiedy wyrzucasz naturalne 20 (k20 pokazuje „20”) i rzut się powiódł, uzyskujesz dodatkowo większy efekt. Jest to podobne do mniejszego efektu, ale na większą skalę. W walce, zadajesz dodatkowe 4 punkty obrażeń, ale znowu, można zamiast tego wybrać jakiś efekt dodatkowy, taki jak przewrócenie wroga, ogłuszenie go, lub wzięcie akcji dodatkowej. Poza walką, większy efekt oznacza, że dzieje się coś korzystnego, w zależności od okoliczności. Dla przykładu, kiedy wspinasz się na ścianę, robisz to dwa razy szybciej. Kiedy rzut daje Ci większy efekt, możesz zamiast tego skorzystać z mniejszego efektu, jeśli taka jest Twoja wola.

W walce (i tylko wtedy) jeśli rzucisz naturalne 17 lub 18 na rzucie na atak, zadajesz – odpowiednio - dodatkowe 1 lub 2 punkty obrażeń. Te rzuty nie dają żadnych innych specjalnych efektów – tylko zwiększają obrażenia.

(Po więcej informacji o specjalnych wynikach rzutów i tym, jak wpływają na walkę i inne akcje, patrz Zasady Gry).

Wyrzucenie naturalnej 1 jest zawsze złe. To oznacza, że MG wprowadza nowe utrudnienie do sceny. 

\section {Słowniczek}\index{Słowniczek}

{\bfseries Mistrz Gry (MG)}: Gracz, który nie ma własnej postaci, a który zamiast tego kieruje całą fabułą i wszystkimi BN-ami.

{\bfseries Bohater Niezależny (BN)}: Postać kierowana przez MG. Myśl o niej jak o pomniejszej postaci w historii, lub jak o złoczyńcy lub oponencie. Wlicza się w to każde każda istota lub potwór.

{\bfseries Drużyna}: Grupa BG (i może jacyś BN-i sojusznicy).

{\bfseries Bohater Gracza (BG)}: Postać odgrywana przez gracza zamiast przez MG. Myśl o BG jak o głównych bohaterach historii.

{\bfseries Gracz}: Gracz, który kieruje BG.

{\bfseries Sesja}: Pojedyncza doświadczenie roleplayowe. Zazwyczaj trwa kilka godzin. Czasami jedną przygodę można ukończyć w czasie jednej sesji. Częściej, jedna przygoda zajmuje kilka sesji.

{\bfseries Przygoda}: Pojedyncza część kampanii z początkiem i końcem. Zazwyczaj zdefiniowana na początku przez wspólny cel BG i na końcu przez to, czy go osiągnęli, czy też nie. 

{\bfseries Kampania}: Seria sesji połączona wspólną historią (lub połączonymi historiami) z tymi samymi BG. Często, lecz nie zawsze, kampania to zbiór przygód.

{\bfseries Postać}: Cokolwiek, co podejmuje akcje w grze. Choć wliczają się w to BG i ludzcy BN-i, technicznie wliczają się w to potwory, kosmici, mutanci, automatony, ruchome rośliny itp. Synonimem jest „istota” bądź „potwór”.

\section {Zasięg i szybkość}\index{Zasięg i szybkość}

Zasięg dzieli się na 4 ogólnikowe kategorie: bliski, średni, daleki i bardzo daleki.

Bliski zasięg to odległość ręki lub paru kroków. Jeśli postać stoi w małym pokoju, wszystko wokół jest w jej bliskim zasięgu. Górna granica bliskiego zasięgu to 3 metry.

Średni zasięg to wszystko, co jest większe od bliskiego, ale mniejsze niż 15 metrów.

Długi dystans to wszystko większe od średniego zasięgu, ale mniejsze niż 30 metrów.

Bardzo długi zasięg to wszystko większe od długiego dystansu, ale nie większe niż 150 metrów. Poza tym zasięgiem, odległości zawsze się ściśle określone – 300 metrów, 1,5 kilometra itp.

Ogólną ideą tego systemu jest to, że nie trzeba dokładnie mierzyć i określać odległości. Bliski zasięg to tutaj, obok postaci. Średni zasięg to blisko postaci. Długi dystans jest dalej, a bardzo długi – znacznie dalej.
Wszystkie bronie i specjalne zdolności korzystają z tych terminów. Dla przykładu, wszystkie bronie do walki wręcz mają bliski zasięg – służą przecież do walki wręcz i można je użyć tylko na osobach, które stoją obok nas. Nóż do rzucania (i większość innych broni rzucanych) mają średni zasięg. Łuk ma długi zasięg. Pocisk Adepta także ma średni zasięg.

Postać może się przemieścić o bliski zasięg jako część innej akcji. Innymi słowy, może ona podejść do panelu kontrolnego i z niego skorzystać. Może przejść przez mały pokój i zaatakować wroga. Mogę otworzyć drzwi i przejść przez nie.

Postać może się przemieścić o średni zasięg jeśli poświęci na to całą akcję w turze. Może także spróbować się przemieścić o długi zasięg w jednej akcji, ale trzeba wykonać rzut, by stwierdzić, czy postać się nie poślizgnęła lub  przewróciła w efekcie tak szybkiego ruchu.

Dla przykładu, jeśli BG walczą z grupą kultystów, każda postać może zaatakować, ogólnie rzecz ujmując, dowolnego kultystę wręcz – wszyscy są w zasięgu. Dokładne pozycje nie są tak ważne. Istoty w walce zawsze się zresztą poruszają. Jednakże, jeden z kultystów został z tyłu by wystrzelić z pistoletu i BG może musieć poświęcić całą akcję, by się do niego dostać. Nie ma większego znaczenia, czy ten kultysta jest 6 metrów od postaci graczy, czy może 12 – po prostu jest w średnim zasięgu. Ma znaczenie, czy kultysta stoi o więcej niż 15 metrów od BG, ponieważ wtedy zasięg by się zwiększył do dalekiego.

(Wiele zasad w tej grze unika konieczności nadmiernej precyzji. Czy naprawdę się liczy to, czy duch jest o 13, czy o 18 stóp od Ciebie? Najpewniej nie. Taki rodzaj niepotrzebnej ścisłości tylko spowalnia rozgrywkę i odciąga uwagę od akcji i fabuły, zamiast być miłym dodatkiem do opowiadanej historii.)

\section {Punkty doświadczenia}\index{Punkty doświadczenia!Wstęp}

Punkty doświadczenia (PD-ki) są nagrodą dawaną graczom, gdy GM wtrąca się narrację (nazywamy to Wtrąceniem MG) z nowym i niespodziewanym wyzwaniem. Dla przykładu, w środku walki, MG może poinformować graczy, że upuszczają oni swoje bronie. Jednakże, aby się wtrącić w taki sposób, MG musi dać graczowi 2 PD-ki. Nagrodzony gracz, z kolei, musi natychmiast dać jednego z owych PD-ków innemu graczowi, uzasadniając to (może ten gracz miał dobry pomysł, powiedział zabawny żart, wykonał akcję, która ocaliła życie jakiegoś BN/BG itp.).

Alternatywnie, gracz może odrzucić Wtrącenie MG. Jeśli on tak uczyni, nie otrzymuje on 2 PD-ków od GM, i musi wydać i PD z posiadanych przez siebie. Jeśli gracz nie ma PD-ków, nie może odrzucić Wtrącenia MG.

MG może także dać graczom PD-ki pomiędzy sesjami, jako nagrody za dokonywanie odkryć podczas gry. Odkrycia to ciekawe fakty, cudowne sekrety, potężne artefakty, odpowiedzi na pytania lub rozwiązania problemów (np.: gdzie porywacze przetrzymują swoje ofiary lub jak gracze naprawią statek kosmiczny). Nie otrzymujesz PD-ków za zabijanie potworów lub przezwyciężanie zwykłych trudności podczas gry. Odkrycia są duszą Cypher System.

Punkty Doświadczenia głównie służą awansowaniu postaci na poziomy (po detale, patrz: rozdział Tworzenie Własnej Postaci), ale gracz może także wydać 1 PD-ek, by przerzucić kość i wybrać leszy z dwóch wyników. 

\section  {Cyphery}\index{Cyphery!Wstęp}

Cyphery to zdolności, z których można skorzystać tylko raz. W wielu kampaniach, cyphery nie są fizycznymi obiektami – mogą być zaklęciem rzuconym na postać, błogosławieństwem od boga, lub po prostu zrządzeniem losu, które daje chwilową przewagę. W pewnych kampaniach, cyphery to obiekty fizyczne które postaci mogą z sobą nosić. Niezależnie od tego, czy cyphery to przedmioty, czy też nie, są częścią postaci (tak jak ekwipunek lub specjalna zdolność). I są czymś, z czego postać może skorzystać podczas gry. Forma, którą przyjmują fizyczne cyphery, zależy od settingu. W świecie fantasy mogą być różdżkami lub eliksirami, ale w grze science fiction mogą być obcymi kryształami lub prototypowymi technologiami.

Postaci często będą znajdowały nowe cyphery, więc gracze powinni równie często z nich korzystać. Ponieważ cyphery zawsze będą odmienne od innych cypherów, postać zawsze będzie miała nowe specjalne zdolności do wypróbowania. 

\section {Inne kości}

W dodatku do k20, potrzebujesz jeszcze k6 (sześciościennej kostki). Czasami będziesz potrzebował k100 (do losowania numerów od 1 do 100), co można osiągnąć, rzucając k20 dwa razy – ostatnia liczba pierwszego rzutu to “dziesiątki” a ostatnia liczba drugiego rzutu to “jedności”. Dla przykładu, rzut 17 i 9 daje nam 79, a 3 i 18 daje nam 38, a rzucenie 20 i 10 daje nam 00 (także znane jako 100). Jeśli masz k10 (dziesięciościenną kostkę) możesz skorzystać z niej zamiast z k20 by losować liczby od 1 do 100.

(k6 jest najczęściej wykorzystywana do rzutów na odzyskiwanie zdrowia i do określania poziomu cypherów).
\input{src/Tworzenie własnej postaci.tex}
\section{Typ}\index{Typ}

Typ postaci to najważniejsza cecha Twojej postaci. Twój typ pozwala określić miejsce postaci w świecie i jej relację z innymi ludźmi. Jest to rzeczownik w zdaniu “Jestem przymiotnik rzeczownik który czasownikuje”.

(W pewnych grach RPG, typ postaci może zostać nazwany klasą postaci.)

Możesz wybrać z 4 typów postaci: Wojownika, Adepta, Odkrywcy i Mówcy. Jednakże, możesz nie chcieć korzystać z tych ogólnikowych nazw na nie. Ten rozdział oferuje parę bardziej specyficznych nazw na każdy typ, które mogą być stosowne, w zależności od świata przedstawionego. Odkryjesz, że nazwy takie jak “Wojownik” czy też “Odkrywca” nie zawsze pasują do gier dziejących się w świecie współczesnym. Jak zawsze, możesz zrobić, co uznasz za stosowne. (Twój typ określa kim jest postać. Powinieneś korzystać z dowolnej nazwy na typ, tak długo, jak pasuje zarówno do postaci, jak i do settingu.)

Ponieważ typ to podstawa, na której się buduje postać, warto się zastanowić, jaka relacja łączy go z settingiem. Aby z tym pomóc, typy to w zasadzie ogólne archetypy. Wojownik, dla przykładu, może być wszystkim, od rycerza w lśniącej zbroi, przez gliniarza na ulicy po cybernetycznego weterana tysiąca futurystycznych wojen.
Aby lepiej dostosować cztery typy do różnych settingów, istnieją różne metody zwane posmakami,  zaprezentowane w stosownym rozdziale, by pomóc w dostosowaniu typów do konwencji fantasy, science fiction, lub innych (lub by dostosować typy do pomysłu na postać).

Dalej, bardziej fundamentalne opcje dla \mytext{dalszej customizacji} są dostępne na końcu tego rozdziału. 

\subsubsection{Wtrącenie Gracza}\index{Wtrącenie Gracza}

Wtrącenie gracza oznacza, że gracz wybiera zmianę czegoś w kampanii, czyniąc rzeczy łatwiejszymi dla jego postaci. Konceptualnie, jest to przeciwieństwo wtrącenia MG: zamiast MG dawać PD graczowi i wprowadzać niespodziewaną komplikacje dla jego postaci, gracz wydaje 1 PD i wprowadza rozwiązanie problemu lub komplikacji. To, co może zrobić wtrącenie gracza, to zmienić świat gry lub obecne okoliczności zamiast bezpośrednio zmieniać postać. Dla przykładu, wtrącenie mówiące, że cypher, z którego właśnie się skorzystało, ma dodatkowe użycie byłoby właściwe, ale wtrącenie uzdrawiające postać nie byłoby. Jeśli gracz nie ma PD-ków do wydania, nie może wprowadzić wtrącenia gracza. 

Parę wtrąceń gracza jest zasugerowanych pod każdym typem. Warto jednak zaznaczyć, że nie każde wtrącenie gracza jest stosowne w każdej sytuacji. MG może zezwolić graczom na inne sugestie wtrąceń, ale ostatecznie to on decyduje, czy dane wtrącenie jest stosowne do typu postaci i danej sytuacji. Jeśli MG odmawia wtrącenia, gracz nie wydaje 1 PD-ka i wtrącenie nie następuje.

Korzystanie z intruzji nie wymaga od postaci akcji, by je zastosować. Po prostu ono następuje.

(Wtrącenie gracza powinno być ograniczone do nie więcej niż jednego wtrącenia na gracza na jedną sesję.)

\subsubsection{Akcje obronne}\index{Akcje obronne}

Akcje obronne występują wtedy, gdy gracz rzuca, by uchronić się od czegoś nieporządanego, co mogłoby się wydarzyć jego BG. Rodzaj akcji obronnej ma znaczenie, gdy rozważamy Wysiłek.

\textbf{Obrona Mocy}: Używa się jej do odporności na trucizny, choroby i wszystko inne, co można przezwyciężyć siłą i zdrowiem.

\textbf{Obrona Szybkości}: Używa się jej do unikana ciosów i uciekana od niebezpieczeństw. To najczęściej wykorzystywany rodzaj akcji obronnej.

\textbf{Obrona Intelektu}: Używa się jej do odpieranie ataków mentalnych i wszystkiego, co może wpłynać na czyiś umysł.
\cleardoublepage

\subsection{Wojownik}\index{Typ!Wojownik}

Fantasy/Baśń: Wojownik, rycerz, barbarzyńca, żołnierz, walkyria.

Współczesność/Horror/Romans: policjant, żołnierz, strażnik, detektyw, ochroniarz, atleta.

Science fiction: oficer bezpieczeństwa, wojownik, żołnierz, najemnik.

Superbohaterowie/Post-apokalipsa: bohater. 

Jesteś dobrym sprzymierzeńcem w potyczce. Wiesz, jak korzystać z broni i chronić siebie. W zależności od konwencji i settingu, może to znaczyć, że nosisz miecz i tarczę na arenie gladiatorów, posiadasz karabin maszynowy i zestaw granatów przydatne w wymianie ognia, lub posiadasz blastera i zasilany pancerz, z których korzystasz na obcej planeci

Rola w grze: Wojownicy są fizyczni i zorientowani na akcję. Cześciej rozwiązują problemy, korzystając z siły, niż na inne sposoby, i często wybierają najprostszą drogę do osiągnięcia swoich celów. 

Rola w drużynie: Wojownicy najczęściej zadają i biorą na klatę najwięcej obrażeń w bitwie. To od nich zależy obrona reszty członków drużyny przed atakami. To czasami oznacza, że wojownicy bywają liderami, przynajmniej w walce i w obliczu innych niebezpieczeństw.

Rola społeczna: Wojownicy nie zawsze są żołnierzami lub najemnikami. Każdy, kto jest gotów do odrobinę przemocy w swoim życiu, lub choćby jej potencjał, może być Wojownikiem, mówiąc ogólnikowo. Wliczają się w to strażnicy, policjanci, marynarze lub ludzie innych profesji, którzy wiedzą, jak się bronić.

Zaawansowani Wojownicy: W miarę, jak wojownicy awansują na poziomy, ich umiejętności bitewne – zarówno obrony, jak i ataku – zwiększają się do niemożliwych poziomów. Na wyższych poziomach, mogą oni często przeciwstawić się grupom wrogów lub stanąć 1-na-1 przeciwko dowolnemu przeciwnikowi.

\subsubsection{Historia Wojownika}

Twój typ pomaga Ci określić połączenie Twojej postaci z settingiem. Rzuć k20 lub wybierz z poniższej listy, by określić pewien fakt odnośnie Twojej historii, który łączy Twoją postać ze światem. Możesz także stworzyć swój własny fakt historyczny.

\begin{table*}[t]
 \centering
 \begin{tabularx}{\textwidth}{| p{0.10\textwidth} | X |}
  \hline
  \textbf{k20} & \textbf{Historia Wojownika}  \\ \hline
    1 & Byłeś w armii i dalej masz przyjaciół, którzy tam są. Twój były dowódca dobrze Cię pamięta. \\ \hline
    2 & Byłeś ochroniarzem bogatej kobiety, która oskarżyła Cię o kradzież. Opuściłeś jej służbę w cieniu podejrzeń. \\ \hline
    3 & Byłeś ochroniarzem lokalnego baru, i właściciele pamiętają Cię.  \\ \hline
    4 & Trenowałeś z szanowanym mentorem. Trzyma on Cię w estymie, ale ma wielu wrogów. \\ \hline
    5 & Trenowałeś w odosobnionym zakonie. Mnisi myślą o Tobie jak o bracie, ale dla wszystkich innych jesteś obcym. \\ \hline
    6 & Nie masz formalnego wyszkolenia. Twoje zdolności po prostu są (naturalnie bądź nie). \\ \hline
    7 & Spędziłeś czas na ulicach i byłeś przez pewien czas w więzieniu. \\ \hline
    8 & Zapisano Cię do służby wojskowej, ale uciekłeś po niedługim czasie. \\ \hline
    9 & Służyłeś jako ochroniarz dla potężnego kryminalisty, który teraz jest Ci winny życie. \\ \hline
    10 & Pracowałeś jako oficer policji lub detektyw. Każdy Cię zna, ale opinie o Tobie są różne. \\ \hline
    11 & Twoje starsze rodzeństwo to niesłynna postać, która żyje w hańbie.  \\ \hline
    12 & Służyłeś jako strażnik komuś, kto dużo podróżował. Znasz ludzi w wielu miejscach. \\ \hline
    13 & Twój najlepszy przyjaciel to nauczyciel lub badacz. Jest on świetnym źródłem wiedzy. \\ \hline
    14 & Ty i Twój przyjaciel palicie ten sam rodzaj rzadkiego, drogiego tytoniu. Spotykacie się co tydzień, by porozmawiać i zapalić. \\ \hline
    15 & Twój wuj prowadzi teatr w mieście. Znasz wszystkich aktorów i masz wolny wstęp na występy. \\ \hline
    16 & Twój przyjaciel-rzemieślnik czasami prosi Cię o pomoc. Jednakże, płaci on dobrze. \\ \hline
    17 & Twój mentor napisał książkę o sztukach walki. Czasami ludzie pragną Cię odszukać i zapytać o jej dziwne zapisy. \\ \hline
    18 & Ktoś, z kim walczyłeś ramię w ramię w armii, teraz jest burmistrzem lokalnego miasteczka. \\ \hline
    19 & Ocaliłeś życie rodziny, gdy jej dom płonął. Mają oni u Ciebie dług, a ich sąsiedzi traktują Ciebie jak bohatera. \\ \hline
    20 & Twój stary trener dalej spodziewa się, że wrócisz i sprzątniesz po jego zajęciach; gdy to robisz, dzieli się on z Tobą okazjonalnie ciekawymi plotkami. \\ \hline
 \end{tabularx}
  \caption {Historia Wojownika}
  \label {Historia Wojownika}
 \end{table*}
 
\subsubsection{Wojownik - Wtrącenia Gracza}

Możesz wydać 1 PD by skorzystać z poniższych wtrąceń gracza, jeśli jest to stosowne do sytuacji, a MG się zgodzi.

Perfekcyjna pozycja: Walczysz przynajmniej z trzema wrogami i każdy z nich stoi w odpowiednim miejscu, możesz więc wykorzystać ruch, który ćwiczyłeś dawno temu, co pozwala Ci zaatakować wszystkich trzech w jednej akcji. Wykonaj odrębne rzuty na atak dla kazego z wrogów. Jesteś ograniczony Wysiłkiem, który możesz wykorzystać w jednej akcji.

Stary Przyjaciel: Towarzysz borni z przeszłości pojawia sięnagle i pomaga w tym, co teraz robisz. Jest on na własnej misji i nie może zostać dłużej niż czas potrzebny na udzielenie pomocy, porozmawianie przez chwilę i być może na wspólne zjedzenie szybkiego posiłku.

Słabość Broni: Broń Twojego przeciwnika ma słaby punkt. Podczas walki, szybko się ona psuje i spada o dwa stopnie w dół na \mytext{liczniku obrażeń przedmiotu}.

\begin{table*}[t]
 \centering
 \begin{tabularx}{\textwidth}{ | X | X |}
  \hline
   \textbf{Statystyka} & \textbf{Początkowa Wartość Puli}  \\ \hline
    Moc & 10  \\ \hline
    Szybkość & 10  \\ \hline
    Intelekt & 8  \\ \hline
 \end{tabularx}
  \caption {Pule Statystyk Wojownika}
  \label {Pule Statystyk Wojownika}
 \end{table*}
 
 Otrzymujesz dodatkowe 6 punktów do podziału pomiędzy Pule, jakkolwiek sobie życzysz.
 
\subsubsection{Wojownik pierwszego poziomu}

Pierwszo-poziomowi wojownicy mają następujące zdolności:

Wysiłek: Twój Wysiłek to 1.

Fizyczna Natura: Masz Skupienie w Mocy 1 i Skupienie w Szybkości 0 lub Skupienie w Mocy 0 i Skupienie w Szybkości 1. Niezależnie od tego, Twoje Skupienie w Intelekcie to 0.

Korzystanie z Cypherów: Możesz nosić dwa Cyphery w danym czasie.

Bronie: Jesteś wytrenowany w lekkich, średnich i ciężkich broniach i nie stosuje siędo Ciebie kara za używanie jakiegokolwiek rodzaju broni. Umożliwienie.

Początkowy Ekwipunek: Odpowiednie ubranie i dwie bronie Twojego wyboru, plus jeden drogi przedmiot, dwa przedmioty średniej ceny i cztery niedrogie.

Specialne Zdolności: Wybierz cztery zdolności z poniższej listy. Nie możesz wybrać tej samej zdolności więcej niż raz, chyba, że jej opis mówi inaczej. Pełny opis wszystkich zdolności znajduje się w rozdziale \mytext{Zdolności}, który także zawiera opis Posmaków i zdolności Specjalizacji w pojedynczym, sporym katalogu.

\begin{itemize}
\item Broń Niepotrzebna
\item Kontrola Bitewna
\item Na Straży
\item Ogłuszenie
\item Szybki Rzut
\item Ulepszone Skupienie
\item Umiejętności Fizyczne
\item Wyszkolony Bez Zbroi
\item Wyszkolony w Zbroi
\item Zamach
\item Zdolności Bojowe
\end{itemize}

\subsubsection{Wojownik Drugiego Poziomu}

Wybierz dwie zdolności z poniższej lisy (lub z niższego poziomu) i dodaj je do swoich zdolności. Dodatkowo, możesz zamienić jedną ze zdolności niższego poziomu na inną z niższego poziomu.

\begin{itemize}
\item Krwawienie
\item Miażdżący Cios
\item Następny Atak
\item Przeładowanie
\item Umiejętny Atak
\item Umiejętna Obrona
\end{itemize}

\subsubsection{Wojownik Trzeciego Poziomu}

Wybierz trzy zdolności z poniższej listy (lub z niższego poziomu) i dodaj je do swoich zdolności. Dodatkowo, możesz zamienić jedną ze zdolności niższego poziomu na inną z niższego poziomu.

\begin{itemize}
\item Chwytaj Moment
\item Cięcie
\item Cios z Wyciągnięciem
\item Czujność
\item Ekspercki Użytkownik Cypherów
\item Furia
\item Odporność na Energię
\item Ostrzał Ciągły
\item Podwójny Strzał
\item Przywykły do Noszenia Zbroi
\item Reakcja
\item Śmiertelna Salwa
\end{itemize}

\subsubsection{Wojownik Czwartego Poziomu}

Wybierz dwie z poniższych zdolności (lub z niższego poziomu) i dodaj je do swoich zdolności. Dodatkowo, możesz zamienić jedną z zdolności niższego poziomu na inną z niższego poziomu.

\begin{itemize}
\item Dodatkowy Wysiłek
\item Doświadczony Obrońca
\item Finta
\item Pęd
\item Przełamanie Obrony
\item Wycelowanie
\item Wyjątkowo Wytrzymały
\item Zręczny Wojownik
\item Zwiększony Efekt
\end{itemize}

\subsubsection{Wojownik Piątego Poziomu}

Wybierz trzy zdolności z poniższej listy (lub z niższego poziomu) i dodaj je do swoich zdolności. Dodatkowo, możesz zamienić jedną ze zdolności niższego poziomu na inną zdolność niższego poziomu.

\begin{itemize}
\item Atak z Wyskoku
\item Blok
\item Mistrzostwo Ataków
\item Mistrzostwo Obrony
\item Mistrzowska Biegłość w Pancerzach
\item Ulepszony Sukces
\item Potrójny Wystrzał
\item Zaawansowany Użytkownik Cypherów
\end{itemize}

(Pamiętaj, że na wyższych poziomach, można wybrać zdolności z niższych poziomów. Czasami jest to najlepszy sposób, by uzyskać dokładnie taką postać, jakiej pragniesz. Jest to zwłaszcza prawdziwe odnośnie zdolności, które zapewniają umiejętności, które zazwyczaj można wybrać więcej niż jeden raz.)

\subsubsection{Wojownik Szóstego Poziomu}

Wybierz dwie ze zdolności z poniższej listy (lub z niższego poziomu) i dodaj je do swoich zdolności. Dodatkowo, możesz zamienić jedną ze zdolności niższego poziomu na inną z niższego poziomu.

\begin{itemize}
\item Broń i Cios
\item Chwila Wspaniałości
\item Morderca
\item Ostateczny Cios
\item Wielokrotny Atak
\item Znowu i Znowu
\end{itemize}

\subsubsection{Przykładowy Wojownik}

Ray chce stworzyć Wojownika do współczesnej kampanii. Decyduje się on na byłego członka armii, który jest silny i szybki. 3 z wolnych punktów idą do Puli Mocy, a pozostałe 3 do Puli Szybkości. Jego Statystyki to teraz Moc 13, Szybkość 13 i Intelekt 8. Jako, że postać jest na 1-szym  poziomie, jej Wysiłek to 1, jej Skupienie w Mocy to 1, a Skupienie w Szybkości i Intelekcie to 0. Jego postać nie jest szczególnie mądra lub charyzmatyczna.

Chce on korzystać z dużego noża bojowego (średnia broń, która zadaje 4 punkty obrażeń) i .357 Magnum (ciężki pistolet, który zadaje 6 punktów obrażeń, ale wymaga dwóch rąk do korzystania). Ray decyduje się na nie noszenie żadnej zbroi, gdyż nie pasuje to do settingu, tak więc, jako swoją pierwszą zdolność wybiera \mytext{Wyszkolony Bez Zbroi}, co ułatwia jego Obronę Szybkości. Jako drugą zdolność wybiera \mytext{Zdolności Bojowe}, by zadawać większe obrażenia swoim wielkim nożem. 

Ray chce być zarówno szybki, jak i wytrzymały, wybiera więc \mytext{Ulepszone Skupienie}. Daje mu to Skupienie w Szybkości na 1. Jako ostatnią zdolność, wybiera \mytext{Umiejętności Fizyczne} i wybiera pływanie i bieganie. 

Wojownik może mieć przy sobie maksymalnie 2 cyphery. GM decyduje, że pierwszy cypher Raya to pigułka, która regeneruje 6 punków Mocy po połknęciu, a jego drugi cypher to mały, łatwy do ukrycia granat, który eksploduje jak ognista bomba, gdy go się rzuci, zadając 3 punkty obrażeń wszystkim w bliskim zasięgu. 

Ray dalej musi wybrać deskryptor i specjalizację. Przeglądając deskryptory, Ray wybiera \mytext{Silnego}, co zwiększa jego Pulę Mocy do 17. Jest także wytrenowany w skakaniu i niszczeniu przedmiotów. (Jeśli Ray wybrałby skakanie jako jedną ze swoich umiejętności fizycznych, teraz dzięki deskryptorowi byłby wyspecjalizowany w skakaniu, zamiast być wytrenowanym). Bycie Islnym daje też Ray’owi dodatkową średnią bądź ciężką broń. Wybiera kij baseballowy, który przechowuje w bagażniku swojego auta. 

Jako swoją specjalizację, Ray wybiera \mytext{Mistrzowsko Posługuje się Bronią}. Daje mu to kolejną broń wysokiej jakości. Wybiera dodatkowy nóż bojowy i pyta MG, czy może z niego korzystać w lewej ręce – nie do wykonywania ataków, lecz jako tarczę. To ułatwi jego rzuty na Obronę Szybkości, jeśli ma obydwie bronie w rękach (“tarcza” liczy się jako atut). MG się zgadza. Podczas gry, ciężko będzie trafić Wojownika Ray’a – kjest wytrenowany w rzutach na Obronę Szybkości, a jego dodatkowy nóż obniża rzuty o kolejny stopień. 

Dzięki jego specjalizacji, zadaje także dodatkowy 1 punkt obrażeń w walce swoją wybranąbronią. Teraz zadaje 6 punktów obrażeń swoim ostrzem. Postać Raya to śmiercionośny wojownik, zapewne rozpoczynający grę z reputacją jako walczący nożami. 

Jako swój motyw fabularny, Ray wybiera \mytext{Pokonać Wroga}. Ten wróg, Ray decyduje, to nikt inny jak jego stary przyjaciel z armii, który wszedł na ścieżkę zła.
\subsection{Adept}\index{Typ!Adept}

Fantasy/baśń: mag, czarodziej, czarnoksiężnik, kleryk, druid, jasnowidz, diabolista, dotknięty przez Fae.

Współczesność/Horror/Romans: psionik, okultysta, wiedźma, praktykujący magię, medium, szalony naukowiec.

Science fiction: psionik, telepata, jasnowidz, skanujący, ESP-er, abominacja.

Superbohaterowie/Post-apokalipsa: mag, czarownik, dzierżący moc, psionik, telepata.

Władasz mocami i zdolnościami poza ludzkim doświadczeniem, zrozumienie i czasami wiarą. Może to być magia, psionika, zdolności mutanta, lub po prostu skomplikowane urządzenia, w zależności od settingu. (“Magia” to termin, który stosujemy tutaj bardzo luźno. To termin na wszystkie wspaniałe, możliwie nadnaturalne rzeczy, które może zrobić Twoja postać, a inne nie mogą. Może to być skutek posługiwania się odpowiednim sprzętem, kontaktu z duchami, mutacji, psioniki, nanotechnologii lub innych źródeł.)

Rola w grze:Adepci to zazwyczaj inteligentni, myślący ludzie. Bardzo często myślą ostrożnie, zanim podejmą akcję i polegają na swoich nadnaturalnych zdolnościach.

Rola w drużynie: Adepci nie są potężni w bezpośredniej walce, choć często posiadają zdolności, które są wspaniałym uzupełnieniem zdolności bojowych ich towarzyszy, zarówno defensywnie, jak i ofensywnie. Czasami posiadają zdolności, które pomagają im przezwyciężać trudności i wyzwania. Dla przykładu, jeśli grupa musi się przedostać przez zamknięte drzwi, Adept może być w stanie je zniszczyć lub przeteleportować wszystkich na ich drugą stronę.

Rola społeczna: W settingach w których moce nadnaturalne są rzadkie, tajemnicze lub wywołują strach, Adepci są zazwyczaj także rzadcy i wywołujący strach. Pozostają wtedy w ukryciu. Kiedy jest inaczej, Adepci są częstsi i bardziej bezpośredni. Mogą nawet zostać liderami swoich społeczności.

Zaawansowani Adepci: Nawet na niższych poziomach, moce Adeptów zapierają dech w piersiach. Na wyższych poziomach, Adepci mogą dokonać prawdziwie wielkich czynów, które mogą przekształcić materię i środowisko wokół nich.
(Adepci prawie zawsze są paranormalni lub nadludzcy w jakimś sensie – czarodzieje, psionicy itp. Jeśli gra, w którą gracie, nie posiada takich postaci, Adept mógłby być szarlatanem, który udaje magiczne zdolności przy pomocy trików i ukrytych urządzeń, lub gadżeciarzem z “przydatnym paskiem” pełnym dziwnych narzędzi. Lub w Twoim świecie może nie być Adeptów. To także jest ok.)

\subsubsection{Adept - Wtrącenia Gracza}

Kiedy grasz Adeptem, możesz wydać 1 PD na jedne z poniższych wtrąceń gracza, jeśli sytuacja jest stosowna i MG się zgodzi.
 
Przydatna Awaria: Urządzenie, z którego korzysta się przeciwko Tobie, ulega awarii. Może ono zranić użytkownika lub jednego z jego sprzymierzeńców w ciągu jednej tury, lub aktywować dramatyczny i rozpraszający efekt uboczny, trwający parę tur.

Nagłe Olśnienie: Doświadczasz nagłego olśnienia, które zapewnia jasną odpowiedz lub sugeruje następne kroki w temacie ważnego pytanie, problemu lub przeszkody na Twojej drodze. 

Cudowna Aktywacja: Nieaktywne, zrujnowane lub najwyraźniej-zniszczone urządzenia chwilowo się aktywuje i wykonuje przydatną akcję w kontekście obecnej sytuacji. Może to kupić Ci trochę czasu na znalezienie lepszego rozwiązania, przezwyciężyć komplikację która wpływa na Twoje moce, lub po prostu umożliwić skorzystanie z zużytego cyphera lub artefaktu jeszcze raz. 

\begin{table*}[t]
 \centering
 \begin{tabularx}{\textwidth}{ | X | X |}
  \hline
  \textbf{ Statystyka} & \textbf{Początkowa Wartość Puli}  \\ \hline
    Moc & 7 \\ \hline
    Szybkość & 9 \\ \hline
    Intelekt & 12 \\ \hline
 \end{tabularx}
  \caption {Pule Statystyk Adepta}
  \label {Pule Statystyk Adepta}
 \end{table*}
 
 Otrzymujesz 6 dodatkowych punktów do podziału pomiędzy Pule statystyk, zgodnie z własną wolą.
 
\subsubsection{Historia Adepta}

Twój typ pomaga Ci określić Twoje miejsce w settingu. Rzuć k20 lub wybierz z poniższej listy, by określić konkretny fakt odnośnie Twojej historii, która łączy Cię z resztą świata. Możesz także stworzyć swój własny fakt. 

 \begin{table*}[t]
 \centering
 \begin{tabularx}{\textwidth}{| p{0.10\textwidth} | X |}
  \hline
  \textbf{d20} & \textbf{Historia Adepta}  \\ \hline
    1 & Służyłeś jako uczeń u Adepta, którego respektowało i bało się wielu ludzi. Teraz nosisz jego brzemię. \\ \hline
    2 & Studiowałeś w szkole słynącej z jej mrocznych nauczycieli i absolwentów. \\ \hline
    3 & Nauczyłeś się swoich zdolności w świątyni mało znanego boga. Jego kapłani i wierni, choć niezbyt liczni, respektują i adorują Twoje talenty i potencjał. \\ \hline
    4 & Kiedy podróżowałeś samotnie, ocaliłeś życie potężnej osoby. Ma ona względem Ciebie dług wieczności. \\ \hline
    5 & Twoja matka była potężnym Adeptem za życia, pomagała też ludziom w okolicy. Patrzą oni na Ciebie ciepło, ale także spodziewają się wiele po Tobie. \\ \hline
    6 & Wisisz pieniądze wielu ludziom i nie masz pieniędzy, by spłacić swój dług. \\ \hline
    7 & Zaliczyłeś gigantyczną klęskę w swoich początkowych studiach z nauczycielem i teraz uczysz się na własną rękę. \\ \hline
    8 & Nauczyłeś się swoich zdolności szybciej, niż Twoi nauczyciele widzieli u któregokolwiek ze swoich uczniów. Potężni tego świata zwrócili na Ciebie swoją uwagę i obserwują Cię intensywnie.  \\ \hline
    9 & Zabiłeś dobrze znanego kryminalistę w samoobronie, zyskując respekt wielu i nieprzyjaźń paru niebezpiecznych ludzi. \\ \hline
    10 & Uczyłeś się na Wojownika, ale Twoje uzdolnienia w kierunku Adepta ostatecznie skierowały Cię na odmienną ścieżkę. Twoi dawni kompani nie rozumieją Cię, ale mimo to Cię szanują. \\ \hline
    11 & Kiedy studiowałeś na Adepta, pracowałeś jako asystant w banku, zaprzyjaźniajac się z właścicielem i klientami. \\ \hline
    12 & Twoja rodzina posiada wielką winnicę niedaleko, znaną ze swojego dobrego wina i uczciwości biznesowej. \\ \hline
    13 & Trenowałeś przez pewien czas z grupą wpływowych Adeptów, którzy dalej darzą Cię przyjaźnią. \\ \hline
    14 & Pracowałeś w ogrodach pałacowych wpływowego szlachcica lub bogatej osoby. Nie pamięta ona Cię, ale zaprzyjaźniłeś się z jej młodą córką. \\ \hline
    15 & Eksperyment, który przeprowadziłeś w przeszłości, kompletnie nie wypalił. Ludzie z tamtej okolicy zapamiętali Cię jako niebezpiecznego i bezmyślnego typka. \\ \hline
    16 & Pochodzisz z dalekiego miejsca, gdzie byłeś dobrze znany i traktowany, ale ludzie tutaj traktują Cię z dużą podejrzliwością. \\ \hline
    17 & Ludzie, których spotykasz, wydają się trzymać na dystans ze względu na dziwne piętna na Twojej twarzy. \\ \hline
    18 & Twój najlepszy przyjaciel to także Adept. Ty i Twój przyjaciel dzielicie się odkryciami i sekretami. \\ \hline
    19 & Znasz lokalnego kupca bardzo dobrze. Ponieważ zapewniłeś mu dużo przychodu, oferuje Ci on zniżki i specjalne traktowanie.  \\ \hline
    20 & Należysz to sekretnego klubu, który spotyka sieco miesiąc, by wypić i porozmawiać. \\ \hline
 \end{tabularx}
  \caption {Historia Adepta}
  \label {Historia Adepta}
 \end{table*}
 
 \subsubsection{Adept Pierwszego Poziomu}
 
 Pierwszo-poziomowi Adepci posiadają następujące zdolności:
 
Wysiłek: Twój Wysiłek to 1.

Geniusz: Masz Skupienie w Intelekcie 1 oraz Skupienie w Mocy i Szybkości 0.

Eksperckie Korzystanie z Cypherów: Możesz nosić 3 cyphery w danym czasie.

Początkowy Ekwipunek: Stosowne ubranie, plus 2 drogie przedmioty, dwa przedmioty średniej ceny i do 4 niedrogich przedmiotów Twojego wyboru.

Bronie: Możesz korzystać z lekkich broni bez żadnej kary. Posiadasz nieumiejętność w średnich i ciężkich broniach – Twoje ataki z średnimi i ciężkimi broniami są utrudnione.

Specjalne zdolności: Wybierz 4 zdolności z poniższej listy. Nie możesz wybrać danej zdolności więcej niż 1 raz, chyba, że jej opis stanowi inaczej. Pełny opis każdej z dostępnych zdolności znajduje się w rozdziale \mytext{Zdolności}, który zawiera także zdolności Posmaków i specjalizacje w jednym, rozbudowanym katalogu. (Zdolności Adepta wymagają przynajmniej jednej wolnej ręki, chyba, że MG mówi inaczej.)

\begin{itemize}
\item Zamglenie
\item Usunięcie Wspomnień
\item Daleki Krok
\item Sztuczki Magiczne
\item Trening Magiczny
\item Pocisk
\item Pchnięcie
\item Pole Renozansowe
\item Skan
\item Strzaskanie
\item Magia Obronna
\end{itemize}

\subsubsection{Adept Drugiego Poziomu}

Wybierz jedną ze zdolności z poniższej listy (lub z niższego poziomu) i dodaj do swoich zdolności. Dodatkowo, możesz zamienić jedną ze zdolności z niższego poziomu na inną zdolność z niższego poziomu.

\begin{itemize}
\item Adaptacja
\item Czytanie Myśli
\item Odzyskanie Wspomnień
\item Ujawnienie
\item Unoszenie Się
\item Tnące Światło
\item Zastój
\end{itemize}

\subsubsection{Adept Trzeciego Poziomu}

Wybierz dwie zdolności z poniższej listy (lub z niższego poziomu) i dodaj do swoich zdolności. Dodatkowo, możesz wybrać jedną ze zdolności niższego poziomu i zamienić na inną z niższego poziomu. 

\cleardoublepage

\subsection{Odkrywca}\index{Typ!Odkrywca}

Fantasy/Baśń: Odkrywca, poszukiwacz przygód, badacz tajemnic.

Współczesność/Horror/Romans: atleta, odkrywca, poszukiwacz przygód, detektyw, badacz, pionier, reporter śledczy.

Science fiction: odkrywca, poszukiwacz przygód, podróżnik, planetolog, ksenobiolog.

Superbohaterowie/Post-apokalipsa: poszukiwacz przygód, stróż prawa.

Jesteś osobą akcji i fizycznych zdolności, bez lęku patrzącą ku nieodkrytemu. Podróżujesz do dziwnych, egzotycznych i niebezpiecznych miejsc, i odkrywasz nowe rzeczy. Oznacza to, że masz duże zdolności fizyczne, ale zapewne także jesteś dobrze wykształcony. 

Rola w grze: Choć Odkrywcy mogą być uczonymi i dobrze wykształconymi, są przede wszystkim zainteresowani akcją. Mierzą się ze śmiertelnymi niebezpieczeństwami i okropnymi przeszkodami praktycznie codziennie.

Rola w drużynie: Odkrywcy czasami pracują sami, ale częściej są częścią zespołu z innymi postaciami. Odkrywca często przoduje i przeciera szlak. Jednakże, często zatrzymują się i badają to, co ich zaintrygowało po drodze. 

Rola społeczna: Nie wszyscy Odkrywcy przedzierają się przez dzicz lub badają stare ruiny. Czasami, Odkrywca to nauczyciel, naukowiec, detektyw lub reporter śledczy. W każdym wypadku, Odkrywca z odwagą zmaga się z nowymi wyzwaniami i zbiera wiedzę, którą może się dzielić z innymi.

Zaawansowani Odkrywcy: Wysokopoziomowi Odkrywcy zyskują więcej umiejętności, trochę zdolności bojowych i dużo zdolności, które pomagają im poradzić sobie z niebezpieczeństwem. W skrócie, stają się uniwersalni, zdolni dać sobie radę z każdym wyzwaniem. 

\subsubsection{Odkrywca - Wtrącenia Gracza}

Kiedy grasz Odkrywcą, możesz wydać 1 PD by skorzystać z poniższych \mytext{wtrąceń gracza}, jeśli sytuacja jest odpowiednia i MG się zgadza.

Szczęśliwa Awaria: Pułapka lub niebezpieczne urządzenie doświadcza awarii, zanim może Ciebie zranić.

Nieoczekiwana Wskazówka: W momencie, gdy myślisz, że kompletnie zgubiłeś drogę, element krajobrazu, drogowskaz, lub po prostu ułożenie terenu sprawia, że odkrywasz najlepszą drogę naprzód, przynajmniej w tym momencie.

Słaba Trucizna: Trucizna lub choroba okazuje się nie być tak poważna, jak na początku wyglądała, i zadaje tylko połowę obrażeń, które zadałaby normalnie. 

\begin{table*}[t]
 \centering
 \begin{tabularx}{\textwidth}{ | X | X  |}
  \hline
   \textbf{Statystyka} & \textbf{Początkowa Wartość Puli} \\ \hline
    Moc & 10  \\ \hline
    Szybkość & 9  \\ \hline
    Intelekt & 9  \\ \hline
 \end{tabularx}
  \caption {Pula Statystyk Odkrywcy}
  \label {Pula Statystyk Odkrywcy}
 \end{table*}
 
 Otrzymujesz dodatkowe 6 punktów, które możesz rozdzielić pomiędzy swoje Pule zgodnie ze swoim życzeniem.
 
 \subsubsection{Historia Odkrywcy}
 
Twój typ pomaga Ci określić połączenie Twojej postaci z settingiem. Rzuć k20 lub wybierz z poniższej listy, by określić konkretny fakt o Twojej historii, który łączy Cię zresztą świata. Możesz także stworzyć swój własny fakt.

 \begin{table*}[t]
 \centering
 \begin{tabularx}{\textwidth}{| p{0.10\textwidth} | X |}
  \hline
  \textbf{k20} & \textbf{Historia Odkrywcy}  \\ \hline
    1 & Byłeś gwiazdą sportu w swoim liceum. Dalej jesteś w dobrej kondycji, ale człowieku, co to było wtedy! \\ \hline
    2 & Twój brat jest głównym śpiewakiem w naprawdę popularnym zespole. \\ \hline
    3 & Dokonałeś szeregu odkryć podczas swoich podróży, ale nie wszystkie okoliczności, by na nich zarobić, jeszcze pojawiły się przed Tobą. \\ \hline
    4 & Byłeś policjantem, ale zrezygnowałeś z pracy po doświadczeniu korupcji w siłach porządkowych. \\ \hline
    5 & Twoi rodzice byli misjonarzami, więc spędziłeś dużą część swojego młodego życia, podróżując do egzotycznych miejsc.  \\ \hline
    6 & Służyłeś w armii z honorem. \\ \hline
    7 & Otrzymałeś pomoc od sekretnej organizacji, która opłaciła Twoją edukację. Teraz ona pragnie znacznie więcej od Ciebie.  \\ \hline
    8 & Uczęszczałeś na prestiżowy uniwersytet dzięki stypendium dla sportowców, ale lśniłeś zarówno na boisku, jak i podczas zajęć.  \\ \hline
    9 & Twój najlepszy przyjaciel z dzieciństwa jest teraz wplywowym członkiem rządu. \\ \hline
    10 & Byłeś nauczycielem. Twoi studenci wspominają Cię miło. \\ \hline
    11 & Przez krótki czas byłeś kryminalistą, który został złapany i poszedł do więzenia – potem próbowałeś wyjść na prostą. \\ \hline
    12 & Twoje największe jak dotąd odkrycie zostało ukradzione przez Twojego rywala.  \\ \hline
    13 & Należysz do ekskluzywnej organizacji Odkrywców, której istnienie nie jest szeroko znane. \\ \hline
    14 & Zostałeś porwany jako dziecko w tajemniczych okolicznościach, are wróciłeś do domu bezpieczny. Media dalej czasami wspominają ową sytuację.  \\ \hline
    15 & Kiedy byłeś młody, byłeś uzależniony od narkotyków, a teraz powoli wstajesz na nogi. \\ \hline
    16 & Kiedy badałeś odległą lokację, dostrzegłeś coś, czego nigdy nie byłeś w stanie wyjaśnić. \\ \hline
    17 & Posiadasz mały bar lub restaurację. \\ \hline
    18 & Opublikowałeś książkę o swoich odkryciach i poczynaniach, która zyskała pewne uznanie. \\ \hline
    19 & Twoja siostra posiada sklep i daje Tobie pokaźną zniżkę. \\ \hline
    20 & Twój ojciec to wysoki rangą oficer w armii i posiada wiele koneksji. \\ \hline
 \end{tabularx}
  \caption {Historia Odkrywcy}
  \label {Historia Odkrywcy}
 \end{table*}
\input{src/Mówca.tex}

\printindex

\listoftables

\end{document}