\documentclass[10pt, a4paper, twocolumn, openright]{book}

\usepackage[pdftex, breaklinks=true]{hyperref}
\usepackage{polski}
\usepackage[utf8]{inputenc}
\usepackage{hyperref}
\usepackage{makeidx}
\usepackage{tabularx}
\usepackage{xcolor }
\usepackage{soul}
\usepackage{afterpage}
\usepackage[T1]{fontenc}
\usepackage{csquotes}
\usepackage{longtable}
\usepackage{supertabular,booktabs}

% TABLES WIDTH !!!
% 0.10 and 0.45

\definecolor{purple}{HTML}{92268F}
\definecolor{red}{HTML}{FF0000}
% \definecolor{gray}{HTML}{D3D3D3}

% \sethlcolor{gray} 

\newcommand{\mytext }[1] {{\color{purple} \textbf{ \texttt {#1}}}}

% \title{Cypher System Reference Document 2024-07-02 (Edycja Polska)}
% \author{Zespół Monte Cook Games\thanks{Strona projektu: \url{https://www.montecookgames.com/cypher-system-open-license/}} \and Szymon ``Kaworu'' Brycki\thanks{\href{mailto:szymon.brycki@gmail.com}{\tt szymon.brycki@gmail.com}}}

\makeindex

\begin{document}

\begin{titlepage}
	\centering
	{\Huge \bfseries \title  _Dokument Referencyjny  \break Cypher System \par}
	\vspace{1cm}
	{\large \itshape 2024-07-02 \par}
	{\large \itshape  Edycja polska \par}
	\vspace{1cm}
	{\normalsize \textbf{Oryginalne zasady}: Monte Cook Games \par}
	{\normalsize \textbf{Polskie tłumaczenie}: Szymon ``Kaworu'' Brycki \par}
	\vspace{1cm}
	{\normalsize Licencja: \bfseries Cypher System Open License\par}
	\vspace{1cm}
	{\large \textbf{Copyright © 2024 Monte Cook Games. Some rights reserved.} \par}
	\vspace{1cm}
	{\large Stworzono w technologii \LaTeX \par}
	\vspace{1cm}
	{\large \today \par}
\end{titlepage}

% \maketitle

\tableofcontents

% here go all the chapters

\chapter {Jak grać w Cypher System}

Zasady Cypher System są całkiem proste i cała rozgrywka bazuje na ledwie kilku podstawowych konceptach.

Ten rozdział zapewnia krótkie wyjaśnienie jak grać w tę grę, i jest przydatny dla dopiero uczących się rozgrywki. Kiedy zrozumiesz już podstawowe koncepty, będziesz pewnie chcieć przeczytać \mytext{Zasady Gry} po więcej szczegółów. 
Cypher System korzysta z kości dwudziestościennej (k20) by określić wynik większości akcji. Za każdym razem, gdy wymagany jest rzut, a nie podano kości, rzuć k20.

Mistrz Gry określa stopień trudności danego zadania. Istnieje 10 stopni trudności. Tak więc, trudność można określić na skali od 1 do 10.

Każda trudność ma minimalny wynik powiązany z sobą. Minimalny wynik (inaczej zwany stopniem trudności) to zawsze 3x poziom trudności, więc stopień trudności 1 ma minimalny wynik 3, a stopień trudności 4 ma minimalny wynik 12. By odnieść sukces, należy wyrzucić minimalny wynik lub więcej danego ST. Patrz Tabela Stopnie Trudności po więcej danych.
Umiejętności postaci, przydatne okoliczności lub doskonały ekwipunek mogą zmniejszyć trudność zadania. Dla przykładu, postać wytrenowana we wspinaczce może zamienić trudność 6 testu wspinaczki na trudność 5. Nazywa się to  Ułatwianiem albo Obniżaniem trudności o jeden stopień (albo po prostu Obniżaniem trudności, gdzie przyjmuje się domyślnie, że dotyczy ona jednego stopnia). Jeśli postać jest wyspecjalizowana we wspinaczce, zamienia ona trudność 6 na trudność 4. Nazywa się to Obniżaniem trudności o dwa stopnie. Obniżanie poziomu trudności może także być nazywane ułatwieniem zadania. Niektóre sytuacje zwiększają, lub Utrudniają, trudność zadania. Jeśli zadanie jest utrudnione, należy zwiększyć jego trudność o jeden poziom.

Umiejętność to kategoria wiedzy, zdolności lub aktywności powiązania z zadaniem, np.: wspinaczka, geografia lub perswazja. Postać, która posiada umiejętność, jest lepsza w powiązanych z nią zadaniach niż postać, która nie posiada danej umiejętności. Posta posiada albo wytrenowaną (do pewnego stopnia) umiejętność, albo wyspecjalizowaną (bardzo dużą).
Jeśli jesteś wytrenowany w umiejętności powiązanej z danym zadaniem, ułatwiasz rzut o stopień. Jeśli jesteś wyspecjalizowany, obniżasz poziom trudności o dwa stopnie. Umiejętność nigdy nie może obniżyć trudności testu o więcej niż dwa stopnie. 

Wszystko inne co obniża trudność danego zadania nazywa się Wysiłkiem. (Wysiłek opisano szczegółowo w rozdziale Zasady Gry).

Podsumowując, trzy rzeczy mogą obniżyć trudność zadania: umiejętności, atuty i Wysiłek. 

Jeśli ułatwisz rzut tak mocno, że jego trudność wynosi 0, wtedy automatycznie uzyskujesz sukces i nie musisz rzucać kośćmi. 

\section {Kiedy rzucać kośćmi?}

Za każdym razem, gdy Twoja postać chce wykonać jakieś zadanie, MG daje mu Poziom Trudności i rzucasz k20 przeciwko Stopniowi Trudności powiązanemu z danym Poziomem Trudności.

Kiedy wyskakujesz z płonącego pojazdu, zamachujesz się toporem na zmutowanąbestię, płyniesz poprzez rwącą rzekę, identyfikujesz dziwne urządzenie, przekonujesz handlarza, by dał Ci niższą cenę, tworzysz obiekt, korzystasz z mocy, by kontrolować umysł przeciwnika lub korzystasz z laserowego działka, by zrobić dziurę w ścianie, wykonujesz rzut k20.
Jednakże, jeśli twój Poziom Trudności ma wartość 0, rzut nie jest konieczny – automatycznie uzyskujesz sukces. Wiele akcji ma trudność 0. Przykłady to przejście przez pokój i otworzenie drzwi, skorzystanie ze specjalnej zdolności lotu, korzystanie z mocy, by ochronić swojego przyjaciela przed promieniowaniem, lub aktywowanie urządzenia (które się rozumie) by stworzyć pole siłowe. To wszystko to są rutynowe akcje i nie wymagają one rzutów.

Korzystając z umiejętności, atutów i Wysiłku, można teoretycznie obniżyć trudność dowolnej akcji do 0 i zlikwidować konieczność rzucania kostką. Przejście po wąskim drewnie jest trudne dla większości ludzi, ale nie dla doświadczonego gimnastyka. Możesz nawet obniżyć trudność ataku na swojego wroga do 0 i odnieść sukces bez rzucania.

Jeśli nie ma rzutu, nie ma szansy, by odnieść porażkę. Jednakże, nie ma także szansy na wyjątkowy sukces (w Cypher System zazwyczaj oznacza to wyrzucenie 19 lub 20, co jest znane jako specjalne rzuty; rozdział Zasady Gry omawia je w szczegółach).

\begin{table*}[t]
 \centering
 \begin{tabularx}{\textwidth}{ | X | X | X | X |}
  \hline
   Poziom trudności & Opis & Stopień trudności & Szczegóły  \\ \hline
    0 & Rutyna & 0 & Każdy może to zrobić zawsze \\ \hline
    1 & Proste & 3 & Większość ludzi może to zrobić przez większość czasu  \\ \hline
 \end{tabularx}
  \caption {Tabela: Trudność zadań}
  \label {Tabela: Trudność zadań}
 \end{table*}
 
\section {Walka}\index{Walka!Wstęp}
Wykonywanie ataków w walce działa tak samo jak inne rzuty – MG określa trudność zadania, a następnie należy rzucić k20 przeciwko Stopniowi Trudności.

Trudność Twojego testu ataku zależy od tego, jak bardzo potężny jest przeciwnik. Istoty mają poziomy od 0 do 10, tak jak i zadania, które może wykonać postać. Zazwyczaj trudność rzutu to ST powiązanie z poziomem istoty. Dla przykładu, atak na bandytę 2 poziomu to zadanie o Poziomie Trudności 2, więc Stopień Trudności wynosi 6. 

Trzeba zaznaczyć, że gracze wykonują wszystkie rzuty w Cypher System. Jeśli gracz atakuje istotę, ten gracz wykonuje rzut na atak. Jeśli istota atakuje gracza, to on wykonuje rzut obronny. 

Obrażenia zadawane przez atak nie są definiowane przez rzut kością – jest to stała wartość bazująca na broni lub ataku. Dla przykładu, włócznia zawsze zadaje 4 punkty obrażeń.

Twój Pancerz redukuje obrażenia które otrzymujesz. Otrzymujesz Pancerz za noszenie fizycznej zbroi (takiej jak skórzana kurtka e współczesnym świecie lub pancerz w świecie fantasy) lub ze specjalnych zdolności. Tak jak wartość obrażeń, Pancerz to stała wartość, nie wynik rzutu. Jeśli jesteś zaatakowany, odejmij swój Pancerz od otrzymanych obrażeń. Dla przykładu, skórzana kurtka daje Ci +1 do Pancerza, co oznacza, że otrzymujesz o 1 mniejsze obrażenia z ataków. Jeśli ktoś trafi Cię atakiem nożem za 2 punkty obrażeń, kiedy ją nosisz, otrzymasz tylko 1 punkt obrażeń. Jeśli Pancerz redukuje obrażenia do 0, wtedy nie otrzymujesz w ogóle żadnych obrażeń. 

Kiedy widzisz w zasadach gry słowo „Pancerz” pisane wielką literą, odnosi się do to statystyki Pancerz – do liczby, o którą obniżasz obrażenia. Kiedy widzisz „pancerz” pisany małą literą, dotyczy to dowolnego fizycznego pancerza, który postać może nosić. 

Fizyczne bronie posiadają 3 kategorie: lekkie, średnie i ciężkie.  

Lekkie bronie zadają tylko 2 punkty obrażeń, ale ułatwiają rzuty na atak, ponieważ są szybkie i łatwe w użyciu. Lekkie bronie to ciosy pięścią, kopnięcia, maczugi, noży, toporki ręczne, rapiery, małe pistolety itp. Bronie, które są małe, są broniami lekkimi.
Średnie bronie zadają 4 punkty obrażeń. Średnie bronie to między innymi miecze, topory bojowe, większe maczugi, kusze, włócznie, pistolety, blastery itp. Większość broni to bronie średnie. Wszystko, co może być użyte w jednej dłoni (nawet, jeśli często korzysta się z dwóch, jak w przypadku kostura i włóczni) jest średnią bronią. 

Ciężkie bronie zadają 6 punktów obrażeń, i trzeba korzystać z obydwu dłoni, by z nich korzystać. Ciężkie bronie to wielkie miecze, młoty bojowe, potężne topory, halabardy, ciężkie kusze, karabinki laserowe itp. Wszystko, z czego trzeba korzystać z obydwu dłoni, to ciężkie bronie.

\section {Specjalne wyniki rzutów}\index{Specjalne wyniki rzutów!Wstęp}

Kiedy wyrzucasz naturalne 19 (k20 pokazuje „19”) i test jest sukcesem, uzyskujesz mniejszy efekt. W walce, mniejszy efekt zadaje dodatkowe 3 do obrażeń, lub, jeśli wolisz efekt specjalny, możesz odrzucić wroga do tyłu, rozproszyć jego uwagę lub coś podobnego. Kiedy nie walczysz, mniejszy efekt może oznaczać, że wykonałeś akcję ze stylem. Przykładowo, gdy przeskakujesz przez płot, lądujesz z gracją na własnych stopach, lub gdy przekonujesz kogoś, wierzy on, że jesteś mądrzejszy, niż jesteś naprawdę. W innych słowach, nie tylko osiągasz zwykły sukces, ale także uzyskujesz pomniejszy bonus. 

Kiedy wyrzucasz naturalne 20 (k20 pokazuje „20”) i rzut się powiódł, uzyskujesz dodatkowo większy efekt. Jest to podobne do mniejszego efektu, ale na większą skalę. W walce, zadajesz dodatkowe 4 punkty obrażeń, ale znowu, można zamiast tego wybrać jakiś efekt dodatkowy, taki jak przewrócenie wroga, ogłuszenie go, lub wzięcie akcji dodatkowej. Poza walką, większy efekt oznacza, że dzieje się coś korzystnego, w zależności od okoliczności. Dla przykładu, kiedy wspinasz się na ścianę, robisz to dwa razy szybciej. Kiedy rzut daje Ci większy efekt, możesz zamiast tego skorzystać z mniejszego efektu, jeśli taka jest Twoja wola.

W walce (i tylko wtedy) jeśli rzucisz naturalne 17 lub 18 na rzucie na atak, zadajesz – odpowiednio - dodatkowe 1 lub 2 punkty obrażeń. Te rzuty nie dają żadnych innych specjalnych efektów – tylko zwiększają obrażenia.

(Po więcej informacji o specjalnych wynikach rzutów i tym, jak wpływają na walkę i inne akcje, patrz Zasady Gry).

Wyrzucenie naturalnej 1 jest zawsze złe. To oznacza, że MG wprowadza nowe utrudnienie do sceny. 

\section {Słowniczek}\index{Słowniczek}

{\bfseries Mistrz Gry (MG)}: Gracz, który nie ma własnej postaci, a który zamiast tego kieruje całą fabułą i wszystkimi BN-ami.

{\bfseries Bohater Niezależny (BN)}: Postać kierowana przez MG. Myśl o niej jak o pomniejszej postaci w historii, lub jak o złoczyńcy lub oponencie. Wlicza się w to każde każda istota lub potwór.

{\bfseries Drużyna}: Grupa BG (i może jacyś BN-i sojusznicy).

{\bfseries Bohater Gracza (BG)}: Postać odgrywana przez gracza zamiast przez MG. Myśl o BG jak o głównych bohaterach historii.

{\bfseries Gracz}: Gracz, który kieruje BG.

{\bfseries Sesja}: Pojedyncza doświadczenie roleplayowe. Zazwyczaj trwa kilka godzin. Czasami jedną przygodę można ukończyć w czasie jednej sesji. Częściej, jedna przygoda zajmuje kilka sesji.

{\bfseries Przygoda}: Pojedyncza część kampanii z początkiem i końcem. Zazwyczaj zdefiniowana na początku przez wspólny cel BG i na końcu przez to, czy go osiągnęli, czy też nie. 

{\bfseries Kampania}: Seria sesji połączona wspólną historią (lub połączonymi historiami) z tymi samymi BG. Często, lecz nie zawsze, kampania to zbiór przygód.

{\bfseries Postać}: Cokolwiek, co podejmuje akcje w grze. Choć wliczają się w to BG i ludzcy BN-i, technicznie wliczają się w to potwory, kosmici, mutanci, automatony, ruchome rośliny itp. Synonimem jest „istota” bądź „potwór”.

\section {Zasięg i szybkość}\index{Zasięg i szybkość}

Zasięg dzieli się na 4 ogólnikowe kategorie: bliski, średni, daleki i bardzo daleki.

Bliski zasięg to odległość ręki lub paru kroków. Jeśli postać stoi w małym pokoju, wszystko wokół jest w jej bliskim zasięgu. Górna granica bliskiego zasięgu to 3 metry.

Średni zasięg to wszystko, co jest większe od bliskiego, ale mniejsze niż 15 metrów.

Długi dystans to wszystko większe od średniego zasięgu, ale mniejsze niż 30 metrów.

Bardzo długi zasięg to wszystko większe od długiego dystansu, ale nie większe niż 150 metrów. Poza tym zasięgiem, odległości zawsze się ściśle określone – 300 metrów, 1,5 kilometra itp.

Ogólną ideą tego systemu jest to, że nie trzeba dokładnie mierzyć i określać odległości. Bliski zasięg to tutaj, obok postaci. Średni zasięg to blisko postaci. Długi dystans jest dalej, a bardzo długi – znacznie dalej.
Wszystkie bronie i specjalne zdolności korzystają z tych terminów. Dla przykładu, wszystkie bronie do walki wręcz mają bliski zasięg – służą przecież do walki wręcz i można je użyć tylko na osobach, które stoją obok nas. Nóż do rzucania (i większość innych broni rzucanych) mają średni zasięg. Łuk ma długi zasięg. Pocisk Adepta także ma średni zasięg.

Postać może się przemieścić o bliski zasięg jako część innej akcji. Innymi słowy, może ona podejść do panelu kontrolnego i z niego skorzystać. Może przejść przez mały pokój i zaatakować wroga. Mogę otworzyć drzwi i przejść przez nie.

Postać może się przemieścić o średni zasięg jeśli poświęci na to całą akcję w turze. Może także spróbować się przemieścić o długi zasięg w jednej akcji, ale trzeba wykonać rzut, by stwierdzić, czy postać się nie poślizgnęła lub  przewróciła w efekcie tak szybkiego ruchu.

Dla przykładu, jeśli BG walczą z grupą kultystów, każda postać może zaatakować, ogólnie rzecz ujmując, dowolnego kultystę wręcz – wszyscy są w zasięgu. Dokładne pozycje nie są tak ważne. Istoty w walce zawsze się zresztą poruszają. Jednakże, jeden z kultystów został z tyłu by wystrzelić z pistoletu i BG może musieć poświęcić całą akcję, by się do niego dostać. Nie ma większego znaczenia, czy ten kultysta jest 6 metrów od postaci graczy, czy może 12 – po prostu jest w średnim zasięgu. Ma znaczenie, czy kultysta stoi o więcej niż 15 metrów od BG, ponieważ wtedy zasięg by się zwiększył do dalekiego.

(Wiele zasad w tej grze unika konieczności nadmiernej precyzji. Czy naprawdę się liczy to, czy duch jest o 13, czy o 18 stóp od Ciebie? Najpewniej nie. Taki rodzaj niepotrzebnej ścisłości tylko spowalnia rozgrywkę i odciąga uwagę od akcji i fabuły, zamiast być miłym dodatkiem do opowiadanej historii.)

\section {Punkty doświadczenia}\index{Punkty doświadczenia!Wstęp}

Punkty doświadczenia (PD-ki) są nagrodą dawaną graczom, gdy GM wtrąca się narrację (nazywamy to Wtrąceniem MG) z nowym i niespodziewanym wyzwaniem. Dla przykładu, w środku walki, MG może poinformować graczy, że upuszczają oni swoje bronie. Jednakże, aby się wtrącić w taki sposób, MG musi dać graczowi 2 PD-ki. Nagrodzony gracz, z kolei, musi natychmiast dać jednego z owych PD-ków innemu graczowi, uzasadniając to (może ten gracz miał dobry pomysł, powiedział zabawny żart, wykonał akcję, która ocaliła życie jakiegoś BN/BG itp.).

Alternatywnie, gracz może odrzucić Wtrącenie MG. Jeśli on tak uczyni, nie otrzymuje on 2 PD-ków od GM, i musi wydać i PD z posiadanych przez siebie. Jeśli gracz nie ma PD-ków, nie może odrzucić Wtrącenia MG.

MG może także dać graczom PD-ki pomiędzy sesjami, jako nagrody za dokonywanie odkryć podczas gry. Odkrycia to ciekawe fakty, cudowne sekrety, potężne artefakty, odpowiedzi na pytania lub rozwiązania problemów (np.: gdzie porywacze przetrzymują swoje ofiary lub jak gracze naprawią statek kosmiczny). Nie otrzymujesz PD-ków za zabijanie potworów lub przezwyciężanie zwykłych trudności podczas gry. Odkrycia są duszą Cypher System.

Punkty Doświadczenia głównie służą awansowaniu postaci na poziomy (po detale, patrz: rozdział Tworzenie Własnej Postaci), ale gracz może także wydać 1 PD-ek, by przerzucić kość i wybrać leszy z dwóch wyników. 

\section  {Cyphery}\index{Cyphery!Wstęp}

Cyphery to zdolności, z których można skorzystać tylko raz. W wielu kampaniach, cyphery nie są fizycznymi obiektami – mogą być zaklęciem rzuconym na postać, błogosławieństwem od boga, lub po prostu zrządzeniem losu, które daje chwilową przewagę. W pewnych kampaniach, cyphery to obiekty fizyczne które postaci mogą z sobą nosić. Niezależnie od tego, czy cyphery to przedmioty, czy też nie, są częścią postaci (tak jak ekwipunek lub specjalna zdolność). I są czymś, z czego postać może skorzystać podczas gry. Forma, którą przyjmują fizyczne cyphery, zależy od settingu. W świecie fantasy mogą być różdżkami lub eliksirami, ale w grze science fiction mogą być obcymi kryształami lub prototypowymi technologiami.

Postaci często będą znajdowały nowe cyphery, więc gracze powinni równie często z nich korzystać. Ponieważ cyphery zawsze będą odmienne od innych cypherów, postać zawsze będzie miała nowe specjalne zdolności do wypróbowania. 

\section {Inne kości}

W dodatku do k20, potrzebujesz jeszcze k6 (sześciościennej kostki). Czasami będziesz potrzebował k100 (do losowania numerów od 1 do 100), co można osiągnąć, rzucając k20 dwa razy – ostatnia liczba pierwszego rzutu to “dziesiątki” a ostatnia liczba drugiego rzutu to “jedności”. Dla przykładu, rzut 17 i 9 daje nam 79, a 3 i 18 daje nam 38, a rzucenie 20 i 10 daje nam 00 (także znane jako 100). Jeśli masz k10 (dziesięciościenną kostkę) możesz skorzystać z niej zamiast z k20 by losować liczby od 1 do 100.

(k6 jest najczęściej wykorzystywana do rzutów na odzyskiwanie zdrowia i do określania poziomu cypherów).
\input{src/Tworzenie własnej postaci.tex}
\section{Typ}\index{Typ}

Typ postaci to najważniejsza cecha Twojej postaci. Twój typ pozwala określić miejsce postaci w świecie i jej relację z innymi ludźmi. Jest to rzeczownik w zdaniu “Jestem przymiotnik rzeczownik który czasownikuje”.

(W pewnych grach RPG, typ postaci może zostać nazwany klasą postaci.)

Możesz wybrać z 4 typów postaci: Wojownika, Adepta, Odkrywcy i Mówcy. Jednakże, możesz nie chcieć korzystać z tych ogólnikowych nazw na nie. Ten rozdział oferuje parę bardziej specyficznych nazw na każdy typ, które mogą być stosowne, w zależności od świata przedstawionego. Odkryjesz, że nazwy takie jak “Wojownik” czy też “Odkrywca” nie zawsze pasują do gier dziejących się w świecie współczesnym. Jak zawsze, możesz zrobić, co uznasz za stosowne. (Twój typ określa kim jest postać. Powinieneś korzystać z dowolnej nazwy na typ, tak długo, jak pasuje zarówno do postaci, jak i do settingu.)

Ponieważ typ to podstawa, na której się buduje postać, warto się zastanowić, jaka relacja łączy go z settingiem. Aby z tym pomóc, typy to w zasadzie ogólne archetypy. Wojownik, dla przykładu, może być wszystkim, od rycerza w lśniącej zbroi, przez gliniarza na ulicy po cybernetycznego weterana tysiąca futurystycznych wojen.
Aby lepiej dostosować cztery typy do różnych settingów, istnieją różne metody zwane posmakami,  zaprezentowane w stosownym rozdziale, by pomóc w dostosowaniu typów do konwencji fantasy, science fiction, lub innych (lub by dostosować typy do pomysłu na postać).

Dalej, bardziej fundamentalne opcje dla \mytext{dalszej customizacji} są dostępne na końcu tego rozdziału. 

\subsubsection{Wtrącenie Gracza}\index{Wtrącenie Gracza}

Wtrącenie gracza oznacza, że gracz wybiera zmianę czegoś w kampanii, czyniąc rzeczy łatwiejszymi dla jego postaci. Konceptualnie, jest to przeciwieństwo wtrącenia MG: zamiast MG dawać PD graczowi i wprowadzać niespodziewaną komplikacje dla jego postaci, gracz wydaje 1 PD i wprowadza rozwiązanie problemu lub komplikacji. To, co może zrobić wtrącenie gracza, to zmienić świat gry lub obecne okoliczności zamiast bezpośrednio zmieniać postać. Dla przykładu, wtrącenie mówiące, że cypher, z którego właśnie się skorzystało, ma dodatkowe użycie byłoby właściwe, ale wtrącenie uzdrawiające postać nie byłoby. Jeśli gracz nie ma PD-ków do wydania, nie może wprowadzić wtrącenia gracza. 

Parę wtrąceń gracza jest zasugerowanych pod każdym typem. Warto jednak zaznaczyć, że nie każde wtrącenie gracza jest stosowne w każdej sytuacji. MG może zezwolić graczom na inne sugestie wtrąceń, ale ostatecznie to on decyduje, czy dane wtrącenie jest stosowne do typu postaci i danej sytuacji. Jeśli MG odmawia wtrącenia, gracz nie wydaje 1 PD-ka i wtrącenie nie następuje.

Korzystanie z intruzji nie wymaga od postaci akcji, by je zastosować. Po prostu ono następuje.

(Wtrącenie gracza powinno być ograniczone do nie więcej niż jednego wtrącenia na gracza na jedną sesję.)

\subsubsection{Akcje obronne}\index{Akcje obronne}

Akcje obronne występują wtedy, gdy gracz rzuca, by uchronić się od czegoś nieporządanego, co mogłoby się wydarzyć jego BG. Rodzaj akcji obronnej ma znaczenie, gdy rozważamy Wysiłek.

\textbf{Obrona Mocy}: Używa się jej do odporności na trucizny, choroby i wszystko inne, co można przezwyciężyć siłą i zdrowiem.

\textbf{Obrona Szybkości}: Używa się jej do unikana ciosów i uciekana od niebezpieczeństw. To najczęściej wykorzystywany rodzaj akcji obronnej.

\textbf{Obrona Intelektu}: Używa się jej do odpieranie ataków mentalnych i wszystkiego, co może wpłynać na czyiś umysł.
\cleardoublepage

\subsection{Wojownik}\index{Typ!Wojownik}

Fantasy/Baśń: Wojownik, rycerz, barbarzyńca, żołnierz, walkyria.

Współczesność/Horror/Romans: policjant, żołnierz, strażnik, detektyw, ochroniarz, atleta.

Science fiction: oficer bezpieczeństwa, wojownik, żołnierz, najemnik.

Superbohaterowie/Post-apokalipsa: bohater. 

Jesteś dobrym sprzymierzeńcem w potyczce. Wiesz, jak korzystać z broni i chronić siebie. W zależności od konwencji i settingu, może to znaczyć, że nosisz miecz i tarczę na arenie gladiatorów, posiadasz karabin maszynowy i zestaw granatów przydatne w wymianie ognia, lub posiadasz blastera i zasilany pancerz, z których korzystasz na obcej planeci

Rola w grze: Wojownicy są fizyczni i zorientowani na akcję. Cześciej rozwiązują problemy, korzystając z siły, niż na inne sposoby, i często wybierają najprostszą drogę do osiągnięcia swoich celów. 

Rola w drużynie: Wojownicy najczęściej zadają i biorą na klatę najwięcej obrażeń w bitwie. To od nich zależy obrona reszty członków drużyny przed atakami. To czasami oznacza, że wojownicy bywają liderami, przynajmniej w walce i w obliczu innych niebezpieczeństw.

Rola społeczna: Wojownicy nie zawsze są żołnierzami lub najemnikami. Każdy, kto jest gotów do odrobinę przemocy w swoim życiu, lub choćby jej potencjał, może być Wojownikiem, mówiąc ogólnikowo. Wliczają się w to strażnicy, policjanci, marynarze lub ludzie innych profesji, którzy wiedzą, jak się bronić.

Zaawansowani Wojownicy: W miarę, jak wojownicy awansują na poziomy, ich umiejętności bitewne – zarówno obrony, jak i ataku – zwiększają się do niemożliwych poziomów. Na wyższych poziomach, mogą oni często przeciwstawić się grupom wrogów lub stanąć 1-na-1 przeciwko dowolnemu przeciwnikowi.

\subsubsection{Historia Wojownika}

Twój typ pomaga Ci określić połączenie Twojej postaci z settingiem. Rzuć k20 lub wybierz z poniższej listy, by określić pewien fakt odnośnie Twojej historii, który łączy Twoją postać ze światem. Możesz także stworzyć swój własny fakt historyczny.

\begin{table*}[t]
 \centering
 \begin{tabularx}{\textwidth}{| p{0.10\textwidth} | X |}
  \hline
  \textbf{k20} & \textbf{Historia Wojownika}  \\ \hline
    1 & Byłeś w armii i dalej masz przyjaciół, którzy tam są. Twój były dowódca dobrze Cię pamięta. \\ \hline
    2 & Byłeś ochroniarzem bogatej kobiety, która oskarżyła Cię o kradzież. Opuściłeś jej służbę w cieniu podejrzeń. \\ \hline
    3 & Byłeś ochroniarzem lokalnego baru, i właściciele pamiętają Cię.  \\ \hline
    4 & Trenowałeś z szanowanym mentorem. Trzyma on Cię w estymie, ale ma wielu wrogów. \\ \hline
    5 & Trenowałeś w odosobnionym zakonie. Mnisi myślą o Tobie jak o bracie, ale dla wszystkich innych jesteś obcym. \\ \hline
    6 & Nie masz formalnego wyszkolenia. Twoje zdolności po prostu są (naturalnie bądź nie). \\ \hline
    7 & Spędziłeś czas na ulicach i byłeś przez pewien czas w więzieniu. \\ \hline
    8 & Zapisano Cię do służby wojskowej, ale uciekłeś po niedługim czasie. \\ \hline
    9 & Służyłeś jako ochroniarz dla potężnego kryminalisty, który teraz jest Ci winny życie. \\ \hline
    10 & Pracowałeś jako oficer policji lub detektyw. Każdy Cię zna, ale opinie o Tobie są różne. \\ \hline
    11 & Twoje starsze rodzeństwo to niesłynna postać, która żyje w hańbie.  \\ \hline
    12 & Służyłeś jako strażnik komuś, kto dużo podróżował. Znasz ludzi w wielu miejscach. \\ \hline
    13 & Twój najlepszy przyjaciel to nauczyciel lub badacz. Jest on świetnym źródłem wiedzy. \\ \hline
    14 & Ty i Twój przyjaciel palicie ten sam rodzaj rzadkiego, drogiego tytoniu. Spotykacie się co tydzień, by porozmawiać i zapalić. \\ \hline
    15 & Twój wuj prowadzi teatr w mieście. Znasz wszystkich aktorów i masz wolny wstęp na występy. \\ \hline
    16 & Twój przyjaciel-rzemieślnik czasami prosi Cię o pomoc. Jednakże, płaci on dobrze. \\ \hline
    17 & Twój mentor napisał książkę o sztukach walki. Czasami ludzie pragną Cię odszukać i zapytać o jej dziwne zapisy. \\ \hline
    18 & Ktoś, z kim walczyłeś ramię w ramię w armii, teraz jest burmistrzem lokalnego miasteczka. \\ \hline
    19 & Ocaliłeś życie rodziny, gdy jej dom płonął. Mają oni u Ciebie dług, a ich sąsiedzi traktują Ciebie jak bohatera. \\ \hline
    20 & Twój stary trener dalej spodziewa się, że wrócisz i sprzątniesz po jego zajęciach; gdy to robisz, dzieli się on z Tobą okazjonalnie ciekawymi plotkami. \\ \hline
 \end{tabularx}
  \caption {Historia Wojownika}
  \label {Historia Wojownika}
 \end{table*}
 
\subsubsection{Wojownik - Wtrącenia Gracza}

Możesz wydać 1 PD by skorzystać z poniższych wtrąceń gracza, jeśli jest to stosowne do sytuacji, a MG się zgodzi.

Perfekcyjna pozycja: Walczysz przynajmniej z trzema wrogami i każdy z nich stoi w odpowiednim miejscu, możesz więc wykorzystać ruch, który ćwiczyłeś dawno temu, co pozwala Ci zaatakować wszystkich trzech w jednej akcji. Wykonaj odrębne rzuty na atak dla kazego z wrogów. Jesteś ograniczony Wysiłkiem, który możesz wykorzystać w jednej akcji.

Stary Przyjaciel: Towarzysz borni z przeszłości pojawia sięnagle i pomaga w tym, co teraz robisz. Jest on na własnej misji i nie może zostać dłużej niż czas potrzebny na udzielenie pomocy, porozmawianie przez chwilę i być może na wspólne zjedzenie szybkiego posiłku.

Słabość Broni: Broń Twojego przeciwnika ma słaby punkt. Podczas walki, szybko się ona psuje i spada o dwa stopnie w dół na \mytext{liczniku obrażeń przedmiotu}.

\begin{table*}[t]
 \centering
 \begin{tabularx}{\textwidth}{ | X | X |}
  \hline
   \textbf{Statystyka} & \textbf{Początkowa Wartość Puli}  \\ \hline
    Moc & 10  \\ \hline
    Szybkość & 10  \\ \hline
    Intelekt & 8  \\ \hline
 \end{tabularx}
  \caption {Pule Statystyk Wojownika}
  \label {Pule Statystyk Wojownika}
 \end{table*}
 
 Otrzymujesz dodatkowe 6 punktów do podziału pomiędzy Pule, jakkolwiek sobie życzysz.
 
\subsubsection{Wojownik pierwszego poziomu}

Pierwszo-poziomowi wojownicy mają następujące zdolności:

Wysiłek: Twój Wysiłek to 1.

Fizyczna Natura: Masz Skupienie w Mocy 1 i Skupienie w Szybkości 0 lub Skupienie w Mocy 0 i Skupienie w Szybkości 1. Niezależnie od tego, Twoje Skupienie w Intelekcie to 0.

Korzystanie z Cypherów: Możesz nosić dwa Cyphery w danym czasie.

Bronie: Jesteś wytrenowany w lekkich, średnich i ciężkich broniach i nie stosuje siędo Ciebie kara za używanie jakiegokolwiek rodzaju broni. Umożliwienie.

Początkowy Ekwipunek: Odpowiednie ubranie i dwie bronie Twojego wyboru, plus jeden drogi przedmiot, dwa przedmioty średniej ceny i cztery niedrogie.

Specialne Zdolności: Wybierz cztery zdolności z poniższej listy. Nie możesz wybrać tej samej zdolności więcej niż raz, chyba, że jej opis mówi inaczej. Pełny opis wszystkich zdolności znajduje się w rozdziale \mytext{Zdolności}, który także zawiera opis Posmaków i zdolności Specjalizacji w pojedynczym, sporym katalogu.

\begin{itemize}
\item Broń Niepotrzebna
\item Kontrola Bitewna
\item Na Straży
\item Ogłuszenie
\item Szybki Rzut
\item Ulepszone Skupienie
\item Umiejętności Fizyczne
\item Wyszkolony Bez Zbroi
\item Wyszkolony w Zbroi
\item Zamach
\item Zdolności Bojowe
\end{itemize}

\subsubsection{Wojownik Drugiego Poziomu}

Wybierz dwie zdolności z poniższej lisy (lub z niższego poziomu) i dodaj je do swoich zdolności. Dodatkowo, możesz zamienić jedną ze zdolności niższego poziomu na inną z niższego poziomu.

\begin{itemize}
\item Krwawienie
\item Miażdżący Cios
\item Następny Atak
\item Przeładowanie
\item Umiejętny Atak
\item Umiejętna Obrona
\end{itemize}

\subsubsection{Wojownik Trzeciego Poziomu}

Wybierz trzy zdolności z poniższej listy (lub z niższego poziomu) i dodaj je do swoich zdolności. Dodatkowo, możesz zamienić jedną ze zdolności niższego poziomu na inną z niższego poziomu.

\begin{itemize}
\item Chwytaj Moment
\item Cięcie
\item Cios z Wyciągnięciem
\item Czujność
\item Ekspercki Użytkownik Cypherów
\item Furia
\item Odporność na Energię
\item Ostrzał Ciągły
\item Podwójny Strzał
\item Przywykły do Noszenia Zbroi
\item Reakcja
\item Śmiertelna Salwa
\end{itemize}

\subsubsection{Wojownik Czwartego Poziomu}

Wybierz dwie z poniższych zdolności (lub z niższego poziomu) i dodaj je do swoich zdolności. Dodatkowo, możesz zamienić jedną z zdolności niższego poziomu na inną z niższego poziomu.

\begin{itemize}
\item Dodatkowy Wysiłek
\item Doświadczony Obrońca
\item Finta
\item Pęd
\item Przełamanie Obrony
\item Wycelowanie
\item Wyjątkowo Wytrzymały
\item Zręczny Wojownik
\item Zwiększony Efekt
\end{itemize}

\subsubsection{Wojownik Piątego Poziomu}

Wybierz trzy zdolności z poniższej listy (lub z niższego poziomu) i dodaj je do swoich zdolności. Dodatkowo, możesz zamienić jedną ze zdolności niższego poziomu na inną zdolność niższego poziomu.

\begin{itemize}
\item Atak z Wyskoku
\item Blok
\item Mistrzostwo Ataków
\item Mistrzostwo Obrony
\item Mistrzowska Biegłość w Pancerzach
\item Ulepszony Sukces
\item Potrójny Wystrzał
\item Zaawansowany Użytkownik Cypherów
\end{itemize}

(Pamiętaj, że na wyższych poziomach, można wybrać zdolności z niższych poziomów. Czasami jest to najlepszy sposób, by uzyskać dokładnie taką postać, jakiej pragniesz. Jest to zwłaszcza prawdziwe odnośnie zdolności, które zapewniają umiejętności, które zazwyczaj można wybrać więcej niż jeden raz.)

\subsubsection{Wojownik Szóstego Poziomu}

Wybierz dwie ze zdolności z poniższej listy (lub z niższego poziomu) i dodaj je do swoich zdolności. Dodatkowo, możesz zamienić jedną ze zdolności niższego poziomu na inną z niższego poziomu.

\begin{itemize}
\item Broń i Cios
\item Chwila Wspaniałości
\item Morderca
\item Ostateczny Cios
\item Wielokrotny Atak
\item Znowu i Znowu
\end{itemize}

\subsubsection{Przykładowy Wojownik}

Ray chce stworzyć Wojownika do współczesnej kampanii. Decyduje się on na byłego członka armii, który jest silny i szybki. 3 z wolnych punktów idą do Puli Mocy, a pozostałe 3 do Puli Szybkości. Jego Statystyki to teraz Moc 13, Szybkość 13 i Intelekt 8. Jako, że postać jest na 1-szym  poziomie, jej Wysiłek to 1, jej Skupienie w Mocy to 1, a Skupienie w Szybkości i Intelekcie to 0. Jego postać nie jest szczególnie mądra lub charyzmatyczna.

Chce on korzystać z dużego noża bojowego (średnia broń, która zadaje 4 punkty obrażeń) i .357 Magnum (ciężki pistolet, który zadaje 6 punktów obrażeń, ale wymaga dwóch rąk do korzystania). Ray decyduje się na nie noszenie żadnej zbroi, gdyż nie pasuje to do settingu, tak więc, jako swoją pierwszą zdolność wybiera \mytext{Wyszkolony Bez Zbroi}, co ułatwia jego Obronę Szybkości. Jako drugą zdolność wybiera \mytext{Zdolności Bojowe}, by zadawać większe obrażenia swoim wielkim nożem. 

Ray chce być zarówno szybki, jak i wytrzymały, wybiera więc \mytext{Ulepszone Skupienie}. Daje mu to Skupienie w Szybkości na 1. Jako ostatnią zdolność, wybiera \mytext{Umiejętności Fizyczne} i wybiera pływanie i bieganie. 

Wojownik może mieć przy sobie maksymalnie 2 cyphery. GM decyduje, że pierwszy cypher Raya to pigułka, która regeneruje 6 punków Mocy po połknęciu, a jego drugi cypher to mały, łatwy do ukrycia granat, który eksploduje jak ognista bomba, gdy go się rzuci, zadając 3 punkty obrażeń wszystkim w bliskim zasięgu. 

Ray dalej musi wybrać deskryptor i specjalizację. Przeglądając deskryptory, Ray wybiera \mytext{Silnego}, co zwiększa jego Pulę Mocy do 17. Jest także wytrenowany w skakaniu i niszczeniu przedmiotów. (Jeśli Ray wybrałby skakanie jako jedną ze swoich umiejętności fizycznych, teraz dzięki deskryptorowi byłby wyspecjalizowany w skakaniu, zamiast być wytrenowanym). Bycie Islnym daje też Ray’owi dodatkową średnią bądź ciężką broń. Wybiera kij baseballowy, który przechowuje w bagażniku swojego auta. 

Jako swoją specjalizację, Ray wybiera \mytext{Mistrzowsko Posługuje się Bronią}. Daje mu to kolejną broń wysokiej jakości. Wybiera dodatkowy nóż bojowy i pyta MG, czy może z niego korzystać w lewej ręce – nie do wykonywania ataków, lecz jako tarczę. To ułatwi jego rzuty na Obronę Szybkości, jeśli ma obydwie bronie w rękach (“tarcza” liczy się jako atut). MG się zgadza. Podczas gry, ciężko będzie trafić Wojownika Ray’a – kjest wytrenowany w rzutach na Obronę Szybkości, a jego dodatkowy nóż obniża rzuty o kolejny stopień. 

Dzięki jego specjalizacji, zadaje także dodatkowy 1 punkt obrażeń w walce swoją wybranąbronią. Teraz zadaje 6 punktów obrażeń swoim ostrzem. Postać Raya to śmiercionośny wojownik, zapewne rozpoczynający grę z reputacją jako walczący nożami. 

Jako swój motyw fabularny, Ray wybiera \mytext{Pokonać Wroga}. Ten wróg, Ray decyduje, to nikt inny jak jego stary przyjaciel z armii, który wszedł na ścieżkę zła.
\subsection{Adept}\index{Typ!Adept}

Fantasy/baśń: mag, czarodziej, czarnoksiężnik, kleryk, druid, jasnowidz, diabolista, dotknięty przez Fae.

Współczesność/Horror/Romans: psionik, okultysta, wiedźma, praktykujący magię, medium, szalony naukowiec.

Science fiction: psionik, telepata, jasnowidz, skanujący, ESP-er, abominacja.

Superbohaterowie/Post-apokalipsa: mag, czarownik, dzierżący moc, psionik, telepata.

Władasz mocami i zdolnościami poza ludzkim doświadczeniem, zrozumienie i czasami wiarą. Może to być magia, psionika, zdolności mutanta, lub po prostu skomplikowane urządzenia, w zależności od settingu. (“Magia” to termin, który stosujemy tutaj bardzo luźno. To termin na wszystkie wspaniałe, możliwie nadnaturalne rzeczy, które może zrobić Twoja postać, a inne nie mogą. Może to być skutek posługiwania się odpowiednim sprzętem, kontaktu z duchami, mutacji, psioniki, nanotechnologii lub innych źródeł.)

Rola w grze:Adepci to zazwyczaj inteligentni, myślący ludzie. Bardzo często myślą ostrożnie, zanim podejmą akcję i polegają na swoich nadnaturalnych zdolnościach.

Rola w drużynie: Adepci nie są potężni w bezpośredniej walce, choć często posiadają zdolności, które są wspaniałym uzupełnieniem zdolności bojowych ich towarzyszy, zarówno defensywnie, jak i ofensywnie. Czasami posiadają zdolności, które pomagają im przezwyciężać trudności i wyzwania. Dla przykładu, jeśli grupa musi się przedostać przez zamknięte drzwi, Adept może być w stanie je zniszczyć lub przeteleportować wszystkich na ich drugą stronę.

Rola społeczna: W settingach w których moce nadnaturalne są rzadkie, tajemnicze lub wywołują strach, Adepci są zazwyczaj także rzadcy i wywołujący strach. Pozostają wtedy w ukryciu. Kiedy jest inaczej, Adepci są częstsi i bardziej bezpośredni. Mogą nawet zostać liderami swoich społeczności.

Zaawansowani Adepci: Nawet na niższych poziomach, moce Adeptów zapierają dech w piersiach. Na wyższych poziomach, Adepci mogą dokonać prawdziwie wielkich czynów, które mogą przekształcić materię i środowisko wokół nich.
(Adepci prawie zawsze są paranormalni lub nadludzcy w jakimś sensie – czarodzieje, psionicy itp. Jeśli gra, w którą gracie, nie posiada takich postaci, Adept mógłby być szarlatanem, który udaje magiczne zdolności przy pomocy trików i ukrytych urządzeń, lub gadżeciarzem z “przydatnym paskiem” pełnym dziwnych narzędzi. Lub w Twoim świecie może nie być Adeptów. To także jest ok.)

\subsubsection{Adept - Wtrącenia Gracza}

Kiedy grasz Adeptem, możesz wydać 1 PD na jedne z poniższych wtrąceń gracza, jeśli sytuacja jest stosowna i MG się zgodzi.
 
Przydatna Awaria: Urządzenie, z którego korzysta się przeciwko Tobie, ulega awarii. Może ono zranić użytkownika lub jednego z jego sprzymierzeńców w ciągu jednej tury, lub aktywować dramatyczny i rozpraszający efekt uboczny, trwający parę tur.

Nagłe Olśnienie: Doświadczasz nagłego olśnienia, które zapewnia jasną odpowiedz lub sugeruje następne kroki w temacie ważnego pytanie, problemu lub przeszkody na Twojej drodze. 

Cudowna Aktywacja: Nieaktywne, zrujnowane lub najwyraźniej-zniszczone urządzenia chwilowo się aktywuje i wykonuje przydatną akcję w kontekście obecnej sytuacji. Może to kupić Ci trochę czasu na znalezienie lepszego rozwiązania, przezwyciężyć komplikację która wpływa na Twoje moce, lub po prostu umożliwić skorzystanie z zużytego cyphera lub artefaktu jeszcze raz. 

\begin{table*}[t]
 \centering
 \begin{tabularx}{\textwidth}{ | X | X |}
  \hline
  \textbf{ Statystyka} & \textbf{Początkowa Wartość Puli}  \\ \hline
    Moc & 7 \\ \hline
    Szybkość & 9 \\ \hline
    Intelekt & 12 \\ \hline
 \end{tabularx}
  \caption {Pule Statystyk Adepta}
  \label {Pule Statystyk Adepta}
 \end{table*}
 
 Otrzymujesz 6 dodatkowych punktów do podziału pomiędzy Pule statystyk, zgodnie z własną wolą.
 
\subsubsection{Historia Adepta}

Twój typ pomaga Ci określić Twoje miejsce w settingu. Rzuć k20 lub wybierz z poniższej listy, by określić konkretny fakt odnośnie Twojej historii, która łączy Cię z resztą świata. Możesz także stworzyć swój własny fakt. 

 \begin{table*}[t]
 \centering
 \begin{tabularx}{\textwidth}{| p{0.10\textwidth} | X |}
  \hline
  \textbf{d20} & \textbf{Historia Adepta}  \\ \hline
    1 & Służyłeś jako uczeń u Adepta, którego respektowało i bało się wielu ludzi. Teraz nosisz jego brzemię. \\ \hline
    2 & Studiowałeś w szkole słynącej z jej mrocznych nauczycieli i absolwentów. \\ \hline
    3 & Nauczyłeś się swoich zdolności w świątyni mało znanego boga. Jego kapłani i wierni, choć niezbyt liczni, respektują i adorują Twoje talenty i potencjał. \\ \hline
    4 & Kiedy podróżowałeś samotnie, ocaliłeś życie potężnej osoby. Ma ona względem Ciebie dług wieczności. \\ \hline
    5 & Twoja matka była potężnym Adeptem za życia, pomagała też ludziom w okolicy. Patrzą oni na Ciebie ciepło, ale także spodziewają się wiele po Tobie. \\ \hline
    6 & Wisisz pieniądze wielu ludziom i nie masz pieniędzy, by spłacić swój dług. \\ \hline
    7 & Zaliczyłeś gigantyczną klęskę w swoich początkowych studiach z nauczycielem i teraz uczysz się na własną rękę. \\ \hline
    8 & Nauczyłeś się swoich zdolności szybciej, niż Twoi nauczyciele widzieli u któregokolwiek ze swoich uczniów. Potężni tego świata zwrócili na Ciebie swoją uwagę i obserwują Cię intensywnie.  \\ \hline
    9 & Zabiłeś dobrze znanego kryminalistę w samoobronie, zyskując respekt wielu i nieprzyjaźń paru niebezpiecznych ludzi. \\ \hline
    10 & Uczyłeś się na Wojownika, ale Twoje uzdolnienia w kierunku Adepta ostatecznie skierowały Cię na odmienną ścieżkę. Twoi dawni kompani nie rozumieją Cię, ale mimo to Cię szanują. \\ \hline
    11 & Kiedy studiowałeś na Adepta, pracowałeś jako asystant w banku, zaprzyjaźniajac się z właścicielem i klientami. \\ \hline
    12 & Twoja rodzina posiada wielką winnicę niedaleko, znaną ze swojego dobrego wina i uczciwości biznesowej. \\ \hline
    13 & Trenowałeś przez pewien czas z grupą wpływowych Adeptów, którzy dalej darzą Cię przyjaźnią. \\ \hline
    14 & Pracowałeś w ogrodach pałacowych wpływowego szlachcica lub bogatej osoby. Nie pamięta ona Cię, ale zaprzyjaźniłeś się z jej młodą córką. \\ \hline
    15 & Eksperyment, który przeprowadziłeś w przeszłości, kompletnie nie wypalił. Ludzie z tamtej okolicy zapamiętali Cię jako niebezpiecznego i bezmyślnego typka. \\ \hline
    16 & Pochodzisz z dalekiego miejsca, gdzie byłeś dobrze znany i traktowany, ale ludzie tutaj traktują Cię z dużą podejrzliwością. \\ \hline
    17 & Ludzie, których spotykasz, wydają się trzymać na dystans ze względu na dziwne piętna na Twojej twarzy. \\ \hline
    18 & Twój najlepszy przyjaciel to także Adept. Ty i Twój przyjaciel dzielicie się odkryciami i sekretami. \\ \hline
    19 & Znasz lokalnego kupca bardzo dobrze. Ponieważ zapewniłeś mu dużo przychodu, oferuje Ci on zniżki i specjalne traktowanie.  \\ \hline
    20 & Należysz to sekretnego klubu, który spotyka sieco miesiąc, by wypić i porozmawiać. \\ \hline
 \end{tabularx}
  \caption {Historia Adepta}
  \label {Historia Adepta}
 \end{table*}
 
 \subsubsection{Adept Pierwszego Poziomu}
 
 Pierwszo-poziomowi Adepci posiadają następujące zdolności:
 
Wysiłek: Twój Wysiłek to 1.

Geniusz: Masz Skupienie w Intelekcie 1 oraz Skupienie w Mocy i Szybkości 0.

Eksperckie Korzystanie z Cypherów: Możesz nosić 3 cyphery w danym czasie.

Początkowy Ekwipunek: Stosowne ubranie, plus 2 drogie przedmioty, dwa przedmioty średniej ceny i do 4 niedrogich przedmiotów Twojego wyboru.

Bronie: Możesz korzystać z lekkich broni bez żadnej kary. Posiadasz nieumiejętność w średnich i ciężkich broniach – Twoje ataki z średnimi i ciężkimi broniami są utrudnione.

Specjalne zdolności: Wybierz 4 zdolności z poniższej listy. Nie możesz wybrać danej zdolności więcej niż 1 raz, chyba, że jej opis stanowi inaczej. Pełny opis każdej z dostępnych zdolności znajduje się w rozdziale \mytext{Zdolności}, który zawiera także zdolności Posmaków i specjalizacje w jednym, rozbudowanym katalogu. (Zdolności Adepta wymagają przynajmniej jednej wolnej ręki, chyba, że MG mówi inaczej.)

\begin{itemize}
\item Zamglenie
\item Usunięcie Wspomnień
\item Daleki Krok
\item Sztuczki Magiczne
\item Trening Magiczny
\item Pocisk
\item Pchnięcie
\item Pole Renozansowe
\item Skan
\item Strzaskanie
\item Magia Obronna
\end{itemize}

\subsubsection{Adept Drugiego Poziomu}

Wybierz jedną ze zdolności z poniższej listy (lub z niższego poziomu) i dodaj do swoich zdolności. Dodatkowo, możesz zamienić jedną ze zdolności z niższego poziomu na inną zdolność z niższego poziomu.

\begin{itemize}
\item Adaptacja
\item Czytanie Myśli
\item Odzyskanie Wspomnień
\item Ujawnienie
\item Unoszenie Się
\item Tnące Światło
\item Zastój
\end{itemize}

\subsubsection{Adept Trzeciego Poziomu}

Wybierz dwie zdolności z poniższej listy (lub z niższego poziomu) i dodaj do swoich zdolności. Dodatkowo, możesz wybrać jedną ze zdolności niższego poziomu i zamienić na inną z niższego poziomu. 

\cleardoublepage

\subsection{Odkrywca}\index{Typ!Odkrywca}

Fantasy/Baśń: Odkrywca, poszukiwacz przygód, badacz tajemnic.

Współczesność/Horror/Romans: atleta, odkrywca, poszukiwacz przygód, detektyw, badacz, pionier, reporter śledczy.

Science fiction: odkrywca, poszukiwacz przygód, podróżnik, planetolog, ksenobiolog.

Superbohaterowie/Post-apokalipsa: poszukiwacz przygód, stróż prawa.

Jesteś osobą akcji i fizycznych zdolności, bez lęku patrzącą ku nieodkrytemu. Podróżujesz do dziwnych, egzotycznych i niebezpiecznych miejsc, i odkrywasz nowe rzeczy. Oznacza to, że masz duże zdolności fizyczne, ale zapewne także jesteś dobrze wykształcony. 

Rola w grze: Choć Odkrywcy mogą być uczonymi i dobrze wykształconymi, są przede wszystkim zainteresowani akcją. Mierzą się ze śmiertelnymi niebezpieczeństwami i okropnymi przeszkodami praktycznie codziennie.

Rola w drużynie: Odkrywcy czasami pracują sami, ale częściej są częścią zespołu z innymi postaciami. Odkrywca często przoduje i przeciera szlak. Jednakże, często zatrzymują się i badają to, co ich zaintrygowało po drodze. 

Rola społeczna: Nie wszyscy Odkrywcy przedzierają się przez dzicz lub badają stare ruiny. Czasami, Odkrywca to nauczyciel, naukowiec, detektyw lub reporter śledczy. W każdym wypadku, Odkrywca z odwagą zmaga się z nowymi wyzwaniami i zbiera wiedzę, którą może się dzielić z innymi.

Zaawansowani Odkrywcy: Wysokopoziomowi Odkrywcy zyskują więcej umiejętności, trochę zdolności bojowych i dużo zdolności, które pomagają im poradzić sobie z niebezpieczeństwem. W skrócie, stają się uniwersalni, zdolni dać sobie radę z każdym wyzwaniem. 

\subsubsection{Odkrywca - Wtrącenia Gracza}

Kiedy grasz Odkrywcą, możesz wydać 1 PD by skorzystać z poniższych \mytext{wtrąceń gracza}, jeśli sytuacja jest odpowiednia i MG się zgadza.

Szczęśliwa Awaria: Pułapka lub niebezpieczne urządzenie doświadcza awarii, zanim może Ciebie zranić.

Nieoczekiwana Wskazówka: W momencie, gdy myślisz, że kompletnie zgubiłeś drogę, element krajobrazu, drogowskaz, lub po prostu ułożenie terenu sprawia, że odkrywasz najlepszą drogę naprzód, przynajmniej w tym momencie.

Słaba Trucizna: Trucizna lub choroba okazuje się nie być tak poważna, jak na początku wyglądała, i zadaje tylko połowę obrażeń, które zadałaby normalnie. 

\begin{table*}[t]
 \centering
 \begin{tabularx}{\textwidth}{ | X | X  |}
  \hline
   \textbf{Statystyka} & \textbf{Początkowa Wartość Puli} \\ \hline
    Moc & 10  \\ \hline
    Szybkość & 9  \\ \hline
    Intelekt & 9  \\ \hline
 \end{tabularx}
  \caption {Pula Statystyk Odkrywcy}
  \label {Pula Statystyk Odkrywcy}
 \end{table*}
 
 Otrzymujesz dodatkowe 6 punktów, które możesz rozdzielić pomiędzy swoje Pule zgodnie ze swoim życzeniem.
 
 \subsubsection{Historia Odkrywcy}
 
Twój typ pomaga Ci określić połączenie Twojej postaci z settingiem. Rzuć k20 lub wybierz z poniższej listy, by określić konkretny fakt o Twojej historii, który łączy Cię zresztą świata. Możesz także stworzyć swój własny fakt.

 \begin{table*}[t]
 \centering
 \begin{tabularx}{\textwidth}{| p{0.10\textwidth} | X |}
  \hline
  \textbf{k20} & \textbf{Historia Odkrywcy}  \\ \hline
    1 & Byłeś gwiazdą sportu w swoim liceum. Dalej jesteś w dobrej kondycji, ale człowieku, co to było wtedy! \\ \hline
    2 & Twój brat jest głównym śpiewakiem w naprawdę popularnym zespole. \\ \hline
    3 & Dokonałeś szeregu odkryć podczas swoich podróży, ale nie wszystkie okoliczności, by na nich zarobić, jeszcze pojawiły się przed Tobą. \\ \hline
    4 & Byłeś policjantem, ale zrezygnowałeś z pracy po doświadczeniu korupcji w siłach porządkowych. \\ \hline
    5 & Twoi rodzice byli misjonarzami, więc spędziłeś dużą część swojego młodego życia, podróżując do egzotycznych miejsc.  \\ \hline
    6 & Służyłeś w armii z honorem. \\ \hline
    7 & Otrzymałeś pomoc od sekretnej organizacji, która opłaciła Twoją edukację. Teraz ona pragnie znacznie więcej od Ciebie.  \\ \hline
    8 & Uczęszczałeś na prestiżowy uniwersytet dzięki stypendium dla sportowców, ale lśniłeś zarówno na boisku, jak i podczas zajęć.  \\ \hline
    9 & Twój najlepszy przyjaciel z dzieciństwa jest teraz wplywowym członkiem rządu. \\ \hline
    10 & Byłeś nauczycielem. Twoi studenci wspominają Cię miło. \\ \hline
    11 & Przez krótki czas byłeś kryminalistą, który został złapany i poszedł do więzenia – potem próbowałeś wyjść na prostą. \\ \hline
    12 & Twoje największe jak dotąd odkrycie zostało ukradzione przez Twojego rywala.  \\ \hline
    13 & Należysz do ekskluzywnej organizacji Odkrywców, której istnienie nie jest szeroko znane. \\ \hline
    14 & Zostałeś porwany jako dziecko w tajemniczych okolicznościach, are wróciłeś do domu bezpieczny. Media dalej czasami wspominają ową sytuację.  \\ \hline
    15 & Kiedy byłeś młody, byłeś uzależniony od narkotyków, a teraz powoli wstajesz na nogi. \\ \hline
    16 & Kiedy badałeś odległą lokację, dostrzegłeś coś, czego nigdy nie byłeś w stanie wyjaśnić. \\ \hline
    17 & Posiadasz mały bar lub restaurację. \\ \hline
    18 & Opublikowałeś książkę o swoich odkryciach i poczynaniach, która zyskała pewne uznanie. \\ \hline
    19 & Twoja siostra posiada sklep i daje Tobie pokaźną zniżkę. \\ \hline
    20 & Twój ojciec to wysoki rangą oficer w armii i posiada wiele koneksji. \\ \hline
 \end{tabularx}
  \caption {Historia Odkrywcy}
  \label {Historia Odkrywcy}
 \end{table*}
\input{src/Mówca.tex}
\subsubsection{Opcje Tworzenia Postaci - Fantasy}\index{Opcje Tworzenia Postaci!Fantasy}

W pewnych przypadkach, poniższe pomysły wymagają pewnych zmian zgodnie z Posmakiem, co opisano w opcjach postaci; powinieneś pracować ze swoim MG w celu aplikacji owych zmian, zgodnie z duchem kampanii. Większość specjalizacji w tej sekcji występuje w Cypher System – specjalizacje z gwiazdką (*) można znaleźć dalej w tym dokumencie. Niektóre z tych opcji sugerują zamianę zdolności z typu na zdolność z Posmaku takiego jak walka, magia lub skradanie się.

Alchemik: W rozumieniu tego, że alchemik to ktoś, kto robi magiczne przedmioty i tym podobne, Adept i Odkrywca to odpowiednie typy dla alchemika-naukowca. Aby stworzyć ogólnego alchemika, który robi mikstury z magicznymi właściwościami, wybierz specjalizację Włada Zaklęciami (zamiast zaklęć, masz eliksiry). Aby stworzyć alchemika, który zamienia się w potężną i niebezpieczną istotę, wybierz Wyje do Księżyca. Dla alchemika, który kocha rzucać bombami, wybierz Nosi Halo Ognia. Aby stworzyć uzdrowiciela, wybierz Uzdrawia.

Barbarzyńca: Barbarzyńca to najpewniej Wojownik lub (jeśli wolisz się skupić nie tylko na walce) Odkrywca. Dobre specjalizacje ,które można wybrać, to: Żyje w Dziczy, Mistrzowsko Posługuje się Bronią, Nie Potrzebuje Broni, Nigdy się Nie Poddaje, Jest Bardzo Silny i Wpada w Furię. 

Bard: Bardowie w fikcji fantasy i grach są trubadurami, minstrelami i opowiadaczami historii, być może z magicznymi zdolnościami. Bardowie to zazwyczaj Odkrywcy lub Mówcy. Odpowiednie specjalizacje to: Zabawia, Pomaga Swoim Przyjaciołom, Infiltruje i Włada Zaklęciami.

Kleryk lub Kapłan: Kapłani z dobrym wykształceniem to zazwyczaj Adepci lub Mówcy, ale wojowniczy są zazwyczaj Wojownikami (możliwe, że z Posmakiem magia). Aby stworzyć typowego kleryka z szerokim wachlarzem zdolności, wybierz specjalizację Otrzymuje Boskie Błogosławieństwo.

\begin{itemize}
\item Kleryk (burza): Ujeżdża Błyskawicę, Grzmi
\item Kleryk (oszustwo): Przyjmuje Zwierzęcy Kształt (patrz także opcje dla łotrzyków) 
\item Kleryk (śmierć): Zadaje się z Martwymi, Mówi z Duchami
\item Kleryk (światło): Jaśnieje Światłem, Otrzymuje Boskie Błogosławieństwo
\item Kleryk (wiedza): Szybko się Uczy, Jest Jasnowidzem, Wolałby Czytać
\item Kleryk (wojna): Mistrzowsko Posługuje się Bronią (patrz także opcje dla wojowników)
\item Kleryk (życie): Chroni Słabszych, Wspiera Społeczność, Uzdrawia
\end{itemize}

Zabójca/Szpieg: Odkrywca i Wojownik są dobrymi typami dla takiej postaci. Stosowne specjalizacje to Mistrzowsko Posługuje się Bronią, Porusza się jak Kot, Morduje i Pracuje w Ciemnych Uliczkach.

Druid: Jako bardzo specyficzny rodzaj kapłana natury, druid to zazwyczaj Adept lub Odkryca (obydwie opcje być może z Posmakiem magii). Typowy druid to ma najpewniej specjalność Otrzymuje Boskie Błogosławieństwo lub Żyje w Dziczy, a;e po bardziej specyficzne opcje, patrz niżej:

\begin{itemize}
\item Druid (transformacja): Jest Stworzony z Kamienia,  Przyjmuje Zwierzęcy Kształt*, Spaceruje w Dzikich Lasach*
\item Druid (więź z naturą): Mówi Głosem Ziemi
\item Druid (zwierzęcy towarzysz): Kontroluje Bestie, Włada Rojem
\item Druid (żywiołak): Jest Stworzony z Kamienia, Nosi Halo Ognia, Porusza się jak Wiatr, Ujeżdza Błyskawicę,  Rides the Lightning, Przywdziewa Połyskliwy Lód
\end{itemize}

Wojownik: Jak sama nazwa wskazuje, wojownik prawie zawsze będzie Wojownikiem, ale niektórzy to Badacze. Typowy wojownik najpewniej posiada bezpośrednią specjalność, taką jak Mistrzowski Posługuje się Bronią lub Dzierży Magiczną Broń*. Po dodatkowe opcje w zależności od specjalizacji, patrz poniżej:

\begin{itemize}
\item Wojownik (strażnik): Nosi Egzotyczną Tarczę, Chroni Wrót, Masters Defense, Nigdy się Nie Poddaje, Jest Jednoosobowym Bastionem.
\item Wojownik (walka na dystans): Ma Licencję na Broń, Rzuca ze Śmiertelną Dokładnością
\item Wojownik (wręcz): Walczy Nieczysto, Walcząc, Porywa Tłum, Szuka Kłopotów, Nie Potrzebuje Broni, Dzierży Dwie Bronie Naraz
\end{itemize}

Rewolwerowiec: Rewolwerowiec to najpewniej Wojownik lub Eksplorer, ale niektórzy są Mówcami z Posmakiem walki. Stosowne specjalności to Ma Licencję na Broń, Mistrzowsko Posługuje się Bronią, Pływał z Piratami i Dzierży Magiczną Broń*.

Inkwizytor: Inkwizytorzy to zazwyczaj Odkrywcy, Mówcy lub Wojownicy, w zależności od tego, czy gracz chce mieć wiele umiejętności, być dobrym w interakcji społecznej, lub w walce. Stosowne specjalności to Infiltruje, Zaprowadza Sprawiedliwość lub Działa pod Przykrywką.

Kupiec: Odkrywca ze specjalizacją skupioną na interakcjach społecznych, taką jak Zabawia lub Przewodzi, mógłby być dobrym kupcem, ale bardziej oczywistym wyborem byłby Mówca.

Mnich lub Mistrz Sztuk Walki: Jako mistrzowie walki bez broni, są to zazwyczaj Wojownicy lub Odkrywcy (możliwe, że z Posmakiem walki). Odpowiednie specjalizacje to Walcząc, Porywa Tłum, Nie Potrzebuje Broni i Rzuca ze Śmiertelną Dokładnością. 

Paladyn/Święty Rycerz: Jako święci wojownicy, którzy mają do dyspozycji magię i moce walki, paladyni to zazwyczaj Wojownicy lub Odkrywcy (w obydwu przypadkach zmodyfikowani Posmakiem magii). Dobre specjalności dla takiej postaci to Chroni Wrót, Chroni Słabszych, Zaprowadza Sprawiedliwość, Zabije Potwory i Dzierży Magiczną Broń. 

Łowca: Łowcy mają mieszankę moc walki i magicznych, i z tego względu są zazwyczaj Odkrywcami (możliwe, że z Posmakiem walki) lub Wojownikiem (możliwe, że z Posmakiem umiejętność i wiedza). Odpowiednie specjalności dla łowcy to: Kontroluje Bestie, Poluje, Żyje w Dziczy, Zabija Potwory, Rzuca ze Śmiertelną Dokładnością i Dzierży Dwie Bronie Naraz.

Łotrzyk lub Złodziej: Większość łotrzyków to Odkrywcy, ale postać skupiona na interakcjach społecznych mogłaby być Mówcą (możliwe, że z Posmakiem skradanie się). Specjalności dobre dla łotrzyka to Bada Ciemne Miejsca, Walczy Nieczysto, Poluje, Infiltruje, Jest Poszukiwany Przez Prawo, Porusza się jak Kot, Pływał z Piratami i Pracuje w Ciemnych Uliczkach.

Zaklinacz: Zaklinacza, to dla naszych potrzeb, magowie, którzy posiadają wrodzoną moc magiczną (w przeciwieństwie do czarodziejów, którzy muszą się tego nauczyć). Większość Zaklinaczy to Adepci, ale niektórzy są Odkrywcami lub Mówcami. Specjalność Włada Zaklęciami daje typowemu zaklinaczowi różne zdolności, a większość specjalności zapewnia zaklęcia tematyczne. Po zaklinaczy z poszczególnych linii krwi, patrz poniżej:

\begin{itemize}
\item Zaklinacz (anioł): Jaśnieje Światłem, Otrzymuje Boskie Błogosławieństwo, Posiada Magicznego Sprzymierzeńca
\item Zaklinacz (przeznaczenie): Ma Szlachetną Krew, Został Przepowiedziany
\item Zaklinacz (smok): Nosi Halo Ognia, Ujeżdża Błyskawicę, Przywdziewa Połyskliwy Lód
\item Zaklinacz (żywiołak): Jest Stworzony z Kamienia, Nosi Halo Ognia, Włada Magnetyzmem, Porusza się jak Wiatr, Ujeżdża Błyskawicę, Przywdziewa Połyskliwy Lód
\item Zaklinacz (fae): Przyjmuje Zwierzęcy Kształt
\item Zaklinacz (demon): Nosi Halo Ognia, Posiada Magicznego Sprzymierzeńca
\item Zaklinacz (nieumarły): Zadaje się z Martwymi, Mówi do Duchów
\end{itemize}

Trikster lub Oszust: Te bystrzaki to zazwyczaj Mówcy, ale niekiedy są Adeptami, jeśli są bardzo magiczni (lub Odkrywcami, jeśli nie są magiczni w ogóle). Wybór specjalności to między innymi Walczy Nieczysto, Pracuje w Ciemnych Uliczkach lub Zabawia.

Czarodziej wojenny: Te nietypowe postaci mieszają korzystanie z broni z magią – wybierz Wojownika z Posmakiem magii lub Odkrywcę z Posmakami magii lub walki. Specjalności, które mogę Cię zainteresować, to Walczą,c Porywa Tłum, Mistrzowsko Włada Broniami, lub Dzierży Magiczną Broń.  

Czarownik lub Wiedźma: Dla celów tej listy, czarownik i wiedźma to magowie, którzy uzyskali moc magiczną z paktu, który zawarli z bytami spoza rzeczywistości. Większość czarowników to Adepci, ale Odkrywcy lub Mówcy (możliwe, że z Posmakiem magia) mogą być ciekawymi opcjami. Odpowiednie specjalności to Tańczy z Czarną Materią, Posiada Magicznego Sprzymierzeńca, Włada Rojem, Izoluje Umysł od Ciała i Został Przepowiedziany. W zależności od patrona i paktu, większość specjalności zaklinacza i czarodzieja będzie ok.

Dziki mag: Ci, którzy korzystają z chaotycznej magii, to zazwyczaj Adepci, ale może t obyć także Odkrywca lub Mówca z Posmakiem magii. Najlepszą specjalizacją byłoby Włada Dziką Magią.

Czarodziej: Dla celów tej listy, czarodzieje uczą się magicznej wiedzy przez wiele lat, aby zdobyć zdolność rzucania zaklęć (w przeciwieństwie do zaklinaczy, czarnoksiężników itp.). Czarodzieje to zazwyczaj Adepci, ale czarodziej zorientowany na ludzi może być Mówcą (być może z Posmakiem magii). Aby stworzyć ogólnego czarodzieja, wybierz specjalizację Włada Zaklęciami. Po bardziej wyspecjalizowanych czarodziejów, patrz niżej

\begin{itemize}
\item Czarodziej (znawca odrzuceń): Absorbuje Energię, Stawia Umysł Ponad Materią, Przywdziewa Połyskliwy Lód
\item Czarodziej (znawca przywołań): Kontroluje Bestie, Posiada Magicznego Sprzymierzeńca
\item Czarodziej (znawca poznań): Szybko się Uczy, Jest Jasnowidzem, Izoluje Umysł od Ciała, Rozwiązuje Zagadki
\item Czarodziej (znawca zauroczeń): Włada Mocami Mentalnymi, Przewodzi
\item Czarodziej (znawca wywołań): Nosi Halo Ognia, Jaśnieje Światłem, Ujeżdza Błyskawicę, Grzmi, Przewdziewa Połyskliwy Lód
\item Czarodziej (iluzjonista): Przebudza Sny, Tworzy Iluzje
\item Czarodziej (nekromanta): Zadaje sięz Umarłymi, Mówi do Duchów
\item Czarodziej (znawca transmutacji): Kontroluje Grawitację, Stawia Umysł Ponad Materią, Przyjmuje Zwierzęcy Kształt
\end{itemize}

\paragraph{Zaklęcia Przygotowane i Spontaniczne}\index{Zaklęcia Przygotowane i Spontaniczne}

Magiczne postaci otrzymują swoje zdolności (które mogą być zaklęciami, rytuałami lub czymś innym) ze swojego typu i specjalności, i mogą korzystać z owych zdolności jak uzn ają za stosowne tak długo, jak wydadzą punkty z Puli. To technicznie czyni ich bardziej jak spontaniczny czarujący. Jeśli wolisz zagrać czymś bardziej jak czarodziej przygotowujący zaklęcia, z większą ilością czarów, z których wybierasz małą ilość każdego dnia, rozważ specjalność skupioną na zaklęciach, taką jak Otrzymuje Boskie Błogosławieństwo, Włada Zaklęciami lub Mówi Głosem Ziemi i rozważ dalszą customizację opcjonalną zasadą rzucania zaklęć. 
\subsubsection{Dalsza Customizacja}\index{Dalsza Customizacja}

Zasady w tej sekcji są bardziej zaawansowane i zawsze zależą od MG. Mogą być użyte, by MG dostosował typ do konwencji lub settingu, lub przez gracza i MG, by dostosować koncept postaci.

\paragraph{Modyfikacja Aspektów Typu}

Poniższe aspekty czterech typów postaci mogą zostać zmodyfikowane podczas tworzenia postaci. Inne zdolności nie powinny być zmienione.

Pule Statystyk: Każda Pula postaci ma wartość startową. Gracz może zamieniać punkty w Pulach kosztem 1-na-1. Dla przykładu, może on przesunąć 2 punkty z Mocy na 2 punkty w Szybkości. Jednakże, żadna początkowa Statystyka nie może być wyższa niż 20.

Skupienie: Gracz może zacząć grę ze Skupieniem w dowolnej Statystyce na 1.

Korzystanie z Cypherów: Jeśli gracz odda zdolność noszenie jednego cyphera, uzyskuje on dodatkową umiejętność swojego wyboru.

Bronie: Pewne typy mają statyczne zdolności pierwszego poziomu które pozwalają im korzystać z lekkich, średnich i/lub ciężkich broni bez kary. Wojownicy mogą korzystać z wszystkich broni, Odkrywcy z lekkich i średnich, a Adepci i Mówcy mogą korzystać tylko z lekkich broni. Każda z tych zdolności może zostać poświęcona, by zyskać trening w odmiennej umiejętności, którą wybierze gracz.

\paragraph{Wady i Kary}

W dodatku do innych opcji customizacji, gracz może wybrać wzięcie kar lub wad, by zyskać dalsze bonusy.

Słabość: Słabość to, esencjalnie, przeciwieństwo Skupienia. Jeśli masz Słabość 1 w Szybkości, wszystkie akcje Szybkości wymagają od Ciebie dodatkowego 1 punktu z Twojej Puli. W każdym momencie, gracz może dać swojej postaci słabość w jednej statystycei otrzymać +1 do Skupienia w jednej z pozostałych dwóch. Tak więc gracz może wziąć słabość 1 w Szybkości i otrzymać +1 do swojego Skupienia w Mocy.

Normalnie, możesz mieć słabość tylko w statystyce, w której Twoje Skupienie wynosi 0. Co więcej, nie możesz mieć więcej niż jednej słabości, i niem ożesz mieć słabości większej niż 1, chyba, że dodatkowa słabość pochodzi z innego źródła (takiego jak zaraza lub niepełnosprawność wynikająca z akcji lub kondycji w grze).

Nieumiejętność: Nieumiejętności są jak negatywne umiejętności. Czynią jeden rodzaj akcji trudniejszym. Jeśli postać wybiera nieumiejętność, zyskuje ona umiejętność swojego wyboru. Normalnie, postać może mieć tylko jedną nieumiejętność, chyba, że pozostałe pochodzą z innego źródła (takiego jak deskryptor, choroba lub niepełnosprawność wynikająca z akcji lub kondycji w grze).

\paragraph{Posmaki}

Posmaki to grupa specjalnych zdolności które MG i gracze mogą wykorzystać, aby zmienić typ postaci – np.: w zgodzie z settingiem lub konwencją. Dla przykładu, jeśli gracz chce stworzyć czarodzieja, który jest też złodziejem, może zagrać Adeptem z Posmakiem w skradaniu się. W settingu science fiction, Wojownik może mieć także wiedzę o maszynach, więc postać może mieć posmak w technologii. 

Na danym poziomie, zdolności z standardowego typu są wymieniane na zdolności z posmaku. Tak więc, aby dodać Zmysł Niebezpieczeństwa z posmaku skradanie się do Wojownika, trzeba poświęcić coś innego – być może Ogłuszenie. Teraz postać może wybrać Zmysł Niebezpieczeństwa, tak jak każdą inną zdolność pierwszego poziomu, ale nigdy nie może wziąć Ogłuszenia.

MG zawsze powinien brać udział w modyfikacji typu za pośrednictwem posmaku. Dla przykładu, może on określić, że w grze science fiction chce stworzyć type zwany “Glam”, czyli Mówcę z pewnymi zdolnościami technologicznymi – konkretniej tymi, które czynią z niego ekstrawaganckiego pilota statków kosmicznych. Tak więc, zamienia on pierwszopoziomowe zdolności Fałszywa Tożsamość i Zainspirowanie Agresji na Implant Wizualnej Identyfikacji i Umiejętności Technologiczne, tak, że postać może połączyć siębezpośrednio ze statkiem i mieć umiejętności komputerowe i pilotażu.

Ostatecznie, posmak to głównie narzędzia dla MG do łatwego tworzenia typów zrobionych pod kampanie, poprzez parę lekkich zmian tu i ówdzie. Choć gracze mogą chcieć skorzystać z posmaków, by stworzyć postać, jakiej pragną, pamiętaj, że mogą oni także dookreślić swojego BG przy pomocy deskryptorów i specjalizacji. 

Dostępne posmaki to: skradanie się, technologia, magia, walka oraz umiejętności i wiedza.
Pełen opis wszystkich zdolności można znaleźć w rozdziale Zdolności, który zawiera także opisy zdolności typów i specjalności w jednym pokaźnym katalogu.


\section{Deskryptor}\index{Deskryptory}

Twój deskryptor definiuje Twoją postać – koloruje wszystko, co robisz. Różnice między Urokliwym Odkrywcą a Złośliwym Odkrywcą są dosyć spore. Deskryptory zmieniają praktycznie wszystkie akcje owych postaci. Twój deskryptor stawia postać w sytuacji (pierwszej przygodzie, która zaczyna kampanię) i pomaga dać jej motywację. Jest to przymiotnik w zdaniu “Jestem przymiotnik rzeczownik który czasownikuje". 

Deskryptory oferują jednorazową paczkę dodatkowych umiejętności, zdolności i modyfikatorów do Twoich Statystyk. Nie wszystkie oferowane przez deskryptor modyfikacje są pozytywne. Dla przykładu, niektóre deskryptory posiadają nieumiejętności – zadania, w których postać sobie nie radzi. Możesz myśleć o nieumiejętnościach jak o negatywnych umiejętnościach – zamiast ułatwiać zadanie o krok, czynią one je o krok trudniejszym. Jeśli zdobywasz trening w sferze, w której masz nieumiejętność, nawzajem się one znoszą. Pamiętaj, że postaci są zdefiniowane równie mocno przez to, z czym sobie nie radzą, jak i przez to, w czym są dobre. 

Deskryptory oferują także krótkie sugestie odnośnie tego, jak postać zapoznała się z resztą grupy na swojej pierwszej przygodzie. Możesz z nich skorzystać, jeśli tylko sobie tego życzysz (lub nie, Twoja wola).

Ta sekcja zawiera 50 deskryptorów. Wybierz jeden z nich dla swojej postaci. Możesz wybrać dowolny deskryptor, niezależnie od swojego typu. Na końcu rozdziału jest parę opcji customizacyjnych, wliczając w to tworzenie rasy jako deskryptora.
(Deskryptor ma największe znaczenie dla początkującej postaci. Benefity (i utrudnienia) związane z deskryptorem będą ostatecznie przyciemnione przez rosnące znaczenie typu i specjalności. Jednakże, deskryptor dalej będzie odgrywał pewne znaczenie w ciągu życia postaci). 

\subsection{Lista Deskryptorów}\index{Deskryptory!Lista}

\subsubsection{Bystrooki}\index{Deskryptory!Bystrooki}

Jesteś percepcyjny i dobrze świadom swojego otoczenia. Dostrzegasz małe detale i zapamiętujesz je. Jesteś trudny do zaskoczenia.

Otrzymujesz poniższe cechy:
\begin{itemize}
\item Umiejętność: Jesteś wyszkolony w inicjatywie.
\item Umiejętność: Jesteś wyszkolony w percepcji.
\item Znaleźć Wadę: Jeśli Twój przeciwnik ma jakaś oczywistą wadę (otrzymuje więcej obrażeń od ognia, nie widzi na lewe oko itp.) MG powie Ci o tym.
\end{itemize}
    
Początki Przygód: Z poniższej listy opcji, wybierz jak Twoja postać wzięła udział w pierwszej przygodzie.

1. Słyszałeś, co się święciło, dostrzegłeś wadę w planach BG i dołączyłeś do nich, by im pomóc.

2. Zauważyłeś, że BG mieli wroga (lub przynajmniej ktoś ich śledził) i nie byli tego świadomi.

3. Dostrzegłeś, że inni BG robili coś ciekawego i się do nich przyłączyłeś.

4. Od pewnego czasu działy się dziwne rzeczy, i to wszystko wydaje się być powiązane.

\subsubsection{Bystry}\index{Deskryptory!Bystry}

Myślisz szybko i bystro. Rozumiesz ludzi i możesz ich oszukiwać, ale sam rzadko kiedy dajesz się omamić. Ponieważ łatwo widzisz rzeczy takimi, jakie są, wchodzisz szybko, oceniasz zagrożenia i sprzymierzeńców, a potem robisz swoją robotę. Może jesteś fizycznie atrakcyjny, lub może korzystasz ze swojego mózgu, by przezwyciężyć fizyczne i psychiczne niedoskonałości. 

Otrzymujesz poniższe cechy:
\begin{itemize}
\item Bystry: +2 do Puli Intelektu.
\item Umiejętność: Jesteś wyszkolony we wszystkich interakcjach typu kłamstwa lub sztuczki.
\item Umiejętność: Jesteś wyszkolony w rzutach na przeciwstawianie się efektom mentalnym.
\item Umiejętność: Jesteś wyszkolony we wszystkich zadaniach identyfikowania lub oceniania niebezpieczeństwa, kłamstw, jakości, ważności, funkcji lub mocy.
\item Nieumiejętność: Nigdy nie byłeś dobry w nauce lub przywoływaniu z pamięci trywialnej wiedzy. Każde zadanie związane z wiedzą lub zrozumieniem jest utrudnione.
\item Dodatkowy ekwipunek: Łatwo dostrzegasz plany innych i czasami przekonujesz ich, by Tobie wierzyli – nawet kiedy nie powinni. Dzięki swojemu bystremu zachowaniu, masz dodatkowy drogi przedmiot.
\end{itemize}

Początki Przygód: Z listy poniższych opcji, wybierz jak Twoja postać wzięła udział w pierwszej przygodzie.

1. Przekonałeś jednego z reszty BG, by powiedział Ci co robią.

2. Z daleka dostrzegłeś, że coś ciekawego ma miejsce.

3. Wplątałeś się w tę sytuację, bo myślałeś, że możesz w ten sposób zarobić trochę pieniędzy.

4. Spodziewałeś się, że bez Ciebie BG nie odniosą sukcesu.

\subsubsection{Chaotyczny}\index{Deskryptory!Chaotyczny}

Niebezpieczeństwo nie ma dla Ciebie dużego znaczenia, głównie dlatego, że nie myślisz o konsekwencjach. Po prawdzie, rozkoszujesz się zaskakiwaniem, żeby tylko zobaczyć, co się stanie. Im bardziej niespodziewany rezultat, tym jesteś szczęśliwszy. Czasami jesteś szczególnie maniakalny i dla swoich towarzyszy powstrzymujesz się przed podejmowaniem akcji, które mogą prowadzić do zagłady. 

Otrzymujesz następujące cechy:
\begin{itemize}
 \item Wzburzony: +4 do Puli Szybkości.
 \item Umiejętność: Jesteś wyszkolony w Obronie Intelektu.
 \item Chaotyczny: Raz po każdym 10-godzinnym rzucie na odzyskanie zdrowia, jeśli nie lubisz swojego wyniku, możesz przerzucić kostkę. Jeśli to zrobisz, niezależnie od wyniku, MG stosuje względem Ciebie wtrącenie MG.
\item Nieumiejętność: Twoje ciało jest nieco zmęczona całą tąchaotycznością. Obrona Mocy jest dla Ciebie utrudniona.
\end{itemize}

Początki Przygód: Z listy poniższych opcji, wybierz jak Twoja postać wzięła udział w pierwszej przygodzie.

1. Inny BG zrekrutował Cię, kiedy dobrze się zachowywałeś, nie zdawszy sobie sprawy z tego, jak bardzo jesteś chaotyczny.

2. Masz powody by wierzyć, że inni BG pomogą Ci kontrolować swoje chaotyczne zachowanie.

3. Inny BG wyzwolił Cię z niewoli, i aby mu podziękować, zaoferowałeś swoją pomoc.

4. Nie masz bladego pojęcia, czemu dołączyłeś do BG. Po prostu to robisz, a odpowiedzi poznasz z czasem.

\subsubsection{Ciekawy}\index{Deskryptory!Ciekawy}

Świat jest wielki i tajemniczy, z cudami i sekretami, które starczą na kilka żywotów. Czujesz wołanie w swoim sercu, zew, by badać pozostałości dawnych cywilizacji, by odkryć nowe ludy, nowe miejsca, i jakiekolwiek dziwne cuda znajdziesz po drodze. Jednakże, choć czujesz potężną chęć, by podróżować po świecie, wiesz, że jest pełen niebezpieczeństw, i musisz się zabezpieczyć na każdą możliwość. Badania, przygotowania i gotowość pomogą Ci żyć dostatecznie długo, by zobaczyć wszystko co chcesz zobaczyć i zrobić wszystko, co chcesz zrobić.

Najpewniej masz tuzin książek i map przy sobie w dowolnym czasie. Kiedy nie podróżujesz i nie chłoniesz widoków, spędzasz czas z nosem w książce, uczać się o miejscu, do którego podrożujesz, żebyś wiedział, czego się spodziewać w tym miejscu.

Otrzymujesz poniższe cechy:
\begin{itemize}
\item  Bystry: +4 do Puli Intelektu.
\item Umiejętność: Jesteś chętny do nauki. Jesteś wyszkolony w każdym zadaniu, które jest nauką czegoś nowego, niezależnie, czy to lokalna informacja, czy przeszukiwanie starych ksiąg wiedzy.
\item  Umiejętność: Uczyłeś się o świecie. Jesteś wyszkolony w każdym zadaniu powiązanych z geografią lub historią.
\item Nieumiejętność: Skupiasz się na detalach, co sprawia, że jesteś nieco zagubiony. Wszelkie zadania w celu usłyszenia lub dostrzeżenia czegoś są dla Ciebie utrudnione.
\item Nieumiejętność: Kiedy widzisz coś interesującego, wahasz się, aavy dostrzec wszystkie detale. Rzuty na inicjatywę (określające kto rusza pierwszy w walce) są dla Ciebie utrudnione.
\item Dodatkowy Ekwipunek: Masz trzy ksiażki o dowolnych tematach, które wybierzesz.
\end{itemize}

Początki Przygód: Z listy poniższych opcji, wybierz jak Twoja postać wzięła udział w pierwszej przygodzie.

1. Jeden z BG podszedł do Ciebie, aby zdobyć informację powiązaną z misją, słysząc, że jesteś ekspertem.

2. Zawsze chciałeś zobaczyć miejsce, do którego podróżuje reszta BG.

3. Zainteresowało Cię to, co chcieli zrobić inni BG i zadecydowałeś, że będziesz im towarzyszyć.

4. Jeden z BG Cię fascynuje, może ze względu na specjalną lub dziwną zdolność, którą on dysponuje.


\subsubsection{Dziki}\index{Deskryptory!Dziki}

Kochasz dziką przyrodę i jesteś przyzwyczajony do życia w trudnych warunkach, wśród dziczy. Najpewniej jesteś uzdolnionych łowcą lub naturalistą. Lata życia w dziczy zostawiły swoje znaki na Twoim ciele – masz znoszone ubrania, burzę dzikich włosów lub blizny. Twoje ubrania są najpewniej znacznie mniej modne niż ubrania ludzi mieszkających w miastach. 

Otrzymujesz poniższe cechy:
\begin{itemize}
 \item Umiejętność: Jesteś wyszkolony w wszystkich zadaniach typu wspinaczka, skakanie, bieganie i pływanie.
\item Umiejętność: Jesteś wyszkolony we wszystkich zadaniach typu trenowanie, ujeżdzanie i uspokajanie naturalnych zwierząt.
\item Umiejętność: Jesteś wyszkolony we wszystkich zadaniach polegających na identyfikacji lub używaniu naturalnych roślin.
\item Nieumiejętność; Nie jesteś bardzo społeczny i wolisz towarzystwo zwierząt niż ludzi. Wszelkie zadania polegające na oczarowaniu innych ludzi, perswazji, etykiecie lub oszustwie są dla Ciebie utrudnione.
\item Dodatkowy Ekwipunek: Posiadasz pakiet odkrywcy, z liną, dwoma dniami racji żywnościowych, matą do spania i innymi narzędziami potrzebnymi do przetrwania na zewnątrz.
\end{itemize}    
    
Początki Przygód: Z listy poniższych opcji, wybierz jak Twoja postać wzięła udział w pierwszej przygodzie.

1. Pomimo Twojego lepszemu osądowi, dołączyłsz do BG, ponieważ widziałeś, że są w niebezpieczeństwie.

2. Jeden z BG przekonał Cię, że dołączenie do grupy jest w Twoim najlepszym interesie.

3. Boisz się, co może się wydarzyć, jeśli BG odniosą klęskę.

4. Jest mowa o nagrodzie, a Ty potrzebujesz pieniędzy.

\subsubsection{Dziwny}\index{Deskryptory!Dziwny}

Nie jesteś taki jak inni, i to w porządku (według Ciebie). Ludzie nie są w stanie Cię zrozumieć – niektórzy nawet się Ciebie boją – ale kogo to obchodzi? Rozumiesz świat lepiej niż oni, ponieważ jesteś dziwny, zupełnie jak świat, w którym żyjesz. Koncept “dziwności” jest Tobie dobrze znany. Dziwne urządzenia, antyczne miejsca, dziwne istoty, burze, które mogą Cię zmutować, żyjące pola energii, konspiracje, obcy i rzeczy, których większość ludzi nie mogłaby nazwać istnieją w tym świecie, a Ty masz się z tym świetnie. Masz specjalną więź z tym wszystkim, a im więcej odkrywasz dziwności w świecie, tym lepiej rozumiesz samego siebie. Dziwne postaci mogą być mutantami lub ludźmi z wrodzonymi dziwnymi cechami, ale czasami zaczęli jako “normalni” i dziwaczeli dopiero w trakcie życia.

Otrzymujesz poniższe cechy:
\begin{itemize}
\item Wewnętrzne Światło: +2 do Puli Intelektu.
\item Fizyczne Znamię Dziwności: Masz unikalną cechę fizyczną, która jest, no cóż, dziwna. W zależności od settingu, mogą to być różne rzeczy. Może masz fioletowe włosy lub metalowe kolce w swojej głowie. Może Twoje ręce nie łączą się z Twoimi ramionami, choć poruszają się tak, jakby były połączone. Może masz trzecie oko na czole, a może bezużyteczne wici wyrastają z Twoich pleców. Cokolwiek to jest, Twoje znamię może być mutacją, cechą nadprzyrodzoną (błogosławieństwem lub klątwą), nie mieć żadnego wyjaśnienia lub być po prostu naprawdę dzikim tatuażem, który przyciąga mnóstwo uwagi.
\item Zmysł Dziwności: Czasami – za uznaniem MG – dziwne rzeczy związane z zjawiskami nadprzyrodzonymi lub ich wpływem na świat coś w Tobie budzą. Możesz je wyczuć z daleka, a jeśli znajdziesz się w dalekim zasięgu od takiej rzeczy, możesz wyczuć czy jest niebezpieczna czy nie.
\item Umiejętność: Jesteś wyszkolony w nadprzyrodzonej wiedzy.
 \item Nieumiejętność: Ludzie uważają Cięza dziwnego. Wszystkie zadania związane z przyjemną interakcją społeczną są dla Ciebie utrudnione.
\end{itemize}
    
Początki Przygód: Z listy poniższych opcji, wybierz jak Twoja postać wzięła udział w pierwszej przygodzie.

1. Wyglądało to na dziwne, czemu więc nie?

2. Niezależnie od tego, czy inni BG zdają sobie z tego sprawę, czy też nie, ich misja jest powiązana z czymś dziwnym, o czym wiesz, więc wziąłeś w niej udział.

3. Jako ekspert od dziwności, zostałeś zrekrutowany przez innych BG.

4. Poczułeś dążenie, by dołączyć do innych BG, ale nie wiesz czemu.

\subsubsection{Empatyczny}\index{Deskryptory!Empatyczny}

Inni ludzie to dla Ciebie otwarte księgi. Możesz mieć talent do odczytania ludzkich uczuć, tych subtelnych ruchów, które zawierają w sobie nastrój i emocje. Lub możesz otrzymywać informacje w bardziej bezpośredni sposób, czując emocje danej osoby jakby były materialne, wrażenia, które odbiera Twój umysł. Twój dar empatii pomaga Ci nawigować w sytuacjach społecznych i kontrolować je, by uniknąć nieporozumień i uniemożliwić erupcje bezsensownych konfliktów. 
Ciągłe bombardowanie emocjami ludzi wokół Ciebie jest jednak męczące. Możesz się dać porwać dominującemu nastrojowi, mieć wahania nastrojów od radości do smutku bez żadnego ostrzeżenia. Lub możesz się zamknąć w sobie i pozostać zagadką dla innych, ze względu na chęć chronienia siebie i podświadomy lęk, że ktoś może się dowiedzieć, jak się naprawdę czujesz.

Otrzymujesz poniższe cechy:
\begin{itemize}
\item Otwarty Umysł: +4 do Puli Intelektu
\item Umiejętność: Jesteś wyszkolony w zadaniach związanych z odczuwaniem emocji innych i przeczuć odnośnie ludzi wokół Ciebie.
\item Umiejętność: Jesteś wyszkolony we wszystkich akcjach związanych ze społecznymi interakcjami, przyjemnymi bądź nie.
\item Nieumiejętność: Bycie tak bardzo otwartym na myśli i nastroje innych czyni Cię narażonym na ataki mentalne. Twoja Obrona Intelektu jest utrudniona.
\end{itemize}

Początki Przygód: Z listy poniższych opcji, wybierz jak Twoja postać wzięła udział w pierwszej przygodzie.

1. Wyczułeś oddanie do zadania innych BG i postanowiłeś im pomóc.

2. Utworzyłeś bliską więź z innym BG i nie chcesz się z nim rozstawać.

3. Wyczułeś coś dziwnego w jednym z BG i zdecydowałeś się dołączyć do ich grupy by sprawdzić, czy wyczujesz to raz jeszcze i odkryjesz prawdę.

4. Dołączyłeś do BG by uciec od nieprzyjemnej relacji lub negatywnego środowiska. 

\subsubsection{Honorowy}\index{Deskryptory!Honorowy}
Jesteś godny zaufania, uczciwy i szczery. Próbujesz zrobić to, co powinno być zrobione, pomagać innym i traktować ich dobrze. Kłamanie i oszukiwanie nie są sposobem, by triumfować w życiu – to wybory słabych, leniwych i godnych potępienia. Najpewniej spędzasz dużo czasu myśląc o Twoim osobistym honorze, jak najlepiej go utrzymać i jak bronić go jeśli jest zagrożony. W walce jesteś bezpośredni i oferujesz litość każdemu z wrogów.

Najpewniej poczucie honoru zaszczepił w Tobie rodzic bądź mentor. Czasami rozróżnienie między tym co jest, a co nie jest honorowe jest zależne od szkoły myślowej, ale ogólnie rzecz ujmując, honorowi ludzie mogą się zgodzić co do większości aspektów znaczenia honoru.

Otrzymujesz następujące cechy:
\begin{itemize}
\item Dzielny: +2 do Puli Mocy.
\item Umiejętność: Jesteś wyszkolony w przyjemnych interakcjach społecznych.
\item Umiejętność: Jesteś wyszkolony w odczytywaniu prawdziwych intencji innych osób i dostrzeganiu kłamstw.
\end{itemize}
    
Początki Przygód: Z listy poniższych opcji, wybierz jak Twoja postać wzięła udział w pierwszej przygodzie.

1. Cele BG wydają się być honorowe.

2. Widzisz, ze to, co chcą zrobić BG, jest niebezpieczne, i chciałbyś im pomóc.

3. Jeden z BG zaprosił Cię, słysząc, że jesteś godny zaufania.

4. Zapytałeś grzecznie, czy mógłbyś dołączyć do BG.

\subsubsection{Idiotyczny}\index{Deskryptory!Idiotyczny}

Nie każdy może być bystry jak rzeka.  Oh nie, nie myślisz o sobie jak o głupim, co to to nie. Po prostu inni zdają się mieć więcej… mądrości. Wejrzenia w rzeczy. Preferujesz ruszać prosto do celu poprzez życie i pozwalasz innym martwić się rzeczami. Zamartwianie się nigdy Ci nie pomogło, więc po co to robić? Bierzesz rzeczy na klatę i nie martwisz się dniem jutrzejszym.

Ludzie swą Cię “idiotą” bądź “głupcem”, ale nie wpływa to jakoś bardzo mocno na Ciebie. 

(Może być bardzo fajnym doświadczeniem odgrywaniem idiotycznej postaci. W pewien sposób zrzuca to presję, by zawsze robić dobre, mądre rzeczy. Z drugiej strony, jeśli grasz taką postacią jako jawnym głupcem w każdej sytuacji, może to być denerwujące dla innych przy stole. Jak ze wszystkim, należy znaleźć złoty środek i rozmawiać z innymi graczami o naszych potrzebach i uwagach krytycznych.)

Zyskujesz poniższe cechy:
\begin{itemize} 
\item Niemądry: -4 do puli Intelektu
\item Wolny: Polegasz na szczęściu bardziej niż na czymkolwiek innym. Za każdym razem, gdy rzucasz kością na jakieś zadanie, rzuć dwa razy i weź wyższy wynik.
\item  Słabość Intelektu: Za każdym razem, gdy wydajesz punkty z Puli Intelektu, kosztuje Cię to o 1 punkt więcej niż normalnie.
\item Nieumiejętność: Twoja Obrona Intelektu jest utrudniona.
\item Nieumiejętność: Każde zadanie, które wymaga dostrzeżenia kłamstwa, iluzji lub pułapki jest utrudnione.
\end{itemize}

Początki Przygód: Z listy poniższych opcji, wybierz jak Twoja postać wzięła udział w pierwszej przygodzie.

1. Kto wie? Wyglądało to jak dobry pomysł.

2. Ktoś poprosił Cię o dołączenie do BG. Ta osoba chciała, byś nie zadawał zbyt wiele pytań, więc tak postąpiłeś.

3. Twój ojciec (lub rodzic/mentor) chciał, żebyś miał zajęcie i może “zdobył trochę rozsądku”.

4. Inni BG potrzebowali siłacza, który nie zastanawiałby się zbyt długo nas swoimi zadaniami.

\subsubsection{Impulsywny}\index{Deskryptory!Impulsywny}

Masz problem z ograniczaniem swojego entuzjazmu. Po co czekać, kiedy można to po prostu zrobić (cokolwiek to jest) i mieć problem z głowy? Radzisz sobie z problemami w momencie, gdy powstają, zamiast planować w przyszłość. Gaszenia małych ognisk sprawia, że potem nie będzie wielkich pożarów. Jesteś pierwszym, który ryzykuje i udziela pomocy, który wstępuje w ciemne korytarze i który znajduje niebezpieczeństwo.

Twoja impulsywność z pewnością sprawia, że niekiedy masz kłopoty. Kiedy inni mogą poświęcić czas na studiowanie przedmiotów, które pozyskali, TY korzystasz z nich bez chwili wahania. Przecież najlepszym sposobem nauki jest akt sprawczy. Kiedy ostrożny odkrywca może rozejrzeć się i poszukać niebezpieczeństw, Ty musisz powstrzymać się siłą przed ruszeniem naprzód. Po co czekać, skoro wokół jest tyle ekscytujących rzeczy?

(Impulsywne postaci ściągają na siebie kłopoty. To ich cecha stała i to jest ok. Ale jeśli ciągle ściągasz innych BG w środek kłopotów (lub, co gorsza, sprawiasz, że są zranieni lub martwi), to będzie to bardzo wkurzające, oględnie mówiąc. Dobrą regułą jest stwierdzenie, że impulsywność nie zawsze oznacza robienie złych rzeczy. Czasami, jest to żądza, by zrobić coś właściwego.)

Otrzymujesz następujące cechy:
\begin{itemize}
\item W Gorącej Wodzie Kąpany: +2 do puli Szybkości.
\item  Umiejętność: Jesteś wyszkolony w rzutach na inicjatywę (by określić, kto działa pierwszy w walce)
\item Umiejętność: Jesteś wyszkolony w Obronie Szybkości.
\item Nieumiejętność: Możesz spróbować wszystkiego jeden raz, ale następne próby Cię nużą. Każde zadanie, które wymaga cierpliwości, siły woli lub dyscipliny jest dla Ciebie utrudnione. 
\end{itemize}

Początki Przygód: Z listy poniższych opcji, wybierz jak Twoja postać wzięła udział w pierwszej przygodzie.

1. Usłyszałeś co planowali inni BG i nagle zadecydowałeś, że do nich dołączysz.

2. Zebrałeś wszystkich razem po usłyszeniu plotki o czymś interesującym, co chcesz zobaczyć lub zrobić.

3. Wydałeś wszystkie swoje pieniądze i teraz potrzebujesz źródła dochodu.

4. Jesteś w kłopotach po podążaniu za głosem swojego serca. Dołączyłeś do BG, bo oferują sposób na wyjście ze swoich problemów.

\subsubsection{Inteligentny}\index{Deskryptory!Inteligentny}

Jesteś dosyć bystry. Twoja pamięć jest ostra jak brzytwa, i z łatwością pojmujesz koncepty, których inni nie rozumieją. Nie oznacza to koniecznie, że masz za sobą lata formalnej edukacji, ale sporo się nauczyłeś w życiu, głównie przy okazji.

Zyskujesz poniższe cechy:
\begin{itemize}
\item Bystry: +2 do Puli Intelektu.
\item Umiejętność:Jesteś wyszkolony w jednej domenie wiedzy swojego wyboru.
\item Umiejętność: Jesteś wyszkolony we wszystkich zadaniach polegającym a zapamiętywaniu tego, czego bezpośrednio doświadczyłeś. Dla przykładu, zamiast przypominać sobie o szczegółach geograficznych, o których czytałeś w książce, możesz pamiętać ścieżkę przez tunele, którymi wcześniej podążałeś.
\end{itemize}

Początki Przygód: Z listy poniższych opcji, wybierz jak Twoja postać wzięła udział w pierwszej przygodzie.

1. Inny z BG zapytał Cię o Twoją opinię o tej misji, wiedząc, że jeśli powiesz, że to dobry pomysł, to zapewne tak będzie.

2. Dostrzegłeś wartość w tym, co robią inni BG.

3. Wierzysz, że to zadanie może prowadzić do ważnych i ciekawych odkryć.

4. Kolega poprosił cię o wzięcie udziału w tej misji jako przysługę, którą mu byłeś winny.

\subsubsection{Intuitywny}\index{Deskryptory!Intuitywny}

Często masz przeczucie co ktoś inny powie, jak zareaguje bądź jak wydarzenia się potoczą. Może masz jakiś zmysł mutanta, może możesz na parę chwil spojrzeć w bliską przyszłość, a może po prostu jesteś w stanie odczytać mimikę i mowę ciała ludzi. Niezależnie od powodu, wielu z tych, którzy patrzą w Twe oczy, natychmiast odwraca wzrok, jakby byli przerażeni tym, co w nich odczytasz.

Zyskujesz poniższe cechy:

\begin{itemize}
\item Intuicja: +2 do Puli Intelektu.
\item Umiejętność: Jesteś wyszkolony w percepcji.
\item Wiesz, co Czynić: Możesz zareagować natychmiast, nawet jeśli to jeszcze nie Twoja tura. Potem, w Twojej następnej normalnej turze, każda akcja, którą wykonujesz, jest utrudniona. Możesz to zrobić jeden raz, ale ta opcja odnawia się po każdym rzucie na odzyskanie zdrowia.
\end{itemize}

Początki Przygód: Z listy poniższych opcji, wybierz jak Twoja postać wzięła udział w pierwszej przygodzie.

1. Po prostu wiedziałeś, że musisz z nimi wyruszyć.

2. Przekonałeś jednego z BG, że Twoja intuicja jest bezcennna.

3. Wyczułeś, że stanie się coś złego, jeśli z nimi nie wyruszysz.

4. Jesteś pewien, że powód, dla którego z nimi wyruszyłeś wkrótce stanie się jasny.

\subsubsection{Kreatywny}\index{Deskryptory!Kreatywny}

Możesz mieć notatnik, w którym notujesz pomysły, które później rozwiniesz. Może wysyłasz samemu sobie maile w momencie, gdy nachodzi Cię inspiracja, by potem je zanotować w elektronicznym dokumencie. A może po prostu siadasz, patrzysz w ekran i niemożliwą siłą woli, tworzysz coś z niczego.

Niezależnie od tego, jak Twój dar działa, jestes kreatywny – kodujesz, piszesz, komponujesz, rzeźbisz, projektujesz, reżyserujesz lub w inny sposób tworzysz narracje, które zachwycająinnych ludzi.

Otrzymujesz następujące cechy:
\begin{itemize}
\item Kreatywny: +2 do Puli Intelektu.
\item Oryginalny: Zawsze robisz coś nowego. Jesteś wyszkolony w każdym zadaniu, związanym z tworzeniem narracji (takiej jak historia, przedstawienie teatralne lub scenariusz). Wlicza się w to oszustwo, jeśli oszustwo jest częścią narracji, którą tworzysz.
\item  Umiejętność: Jesteś naturalnym twórcą. Jesteś wyszkolony w jednej konkretnej umiejętności kreatywnej Twojego wyboru – pisaniu, tworzeniu oprogramowania, komponowaniu muzyki, malowaniu, rysowaniu itp.
\item Umiejętność: Kochasz rozwiązywać zagadki itp. Jesteś wyszkolony w rozwiązywaniu zagadek.
\item Umiejętność: Bycie kreatywnym wymaga ciągłego zdobywania wiedzy. Jesteś wyszkolony w każdym zadaniu związanym ze zdobywaniem nowych informacji, tak jak wtedy, gdy przekopujesz się przez bibliotekę, dane bankowe, archiwum newsów lub malą kolekcję źródeł wiedzy.
\item  Nieumiejętność: Jesteś kreatywny, ale nie urokliwy. Wszystkie zadania związane z przyjemną interakcją społeczną są dla Ciebie utrudnione.
 \end{itemize}
 
Początki Przygód: Z listy poniższych opcji, wybierz jak Twoja postać wzięła udział w pierwszej przygodzie.

1. Robiłeś badania związane z projektem i przekonałeś BG, by Cię z sobą zabrali.

2. Szukałeś nowych rynków zbytu na swój kreatywny output.

3. Wpadłeś w niewłaściwe towarzystwo, ale zaczęli Cię lubić.

4. Kreatywne życie zazwyczaj oznacza problemy finansowe. Dołączyłeś do BG, bo miałeś nadzieję na zarobienie pieniędzy.

\subsubsection{Mechaniczny}\index{Deskryptory!Mechaniczny}

Masz specjalny talent do maszyn wszelkiego rodzaju i jesteś dobry jeśli chodzi o zrozumienie i, jeśli zajdzie potrzeba, naprawę owych maszyn. Może jesteś nieco wynalazcą, tworzącym nowe maszyny od czasu do czasu. Jesteś nazywany rożnymi nickami, w tym “złotą rączką”. Mechanicy zazwyczaj noszą praktyczne stroje i noszą ze sobą dużo narzędzi.

Zyskujesz poniższe cechy:
\begin{itemize}
\item Bystry: +2 do Puli Intelektu.
\item Umiejętność: Jesteś wyszkolony we wszystkich akcjach związanych z identyfikacją i zrozumieniem maszyn.
\item Umiejętność: Jesteś wyszkolony we wszystkich zadaniach związanych z używaniem, naprawą lub tworzeniem maszyn.
\item  Dodatkowy Ekwipunek: Zaczynasz grę z rożnymi narzędziami do naprawy maszyn.
\end{itemize}    
    
Początki Przygód: Z listy poniższych opcji, wybierz jak Twoja postać wzięła udział w pierwszej przygodzie.

1. Kiedy naprawiałeś pobliską maszynę, podsłuchałeś rozmawiających BG.

2. Potrzebujesz pieniędzy na części i narzędzia.

3. Było oczywiste, że misja nie uda się bez Twoich umiejętności i wiedzy.

4. Inny BG poprosił Cię ,byś do nich dołączył.

\subsubsection{Milkliwy}\index{Deskryptory!Milkliwy}

Nigdy nie byłeś zbyt rozmowny. Kiedy jesteś zmuszony, by wejść w interakcję społeczną, nie masz pomysłu, jakich słów użyć – zawodzą one Ciebie lub wychodzą nie takie, jak trzeba. Zazwyczaj mówisz dokładnie złą rzecz i przez przypadek kogoś znieważasz. Przez większość czasu, po prostu jesteś cichy. Czyni to z Ciebie słuchacza – uważnego obserwatora. To oznacza także, że jesteś lepzy w robieniu rzeczy, niż w mówieniu. Jesteś szybki, by zacząć działać.

Zyskujesz poniższe cechy:
\begin{itemize}
\item Akcja, nie Słowa: +2 do puli Mocy i +2 do Puli Szybkości.
\item Umiejętność: Jesteś wyszkolony w percepcji.
\item  Umiejętność: Jesteś wyszkolony w inicjatywie (chyba, że starcie jest społeczne).
\item  Nieumiejętność: Wszystkie zadania związane ze społeczną interakcją są dla Ciebie utrudnione.
\item  Nieumiejętność: Wsyzstkie zadania powiązane z komunikacją werbalną lub przekazywaniem informacji są dla Ciebie utrudnione.
\end{itemize}

Początki Przygód: Z listy poniższych opcji, wybierz jak Twoja postać wzięła udział w pierwszej przygodzie.

1. Po prostu się dołączyłeś i nikt nie powiedział Ci, żebyś sobie poszedł.

2. Widziałeś coś ważnego, czego inni BG nie dostrzegli i (z pewnym wysiłkiem) powiedziałeś im o tym.

3. Zainterweniowałeś, by ocalić jednego z innych BG kiedy byli w niebezpieczeństwie.

4. Jeden z innych BG zrekrutował Cię ze względu na Twoje talenty.

\subsubsection{Mistyczny}\index{Deskryptory!Mistyczny}

Myślisz o sobie jako o mistycznym, dostrojonym do tego co paranormalne i tajemnicze. Twoje prawdziwe talenty leżą w nadnaturalnym.  Najpewniej masz doświadczenie w wiedzy tajemnej, i możesz wyczuć i dzierżyć nadnaturalne – możliwe, że jako “magię”, “psionikę” lub coś odmiennego, zależy to od Ciebie i tych wokół Ciebie. Mistyczne postaci często noszą biżuterię, taką jak pierścienie i amulety, lub mają tatuaże i inne oznaczenia swoich zainteresowań.

Zyskujesz poniższe cechy:
\begin{itemize}
\item  Bystry: +2 do Puli Intelektu.
\item  Umiejętność: Jesteś wyszkolony w akcjach powiązanych z identyfikowaniem i rozumieniem nadnaturalnego.
\item  Wyczucie Magii: Możesz wyczuć, czy zjawiska nadprzyrodzone są aktywne w sytuacjach, gdzie ich obecność nie jest oczywista. Musisz badać obiekt lub lokację czujnie przez minutę, by odczuć, czy jest tu magia. 
\item  Zaklęcie: Możesz dokonywać Sztuczki Magiczne jako zaklęcie gdy masz wolną dłoń i możesz zapłacić koszt w Intelekcie. 
 \item Nieumiejętność: Posiadasz pewną aurę, która nieco zbija z tropu innych. Każde zadanie wymagające uroku, perswazji lub oszustwa jest dla Ciebie utrudnione. 
\end{itemize}

Początki Przygód: Z listy poniższych opcji, wybierz jak Twoja postać wzięła udział w pierwszej przygodzie.

1. Miałeś proroczy sen.

2. Potrzebujesz pieniędzy, by sfinansować swoje studia.

3. Wierzyłeś, że ta misja byłaby świetnym sposobem, by dowiedzieć się więcej o zjawiskach nadprzyrodzonych.

4. Różne znaki i omeny przywiodły tu Ciebie.

\subsubsection{Miły}\index{Deskryptory!Miły}

To zawsze było dla Ciebie proste, by dostrzec rzeczy z perspektywy innych ludzi. Ta zdolność uczyniła z Ciebie sympatyczną osobę, która rozumie czego inni pragną lub chcą. Z Twojej perspektywy, po prostu stosujesz stare przysłowie “łatwiej jest złapać muchy na miód niż na ocet” ale inni patrzą na Twoje zachowanie jak na bycie miłym. Oczywiście, bycie miłym wymaga czasu, a Twój jest ograniczony. Nauczyłeś się, żepewna czesć ludzi nie zasługuje na Twoje miłosierdzie – prawdziwi sadyści, narcyzy i tym podbni tylko marnują Twoją energię. Tak więc załatwiasz ich szybko, oszczędzając swoje bycie miłym na tych, którzy tego zasługują i na tym skorzystają. 

Zyskujesz poniższe cechy:
\begin{itemize}
\item  Emocjonalnie Intuitywny: +2 do Puli Intelektu.
\item Umiejętność: Wiesz, co to znaczy przejść milę w czyichś butach. Jesteś wyszkolony we wszystkich zadaniach związanych z przyjemną interakcją społeczną i wyczuwaniem emocji innych ludzi.
\item Karma: Czasami, obcy po orstu Ci pomagają. By zyskać pomoc obcego człowieka, musisz zużyć jeden rzut na odzyskiwanie zdrowia – taki, który zajmuje akcję, 10 minut lub 10 godzin (bez regeneracji zdrowia) i MG określa naturę pomocy, którą otrzymujesz. Zazwyczaj, akt dobroci nie jest dostateczny, by kompletnie przezwyciężyć trudną sytuację, ale może w niej pomóc i prowadzić do nowych okazji. Dla przykładu, jeśli jesteś pojmany, strażnik może nieco poluzować Twoje więzy, przynieść Ci wodę lub przekazać wiadomość.
\item Nieumiejętność: Bycie miłym nosi z sobą pewne ryzyko. Wszystkie zadania związane z odczytywaniem kłamstw są utrudnione. 
\end{itemize}    
    
Początki Przygód: Z listy poniższych opcji, wybierz jak Twoja postać wzięła udział w pierwszej przygodzie.

1. BG potrzebował Twojej pomocy i zgodziłeś się mu towarzyszyć.

2. Dałeś złej osobie dostęp do swoich pieniędzy i teraz musisz je odzyskać.

3. Jesteś gotów zanieść swe miłosierdzie wzdłuż i w dal i pomóc większej ilości ludzi, w którym to celu dołączyłeś do BG.

4. Twoja praca, która wyglądała początkowo na świetną, nie jest taka. Dołączyłeś do BG, by od niej uciec.

\subsubsection{Naiwny}\index{Deskryptory!Naiwny}

Wychowywałeś się pod kloszem. Twoje dzieciństwo było bezpieczne i nie miałeś szansy, by nauczyć się czegoś o świecie – a nawet mniejszą szansę, by go doświadczyć. Mogłeś być szkolony w w jakimś kierunku, trzymać nos w książkach, lub po prostu byłeś w odosobnionym miejscu, tak więc nie zrobiłeś zbyt wiele, nie spotkałeś wielu ludzi i nie widziałeś zbyt wielu interesujących rzeczy. Pewnie to się szybko zmieni, ale jak wchodzisz w większy, obcy świat, robisz to bez pewnego zrozumienia, które posiadają inni ludzie wokół. 

Otrzymujesz poniższe cechy:
\begin{itemize}
\item Świeży: Dodajesz +1 do swoich rzutów na odzyskanie zdrowia.
\item Szlachetny Umysł: Jesteś wyszkolony w Obronie Intelektu i we wszystkich zadaniach powiązanych z odpieraniem się kuszeniu.
\item Umiejętność: Wszystko dla Ciebie jest nowe i ciekawe. Jesteś wyszkolony w percepcji.
\item Nieumiejętność: Każde zadanie polegające na rozpoznaniu fałszu lub odkryciu czyisć sekretnych motywów jest dla Ciebie utrudnione.
\end{itemize}   
    
Początki Przygód: Z listy poniższych opcji, wybierz jak Twoja postać wzięła udział w pierwszej przygodzie.

1. Ktoś powiedział Ci, że powinieneś się w to wmieszać.

2. Potrzebujesz pieniędzy, i to wyglądało na dobry sposób ich zarobku.

3. Wierzysz, że możesz się dużo nauczyć od jednego z BG.

4. To brzmi jak dobra zabawa.

\subsubsection{Niehonorowy}\index{Deskryptory!Niehonorowy}

Złodzieje nie mają honoru – i zdrajcy, wbijający noże w plecy, kłamcy i oszuści. Jesteś nimi wszystkimi i albo nie prześladuje Cię to po nocach, albo okłamujesz samego siebie i innych. Niezależnie, jesteś zdolny zrobić wszystko, by było tak, jak tego pragniesz. Honor, etyka i zasady to ledwie słowa. Według Ciebie, nie mają one miejsca w prawdziwym świecie.

Otrzymujesz poniższe cechy:
\begin{itemize}
\item  Ukradkowy: +4 do Puli Szybkości.
\item Po Obiedzie: Kiedy MG daje innemu graczowi punkt doświadczenia jako nagrodą za Wtrącenie MG, ten gracz nie może dać Tobie drugiego punktu.
\item  Umiejętność: Jesteś wyszkolony w oszustwie.
\item  Umiejętność: Jesteś wyszkolony w skradaniu się.
\item  Umiejętność: Jesteś wyszkolony w zastraszaniu.
\item  Nieumiejętność: Ludzie nie lubią Cię i Ci nie ufają. Przyjemna interakcje społeczne są dla Ciebie utrudnione. 
 \end{itemize}
    
Początki Przygód: Z listy poniższych opcji, wybierz jak Twoja postać wzięła udział w pierwszej przygodzie.

1. Jesteś zainteresowany tym, co robią BG, więc im nakłamałeś, aby Cię przyjęli do grupy.

2. Przypadkiem podsłuchałeś BG omawiających swoje plany i zdałeś sobie sprawę, że chcesz do nich dołączyć.

3. Jeden z innych BG zaprosił Cię, nie mając pojęcia jaki naprawdę jesteś.

4. Wprosiłeś się przy pomocy zastraszania i pogróżek.

\subsubsection{Niezręczny}\index{Deskryptory!Niezręczny}

Bez gracji ruchów i niezręczny, wszysci Ci mówili, że “kiedyś to minie”, ale nigdy tak się nie stało. Często upuszczasz rzeczy, potykasz sięo własne stopy, lub przewracasz rzeczy (i ludzi). Niektórych ludzi to irytuje, lecz dla większości jest to zabawne, a może nawet urocze.

(Pewnie gracze mogą nie chcieć być definiowani przez “negatywny” deskryptor jak Niezręczny, ale po prawdzie, nawet ten rodzaj deskryptora ma tyle zalet, że czyni postać zdolną i utalentowaną. Negatywne deskryptory czynią postać bardziej interesującą i wieloaspektową i często świetnie się nimi gra.)

Zyskujesz poniższe cechy:
\begin{itemize}
\item Maślane palce: -2 do Puli Szybkości.
\item  Umięśniony: +2 do Puli Mocy.
\item Uroczy: Masz pewien wrodzony urok. Jesteś wyszkolony we wszystkich przyjemnych interakcjach społecznych kiedy wykazujesz lekkie podejście na swojej niezręczności.
\item  Głupie Szczęście: MG może CI ofiarować Wtrącenie MG, bazujące na Twojej niezręczności, bez dawania Ci PD-ka (jakbyś wylosował 1 na k20). Jednakże, jeśli to się wydarzy, istnieje szansa 50\%, że Twoja niezręczność zadziała na Twoją korzyść. Zamiast zranić Ciebie (mocno), pomaga to Tobie lub rani Twoich wrogów. Poślizgujesz się, ale w odpowiednim momencie, by uniknąć ataku. Upadasz, ale przewracasz też swoich wrogów, podcinając im nogi. Odwracasz się zbyt szybko, ale w końcu wytrącasz wrogowi broń z rąk. Powinieneś pracować razem z MG by określić szczegóły. Jeśli MG sobie tego życzy, może on zastosować Wtrącenia MG związane z Twojąniezdarnością korzystając z normalnych zasad (przyznając PD-ki).
\item Umiejętność: Masz pewne cechy byka. Jesteś wyszkolony w zadaniach powiązanych z łamaniem i niszczeniem rzeczy.
\item Nieumiejętność: Wszelkie zadania związane z równowagą, gracją i koordynację ręka-oko są dla Ciebie utrudnione. 
\end{itemize}

Początki Przygód: Z listy poniższych opcji, wybierz jak Twoja postać wzięła udział w pierwszej przygodzie.

1. Byłeś we właściwym miejscu o właściwej porze.

2. Miałeś informację, której inni BG potrzebowali, by ruszyć do przodu.

3. Rodzeństwo zarekomendowało Cię innym BG.

4. Natknąłeś się na BG, gdy Ci dyskutowali o swojej misji, i polubili Cię.

\subsubsection{Odporny}\index{Deskryptory!Odporny}

Możesz wytrzymać wiele, zarówno fizycznie, jak i przychicznie, i dalej stać na nogach. Trzeba wiele, by Cię powstrzymać. Ani fizyczne, ani mentalne trudy i obrażenie nie mają na Ciebie długotrwałego wpływu. Jesteś wytrzymały. Niezłomny. Niepowstrzymany. 

Otrzymujesz poniższe cechy:
\begin{itemize}
    \item  Odporny: +2 do Puli Mocy, +2 do Puli Intelektu.
    \item  Odzyskiwanie zdrowia: Możesz wykonać dodatkowy rzut na odzyskanie zdrowia każdego dnia. Ten rzut to tylko jedna akcja. Tak więc możesz wykonać dwa rzuty, które trwają 1 akcję, 1 1-minutowy, i czwarty, który zajmuje 1 godzinę, w końcu zaś piąty, który zajmuje 10 godzin odpoczynku.
    \item  Umiejętność: Jesteś wyszkolony w Obronie Mocy.
    \item  Umiejętność: Jesteś wyszkolony w Obronie Intelektu.
    \item  Nieumiejętność: Jesteś wytrzymały, ale niekoniecznie silny. Wszelkie akcje związane z przemieszczeniem, naginaniem lub niszczeniem rzeczy są dla Ciebie utrudnione.
    \item  Nieumiejętność: Masz mnóstwo siły woli, ale niekoniecznie jesteś mądry. Wszelkie akcje powiązane z wiedzą i rozwiązywaniem problemów są dla Ciebie utrudnione.
\end{itemize}

Początki Przygód: Z listy poniższych opcji, wybierz jak Twoja postać wzięła udział w pierwszej przygodzie.

1. Dostrzegłeś, że BG z pewnością potrzebują kogoś takiego jak Ty.

2. Ktoś poprosił Cię, byś miał na oku jednego z BG w szczególności, i przychyliłeś się do tej prośby.

3. Jesteś znużony i w desperackiej potrzebie jakiegoś wyzwania.

4. Przegrałeś zakład – nie fair, wierzysz – i musiałeś zająć czyjeś miejsce w tej misji.

\subsubsection{Odporny Psychicznie}\index{Deskryptory!Odporny Psychicznie}

Jesteś osobą o silnej woli i bardzo niezależną. Nikt nie może Ci niczego wmówić lub zmienić Twojego zdania kiedy nie chcesz sam go zmienić. Ta cechan iekoniecznie czyni Cięmądrym, ale na pewno jesteś bastionem siły woli. Możliwe, że sięubierasz i działasz z unikalnym stylem i mocą, nie dbając o to, co myślą inni. 

Otrzymujesz poniższe cechy:
\begin{itemize}
    \item  Wytrzymały psychicznie: +4 do Puli Intelektu.
    \item  Umiejętność: Jesteś wyszkolony w opieraniu się efektom mentalnym.
    \item  Umiejętność: Jesteś wyszkolony w zadaniach wymagających wielkiego skupienia i koncentracji.
    \item  Nieumiejętność: Silna wola nie oznacza bycia bystrym. Wszelkie zadania powiązane z zagadkami lub problemami, zapamiętywaniem rzeczy lub korzystania z wiedzy są dla Ciebie utrudnione. 
\end{itemize}

Początki Przygód: Z listy poniższych opcji, wybierz jak Twoja postać wzięła udział w pierwszej przygodzie.

1. Pomimo Twojego osądu, dołączyłeś do BG ponieważ byli w niebezpieczeńtwie.

2. Jeden z BG przekonał Cię, że dołączenie do grupy byłoby w Twoim najlepszym interesie.

3. Boisz się, co się stanie, jeśli reszta BG zawiedzie.

4. W tle jest nagroda, a Ty potrzebujesz pieniędzy.

\subsubsection{Okrutny}\index{Deskryptory!Okrutny}

Nieszczęścia i cierpienie Cię nie obchodzą. Kiedy ktoś inny przechodzi w swoim życiu trudy, ciężko Ci się przejąć, co więcej, jego cierpienie i trudności mogą Cię ucieszyć, jeśli w przeszłości zaszedł Ci on za skórę. Twoje okrucieństwo może wynikać z rozżalenie światem i własnymi problemami. Możesz być twardym pragmatystą, robiąc to, co jak czujesz musisz zrobić ,nawet jeśli inni przez to cierpią. Lub możesz być sadystą, rozkoszującym się cierpieniem, które zadajesz. 

Bycie okrutnym niekoniecznie czyni z Ciebie złoczyńcę. Twoje okrucieństwo może być zarezerwowane tych, którzy staną na Twojej drodze lub innych ludzi, gdzie będzie to miało swoje użycie. Mogłeś stać się okrutny po bardzo smutnym wydarzeniu życiowym. Przemoc i tortury, na ten przykład, mogą odebrać litość dla innych istot żywych.

Podobnie, nie musisz być okrutny w każdej sytuacji. Po prawdzie, możesz być lubiany, przyjazny i nawet pomocny. Ale kiedy jesteś wściekły lub sfrustrowany, Twoja podwójna natura daje o sobie znać i Ci, którzy cię wkurzyli, z pewnością tego pożałują.

Otrzymujesz poniższe cechy:
\begin{itemize}
    \item  Bystry: +2 do Puli Intelektu.
    \item  Okrucieństwo: Kiedy korzystasz ze swojej siły, możesz wybrać okaleczanie swojego wroga lub zadawanie mu bolesnych obrażeń, by przedłużyć jego cierpienie. Kiedykolwiek zadajesz obrażenia, możesz wybrać zadanie 2 punktów obrażeń mniej, by ułatwić Twój następny atak przeciwko temu wrogowi.
    \item  Umiejętność: Jesteś wyszkolony w zadaniach powiązanych z oszukiwaniem, zastraszaniem i perswazją, gdy wchodzisz w interakcję z postaciami cierpiącymi fizycznie lub emocjonalnie. 
    \item  Nieumiejętność: Masz problemy w wyciąganiu ręki do innych ludzi, rozumieniu ich pobudek lub dzieleniu się swoimi uczuciami. Wszystkie zadania powiązane z zrozumieniem motywów, uczuć i stanu umysłu są dla Ciebie utrudnione.
    \item  Dodatkowy Ekwipunek: Masz cenną pamiątkę o ostatniej osobie, którą zniszczyłeś. Ta pamiątka jest średnio wielkiej ceny i możesz jąsprzedać lub wymienić na przedmiot o takiej samej lub niższej wartości.
\end{itemize}

Początki Przygód: Z listy poniższych opcji, wybierz jak Twoja postać wzięła udział w pierwszej przygodzie.

1. Uważasz, że możesz wiele zyskać w dłuższym czasie, jeśli pomożesz BG i możesz wykorzystać ową przewagę przeciwko Twoim wrogom.

2. Poprzez dołączanie do BG, widzisz szansę na zwiększenie swojej własnej potęgi i statusu kosztem innych.

3. Pragniesz uczynić życie innego BG ciężkim, poprzez dołączenie do grupy.

4. Dołączenie do BG daje Ci szansę na ucieczkę od sprawiedliwości za przestępstwo, które popełniłeś. 

\subsubsection{Pewny Siebie}\index{Deskryptory!Pewny Siebie}

Jesteś pewien własnych zdolności, energetyczny, i najpewniej nieco prześmiewczy odnośnie pomysłów, z którymi się nie zgadzasz.  Niektóry nazywają Cię odważnym, ale Ci, którym pokazałeś ich miejsce, zwą Cię aroganckim. Ojtam. To nie w Twojej naturze przejmować się co inni myślą o Tobie – chyba, że to rodzina lub przyjaciele. Nawet ktoś tak pewny siebie jak tywie, że przyjaciele czasami muszą być na pierwszym miejscu.

Otrzymujesz poniższe cechy:
\begin{itemize}
    \item Energetyczny: +2 do Puli Szybkości.
    \item Umiejętności: Jesteś wyszkolony w inicjatywie. 
    \item Odważny:Jesteś wyszkolony we wszystkich akcjach które polegają na przezwyciężaniu strachu i zastraszania. 
\end{itemize}

Początki Przygód: Z listy poniższych opcji, wybierz jak Twoja postać wzięła udział w pierwszej przygodzie.

1. Dostrzegłeś, ze dzieje się coś dziwnego, i bez większego namysłu, wkroczyłeś do akcji.

2. Pokazałeś się tam i wtedy, ponieważ ktoś stwierdził, że tego nie zrobisz – hej, musiałeś pokazać, że nie miał racji!

3. Ktoś Cię wyzwał, ale zamiast zacząć walkę, zaczęło to Twoją obecną przygodę.

4. Powiedziałeś przyjacielowi, że nic niem oże Cię przerazić, i że nic, co zobaczysz, nie zmieni Twojego zdania. To zaprowadziło Cię do wydarzeń obecnych. 

\subsubsection{Pełen Wdzięku}\index{Deskryptory!Pełen Wdzięku}

Masz perfekcyjny zmysł równowagi, poruszasz się i mówisz z gracją i pięknem. Jesteś szybki i zwinny. Twoja cieło nadaje się perfekcyjnie do tańca, i używasz tej przewagi podczas walki do unikania ciosów. Możesz nosisz ubrania, które umożliwiają Twoje zwinne ruchy i stanowią o Twoim wyczuciu stylu.

Otrzymujesz poniższe cechy:
\begin{itemize}
    \item Zwinny: +2 do Puli Szybkości.
    \item Umiejętność: Jesteś wyszkolony we wszystkich zadaniach powiązanych z równowagą i ostrożnymi ruchami.
    \item Umiejętność: Jesteś wyszkolony we wszystkich akcjach powiązanych z fizycznymi sztukami performatywnymi.
    \item Umiejętność: Jesteś wyszkolony w Obronie Szybkości.
\end{itemize}

Początki Przygód: Z listy poniższych opcji, wybierz jak Twoja postać wzięła udział w pierwszej przygodzie.

1. Pomimo Twojego lepszego osądu, dołączyłeś do innych BG, bo dostrzegłeś, że są w niebezpieczeństwie.  

2. Inny z BG przekonał Cię, że dołączenie do grupy będzie w Twoim najlepszym interesie.

3. Boisz się, co się stanie, jeśli reszta BG poniesie klęskę.

4. Można wygrać nagrodę, a Ty potrzebujesz pieniędzy.

\subsubsection{Piękny}\index{Deskryptory!Piękny}

Jesteś atrakcyjny dla innych, ale co być może ważniejsze, jesteś lubiany i charyzmatyczny. Masz to “specjalne coś” co przyciąga do Ciebie innych. Wiesz zazwyczaj, co powiedzieć, by kogoś rozbawić, uspokoić lub zmusić do akcji. Ludzie lubią Cię, pragną Ci pomóc i chcą być Twoimi przyjaciółmi. 

Otrzymujesz poniższe cechy:
\begin{itemize}
    \item Charyzmatyczny: +2 do Puli Intelektu.
    \item Umiejętność: Jesteś wyszkolony w przyjemnych interakcjach społecznych.
    \item Odporny na Wdzięki: Jesteś siadom, jak inni mogą manipulować i czarować ludzi i dostrzegasz, gdy ta taktyka jest używana na Tobie. Z powodu tej świadomości, jesteś wyszkolony w odpieraniu perswazji i uwodzenia.
\end{itemize}

Początki Przygód: Z listy poniższych opcji, wybierz jak Twoja postać wzięła udział w pierwszej przygodzie.

1. Spotkałeś kompletnego obcego (jednego z BG) i tak go oczarowałeś, że zaprosił Cię do grupy.

2. BG szukali kogoś innego, ale przekonałeś ich, że jesteś perfekcyjnym kandydatem.

3. Czysty przypadek – ponieważ ruszasz z prądem i wszystko zazwyczaj działa, jak powinno.

4. Twoja charyzma pomogła jednego BG wyjść z trudnej sytuacji dawno temu i zawsze teraz prosi Cię on o wzięcie udziału w nowych przygodach.

\subsubsection{Pomocny}\index{Deskryptory!Pomocny}

Pomaganie innym to Twoje powołanie. To dlatego tutaj jesteś. Inni doceniają twoją przyjazną i pomocną naturę, a Ty rozkoszujesz się ich szczęściem. Kochasz pomagać ludziom, czy to poprzez wyjaśnianie im, jak mogą najlepiej poradzić sobie z problemem, czy też demonstrując im to.

Zyskujesz poniższe cechy:
\begin{itemize}
    \item Życzliwy: Wszyscy, którzy spędzili z Tobą cały poprzedni dzień, zyskują +1 do swoich rzutów na odzyskanie zdrowia.
    \item Altruistyczny: Jeśli stoisz obok istoty, która otrzymuje obrażenia, możesz przejąć 1 punkt tych obrażeń na samego siebie (redukując obrażenia istoty o 1). Jeśli masz Pancerz, nie korzystasz z niego przy korzystaniu z tej zdolności.
    \item Umiejętność: Jesteś wyszkolony we wszystkich zadaniach związanych z przyjemną interakcją społeczną, pomaganiu się innym zrelaksować i pozyskiwaniu zaufania.
    \item Pomocny: Zawsze, gdy pomagasz innej postaci, ta postać zyskuje bonus, jakbyś był wyszkolony, nawet, jeśli ni jesteś wyszkolony lub wyspecjalizowany w danym zadaniu.
    \item Nieumiejętność: Gdy jesteś sam, wszystkie zadania Intelektu i Szybkości są dla Ciebie utrudnione.
\end{itemize}

Początki Przygód: Z listy poniższych opcji, wybierz jak Twoja postać wzięła udział w pierwszej przygodzie.

1. Pomimo tego, że nie znałeś większości BG przedtem, wprosiłeś się w ich drużynę.

2. Widziałeś, że BG mają kłopot i ruszyłeś im na pomoc.

3. Jesteś pewien, że BG zaliczą klęskę bez Ciebie.

4. Twój wybór to było  jednej strony.

\subsubsection{Ryzykujący}\index{Deskryptory!Ryzykujący}

Jest częścią Twojej natury kwestionowanie co inni sądzą w temacie tego, co nie powinno być robione. Nie jesteś szalony, oczywiście – nie chciałbyś przeskoczyć przez przepaść o szerokości mili ylko dlatego, że ktoś Cię wyzwał. Istnieją rzeczy niemożliwe i prawie niemożliwe. Lubisz próbować swoich sił w tych drugich, ponieważ daje Ci to satysfakcję i przyjemność, gdy odnosisz sukces. Im większy Twój sukces, tym bardziej szukasz kolejnego ryzykownego wyzwania.

Otrzymujesz poniższe cechy:
\begin{itemize}
    \item Zwinny: +4 do Puli Szybkości.
    \item Umiejętność: Jesteś dobry w ocenaniu ryzyka, i wyszkolony w zadaniach, które mają pewien element losowości, jak granie w gry hazardowe lub wybieranie pomiędzy dwoma lub trzema podobnymi opcjami.
    \item Szczęście: Możesz wybrać automatyczny sukces na jednym teście bez rzucania kością, tak długo, jak zadanie ma trudność nie większą niż 6. Kiedy to jednak robisz, MG dokonuje Wtrącenia, jakbyś wylosował 1. Wtrącenie nie neguje sukcesu, ale może na niego wpłynąć. Możesz to zrobić jeden raz, lecz zdolność odnawia się po każdym 10-godzinnym rzucie na odzyskanie zdrowia.
    \item Nieumiejętność: Jesteś zwinny, ale nie ukradkowy. Zadania powiązane z zakradaniem się i byciem cicho są dla Ciebie utrudnione.
\end{itemize}

Początki Przygód: Z listy poniższych opcji, wybierz jak Twoja postać wzięła udział w pierwszej przygodzie.

1. Wyglądało na to, że była szansa 50/50 na to, że BG odniosą sukces, co dla Ciebie było odpowiednio dużą.

2. Myślisz, że zadanie będzie miało wyjątkowo i satysfakcjonujące wyzwania.

3. Jedno z Twoich ryzyk się nie udało, i potrzebujesz pieniędzy, by je spłacić.

4. Chwaliłeś się, że nigdy nie widziałeś ryzyka, którego byś nie lubił, co doprowadziło Cię do sytuacji bieżącej. 

\subsubsection{Sceptyczny}\index{Deskryptory!Sceptyczny}

Jesteś bardzo sceptyczny odnośnie twierdzeń, które inni biorą za ozywisa prawdę. Nie jesteś koniecznie “wątpiącym Tomaszem” (sceptykiem, który wątpi we wszystko bez stuprocentowych dowodów), ale często odnosiłeś korzyści, kwestionując twierdzenia, opinie i wiedzę innych ludzi.

Otrzymujesz poniższe cechy:
\begin{itemize}
    \item Myśliciel: +2 do Puli Intelektu.
    \item  Umiejętność: Jesteś wyszkolony w identyfikowaniu.
    \item  Umiejętność: Jesteś wyszkolony w każdej akcji, która polega na przejrzeniu trików, iluzji bądź sztuczek retorycznych które unikają prawdy lub jawnego kłamstwa. Dla przykładu, jesteś lepszy w stwierdzeniu, który kubek zawiera ukrytą kulkę, wyczuwaniu iluzji, lub zorientowaniu się, że ktoś Cię okłamuje (ale tyko, jeśli się skoncentrujesz i użyjesz tej zdolności).
\end{itemize}    

Początki Przygód: Z listy poniższych opcji, wybierz jak Twoja postać wzięła udział w pierwszej przygodzie.

1. Podsłuchałeś innych BG rozmawiających o czymś, odnośnie czego byłeś sceptyczny, więc podszedłeś do nich i poprosiłeś o dowód.

2. Śledziłeś jednego BG, którego o coś podejrzewałeś, co wpłątało Cię w wydarzenia.

3. Twoja teoria o nieistnieniu nadnaturalnego może być udowodniona jako fałszywa tylko przez Twoje własne zmysły, więc wyruszyłeś w podróż z BG.

4. Potrzebujesz pieniędzy, by sfinansować własne badania.

\subsubsection{Silny}\index{Deskryptory!Silny}
Jesteś wyjątkowo silny i potężny fizycznie i korzystasz właściwie z tych cech, czy to przy pomocy przemocy, czy podnoszenia ciężarów. Najpewniej jesteś silnie zbudowany i masz okazałą muskulaturę.

Otrzymujesz poniższe cechy
\begin{itemize}
    \item  Bardzo Silny: +4 do Puli Mocy.
    \item  Umiejętność: Jesteś wyszkolony we wszystkich zadaniach niszczenia nieruchomych obiektów.
    \item  Umiejętność: Jesteś wyszkolony w skakaniu.
    \item Dodatkowy Ekwipunek: Masz dodatkową średnią lub ciężką broń.
\end{itemize}

Początki Przygód: Z listy poniższych opcji, wybierz jak Twoja postać wzięła udział w pierwszej przygodzie.

1. Pomimo Twojego lepszego osądu, dołączyłeś do reszty BG ponieważ byli w niebezpieczeństwie.

2. Inny z BG przekonał Cię, że dołączenie byłoby w Twoim najlepszym interesie.

3. Boisz się, co się wydarzy, jeśli BG poniosą klęskę.

4. Jest oferowana nagroda, a Ty potrzebujesz pieniędzy.

\subsubsection{Skryty}\index{Deskryptory!Skryty}

Ukrywasz swoją prawdziwą naturę za maską i nie chcesz, by ktokolwiek poznał, kim naprawdę jesteś. Chronienie siebie, fizycznie i emocjonalnie, jest tym, na czym Ci zależy, i trzymasz wszystkich na bezpieczny dystans. Możesz być podejrzliwy względem każdego, kogo spotykasz, spodziewać się najgorszego od ludzi, tak więc nie jesteś zaskoczony, gdy okazuje się, że masz rację. Lub po prostu możesz być nieco zdystansowany, ostrożny odnośnie pozwolenia ludziom na przejrzenie kim naprawdę jesteś.

Nikt nie może być tak zdystansowany jak Ty i mieć wielu przyjaciół. Najpewniej posiadasz szorstką osobowość i masz pesymistyczne spojrzenie na życie. Możliwe, że rozdrapujesz rany i odkryłeś, że jedyny sposób, w jaki możesz iść do przodu, to zamknąć się w sobie.

Otrzymujesz poniższe cechy:
\begin{itemize}
    \item  Podejrzliwy: +2 do Puli Intelektu.
    \item  Umiejętność: Jesteś wyszkolony w Obronie Intelektu.
    \item  Umiejętność: Jesteś wyszkolony w zadaniach polegających na ustalaniu prawdy, dostrzeganiu przebrań i rozpoznawaniu kłamstw i innych oszustw.
    \item  Nieumiejętność: Twoja podejrzliwa natura czyni Cię nielubianym. Wszystkie zadania związana z oszustwem lub perswazją są dla Ciebie utrudnione.
\end{itemize}

Początki Przygód: Z listy poniższych opcji, wybierz jak Twoja postać wzięła udział w pierwszej przygodzie.

1. Jeden z BG zdołał się przebić przez Twoją skrytość i zostać Twoim przyjacielem.

2. Chcesz odkryć, co robią BG, więc im towarzyszysz, chcąc ich złapać w akcie.

3. Masz paru wrogów i dołączyłeś do BG w celach obronnych.

4. BG to jedyni ludzie, którzy mogą z Tobą wytrzymać.

\subsubsection{Skazany na Zagładę}\index{Deskryptory!Skazany na Zagładę}

Dobrze wiesz jaki los został Ci przepowiedziany – nieuchronny, okropny koniec. To przeznaczenie może być tylko Twoje, lu może też dotyczyć Twoich najbliższych.

Otrzymujesz poniższe cechy:
\begin{itemize}
    \item  Nerwowy: +2 do Puli Szybkości.
    \item  Umiejętność: Zawsze rozglądasz sieza niebezpieczeństwem, jesteświęc wyszkolony w percepcji.
    \item  Umiejętność: Koncentrujesz się na obronie, jesteś więc wyszkolony w Obronie Szybkości.
    \item  Umiejętność: Jesteś cyniczny i zawsze spodziewasz się najgorszego. Tak więc, jesteś odporny na szoki mentalne. Jesteś wyszkolony w Obronie Intelektu, która polega na unikaniu szaleństwa i osiąganiu spokoju umysłu.
    \item  Zagłada: Co drugi raz, gdy MG używa Wtrącenia MG na Twojej postaci, nie możesz odmówić i nie uzyskujesz z tego tytułu PD (ale dalej dajesz PD innemu graczowi). Dzieje się tak, gdyż jesteś skazany na zagładę. Kosmos to zimne, nieczułe miejsce, a Twoje wysiłki są daremne.
\end{itemize}

Początki Przygód: Z listy poniższych opcji, wybierz jak Twoja postać wzięła udział w pierwszej przygodzie.

1. Chciałeś uniknąć tego losu, ale wydarzenia zdawały się spiskować przeciwko Tobie.

2. Czemu nie? To bez znaczenia. I tak jesteś skazany na zagładę, niezależnie od wszystkiego.

3. Jeden z BG ocalił Twoje życie, a teraz spłacasz dług, pomagając mu w różnych zadaniach.

4. Podejrzewasz, że jedyna nadzieja jaką masz, by uniknąć swojemu losowi, znajduje się na obranym przez Ciebie szlaku.

\subsubsection{Spokojny}\index{Deskryptory!Spokojny}

Poświęciłeś całe swój czas na życie w spokoju – książki, filmy, hobby itp. - zamiast poświęcić je na jakieś aktywności. Dobrze znasz sięna wielu problemach naukowych lub intelektualnych, ale nie na czymś fizycznej natury. Nie jesteś słaby czy coś, raczej (choć to dobry deskryptor dla postaci, które są starszymi osobami) ale nie masz doświadczenia w fizycznych aktywnościach.

(Spokojny to świetny deskryptor dla postaci, które nigdy nie chciały ruszyć na przygody, lecz życie ich do tego zmusiło – wątek powszechny we współczesnych grach, zwłaszcza w horrorach.)

Otrzymujesz poniższe cechy:
\begin{itemize}
    \item  Z nosem w książkach: +2 do puli Intelektu.
    \item  Umiejętności: Jesteś wyszkolony w 4 niefizycznych umiejętnościach Twojego wyboru.
    \item  Trivia:Kiedy sobie tego życzysz, możesz podać z pamięci fakty przydatne w danej sytuacji. Zawsze są to fakty, an ie przesądy lub wierzenia, i musi to być coś, o czym mógłbyś przeczytać lub widzieć w przeszłości. Możesz zrobić tak jedne raz, ale zdolność odświeża się po wykonaniu rzutu na odzyskanie zdrowia.
    \item Nieumiejętność: Nie jesteś wojownikiem. Wszystkie Twoje fizyczne akcje są utrudnione. 
    \item Nieumiejętność: Nie lubisz wychodzić na zewnątrz. Wszystkie zadania związane ze wspinaczką, bieganiem, skakaniem i pływaniem są dla Ciebie utrudnione.
\end{itemize}

Początki Przygód: Z listy poniższych opcji, wybierz jak Twoja postać wzięła udział w pierwszej przygodzie.

1. Czytałeś o obecnej sytuacji i zadecydowałeś, że chcesz są sprawdzić osobiście.

2. Byłeś w złym (dobrym?) miejscu o złej (dobrej?) porze.

3. Kiedy unikałeś kompletnie odmiennej sytuacji, napotkałeś na obecny problem.

4. Jeden z innych BG wmieszał Cię w tą sytuację.

\subsubsection{Szalony}\index{Deskryptory!Szalony}

Zagłębiłeś się zbyt głęboko w rzeczy, o których ludzkość nie powinna wiedzieć. Jesteś dobrze wykształcony w ezoterycznych tematach dostępnych niewielu, ale ta wiedza miała przerażającą cenę. Zapewne jesteś w złym stanie fizycznym i czasami masz tiki nerwowe. Czasami mówisz sam do siebie bez zdania sobie z tego sprawy.

Otrzymujesz poniższe cechy:
\begin{itemize}
    \item Ten, Który Wie: +4 do Puli Intelektu.
    \item Przebłysk Olśnienia: Kiedykolwiek posiadanie takiej wiedzy jest stosowne, GM może dać ci ją, choć nie istnieje jasne wyjaśnienie skąd masz ową wiedzę. Leży to w kwestii MG, ale powinno się to zdarzać przynajmniej raz na sesję.
    \item  Dziwne Zachowanie: Masz tendencję do dziwnych, irracjonalnych zachowań. Kiedy jesteś w przededniu wielkiego odkrycia lub poddany wielkiemu stresowi (no: podczas zagrożenia fizycznego) MG może wprowadzić Wtrącenie MG bez nagradzania PD, które pokieruje Twoja następną akcją. Możesz dalej wydać 1 PD, by odmówić. Wpływ MG to manifestacja Twojego szaleństwa i zawsze jest czymś, czego normalnie byś nie zrobił,, ale nie jest ono bezpośrednio, oczywiście szkodliwe, chyba, że zajdą specjalne okoliczności (Dla przykładu, jeśli wróg nagle wyskakuje z ciemności, możesz spędzić pierwszą turę mrucząc coś pod nosem albo krzycząc imię swojej pierwszej prawdziwej miłości). 
    \item  Umiejętność: Jesteś wyszkolony w jednej dziedzinie wiedzy (najpewniej czymś dziwnym lub ezoterycznym).
    \item  Nieumiejętność: Twó umysł jest bardzo wrażliwy. Zadania związane z odpieraniem ataków mentalnych są dla Ciebie utrudnione.
\end{itemize}

Początki Przygód: Z listy poniższych opcji, wybierz jak Twoja postać wzięła udział w pierwszej przygodzie.

1. Głosy w Twojej głowie kazały ci to uczynić.

2. Rozpocząłeś całą przygodę i nakłoniłeś innych, by do Ciebie dołączyli.

3. Jeden z innych BG zyskał księgę wiedzy dla Ciebie, a teraz spłacasz dług.

4. Czułeś się kuszony przez niejasną intuicje.

\subsubsection{Szczęśliwy}\index{Deskryptory!Szczęśliwy}

Polegasz na szansach i szczęściu w wielu sytuacjach. Kiedy ludzie mówią, że ktoś był urodzony pod szczęśliwą gwiazdą, mają na myśli Ciebie. Kiedy próbujesz czegoś nowego, niezależnie od tego, jak bardzo nie wiesz, co robisz, częściej niż nie odnosisz sukces. Nawet gdy nadciąga katastrofa, rzadko jest ona tak zła, jak mogłaby być. Bardzo często, drobnostki idą po Twojej myśli, wygrywasz konkursy i znajdujesz się we właściwym miejscu o właściwej porze.

Otrzymujesz poniższe cechy:
\begin{itemize}
    \item  Pula Szczęścia: Masz dodatkową pule zwaną Szcześciem, która zaczyna się od 3 punktów i ma maksymalna wartość 3 punktów. Kiedy wydajesz punkty z dowolnej innej Puli, możesz wziąć jeden, trochę lub wszystkie punkty ze swojej Puli Szczęścia. Kiedy wykonujesz rzut na odzyskanie zdrowia, Twoja Pula Szczęście regeneruje się o tyle samo punktów. Kiedy Twoja Pula Szczęścia zawiera 0 punktów, nie wlicza się to do obniżenia licznika obrażeń. 
    \item  Przewaga: Gdy wydajesz 1 PD, by przerzucić k20 na każdym rzucie, który dotyczy tylko Ciebie, dodajesz +3 do wartości ponownie rzuconej kości. 
\end{itemize}

Początki Przygód: Z listy poniższych opcji, wybierz jak Twoja postać wzięła udział w pierwszej przygodzie.

1. Wiedząc, że szczęściarze dostrzegają i wykorzystują sposobność, wziąłeś udział w pierwszej przygodzie z wyboru. 

2. Dosłownie wpadłeś na kogoś na tej przygodzie poprzez czyste szczęście.

3. Odkryłeś teczkę leżącą na drodze. Była znoszona, ale w  środku znalazłeś dużo dziwnych dokumentów, które tu Cię 
przywiodły.

4. Twoja szczęście Cię ocaliło, gdy uniknąłeś pędzącego pojazdu poprzez wpadnięcie w dziurę w ziemi. Poniżej poziomu gruntu, odnalazłeś coś, czego nie mogłeś zignorować. 

\subsubsection{Szlachetny}\index{Deskryptory!Szlachetny
}
Robienie tego, co słuszne, to sposób życia. Żyjesz zgodnie z kodeksem, a ten kodeks to ,coś, co dyktuje Twoje codzienne życie. Kiedy zbaczasz z drogi, bijesz się w pierś za swoją słabość i zaczynasz od nowa. W Twój kodeks najpewniej wlicza się umiarkowanie, szacunek dla innych, czystość i inne charakterystyki, które większość ludzi uzna za cnoty. Odrzucasz także ich przeciwieństwa: lenistwo, rządzę pieniądza, obżarstwo itp.

Otrzymujesz poniższe cechy:
\begin{itemize}
    \item  Silny: Otrzymujesz +2 do Puli Mocy.
    \item  Umiejętność: Jesteś wyszkolony w ocenaniu prawdziwych motywów innych ludzi i dostrzeganiu kłamstw.
    \item  Umiejętność: Twoje podążanie ścisłym kodeksem moralnym uodporniło Twój umysł na strach, wątpliwości i zewnętrzne wpływy. Jesteś wyszkolony w Obronie Intelektu.
\end{itemize}

Początki Przygód: Z listy poniższych opcji, wybierz jak Twoja postać wzięła udział w pierwszej przygodzie.

1. BG robią coś szlachetnego, i wspierasz ich całym sercem.

2. BG są na drodze do zatracenia i uważasz, że musisz ich sprowadzić na właściwą drogę.

3. Inny z BG zaprosił Ciędo przygody, słysząc o Twej szlachetnej naturze.

4. Postawiłeś cnotę przed zdrowym rozsądkiem i broniłeś czyjegoś honoru w obliczu organizacji lub mocy znacznie większej od Ciebie samego. Dołączyłeś do BG, ponieważ zaoferowali Ci pomoc i przyjaźń wtedy, gdy z obawy przed konsekwencjami, nikt inny tego nie zrobił.

\subsubsection{Szpetny}\index{Deskryptory!Szpetny}

Jesteś ohydny zgodnie z prawie każdym ludzkim standardem. Może miałęś tragiczny wypadek, szkodliwą mutację, lub złe szczęście do genów, ale jesteś niemożliwie wprost szpetny.

Nadrabiasz wygląd innymi cechami. Ponieważ musisz ukrywać swój wygląd, jesteś dobry w skradaniu się i przebieraniu. Ale co być może najważniejsze, bycia ofiarą ostracyzmu, gdy inni się socjalizowali, dało Ci czas na dorośnięcie  - wyrosłeś na silnego lub szybkiego – a może masz bystry umysł? 

Otrzymujesz poniższe cechy:
\begin{itemize}
    \item  Uniwersalny: Masz dodatkowe 4 punkty do rozdzielenia między swoje Statystyki.
    \item  Umiejętność: Jesteś wyszkolony w zastraszaniu i innych akcjach bazujących na lęku, jeśli pokażesz swoją prawdziwą twarz.
    \item  Umiejętność: Jesteś wyszkolony w skradaniu się i przebieraniu się.
    \item  Nieumiejętność: Wszystkie zadania związane z przyjemną społeczną interakcją śa dla Ciebie utrudnione.
\end{itemize}    

Początki Przygód: Z listy poniższych opcji, wybierz jak Twoja postać wzięła udział w pierwszej przygodzie.

1. Jeden z BG podszedł do Ciebie kiedy byłeś w przebraniu i Cię zrekrutował, wierząc, że jesteś kimś innym.

2. Kiedy się przekradałeś, usłyszałeś jak BG planowali swoją przygodę i zdałeś sobie sprawę, że chcesz do niej dołączyć.

3. Jeden z BG zaprosił Cię, ale zastanawiasz się ,czy nie zrobił tego z litości.

4. Pogróżkami zdobyłeś sobie miejsce w drużynie.

\subsubsection{Szybki}\index{Deskryptory!Szybki}

Jesteś naprawdę szybki. Z tego powodu, możesz wykonać niektóre zadania szybciej od innych. Nie tylko masz szybkie stopy, ale i dłonie, myślisz także i reagujesz szybko. Nawet mówisz z pośpiechem. 

Otrzymujesz poniższe cechy:
\begin{itemize}
    \item  Energetyczny: +2 do Puli Szybkości.
    \item  Umiejętność: Jesteś wyszkolony w bieganiu.
    \item  Szybki: Możesz się poruszyć na średni dystans i dalej dokonać akcji w tej samej rundzie, lub możesz poruszyć się na daleki dystans bez konieczności wykonania rzutu.
    \item  Nieumiejętność: Jesteś sprinterem, ale nie biegaczem długodystansowym. Nie masz wielkiej kondycji. Obrona Mocy jest dla Ciebie utrudniona.
\end{itemize}

Początki Przygód: Z listy poniższych opcji, wybierz jak Twoja postać wzięła udział w pierwszej przygodzie.

1. Ruszyłeś by ocalić jednego z BG, który był w wielkiej potrzebie.

2. Jeden z innych BG zrekrutował Cię ze względu na Twoje wyjątkowe talenty.

3. Jesteś impulsywny, i w tamtym momencie wyglądało to na dobry pomysł.

4. Ta misja wiąże się z Twoim osobistym celem.

\subsubsection{Tajemniczy}\index{Deskryptory!Tajemniczy}

Mroczna figura czająca się w kącie? To Ty. Nikt naprawdę nie wie skąd przybyłeś lub jakie są Twoje motywy – chowasz to w sekrecie. Twoja maniera konfunduje innych, ale nie czyni to z Ciebie złego przyjaciela lub sprzymierzeńca. Jesteś po prostu dobry w zachowywaniu pewnych wiadomości dla siebie, poruszaniu się tak, by nikt Cię nie dostrzegł i ukrywaniu swojej obecności i tożsamości.

Otrzymujesz poniższe cechy:
\begin{itemize}
    \item  Umiejętność: Jesteś wyszkolony w skradaniu się.
    \item  Umiejętność: Jesteś wyszkolony w opieraniu się przesłuchaniom i sztuczkom, by wydobyć z Ciebie prawdę.
    \item  Konfuzja: Masz talenty i zdolności Bóg wie skąd. Możesz spróbować wykonać jedno zadanie, do którego nie masz przeszkolenia, jakbyś był wyszkolony, zadanie,w którym jesteś wyszkolony, jakbyś był wyspecjalizowany, lub zyskać darmowy poziom Wysiłku dla zadania, w którym jesteś wyspecjalizowany. Ta zdolność restartuje się za każdym razem, gdy wykonasz rzut na odzyskanie zdrowia, ale jej użycia nigdy się nie kumulują.
    \item  Nieumiejętność: Ludzie nigdy nie wiedzą, co Cięz nimi łączy. Wszelkie zadania, mające na celu sprawienie, by ludzie Ci uwierzyli lub Ci zaufali, są dla Ciebie utrudnione.
\end{itemize}

Początki Przygód: Z listy poniższych opcji, wybierz jak Twoja postać wzięła udział w pierwszej przygodzie.

1. Po prostu się stawiłeś pewnego dnia.

2. Przekonałeś jednego z BG, że masz cenne umiejętności.

3. Równie tajemnicza osoba powiedziała Ci gdzie być i kiedy (ale nie dlaczego) aby dołączyć do grupy.

4. Coś – przeczucie lub być może sen – powiedziało Ci, aby być w danym miejscu i dołączyć do BG.

\subsubsection{Tchórzliwy}\index{Deskryptory!Tchórzliwy}

Odwaga zawodzi Cię za każdym razem. Nie masz siły woli i mocnego postanowienia, by spojrzeć twarzą w twarz niebezpieczeństwu. Strach pożera Twe serce. Słuchanie swoich lęków pomogło Ci w ucieczce od niebezpieczeństwa i unikaniu zbędnego ryzyka. Inni mogli cierpieć w Twoim miejscu, i możesz być pierwszym, by się do tego przyznać, ale w sekrecie czujesz wielką ulgę, uniknąwszy niewyobrażalnego i strasznego losu.

(Deskryptory takie jak Tchórzliwy, Okrutny lub Niehonorowy mogą nie być stosowne dla każdej grupy. Są to cechy złoczyńców i niektórzy ludzie nie chcą, by ich BG nie byli niczym mniej niż w 100\% herosami. Ale inni nie mają nic przeciwko odrobinie molarnej szarości. Inni z kolei patrzą na Okrutnego i Tchórzliwego jak na cechy, które trzeba przezwyciężyć w trakcie gry (najpewniej zyskując po drodze odmienne deskryptory).)

Otrzymujesz poniższe cechy:
\begin{itemize}
    \item  Ukradkowy: +2 do Puli Szybkości.
    \item  Umiejętność: Jesteś wyszkolony w skradaniu się.
    \item  Umiejętność: Jesteś wyszkolony w bieganiu.
    \item  Umiejętność: Jesteś wyszkolony w każdej akcji polegającej na uciekaniu od niebezpieczeństwa, groźnej sytuacji lub unikaniu kłopotów. 
    \item  Nieumiejętność: Nie wchodzisz z własnej woli w niebezpieczne sytuacje. Wszelkie rzuty na inicjatywę (by określić kto pierwszy działa w walc) są dla Ciebie utrudnione.
    \item  Nieumiejętność: Trzęsiesz się jak galareta gdy masz wykonać potencjalnie niebezpieczne zadanie sam. Wszelkie takie akcje (takie jak atakowanie istoty, gdy jesteś sam) są dla Ciebie utrudnione.
    \item  Dodatkowy Ekwipunek: Masz amulet przynoszący szczęście lub urządzenie, chroniące Cię od niebezpieczeństw. 
\end{itemize}

Początki Przygód: Z listy poniższych opcji, wybierz jak Twoja postać wzięła udział w pierwszej przygodzie.

1. Wierzysz, że jesteś nawiedzany i zatrudniłeś jednego z BG jako swojego obrońcę.

2. Chcesz uciec od wstydu i zebrałeś zdolne jednostki w nadziei na poprawienie swojej reputacji.

3. Jeden z BG zmusił Cię groźba, byś im towarzyszył.

4. Grupa odpowiedziała na Twoje krzyki o pomoc, kiedy miałeś kłopoty.

\subsubsection{Twardy}\index{Deskryptory!Twardy}

Jesteś silny i możesz wytrzymać naprawdę wiele. Możesz być wielki i mieć kanciaste szczęki. Twarde postaci często mają widoczne blizny.

Otrzymujesz poniższe cechy:
\begin{itemize}
    \item  Odporny: +1 do Pancerza
    \item  Zdrowy: Dodaj 1 do punktów które odzyskujesz gdy rzucasz na odzyskanie zdrowia.
    \item  Umiejętność: Jesteś wyszkolony w Obronie Mocy.
    \item  Dodatkowy Ekwipunek: Masz dodatkową lekką broń.
\end{itemize}

Początki Przygód: Z listy poniższych opcji, wybierz jak Twoja postać wzięła udział w pierwszej przygodzie.

1. Jesteś ochroniarzem jednego z innych BG.

2. Jeden z BG to Twoja rodzeństwo i masz na nie oko.

3. Potrzebujesz pieniędzy, gdyż Twoja rodzina jest zadłużona.

4. Ochroniłeś innego BG, gdy ten był zagrożony. Kiedy rozmawiałeś z nim potem, usłyszałeś o celu grupy. 

\subsubsection{Uczony}\index{Deskryptory!Uczony}

Uczyłeś się dużo, albo samodzielnie, albo z nauczycielem. Wiesz wiele rzeczy i jesteś ekspertem w kilku tematach, takich jak historia, biologia, geografia, mitologia, nature lub inne dziedziny wiedzy. Uczone postaci zazwyczaj mają przy sobie parę książek i spędzają wolny czas na czytaniu.

Otrzymujesz poniższe cechy:
\begin{itemize}
    \item  Bystry: +2 do Puli Inteligencji.
    \item  Umiejętność: Jesteś wyszkolony w trzech dziedzinach wiedzy własnego wyboru.
    \item  Nieumiejętność: Masz mało uroku osobistego. Wszelkie zadania powiązane z byciem miłym, perswazją i etykietą są dla Ciebie utrudnione.
    \item  Dodatkowy Ekwipunek: Masz dwie książki traktujące o tematach własnego wyboru.
\end{itemize}

Początki Przygód: Z listy poniższych opcji, wybierz jak Twoja postać wzięła udział w pierwszej przygodzie.

1. Jeden z BG poprosił Cię o pomoc ze względu na Twoją wiedzę.

2. Potrzebujesz pieniędzy, by sfinansować swoje badania.

3. Wierzyłeś, że to zadanie doprowadzi do ważnych i interesujących odkryć.

4. Kolega po fachu poprosił Cię o wzięcie udziału w misji.

\subsubsection{Ukradkowy}\index{Deskryptory!Ukradkowy}

Jesteś zwinny, śliski i szybki. Te talenty pomagają Ci się ukrywać, poruszać cicho i wykonywać sztuczki z dłońmi. Najpewniej jesteś mały. Jednakże, zły z Ciebie sprinter – jesteś bardziej zwinny niż szybko biegnący.

Otrzymujesz poniższe cechy:
\begin{itemize}
    \item  Szybki: +2 do Puli Szybkości.
    \item  Umiejętność: Jesteś wyszkolony w skradaniu się.
    \item  Umiejętność: Jesteś wyszkolony w interakcjach polegających na kłamaniu lub oszustwie.
    \item  Umiejętność: Jesteś wyszkolony we wszystkich specjalnych zdolnościach powiązanych z iluzjami lub oszustwami.
    \item  Nieumiejętność:  Jesteś ukradkowy, ale nie szybki. Wszystkie zadania polegające na przemieszczaniu się są dla Ciebie utrudnione. 
\end{itemize}

Początki Przygód: Z listy poniższych opcji, wybierz jak Twoja postać wzięła udział w pierwszej przygodzie.

1. Chciałeś ukraść coś innemu BG. Złapał on Cię i zmusił, byś udał się z nim.

2. Śledziłeś jednego z BG z sobie znanych powodów, co wplątało Cię w akcję.

3. Szef (BN) w sekrecie zapłacił Ci, byś wziął udział w przygodzie.

4. Podsłuchałeś BG mówiących o temacie, który Cię zainteresował, więc podszedłeś do nich.

Urokliwy
Jesteś prawdziwym czarusiem. Niezależnie od tego, czy masz najwyraźniej nadprzyrodzone zdolności, czy po prostu jesteś dobry w słowa, możesz przekonywać innych ludzi do swoich racji. Najpewniej jesteś fizycznie atrakcyjny lub przynajmniej charyzmatyczny, a inni lubią słuchać Twojego głosu. Najpewniej dużo uwagi poświęcasz swojemu wyglądowi. Łatwo Ci nawiązywać nowe przyjaźnie. Ten deskryptor skupia się na Intelekcie jako mierze osobowości postaci – inteligencja nie jest Twoją mocną stroną. Jesteś przyjemny w obyciu, ale niekoniecznie uczony lub o silnej woli. 
Otrzymujesz poniższe cechy:
    • Lubiany: +2 do Puli Intelektu.
    • Umiejętność: Jesteś wyszkolony we wszystkich zadaniahc polegających na pozytywnej lub przyjemnej interakcji społecznej.
    • Umiejętność: Jesteś wyszkolony w używaniu specjalnych zdolności, które wpływają na umysł innych.
    • Informator: Posiadasz znajomego, który pełni ważną funkcję, np.: pomniejszego szlachcica, kapitana straży miejskiej lub głową dużego gangu złodziei. Ty i MG powinniście popracować wspólnie nad stworzeniem tej postaci.
    • Nieumiejętność: Nigdy nie byłeś dobry w szkole. Wszelkie zadania powiązanie z wiedzą, nauką i zrozumieniem są dla Ciebie utrudnione.
    • Nieumiejętność: Twoja siła woli nie jest najlepsza. Obronne akcje w celu unikania mentalnych ataków są dla Ciebie utrudnione.
    • Dodatkowy Ekwipunek: Zdołałeś pozyskać ostatnio pewne porządne zniżki. W efekcie, masz dostatecznie dużo pieniędzy, by zakupić jeden przedmiot średniej ceny. 
Początki Przygód: Z listy poniższych opcji, wybierz jak Twoja postać wzięła udział w pierwszej przygodzie.
1. Przekonałeś jednego z BG, by powiedział Ci, co zamierza zrobić.
2. Zapoczątkowałeś całą tę rzecz i nakłoniłeś innych, by do Ciebie dołączyli.
3. Jeden z innych BG zrobił Ci przysługę i teraz spłacasz zobowiązanie, pomagając mu.
4. Oferowana jest nagroda, a Ty potrzebujesz pieniędzy.
Uważny
Mało rzeczy Ci umyka. Dostrzegasz najmniejsze detale w świecie wokół Ciebie i umiejętnie dedukujesz na bazie informacji, które zdobywasz. Twoje talenty czynią z Ciebie wyśmienitego detektywa, bystrego naukowca, lub utalentowanego zwiadowcę.
Jednakże, masz trudności ze zrozumieniem społecznych interakcji. Nie zwracasz uwagi na to, kogo Twoje dedukcje obrazą, lub jak niekomfortowo inni się czują, gdy ich badasz. Masz w zwyczaju traktować innych jak ludzi mniejszego intelektu w porównaniu z Tobą, co wykorzystujesz, gdy tego potrzebujesz.
Otrzymujesz poniższe cechy:
    • Bystry: +2 do Puli Intelektu.
    • Umiejętność: Masz oko do zczegółów. Jesteś wyszkolony w każdym zadaniu, które polega na znajdywaniu lub dostrzeganiu małych detali.
    • Umiejętność: Wiesz co nieco o wszystkim. Jesteś wyszkolony w każdym zadaniu, które polega na identyfikowaniu obiektów lub przywoływaniu z pamięci detali i szczegółów oraz faktów.
    • Umiejętność: Twoja umiejętność wyciągania wniosków jest zatrważająca. Jesteś wyszkolony w zadaniach polegających na zastraszaniu innych istot.
    • Nieumiejętność: Twoja pewność siebie jest przez innych uważana za arogancję. Wszelkie zadania związana z pozytywną interakcją społeczną są dla Ciebie utrudnione.
    • Dodatkowy Ekwipunek: Masz torbę lekkich narzędzi.
Początki Przygód: Z listy poniższych opcji, wybierz jak Twoja postać wzięła udział w pierwszej przygodzie.
1. Podsłuchałeś, jak inni BG mówili o swojej misji i zgłosiłeś się na ochotnika.
2. Jeden z BG poprosił Cię ,byś im towarzyszył, wierząc, że Twoje talenty okażą się ważne podczas tej misji. 
3. Wierzysz że misja BG jest jakoś powiązana z jednym z Twoich śledztw.
4. Osoba trzecia zrekrutowała Cię, byś podążał za BG i zobaczył, co planują.
Wygnany
Podążasz długą i samotną drogą, zostawiwszy swój dom i życie za sobą. Może popełniłeś okropną zbrodnię, coś tak okropnego, że Twoi ludzie Cię wygnali, pod groźbą śmierci. Mogłeś zostać oskarżony o zbrodnię, której nie popełniłeś, i teraz płacisz cenę błędów innej osoby. Twoje wygnanie może być skutkiem gafy – może zawstydziłeś swoją rodzinę lub przyjaciela, lub samego siebie w obliczu kolegów, autorytetu lub kogoś, kogo szanujesz. Niezależnie od powodu, zostawiłeś stare życie z tyłu i teraz starasz się stworzyć dla siebie nowe.
Otrzymujesz poniższe cechy:
    • Samowystarczalny: +2 do Puli Mocy.
    • Samotnik: Nie uzyskujesz korzyści, gdy uzyskujesz pomoc od innej postaci, która jest wyszkolona lub wyspecjalizowana w danym zadaniu.
    • Umiejętność: Jesteś wyszkolony w skradaniu się.
    • Umiejętność: Jesteś wyszkolony w zbieractwie, polowaniu i znajdowaniu bezpiecznych miejsc do odpoczynku lub ukrycia się.
    • Nieumiejętność: Życie na własną rękę tak długo sprawia, że trudniej Ci ufać innym i wejść w sytuacje społeczne. Wszelkie zadania polegające na społecznej interakcji są dla Ciebie utrudnione.
    • Dodatkowy Ekwipunek: Masz pamiątkę ze swojej przeszłości – stare zdjęcie, medalik z puklem włosów, lub zapalniczkę daną Ci przez kogoś ważnego. Trzymasz ten obiekt blisko siebie i wyciągasz, by powspominać stare, dobre czasy.
Początki Przygód: Z listy poniższych opcji, wybierz jak Twoja postać wzięła udział w pierwszej przygodzie.
1. Inny BG pozyskał Twoje zaufanie pomagajac Ci, gdy tego potrzebowałeś. Towarzyszysz mu, by spłacić dług.
2. Kiedy podróżowałeś samotnie, odkryłeś coś dziwnego. Kiedy trafiłeś do osady, BG byli jedynymi, którzy Ci uwierzyli, podróżując z Tobą, by dać sobie radę z owym problemem.
3. Inny z BG przypomina Ci kogoś z Twojej przeszłości.
4. Masz dosyć swojej izolacji. Dołączenie do BG dale Ci szansę, by na nowo mieć przyjaciół.
Wytrzymały
Twoje ciało jest stworzone, by wytrzymać wiele. Niezależnie od tego, czy upijasz się w barze lub wymieniasz się ciosami z członkiem gangu, idziesz naprzód, a trudy spływają po Tobie jak po kaczce. Ani głód, ani pragnienie, pocięte ciało lub złamane kości – ni c z tego nie może Cię zatrzymać. Po prostu ignorujesz ból i przesz naprzód. 
Pomimo cieszenia się kondycją i zdrowiem, znaki zużycia malują się na Twoim ciele milionem mniejszych blizn, potrójnie złamanym nosem, poszarpanymi uszami i innymi oznakami fizycznymi, które nosisz z dumą.
Otrzymujesz poniższe cechy:
    • Potężny: +4 do Puli Mocy.
    • Szybkie Leczenie: Dzielisz na pół czas potrzebny, by wykonać rzut na odzyskanie zdrowia (co najmniej jedna tura).
    • Prawie Niepowstrzymany: Kiedy znajdujesz się na stopniu zranionego na liczniku obrażeń, funkcjonujesz jakbyś był w pełni zdrowia. Kiedy jesteś krytycznie ranny, funkcjonujesz jakbyś był zraniony. W innych słowach, konsekwencje była zranionym dotyczą Ciebie tylko gdy stajesz się krytycznie ranny i nigdy nie ponosisz konsekwencji bycia krytycznie rannym. Dalej umierasz, jeśli wszystkie Twoje punkty Pul spadną do 0.
    • Umiejętność: Jesteś wyszkolony w Obronie Mocy.
    • Nieumiejętność: Twoje silne, wielkie ciało reaguje wolno. Rzuty na inicjatywęsą dla Ciebie utrudnione.
    • Niezgrabny: Gdy stosujesz Wysiłek na rzucie Szybkości, musisz wydać 1 dodatkowy punkt ze swojej Puli Szybkości.
Początki Przygód: Z listy poniższych opcji, wybierz jak Twoja postać wzięła udział w pierwszej przygodzie.
1. BG zrekrutowali Cię, usłyszawszy o Twojej reputacji.
2. Dołączyłeś do BG, bo potrzebowałeś pieniędzy.
3. BG oferowali wyzwanie równe Twojej mocy.
4. Wierzysz, że BG przetrwają tylko wtedy, jeśli będziesz ich bronił.
Zabawny
Jesteś radosny i przyjazny. Pomagasz się innym zrelaksować za pomocą uśmiechu i dowcipu, możliwe, że z samego siebie, choć także lekko z Twoich towarzyszy.  Czasami ludzie mówią, że niczego nie bierzesz na serio. Nie jest to prawdą, oczywiście, ale nauczyłeś się, że rozmyślanie o złych rzeczach zbyt długo nie prowadzi do niczego dobrego. Zawsze masz nowy dowcip i kolekcjonujesz je tak, jak niektórzy kolekcjonują butelki wina.
Zyskujesz poniższe cechy:
    • Zabawny: +2 do Puli Intelektu.
    • Umiejętność: Jesteś biesiadny i większość ludzi lubi z Tobą przebywać. Jesteś wyszkolony we wszystkich zadaniach związanych z przyjemną interakcją społeczną.
    • Umiejętność: masz przewagę w odkrywaniu puent dowcipów, których nigdy wcześniej nie słyszałeś. Jesteś wyszkolony we wszystkich zadaniach związanych z rozwiązywaniem zagadek i łamigłówek.
Początki Przygód: Z listy poniższych opcji, wybierz jak Twoja postać wzięła udział w pierwszej przygodzie.
1. Rozwiązałeś łamigłówkę zanim zdałeś sobie sprawę, że odpowiedź rozpoczęłaby obecną przygodę.
2. Inni BG uznali, że przyniesiesz z sobą bardzo im potrzebny humor na misji.
3. Uznałeś, że sama zabawa i zero pracy nie jest najlepszym sposobem, by przeżyć życie, więc dołączyłeś do innych BG.
4. Miałeś wybór – pójść z BG, lub zmierzyć sięz czymś, co nie jest zabawne w ogóle. 
Zwinny
Poruszasz się szybko, jesteś dobrym sprinterem i pracujesz umiejętnie swoimi dłońmi. Jesteś świetny w bieganiu, ale nie zawsze wychodzi Ci to z gracją. Najpewniej jesteś chudy i muskularny.
Otrzymujesz poniższe cechy:
    • Szybki: +4 do puli Szybkości.
    • Umiejętność: Jesteś wyszkolony w rzutach na inicjatywę (by określić kto pierwszy działa w turze).
    • Umiejętność: Jesteś wyszkolony w bieganiu.
    • Nieumiejętność: Jesteś szybki, ale niekoniecznie pełen gracji. Wszelkie zadania związane z równowagą są dla Ciebie utrudnione. 
Początki Przygód: Z listy poniższych opcji, wybierz jak Twoja postać wzięła udział w pierwszej przygodzie.
1. Pomimo swojego lepszego osądu, dołączyłeś do BG ponieważ byli w niebezpieczeństwie. 
2. Jeden z BG przekonał Cię, byś dołączył do grupy, gdyż byłoby to w Twoim najlepszym interesie.
3. Boisz się co się może zdarzyć, jeśli BG odniosą klęskę.
4. W tle jest nagroda, a Ty potrzebujesz pieniędzy.
Złośliwy
Próbujesz ukryć swoje wnętrze, krzyk, by się wyzwolić i sprawić, by zapłacili, cierpieli i krwawili. Czasami udaje Ci się to dla Twoich przyjaciół – uśmiechać tak jak oni, śmiać, gdy oni się śmieją, a czasami nawet odczuwać inne emocje. Ale to zawsze czeka w środku, to poczucie szaleńczej radości zmieszane z nienawiścią, czasami wyskakujące, gdy konfrontujesz się z wrogiem. Twoi przyjaciele mogą tolerować przemoc, ale czasami martwisz się, że odkryją, że jesteś także okrutny.
Otrzymujesz poniższe cechy:
    • Umiejętność: Jesteś wyszkolony w śledzeniu istot. Jeśli istota Ciękrzywdziła, to zadanie jest ułatwione.
    • Żądny krwi: Kiedy zaczynasz walkę, widzisz tylko czerwień. Zadajesz 2 dodatkowe punkty obrażeń dowolnym atakiem. 
    • Berserk: Kiedy już zaczniesz walkę, ciężko Ci przestać. Po prawdzie, jest to zadanie Intelektu trudności 2 , nawet, gdy wróg się podda lub gdy wrogowie Ci się skończą. Jeśli nastąpi ta druga sytuacja, atakujesz najbliższego sprzymierzeńca w zasięgu.
    • Dodatkowy Ekwipunek: Posiadasz notatki z listą wszystkich, którzy zrobili Ci krzywdę.
Początki Przygód: Z listy poniższych opcji, wybierz jak Twoja postać wzięła udział w pierwszej przygodzie.
1. Inny BG widział, jak poradziłeś sobie z pijakiem w karczmie, nie wiedząc, że to Ty zacząłeś bójkę.
2. Chciałeś uciec od złej sytuacji, więc ruszyłeś z BG.
3. Chcesz się zmienić, i masz nadzieję ,że przebywanie w obecności reszty BG pomoże Ci zaznać spokoju.
4. Jeden z BG poprosił Cię, byś im towarzyszył, wierząc że Twoja złość może zostać opanowana na potrzeby misji.
\section{Specjalizacje}\index{Specjalizacje}

Specjalizacja czyni postać wyjątkową. Specjalziacje postaci w grupie nie powinny się powtarzać. Specjalizacja daje postaci korzyści podczas tworzenia postaci i z każdym następnym poziomem. Jest to czasownik w zdaniu ``Jestem przymiotnikiem rzeczownikiem który czasownikuje''.

Ten rozdział zawiera około 100 przykładowych specjalizacji, takich jak Nosi Halo Ognia, Wolałby Czytać i Pilotuje Statki Kosmiczne. Te specjalizacje mogą być wybrane i użyte przez gracza, lub przez MG, który dodaje je do listy dostępnych specjalizacji dla swoich graczy w następnej kampanii. 

Dodatkowo, dalsza część rozdziału zapewnia narzędzia dla MG, który chciałby stworzyć swoją własną specjalizację, tak, by pasowała do wymogów danej gry lub kampanii.

\subsection{Wybieranie specjalizacji}\index{Specjalizacje!Wybieranie specjalizacji}

Nie wszystkie specjalizacje pasują do każdej konwencji. Rozdział Konwencja zapewnia porady i pomoc, ale tutaj pójdą pewne uogólnienia. Oczywiście, MG może zezwolić na każdą specjalizację w swoim settingu. Specjalizacje są ważną jego częścią, gdyż np.: obecność na liście dozwolonych specjalizacji Włada Mocami Mentalnymi oznacza, że w tym świecie istnieją moce psioniczne, Wyje do Księżyca oznacza istnienie w nim likantropów, a Pilotuje Statki Kosmiczne sprawia, że w settingu, no cóż, istnieją statki kosmiczne. 

Kiedy wybiera się specjalizację dla postaci, otrzymuje ona specjalne połączenie z jedną lub więcej innych postaci graczy, zdolność pierwszego poziomu i być może dodatkowy ekwipunek, niezbędny, by korzystać z mocy zapewnianych przez specjalizację, lub który dobrze z nią współgra. Dla przykładu, postać będąca rzemieślnikiem może potrzebować wielu narzędzi. Postać, która ciągle płonie, potrzebuje ubrań odpornych na ogień. Postać rysująca magiczne runy może potrzebować pędzelka i farm. Postać zabijająca potwory mieczem potrzebuje miecza. I tak dalej. Jednakże, wiele specjalizacji nie wymaga dodatkowego ekwipunku. Każda specjalizacja oferuje także jedne lub dwie sugestie na Wtrącenia MG z listą możliwych konsekwencji naprawdę dobrych i złych rzutów kością.  
 
Parę specjalizacji w tym rozdziale zapewnia “zamianę z typem”, która pozwala zamienić zdolność typu na zdolność specjalizacji. Gracz nie musi poczynić tej zamiany, ale ma taką możliwość. Dla przykładu, specjalizacja Kocha Pustkę zapewnia opcję zyskania zdolności Mam Kombinezon Kosmiczny, Będę Podróżnikiem zamiast zdolności typu.

W miarę, jak postać zyskuje nowe poziomy, specjalizacja daje więcej zdolności. Każdy poziom jest zazwyczaj oznaczony jako Akcja lub Umożliwienie, czyni inne akcje lepszymi lub zapewnia jakieś inne korzyści, ale nie akcję. Zdolność, która pozwala bohaterowi razić wrogów laserami jest Akcją. Zdolność, która daje więcej dodatkowych obrażeń, kiedy się wykonuje akcję, jest Umożliwieniem. Umożliwienie jest wykorzystywane w tej samej turze co inna akcja, i często jest częścią innej akcji. Korzyści każdego poziomu są niezależne i kumulują się z korzyściami z innych poziomów (chyba, że zaznaczono inaczej). Tak więc jeśli zdolność pierwszego poziomu daje +1 do Pancerza, a zdolność czwartego poziomu także daje +1 do Pancerza, to postać na czwartym poziomie ma w sumie pancerz na +2.
Na poziomach trzecim i szóstym, postać może wybrać zdolność z dwóch opcji.

Możesz także wybrać, czy chcesz rozwinąć historie będącą opisem danej specjalizacji (choć nie jest to wymagane).

\subsection{Połączenia z innymi BG}\index{Specjalizacje!Połączenia z innymi BG}

Wybierz połączenie, które pasuje do specjalności. Jeśli jesteś MG wybierającym (lub tworzącym) jeden lub kilka specjalizacji dla swoich postaci, wybierz do 4 z poniższych połączeń.

\begin{itemize}
\item Wybierz innego BG. Z nieznanych Tobie powodów, ta postać jest kompletnie odporna na Twoje zdolności specjalizacji, niezależnie od tego, czy pragniesz jej nimi pomóc, czy zagrozić.
\item Wybierz innego BG. Wiedziałeś o jego istnieniu przez lata, ale nie sądziłeś, że on Ciebie znał.
\item Wybierz innego BG. Zawsze chcesz mu zaimponować, ale sam nie wiesz czemu.
\item Wybierz innego BG. Ta postać ma nawyk, który Cię wkurza, ale poza tym jesteś pod wrażeniem jej zdolności.
\item Wybierz innego BG. Ta postać posiada potencjał w okiełznaniu Twojego stylu walki, paradygmatu lub innej zdolności zapewnianej przez Twoją specjalizację. Chciałbyś ją potrenować, ale nie jesteś dobrze przygotowany do nauczania (być może) a ona może nie być zainteresowana (znowu: być może).
\item Wybierz innego BG. Jeśli znajduje się on w bliski zasięgu kiedy walczysz, czasami zapewnia on atut, a czasami utrudnia Twój test na atak (szansa 50% na jedno lub drugie, rzucane raz na walkę).
\item Wybierz innego BG. Kiedyś ocaliłeś mu życie i teraz ma dług wdzięczności. Nie jesteś z tego jakoś bardzo zachwycony – zrobiłeś, co należało zrobić i tyle.
\item Wybierz innego BG. Ta postać ostatnio Cię wyśmiała, co naprawdę Cię zraniło. Jak zamierzasz sobie z tym poradzić (jeśli w ogóle) zależy od Ciebie.
\item Wybierz innego BG. Ta postać wie, że cierpiałeś z powodu robotów w przeszłości. To, czy nienawidzisz robotów, zależy od Ciebie, co może wpłynąć na Twoją relację z tym BG, jeśli jest on przyjacielem robotów lub posiada robotyczne protezy.
\item Wybierz innego BG. Ta postać pochodzi z tego samego miejsca co ty i znaliście się jako dzieci. 
\item Wybierz innego BG. W przeszłości, nauczył on Cię paru trików dy wykorzystania w walce.
\item Wybierz innego BG. Ta postać nie pochwala Twoich metod.
\item Wybierz innego BG. Dawno temu, byliście po przeciwnych stronach barykady w starciu. Wygrałeś, choć w jego oczach “oszukiwałeś” (ale z Twojej perspektywy wszystko jest ok). Może on chcieć ponownego starcia, choć to zależy od niego.
\item Wybierz innego BG. Zawsze próbujesz zachwycić tą postać swoimi umiejętnościami, sprytem, wyglądem lub odwagą. Może jest ona Twoim rywalem, może pragniesz jej szacunku, a może jest ona Twoim obiektem westchnień.
\item Wybierz innego BG. Boisz się, że jest on zazdrosny o Twoje zdolności i martwisz się, że może to doprowadzić do problemów. 
\item Wybierz innego BG. Przypadkowo został złapany w pułapkę, którą założyłeś i musiał się wydostać z niej o własnych siłach.
\item Wybierz innego BG. Kiedyś zostałeś zatrudniony, by wyśledzić kogoś, kto był blisko tej postaci.
\item Wybierz dwóch BG (najlepiej takich, którzy mogą znaleźć się na trajektorii Twoich ataków). Kiedy nie trafiasz atakiem i MG decyduje, że atak uderzył w kogoś innego niż w Twój cel, trafia on w jedną tych dwóch postaci.
\item Wybierz jednego BG. Nie jesteś pewien jak ani skąd, ale ta postać posiada butelki rzadkiego alkoholu i może go dla Ciebie sprowadzić za pół ceny.
\item Wybierz jednego BG. Ostatnio straciłeś posiadany przedmiot i przekonałeś samego siebie, że to ten BG go skradł. Czy tak jest w istocie zależy od niego.
\item Wybierz jednego BG. On zawsze zdaje się wiedzieć, gdzie jesteś, lub przynajmniej w którym kierunku się znajdujesz w stosunku do niego.
\item Wybierz jednego BG. Patrzenie jak używasz swoich zdolności specjalizacyjnych wydaje się budzić w nim nieprzyjemne wspomnienia. To wspomnienie leży w gestii tego BG, choć może on nie być w stanie przywołać je na poziomie świadomości.
\item Wybierz jednego BG. Coś sprawia, że jego obecność przeszkadza w Twoich zdolnościach. Kiedy stoi on obok Ciebie, Twoje zdolności specjalizacyjne kosztują 1 dodatkowy punkt więcej.
\item Wybierz jednego BG. Coś sprawia, że uzupełnia on Twoje zdolności. Kiedy stoi on obok Ciebie, pierwsza zdolność specjalizacji, z której korzystasz w ciągu danej doby, kosztuje 2 punkty mniej.
\item Wybierz jednego BG. Znasz tą postać już jakiś czas, i pomogła ona Ci zyskać kontrolę nad Twoimi zdolnościami specjalizacyjnymi.
\item Wybierz jednego BG. Kiedyś w przeszłości tej postaci, miała ona dewastujące doświadczenie, kiedy próbowała zrobić coś, co Tobie przychodzi łatwo dzięki Twojej specjalizacji. To od niej zależy, czy powiedziała Ci o tym.
\item Wybierz jednego BG. Jego okazjonalna niezdarność i głośne zachowania irytują Cię.
\item Wybierz jednego BG. W niedalekiej przeszłości, gdy trenowaliście, przypadkowo trafiłeś go swoim atakiem, poważnie go raniąc. To od niego zależy, czy żywi urazę, czy też może wybaczył tobie.
\item Wybierz jednego BG. Wisi Ci on dużą sumę pieniędzy.
\item Wybierz jednego BG. W niedalekiej przeszłości, kiedy uciekaliście od jakiegoś zagrożenia, przypadkowo zostawiłeś tą postać z tyłu, by dała sobie radę sama. Przetrwała ona, ale ledwie. Od gracza tamtego BG zależy, czy jego postać dalej jest zła, czy może Ci wybaczyła.
\item Wybierz jednego BG. Niedawno, przypadkowo (lub celowo) sprawił on, że znalazłeś się w niebezpieczeństwie. Wszystko z Tobą teraz w porządku, ale masz się na baczności w jego obecności.
\item Wybierz jednego BG. Z Twojego punktu widzenia, wydaje się on nerwowy w związku z pewną ideą, osobą lub sytuacją. Chciałbyś nauczyć go jak być bardziej wyluzowanym (jeśli tylko Ci na to pozwoli).
\item Wybierz jednego BG. Kiedyś nazwał Cię on tchórzem.
\item Wybierz jednego BG. Ta postać zawsze rozpoznaje Cię i Twoje ślady, nawet kiedy jesteś w przebraniu lub uciekłeś z danego miejsca dawno temu.
\item Wybierz jednego BG. Niechcący spowodowałeś wypadek, który sprawił, że zapadł on w sen tak głęboki, że nie obudził się przez trzy dni. To, czy Ci wybaczył, czy też nie, zależy od niego.
\item Wybierz jednego BG. Jesteś przekonany, że jesteście w jakiś sposób spokrewnieni.
\item Wybierz jednego BG. Przypadkowo dowiedziałeś się czegoś, co on próbuje utrzymać w tajemnicy. 
\item Wybierz jednego BG. Jest on szczególnie wrażliwy na co bardziej widocznie zdolności Twojej specjalizacji, i czasami doznaje szoku trwającego parę rund, co utrudnia jego akcje. 
\item Wybierz jednego BG. Najwyraźniej posiada on cenny przedmiot, który kiedyś był Twój, a który przegrałeś w grach hazardowych lata temu.
\item Wybierz jednego BG. Gdyby nie Ty, ta postać oblałaby w przeszłości test zdolności umysłowych.
\item Wybierz jednego BG. Bazując na paru komentarzach, które podsłuchałeś, podejrzewasz, że nie darzy on Twojej strefy kompetencji lub ulubionego hobby wielką estymą. 
\item Wybierz jednego BG, którego specjalizacja jest powiązana z Twoją. Połączenie wpływa na nie w pewien sposób. Dla przykładu, jeśli postać korzysta z broni, Twoja zdolność specjalizacyjna czasami ulepsza ten atak w pewien sposób.
\item Wybierz jednego BG. Panicznie boi się on wysokości. Chciałbyś nauczyć go, jak być bardziej wyluzowanym na wysokościach. To od niego zależy decyzja, czy się zgodzi zaakceptować Twoją pomoc.
\item Wybierz jednego BG. Jest on skeptyczny odnośnie Twoich twierdzeń o czymś ważnym, co się przytrafiło Tobie w przeszłości. Może on nawet chcieć cię zdyskredytować lub odkryć “tajemnicę” Twojej historii, choć to zależy od niego.
\item Wybierz jednego BG. Ma on talent do dostrzegania, gdzie Twoje plany mają słabe punkty.
\item Wybierz jednego BG. Twarz tej postaci jest tak intrygująca z powodów, których nie rozumiesz, że czasami ją szkicujesz w piasku lub innym medium, do którego masz dostęp.
\item Wybierz jednego BG. Ta postać ma dodatkowy zwykły przedmiot od Ciebie – może to być coś, co zrobiłeś lub po prostu coś, co jej ofiarowałeś. (Dany gracz wybiera przedmiot.)
\item Wybierz jednego BG. Wynajął on Cię, byś wykonał dla niego pewną robotę. Otrzymałeś zapłatę, ale jeszcze nie wykonałeś tej pracy.
\item Wybierz jednego BG. Pracowaliście razem kiedyś, i skończyło się to źle.
\item Wybierz jednego BG. Kiedy stoi on obok Ciebie i poświęca swoją akcję na skoncentrowanie się, by ci pomóc,  jedna z Twoich zdolności specjalizacji ma podwojony zasięg.
\end{itemize}

\subsection{O specjalizacji}

Specjalizacje w tej książce celowo mają opis w ledwie kilku zdaniach, by można było je zastosować w wielu konwencjach. Zdanie lub dwa podsumowuje każdy z nich. Po wyborze przez Ciebie specjalizacji, masz opcję rozszerzenia jej opisu, tak, by pasowała do settingu lub postaci.

Dla przykładu, jeśli wybierasz Działa pod Przykrywką, opis tej specjalizacji to “Udając kogoś innego, poszukujesz odpowiedzi, których potężni tego świata nie chcą wyjawić”. Jeśli wybierasz Tworzy Dziwną Naukę, opis brzmi “Twoje nadnaturalne wejrzenie i zdolności tworzą z Ciebie naukowca zdolnego tworzyć cuda”. Te opisy zapewniają czego potrzebujesz, by korzystać z Specjalności.

Jednakże, jeśli sobie życzysz (i tylko, jeśli sobie życzysz – nie ma takiego obowiązku) możesz dodać więcej do tych opisów w sposób, który pasuje do Twojej gry. Dla przykładu, jeśli wybierasz Działa pod Przykrywką i Tworzy Dziwną Naukę dla sesji współczesnej, takiej jak horror, urban fantasy, sesja szpiegowska lub coś podobnego, możesz rozbudować opisy, jak pokazano w poniższych przykładach.

\textbf{Działa pod Przykrywką}: Szpiegostwo nie jest czymś, o czym masz jakąkolwiek wiedzę. Przynajmniej chcesz, by wszyscy wokół w to wierzyli, ponieważ naprawdę, zostałeś wytrenowany jako szpieg lub tajny agent. Możesz pracować dla rządu lub dla siebie. Możesz być funkcjonariuszem policji lub przestępcą. Możesz nawet być dziennikarzem śledczym.

Niezależnie od okoliczności, pozyskujesz informacje, które inni chcieliby zachować w tajemnicy. Zbierasz szeptane plotki, historie i dowody, i wykorzystujesz tę wiedzę w swoich własnych misjach oraz, jeśli to stosowne, zapewniasz swoim mocodawcom informacje, których pożądają. Alternatywnie, możesz sprzedać wiedzę, którą pozyskałeś, tym, którzy płacą najwięcej.

Najpewniej nosisz ciemne kolory – czarny, szarości lub ciemny błękit – by pomóc Ci wmieszać się w cienie, chyba, że przebrałeś się akurat za kogoś innego.

\textbf{Tworzy Dziwną Naukę}: Możesz być szanowanym naukowcem, publikującym w naukowych czasopismach. Lub możesz być uznawany za szaleńca przez innych, podążając za dziwnymi teoriami, którzy inni uznają za niedostatecznie dowiedzione. Prawdą jest jednak, że masz szczególny dar do przesuwania granic tego, co możliwe. Możesz pozyskać nową perspektywę i odblokować dziwne zjawiska dzięki swoim eksperymentom. Tam, gdzie inni widzą masę bzdur, Ty przeczesujesz teorie spiskowe dla olśnienia. Możesz robić swoje badania jako badacz rządowy, uniwersytecki, naukowiec korporacyjny, lub z wnętrza swojego własnego garażu. Zawsze jednak przesuwasz granice tego, co możliwe. 

Najpewniej dbasz o swoją pracę bardziej niż o trywialności pokroju własnego wyglądu, miłe zachowanie, lub społeczne normy, jednakże, ekscentryk Twojego pokroju nawet tutaj może się wyłamywać stereotypom. 

Jeśli chcesz pójść dalej, możesz także określić skąd zdolności Twojej specjalizacji się biorą. W zależności od konwencji, mogą ona brać się z treningu, magicznych run, poprzez zdolności cybernetyczne, dziedzictwo genetyczne lub ponieważ masz dostęp do zaawansowanej technologii. Dla przykładu, postać może być w stanie atakować błyskawicami ponieważ dostała się pod wpływ dziwnego promieniowania lub ponieważ posiada blaster elektryczny. Z drugiej strony, może tak się dziać, ponieważ intensywny trening odblokował dla niej dostęp do magii błyskawic. Możliwości są prawie nieskończone, i od Ciebie zależy, czy je wylistujesz, czy też nie.  Niezależnie od tego, jak zdolności zostały pozyskane, wystarczy, że działają.

\subsection{Specializacje}

Pełen opis wszystkich zdolności można znaleźć w odpowiednim rozdziale, który ma opisy typów, posmaków i zdolności w jednym bogatym katalogu.

\subsubsection{Absorbuje Energię}\index{Specjalizacje!Lista!Absorbuje Energię}

Władasz energią i zamieniasz ją na inne jej rodzaje.

Poziom 1: Absorpcja Energii Kinetycznej

Poziom 1: Wyzwolenie Energii

Poziom 2: Zasilenie Przedmiotu

Poziom 3: Absorpcja Czystej Energii lub Ulepszona Absorpcja Energii Kinetycznej

Poziom 4: Przeładowanie Energii

Poziom 5: Zasilenie istoty

Poziom 6: Zasilenie Tłumu lub Przeładowanie Urządzenia

Wtrącenia MG: Energia wyładowuje się w destruktywny sposób. Pewni drapieżcy żywią się czystą energią. Przypadkowy przedmiot zostaje wyssany z energii.

\subsubsection{Bada Ciemne Miejsca}\index{Specjalizacje!Lista!Bada Ciemne Miejsca}

Jesteś archetypowym łowcą skarbów i znalazcą zgubionych rzeczy.

Poziom 1: Wspaniały Odkrywca

Poziom 2: Wspaniały Infiltrator

Poziom 3: Dostosowanie Oczu

Poziom 3: Nocne Uderzenie lub Śliski Klient

Poziom 4: Ciężko Zapracowana Odporność

Poziom 5: Eksplorator Ciemności

Poziom 6: Oślepiający Atak lub W Objęciach Mroku

Wtrącenia MG: Przedmioty wypadają Ci z kieszeni lub torby w mroku, mapy Ci się gubią, pozyskane informacje nie zawierają istotnego szczegółu.  

\subsubsection{Buduje Roboty}\index{Specjalizacje!Lista!Buduje Roboty}

Twoje robotyczne twory robią to, czego od nich zażądasz. 

(Słowo “robot” użyte w tej specializacji jest używane, nawet jeśli roboty tworzone przed Ciebie mogą być odmienne od tych tworzonych przez kogoś innego, w zależności on konwencji. Roboty steampunkowe, organiczne lub nawet magiczne golemy – do 
tego wszystkiego odnosi się tutja słowo “robot”.)

Poziom 1: Robot-Asystent

Poziom 1: Twórco Robotów

Poziom 2: Kontrola Robotów

Poziom 3: Kompan-Ekspert lub Umiejętna Obrona

Poziom 4: Unowocześnienie Robota

Poziom 5: Armia Robotów

Poziom 6: Robotyczna Ewolucja lub Unowocześnienie Robota

Wtrącenia MG: Robot zostaje zhackowany, działa randomowo lub niespodziewanie wybucha.

\subsubsection{Chroni Słabszych}\index{Specjalizacje!Lista!Chroni Słabszych}

Pomagasz słabszym, pragnącym pomocy i bezsilnym.

Poziom 1: Odwaga

Poziom 1: Tarcza Obronna

Poziom 2: Wierny Obrońca

Poziom 2: Empatia

Poziom 3: Podwójni Bronieni lub Prawdziwy Strażnik

Poziom 4: Wyzwanie Bojowe

Poziom 5: Chętna Ofiara

Poziom 6: Resuscytacja lub Prawdziwy Obrońca

Wtrącenia MG: Postać skupiona na ochronie innych może czasami wystawić samą siebie do ataku.

\subsubsection{Chroni Wrota}\index{Specjalizacje!Lista!Chroni Wrota}

Każdy chce mieć Ciebie po swojej stronie w walce, ponieważ nic Cię nie omija.

Poziom 1: Ufortyfikowana Pozycja

Poziom 1: Do Mnie!

Poziom 2: Moc i Umysł

Poziom 3: Budujący Umocnienia lub Odbicie Ataków

Poziom 4: Większa Ulepszona Moc

Poziom 5: Pole Wzmacniające

Poziom 6: Generacja Pola Siłowego lub Atak Oszałamiający

Wtrącenia MG: Strategicznie ważna struktura się zapada. Wróg atakuje z niespodziewanej strony.

\subsubsection{Dotyka Nieba}\index{Specjalizacje!Lista!Dotyka Nieba}

Kontrolujesz pogodę.

Poziom 1: Unoszenie się

Poziom 2: Zbroja Wiatru

Poziom 3: Promienie Mocy lub Przywołanie Burzy

Poziom 4: Jeździec Wiatru

Poziom 5: Emisja Zimna

Poziom 6: Kontrola Pogody lub Rydwan Wiatru

Wtrącenia MG: Sojusznik jest przypadkowo trafiony przez błyskawicę. Niespodziewane uziemienie zadaje obrażenia. Pogoda jest zmieniona w niewłaściwy sposób i burza wyrywa się spod kontroli. 

\subsubsection{Działa pod Przykrywką}\index{Specjalizacje!Lista!Działa pod Przykrywką}

Pod przebraniem kogoś innego, szukasz odpowiedzi, które potężni tego świata pragną zachować dla siebie. 

(Ktoś kto Działa pod Przykrywką może mieć zestaw do przebierania się jako dodatkowy ekwipunek).

Poziom 1: Śledztwo

Poziom 2: Przebranie

Poziom 3: Agent-Prowokator lub Bieg i Walka

Poziom 4: Niezłe Oszustwo

Poziom 5: Korzystanie z Dostępnych Opcji

Poziom 6: Zaufaj Swemu Szcześciu lub Śmiertelny Cios

Wtrącenia MG: Pech może zepsuć najlepszy plan. Przebrenie zawodzi. Sprzymierzeńcy okazują się również być agentami.

\subsubsection{Dzierży Dwie Bronie Naraz}\index{Specjalizacje!Lista!Dzierży Dwie Bronie Naraz}

Dzierżysz stal w obydwu rękach, gotowy stanąć naprzeciwko każdego wroga. 

Poziom 1: Podwójne Władanie Lekkimi Broniami 

Poziom 2: Podwójny Cios

Poziom 2: Infiltrator

Poziom 3: Podwójne Władanie Średnią Bronią lub Precyzyjne Cięcie

Poziom 4: Podwójna Obrona

Poziom 5: Podwójne Rozproszenie Uwagi

Poziom 6: Rozbrojenie lub Wielokrotny Atak

Wtrącenia MG: Ostrze łamie się w połowie lub broń wypada z dłoni swego nosiciela. 

\subsubsection{Dzierży Magiczną Broń}\index{Specjalizacje!Lista!Dzierży Magiczną Broń}

Posiadasz broń o dziwnych właściwościach i Twoja wiedza o jej mocy pozwoliła Ci stworzyć unikalny styl walki.

Poziom 1: Zaczarowana Broń

Poziom 1: Wrodzona Moc

Poziom 1: Naładowanie Broni

Poziom 2: Uderzenie Mocy

Poziom 3: Szybki Atak lub Rzut Zaczarowaną Bronią

Poziom 4: Broń Defensywna

Poziom 5: Zaczarowany Ruch

Poziom 6: Smiertelny Cios lub Wielokrotny Atak

Wtrącenia MG: Broń się psuje lub zostaje upuszczona. Postać traci połąćzenie ze swoją bronią aż do czasu, gdy wykorzysta swoj akcję, by odnowić połączenie. Energia broni rozładowuje się w niespodziewany sposób. 

\subsubsection{Fruwa Szybciej Niż Pocisk}\index{Specjalizacje!Lista!Fruwa Szybciej Niż Pocisk}

Możesz latać i jesteś supersilny, ciężki w uszkodzeniu, a także szybki. Czy jest coś, czego nie możesz zrobić?

Poziom 1: Unoszenie się

Poziom 2: Większy Ulepszony Potencjał

Poziom 3: Ukryta Siła lub Rentgen w Oczach

Poziom 4: W Mgnieniu Oka

Poziom 4: Rozpęd

Poziom 5: Jeszcze Żywy

Poziom 6: palące światło lub Zignorowanie Przeszkody

Wtrącenia MG: Nemezis Cię odnajduje. Odnaleziono dziwny materiał, któy niweluje moce postacu. 

\subsubsection{Gra w Zbyt Wiele Gier}\index{Specjalizacje!Lista!Gra w Zbyt Wiele Gier}

Lekcje, refleks i strategie, których się nauczyłeś, grając w zbyt wiele gier, mają zastosowanie w prawdziwym życiu, gdzie ludzie, którzy nie grają dostatecznie dużo muszą się szczególnie męczyć. 

Poziom 1: Lekcje z Gier

Poziom 1: Gamer

Poziom 2: Oczy Przyzwyczajone do Ciemności

Poziom 2: Odporność na Sztuczki

Poziom 3: Cel Snipera lub Ulepszone Skupienie w Szybkości

Poziom 4: Gierki Umysłowe

Poziom 4: Ulepszony Intelekt

Poziom 5: Wytrzymałość Gracza

Poziom 6: Regeneracja Umysłu lub Bóg Gier

Wtrącenia MG: Chybiony atak trafie nie ten cel. Ekwipunek siępsuje. Czasami ludzie reaguję nagatywnie na kogoś, kto przeżył większość swego życia w wyimaginowanych światach gier.

\subsubsection{Grzmi}\index{Specjalizacje!Lista!Grzmi}

Emitujesz destruktywne dźwięki i manipulujesz nimi.

Poziom 1: Promień Grzmotu

Poziom 2: Bariera Konwersji Dźwięku

Poziom 3: Tłumienie Dźwięków lub Echolokacja

Poziom 4: Okrzyk Roztrzaskania

Poziom 5: Subsoniczny Hałas

Poziom 5: Wzmocnienie Dźwięku

Poziom 6: Trzęsienie Ziemi lub Śmiertelna Wibracja

Wtrącenia MG: Głośne hałasy przyciągają uwagę.

\subsubsection{Ignoruje Fizyczny Dystans}\index{Specjalizacje!Lista!Ignoruje Fizyczny Dystans}

Możesz się teleportować w jedno miejsce z drugiego poprzez krótki pobyt w równoległym wymiarze.

Poziom 1: Wymiarowy Ścisk

Poziom 2: Oportunista

Poziom 3: Obronna Teleportacja lub Skoki Teleportacyjne

Poziom 4: Krótka Teleportacja

Poziom 5: Średnia Teleportacja

Poziom 6: Teleportacja lub Rana Teleportacyjna

Wtrącenia MG: Teleportacja kończy się źle, umieszczając postać w niebezpiecznym miejscu. Bezwładność (np.: wskutek spadania) trwa podczas teleportacji, raniąc postać. 

\subsubsection{Infiltruje}\index{Specjalizacje!Lista!Infiltruje}

Subtelność, chytrość i ukradkowość pozwalają Ci na dostęp tam, gdzie inni nie mogą.

Poziom 1: Umiejętności Złodzieja

Poziom 1: Wyczucie Pobudek

Poziom 2: Impersonacja

Poziom 2: Ucieczka, nie Walka

Poziom 3: Świadomość lub Umiejętny Atak

Poziom 4: Niewidzialność

Poziom 5: Unik

Poziom 6: Pranie Mózgu lub Odsunięcie się

Wtrącenia MG: Szpiegów traktuje się okrutnie, gdy się ich złapie. Ich sprzymierzeńcy się ich wypierają. Pewnych sekretów lepiej nigdy nie poznać.

\subsubsection{Interpretuje Prawo}\index{Specjalizacje!Lista!Interpretuje Prawo}

Jest twoją rzeczą naginanie innych do swoich poglądów.

Poziom 1: Dyplomata

Poziom 1: Wiedza Prawnicza

Poziom 2: Debata

Poziom 3: Przydatna Pomoc lub Ulepszone Skupienie w Inteligencji

Poziom 4: Przerażenie

Poziom 5: Nikt nie Wie Lepiej

Poziom 6: Większy Ulepszony Potencjał lub Prawnik-Stażysta

Wtrącenia MG: Ludzie nie lubią wszystkowiedzących. Rozproszenie lub przeszkodzenie przeszkadza w argumencie prawniczym.

\subsubsection{Istnieje Częściowo Poza Fazą}\index{Specjalizacje!Lista!Istnieje Częściowo Poza Fazą}

Częściowo przezroczysty, jesteś w części poza fazą i możesz się przemieszczać przez ciała stałe. 

Poziom 1: Przechodzenie Przez Ściany

Poziom 2: Defensywne Znikanie

Poziom 3: Atak Fazowy lub Drzwi Fazowe

Poziom 4: Duch

Poziom 5: Nietykalny

Poziom 6: Ulepszony Atak Fazowy lub Wyfazowanie Wroga

Wtrącenia MG: Postać wyfazowuje się w nieznany wymiar. Postać gubi się w dużym ciele stałym. 

\subsubsection{Istnieje w Dwóch Miejscach Naraz}\index{Specjalizacje!Lista!Istnieje w Dwóch Miejscach Naraz}

Istniejesz w dwóch miejscach w tym samym czasie.

Poziom 1: Kopia

Poziom 2: Dzielone Zmysły

Poziom 3: Ulepszona Kopia lub Odporna Kopia

Poziom 4: Przekaz Obrażeń

Poziom 5: Skoordynowany Wysiłek

Poziom 6: Wielość lub Odporna Kopia

Wtrącenia MG: Obserwacja świata z dwóch odmiennych perspektyw dezorientuje postać, powodując zawroty głowy, wymioty lub konfuzję.

\subsubsection{Izoluje Umysł od Ciała}\index{Specjalizacje!Lista!Izoluje Umysł od Ciała}

Twój umysł opuszcza Twoje ciało, by widzieć odległe miejsca i poznawać sekrety, których nie da się poznać inaczej.

Poziom 1: Trzecie Oko

Poziom 2: Otwarty Umysł

Poziom 2: Wyostrzone Zmysły

Poziom 3: Zdalne Trzecie Oko lub Odnalezienie Ukrytych

Poziom 4: Sensor

Poziom 5: Psioniczny Pasażer

Poziom 6: Projekcja Mentalna lub Ulepszony Sensor

Wtrącenia MG: Ponowne połączenie ciała i umysłu może czasami być dezorientujące i wymagać od postaci spędzenia paru minut na dostrajaniu się. 

\subsubsection{Jaśnieje Światłością}\index{Specjalizacje!Lista!Jaśnieje Światłością}

Możesz tworzyć światło, kształtować je, naginać lub gromadzić jako broń. 

Poziom 1: Oświecony

Poziom 1: Dotyk Oświecenia

Poziom 2: Oszałamiające Światło

Poziom 3: Palące Światło lub Umiejętna Obrona

Poziom 4: Światło Słońca 

Poziom 5: Niewidzialność

Poziom 6: Żywe Światło lub Pole Obronne

Wtrącenia MG: Sprzymierzeńcy są przypadkowo oszołomieni lub oślepieni. Jasne błyski przywołują strażników. 

\subsubsection{Jest Bardzo Silny}\index{Specjalizacje!Lista!Jest Bardzo Silny}

Jesteś umięśniony, możesz podnosić wielkie ciężary i przebijać się przez drzwi.

Poziom 1: Atleta

Poziom 1: Ulepszone Skupienie w Mocy

Poziom 2: Pokaz Siły

Poziom 3: Żelazne Pięści lub Rzut

Poziom 4: Większa Ulepszona Moc

Poziom 5: Brutalne Uderzenie

Poziom 6: Większa Ulepszona Moc lub Atak z Wyskoku

Wtrącenia MG: Łatwo jest zniszczyć delikatne przedmioty lub kogoś przypadkowo zranić

\subsubsection{Jest Idolem Milionów}\index{Specjalizacje!Lista!Jest Idolem Milionów}

Jesteś celebrytą i większość ludzi Cię uwielbia.

Poziom 1: Świta

Poziom 1: Talent Celebryty

Poziom 2: Zalety Sławy

Poziom 3: Ulepszone Zdrowie lub Umiejętny Atak

Poziom 4: Zachwyt Światła Gwiazd

Poziom 4: Kompan-Ekspert

Poziom 5: Czy Ty Wiesz W Ogóle Kim Jestem?

Poziom 6: Oratorska Inspiracja lub Ulepszony Kompan

Wtrącenie GM: Fani są w niebezpieczeństwie lub odnieśli obrażenia ze względu na Ciebie. Ktośw Twojej świcie Cię zdradza. Twój show, tour, kontakt lub inne wydarzenie jest odwołane. Media publikują zdjęcia z Tobą we wstydliwych sytuacjach. 

\subsubsection{Jest Jasnowidzem}\index{Specjalizacje!Lista!Jest Jasnowidzem}

Posiadasz psioniczny dar, który  pozwala Ci widzieć to, czego inni nie mogą.

Poziom 1: Postrzeganie Niewidocznego

Poziom 2: Rentgen w Oczach

Poziom 3: Odnalezienie Ukrytych lub Sensor

Poziom 4: Widzenie na Odległość

Poziom 5: Postrzeganie Czasu

Poziom 6: Projekcja Mentalna lub Całkowita Świadomość

Wtrącenia MG: Pewne sekrety są zbyt okropne, by je poznać.

\subsubsection{Jest Jednoosobowym Bastionem}\index{Specjalizacje!Lista!Jest Jednoosobowym Bastionem}

Twoja zbroja, wraz z Twoim rozmiarem, siłą, treningiem lub wszczepami bionicznymi, czyni Ciętrudnym do przemieszczenia lub zaatakowania.

(Pewne postaci, które Są Jednoosobowym Bastionem, mogą już być ekspertami w pancerzach. Mogą one wybrać inną zdolność1-szego poziomu niż Wyszkolony w Zbroi)

Poziom 1: Wyszkolony w Zbroi

Poziom 1: Doświadczony Obrońca

Poziom 2: Odporność na Żywioły

Poziom 3: Nieporuszalny

Poziom 3: Większa Ulepszona Moc lub Wyszkolony we Wszystkich Broniach

Poziom 4: Żywa Ściana

Poziom 5: Wytrzymały

Poziom 5: Mistrzowska Biegłość w Pancerzach

Poziom 6: Śmiertelne Obrażenia lub Wyszkolony w Tarczach

Wtrącenia MG: Zbroja się uszkadza. Mali wrogowie atakują Cię w sprytny sposób. 

\subsubsection{Jest Mistrzem Obrony}\index{Specjalizacje!Lista!Jest Mistrzem Obrony}

Korzystasz z odpowiedniego ekwipunku i wyszkolenia, by uniknąć zranienia w walce.

Poziom 1: Mistrz Tarcz

Poziom 2: Twardy

Poziom 2: Wyszkolony w Zbroi

Poziom 3: Unik i Odporność lub Unik i Rewanż

Poziom 4: Wieża Siły Woli

Poziom 4: Przywykły do Noszeni Zbroi

Poziom 5: Nic Tylko Obrona

Poziom 6: Mistrz Obrony lub Jak Druga Skóra

Wtrącenia MG: Tarcza pęka przy trafieniu, jak i bronie, którymi się blokuje. Paski od pancerza pękają.

\subsubsection{Jest Poszukiwany Przez Prawo}\index{Specjalizacje!Lista!Jest Poszukiwany Przez Prawo}

Plakaty "POSZUKIWANY, ŻYWY LUB MARTWY" posiadają Twoje podobieństwo. To od Ciebie zależy, czy to koszmarna pomyłka, która się wymknęła spod kontroli, czy może potrafisz kogoś zabić, bo na Ciebie krzywo spojrzał. 

Poziom 1: Ulepszona Szybkość

Poziom 1: Zmysł Niebezpieczeństwa

Poziom 2: Atak z Zaskoczenia

Poziom 3: Reputacja Spoza Prawa lub Następny Atak

Poziom 4: Szybkie Zabójstwo

Poziom 5: Drużyna Desperados

Poziom 6: Jeszcze Żywy lub Śmiertelne Obrażenia

Wtrącenia MG: Większość ludzi nie reaguje dobrze na poszukiwanego listem gończym w swoich szeregach.

\subsubsection{Jest Stworzony z Kamienia}\index{Specjalizacje!Lista!Jest Stworzony z Kamienia}

Twoje ciało jest stworzone z twardego minerału, czyniąc się twardym, ciężkim do zranienia humanoidem.

Poziom 1: Ciało Golema

Poziom 1: Uzdrawianie Golema

Poziom 2: Chwyt Golema

Poziom 3: Wyszkolony Miażdżyciel

Poziom 3: Tupnięcie Golema lub Uzbrojenie

Poziom 4: Głębokie Rezerwy

Poziom 5: Wyspecjalizowany Pięściarz

Poziom 5: Jak Posąg

Poziom 6: Ultra Wzmocnienie lub Regeneracja Umysłu

Wtrącenia MG: Istoty z kamienia czasami zapominają o własnej sile lub wadze. Chodzący posąg może przerazić zwykłych ludzi.

\subsubsection{Jeździ Jak Maniak}\index{Specjalizacje!Lista!Jeździ Jak Maniak}

Niezależnie od tego, czy balansujesz na dwóch kołach, przeskakujesz między pojazdami lub ruszasz na przód ku niebezpiezeństwu, nie myślisz zbyt dużo o własnym bezpieczeństwie, gdy jesteś za kierownicą.  

(Ktoś to Jeździ Jak Maniak potrzebuje dostępu do pojazdu.)

Poziom 1: Kierowca

Poziom 1: Atak Podczas Kierowania

Poziom 2: Surfer Aut

Poziom 2: Pojedynek Spojrzeń

Poziom 3: Doświadczony Kierowca lub Ulepszone Skupienie w Szybkości

Poziom 4: Bystrooki

Poziom 4: Ulepszona Szybkość

Poziom 5: Coś na Drodze

Poziom 6: Uzdolniony Kierowca lub Śmiertelne Obrażenia

Wtrącenia MG: Silnik odmawia posłuszeństwa. Most na końcu drogi jest wyłączony z ruchu. Przednia szyba się roztrzaskuje. Ktoś nagle wyskakuje na przód pojazdu.

\subsubsection{Kocha Pustkę}\index{Specjalizacje!Lista!Kocha Pustkę}

Kiedy jesteś tylko Ty, Twój skafander kosmiczny i panorama niekończących się gwiazd, osiągasz stan spokoju.

Opcja do podmiany: Mam Kombinezon Kosmiczny, Będę Podróżnikiem 

Poziom 1: Umiejętności Kosmiczne

Poziom 1: Przyzwyczajony do Mikrograwitacji

Poziom 2: Ulepszone Skupienie w Szybkości

Poziom 2: Ulepszona Muskulatura

Poziom 3: Walka w Kosmosie lub Zbroja Fuzyjna

Poziom 4: Cichy jak Kosmos

Poziom 4: Odepchnięcie i Rzut

Poziom 5: Uniki w Mikrograwitacji

Poziom 6: Wystrzał Mikrograwitacyjny lub Pole Reakcyjne

Wtrącenia MG: Kombinezony kosmiczne mogą się zepsuć. Wskaźniki poziomu tlenu czasami mogą być mylące. Mikrometeoryty są powszechne w kosmosie.

\subsubsection{Kontroluje Bestie}\index{Specjalizacje!Lista!Kontroluje Bestie}

Masz rzadką zdolność komunikowania się i przewodzenia bestiom.

Poziom 1: Zwierzęcy Kompan

Poziom 2: Ukojenie Dzikiego

Poziom 2: Komunikacja

Poziom 3: Rumak lub Silniejsi Razem

Poziom 4: Oczy Bestii

Poziom 4: Ulepszony Kompan

Poziom 5: Zew Dziczy

Poziom 6: Jak Jedna Istota lub Kontrola Dzikiej Bestii

Wtrącenia MG: Społeczność jest niechętna dzikim zwierzęciom. Bestia wyrwane spod kontroli stają się prawdziwym niebezpieczeństwem. 

\subsubsection{Kontroluje Grawitację}\index{Specjalizacje!Lista!Kontroluje Grawitację}

Manipulujesz siłami grawitacyjnymi.

Opcja do podmiany: Ciężki

Poziom 1: Unoszenie się

Poziom 2: Ulepszone Skupienie w Szybkości

Poziom 3: Definiowanie Dołu lub Szarpnięcie Grawitacyjne

Poziom 4: Pole Grawitacyjne

Poziom 5: Lot

Poziom 6: Ulepszone Szarpnięcie Grawitacyjne lub Ciężar Świata

Wtrącenia MG: Świadkowie reagują nierozsądnym strachem. Dziwna interakcja posyła obiekt lub sprzymierzeńca ku przestworzom.

\subsubsection{Kopiuje Supermoce}\index{Specjalizacje!Lista!Kopiuje Supermoce}

Możesz kopiować umiejętności, zdolności i supermoce innych. 

Poziom 1: Skupienie na Umiejętności

Poziom 1: Skupienie na Umiejętności

Poziom 2: Skopiuj Moc

Poziom 3: Kradzież Mocy lub Dzikie Zdolności

Poziom 4: Ulepszone Skopiowanie Mocy

Poziom 5: Pamięć Mocy

Poziom 6: Cudowne Kopiowanie lub Wielość Kopii

Wtrącenia MG: Skopiowana moc przestaje nagle działać lub wymyka się z kontroli. Skopiowana moc nie posiada drugorzędnych mocy (np.: superszybkość bez odporności na pęd powietrza lub bycie odpornym na żar własnych ognistych pocisków). 

\subsubsection{Leci na Wspaniałych Skrzydłach}\index{Specjalizacje!Lista!Leci na Wspaniałych Skrzydłach}

Wielu superbohaterów może latać, niektórzy z nich nawet mają skrzydła. Możesz korzystać ze swoich skrzydeł do poruszania się, atakowania i obrony.

Poziom 1: Unoszenie się

Poziom 1: Krótki Lot

Poziom 2: Skrzydła-Bronie

Poziom 3: Akrobatyczny Atak lub or Latający Kompan

Poziom 4: Trudny do Trafienia

Poziom 5: Rozpęd

Poziom 6: Trudny Cel lub Mistrz Obrony

Wtrącenia MG: Skrzydło może być zranione lub nie mieć dość miejsca, przez co bohater upada. Latanie wysoko czyni postać wyraźnym celem dla niespodziewanego wroga. 

\subsubsection{Łamie Systemy}\index{Specjalizacje!Lista!Łamie Systemy}

Wykorzystujesz słabości sztucznych systemów, wliczając (ale nie ograniczając się) do programów komputerowych.

Poziom 1: Hakowanie Niemożliwości

Poziom 1: Programowanie

Poziom 2: Połączenia

Poziom 3: Sprawny Oszust lub Umiejętny Atak

Poziom 4: Skonfunduj Wroga

Poziom 5: Wsparcie Przyjaciela

Poziom 6: Przysługa lub Większy Ulepszony Potencjał

Wtrącenia MG: Twoje kontakty czasami mają swoje własne motywacje. Niekiedy urządzenia mają zabezpieczenia lub nawet pułapki.

\subsubsection{Łączy Ciało i Stal}\index{Specjalizacje!Lista!Łączy Ciało i Stal}

Twoje ciało jest częściowo maszyną.

Poziom 1: Ulepszone Ciało

Poziom 2: Interfejs

Poziom 3: Pakiet Sensoryczny lub Uzbrojenie

Poziom 4: Fuzja

Poziom 5: Głębokie Reserwy

Poziom 6: Regeneracja Umysłu lub Ultra Wzmocnienie

Wtrącenia MG: Ludzie w większości społeczności boją się kogoś, kto ma w sobie mechaniczne części.

\subsubsection{Łączy Umysł i Maszynę}\index{Specjalizacje!Lista!Łączy Umysł i Maszynę}

Elektroniczne implanty w Twoim mózgu czynią Cię supermyślicielem.

Poziom 1: Ulepszony Intelekt

Poziom 1: Umiejętności Wiedzy

Poziom 2: Kwerenda

Poziom 3: Procesor Akcji lub Telepatia Maszyn

Poziom 4: Większy Ulepszony Intelekt

Poziom 4: Umiejętności Wiedzy

Poziom 5: Wizja Przyszłości

Poziom 6: Ulepszenie Maszyny lub Regeneracja Umysłu

Wtrącenia MG: Maszyny się psują. Potężne maszyny myślące mogą przejąć kontrolę nad mniejszymi maszynami. Pewni ludzie nie ufają komuś, kto nie jest w pełni organiczny. 

\subsubsection{Ma Szlachetną Krew}\index{Specjalizacje!Lista!Ma Szlachetną Krew}

Dziedzic bogactwa i mocy, masz tytuł szlachecki i zdolności przyznae przez uprzywilejowane wychowanie. 

Opcja Zamiany Typu: Służba

Poziom 1: Przywileje Szlachty

Poziom 2: Wyszkolony Dyskutant

Poziom 3: Zaawansowany Rozkaz lub Odwaga Szlachcica

Poziom 4: Kompan-Ekspert

Poziom 5: Potwierdzenie Własnego Przywileju

Poziom 6: Przydatna Pomoc lub Umysł Lidera

Wtrącenia MG: Długi rodziny szlacheckiej są problemem bohatera. Dawno zagubione rodzeństwo chce się pozbyć swego rywala. Zabójca odnajduje postać. 

\subsubsection{Ma Tysiąc Twarzy}\index{Specjalizacje!Lista!Ma Tysiąc Twarzy}

Możesz zmienić swój wygląd, by wyglądać jak zupełnie inna osoba. 

Poziom 1: Morficzna Twarz

Poziom 1: Umiejętności Międzyludzkie

Poziom 2: Zmiana Ciała

Poziom 2: Ciało Bitewne

Poziom 3: Przebranie Innej Osoby lub Odporność

Poziom 4: Nieśmiertelny

Poziom 4: Przemyślenie Problemów

Poziom 5: Pamięć w Czyn

Poziom 6: Rozdzielenie Jaźni lub Odczytanie Myśli

Wtrącenia MG: Część przebrania zawodzi. BN myśli, że przebrana postać to ktoś, kogo zna bardzo dobrze.

\subsubsection{Mistrzowsko Posługuje się Bronią}\index{Specjalizacje!Lista!Mistrzowsko Posługuje się Bronią}

Jesteś mistrzem w używaniu pewnego rodzaju broni, czy to mieczy, biczy, noży, pistoletów, czy czegoś innego.

(Ktoś, kto Mistrzowsko Posługuje się Bronią, może mieć dodatkowy ekwipunek, wliczając broń wysokiej jakości.)

Poziom 1: Mistrz Broni

Poziom 1: Twórca Broni

Poziom 2: Obrona Bronią

Poziom 3: Szybki Atak lub Cios Rozbrajający

Poziom 4: Bez Wpadek

Poziom 5: Wyjątkowe Mistrzostwo

Poziom 6: Morderca lub Śmiertelny Cios

Wtrącenia MG: Bronie siępsują. Broniem ogą zostać ukradzione. Bronie można upuścić lub zostać rozbrojonym. 

\subsubsection{Morduje}\index{Specjalizacje!Lista!Morduje}

Jesteś asasynem, z profesji, chęci lub ponieważ na tym świecie mordujesz lub zostajesz zamordowany.

(Ktoś kto Morduje może mieć dodatkowy ekwipunek, wliczywszy 3 dawki trucizny 2 poziomu która zadaje 5 punktów obrażeń). 

Poziom 1: Atak z Zaskoczenia

Poziom 1: Umiejętności Zabójcy

Poziom 2: Szybka Śmierć

Poziom 2: Infiltrator

Poziom 3: Świadomość lub Warzenie Trucizn

Poziom 4: Lepszy Atak z Zaskoczenia

Poziom 5: Dodatkowe Obrażenia

Poziom 6: Plan Ucieczki lub Morderca

Wtrącenia MG: Większość ludzi nie reaguje dobrze na profesjonalnego zabójcę.

\subsubsection{Mówi Głosem Ziemi}\index{Specjalizacje!Lista!Mówi Głosem Ziemi}

Twoje duchowe połączenie z naturą i środowiskiem daje Ci mistyczne moce.

Poziom 1: Nasiona Furii

Poziom 1: Wiedza o Dziczy

Poziom 2: Chwytające Zielska

Poziom 3: Ukojenie Dzikiego lub Komunikacja

Poziom 4: Księżycowa Zmiana Kształtu

Poziom 5: Erupcja Insektów

Poziom 6: Wezwanie Burzy lub Trzęsienie Ziemi

Wtrącenia MG: Ranna naturalna (lecz niebezpieczna) istota jest odkryta. Ktoś poluje dla skór, zostawiając zwłoki, by gniły. Drzewo upada w lesie, jedno z ostatnich tak wielkich.

\subsubsection{Mówi do Duchów}\index{Specjalizacje!Lista!Mówi do Duchów}

Niespokojne dusze, duchy natury i żywiołaki wspomagają Cię.

(W pewnych settingach, Specjalizacja Mówi do Duchów dotyczy tylko jednego rodzaju duchów, takich jak duchy martwych, duchy natury itp.)

Poziom 1: Przepytanie Ducha

Poziom 2: Duch Kompan

Poziom 3: Rozkazywanie Duchom lub Wyczulone Zmysły

Poziom 4: Płaszcz Gniewu

Poziom 5: Wezwanie Ducha

Poziom 6: Wezwanie Międzywymiarowego Ducha lub Absorpcja Ducha

Wtrącenia MG: Niektórzy nie ufają tym, którzy się zadają z duchami. Martwi czasami wcale nie chcą rozmawiać. 

\subsubsection{Mówi do Maszyn}\index{Specjalizacje!Lista!Mówi do Maszyn}

Używasz swojego organicznego mózgu jak komputera, bezprzewodowo łącząć się z dowolnym urządzeniem elektronicznym. Możesz je kontrolować i wpływać na nie w sposób, w jaki inni nie mogą. 

Poziom 1: Umiłowanie do Maszyn

Poziom 1: Interfejs Zasięgowy

Poziom 2: Ulepszenie Maszyny

Poziom 2: Zauroczenie Maszyny

Poziom 3: Inteligentny Interfejs lub Rozkazywanie Maszynom

Poziom 4: Kompan-Maszyna

Poziom 4: Walczący z Robotami

Poziom 5: Zbieranie Informacji

Poziom 6: Kontrola Maszyny lub Ulepszony Kompan-Maszyna

Wtrącenia MG: Maszyna się psuje lub działa w nieprzewidziany sposób.

\subsubsection{Nie Potrzebuje Broni}\index{Specjalizacje!Lista!Nie Potrzebuje Broni}

Potężne ciosy, kopnięcia, zamachy łokciami i kolanami oraz ruchy całego ciała są wszystkimi broniami, których potrzebujesz. 

Poziom 1: Pięści Furii

Poziom 1: Ciało z Kamienia

Poziom 2: Atak z Rozbrojeniem

Poziom 2: Sztuki Walki

Poziom 3: Gibkość Niczym Woda lub Większy Ulepszony Potencjał

Poziom 4: Odbicie Ataków

Poziom 5: Atak Oszałamiający

Poziom 6: Mistrz Sztuk Walki lub Śmiertelne Obrażenia

Wtrącenia MG: Uderzanie pewnych wrogów boli Cię tak mocno, jak ich ranisz. Wrogowie z broniami mają większy zasięg. Skomplikowane ruchy sztuk walki mogą Cię wytrącić z równowagi.

\subsubsection{Nie Robi Zbyt Dużo}\index{Specjalizacje!Lista!Nie Robi Zbyt Dużo}

Jesteś obibokiem, ale wiesz coś o wielu rzeczach. 

Poziom 1: Lekcje Życiowe

Poziom 2: Wyluzowanie

Poziom 3: Umiejętny Atak lub Improwizacja

Poziom 4: Lekcje Życiowe

Poziom 4: Większa Umiejętność Obrony

Poziom 5: Większy Ulepszony Potencjał

Poziom 6: Korzystając z Doświadczenia Życiowego lub Szybki Umysł

Wtrącenia MG: Nowe sytuacje są konfundujące i stresujące. Przeszłe akcja (lub ich brak) wracają, by gnębić postać. 

\subsubsection{Nigdy się nie Poddaje}\index{Specjalizacje!Lista!Nigdy się nie Poddaje}

Nigdy się nie poddajesz, radzisz sobie z każdą raną, i zawsze jesteś gotowy na więcej.

Poziom 1: Ulepszone Odzyskanie Zdrowia

Poziom 1: Parcie Dalej

Poziom 2: Zignorowanie Bólu

Poziom 3: Gorączka Krwi lub Ukryta Siła

Poziom 4: Determinacja lub Wytrzymalszy Niż Wróg

Poziom 5: Jeszcze Żywy

Poziom 6: Ostateczne Zaprzeczenie lub Zignorowanie Przeszkody

Wtrącenia MG: Czasami, to ekwipunek i broń się poddają. 

\subsubsection{Nosi Egzotyczną Tarczę}\index{Specjalizacje!Lista!Nosi Egzotyczną Tarczę}

Posiadasz wspaniałą tarczę czystej mocy, która zapewnia obronę i pewne zdolności ataku.

Poziom 1: Tarcza Pola Siłowego

Poziom 1: Uderzenie Mocy

Poziom 2: Ulepszona Tarcza

Poziom 3: Leczący Puls lub Rzut Tarczą Siłową

Poziom 4: Zasilona Tarcza

Poziom 5: Ściana Mocy

Poziom 6: Skacząca Tarcza lub Wybuch Tarczy

Wtrącenia MG: Tarcza jest chwilowo nieaktywna. Wróg chwilowo przejmuje kontrolę nad tarczą.

\subsubsection{Nosi Halo Ognia}\index{Specjalizacje!Lista!Nosi Halo Ognia}

Możesz pokryć swe ciało płomieniami, co chroni Ciebie i rani Twoich wrogów. 

Poziom 1: Płaszcz Ognia

Poziom 2: Macki Płomieni

Poziom 3: Skrzydła Ognia lub Ognista Ręka Zguby

Poziom 4: Ostrze Ognia

Poziom 5: Ogniste Macki

Poziom 6: Ognisty Sługa lub Piekielny Szlak

Wtrącenia MG: Ogień pali łatwopalne materiały. Płomienie wyzwalają się spod kontroli. Prymitywne istoty boją się ognia i często atakują źródło swoich lęków. 

\subsubsection{Nosi Zasilany Pancerz}\index{Specjalizacje!Lista!Nosi Zasilany Pancerz}

Poziom 1: Zasilany Pancerz

Poziom 1: Ulepszona Moc

Poziom 2: Wyświetlacz w Hełmie

Poziom 3: Zbroja Fuzyjna lub Ulepszone Zdrowie

Poziom 4: Wystrzał Mocy

Poziom 5: Pole Mocy Zasilanego Pancerza

Poziom 6: Mistrzowska Modyfikacja Pancerza (Krótki Lot) lub Mistrzowska Modyfikacja Pancerza (Pojemnik na Cypher)

Wtrącenia MG: Nie możesz zdjąć pancerza. Pancerz działa samodzielnie. Pancerz chwilowo się wyłącza. BN-i boją się pancerza. 

\subsubsection{Oblicza Nieobliczalne}\index{Specjalizacje!Lista!Oblicza Nieobliczalne}

Nadludzkie zdolności matematyczne pozwalająci Ci na modelowanie świata na bieżąco, dając Ci przewagę nad innymi. 

Poziom 1: Prorocze Równanie

Poziom 1: Wyższa Matematyka

Poziom 2: Proroczy Model

Poziom 3: Podświadoma Obrona lub Ulepszony Intelekt

Poziom 4: Obliczenia Bitewne

Poziom 5: Większy Ulepszony Intelekt

Poziom 5: Najwyższa Matematyka

Poziom 6: Wiedza o Nieznanym lub Większy Ulepszony Intelekt 

Wtrącenia MG: Zbyt wiele przewidzianych wyników przeraża lub przeciąża i oszałamia postać. Wynik wskazuje na nadchodzącą klęskę. 

\subsubsection{Otrzymuje Boskie Błogosławieństwa}\index{Specjalizacje!Lista!Otrzymuje Boskie Błogosławieństwa}

Jako oddany wyznawca boskiej istoty, posiadasz pewne moce swego bóstwa, by czynić cuda. 

Poziom 1: Błogosławieństwo Bóstw

Poziom 2: Ulepszony Intelekt

Poziom 3: Boski Blask lub Kwiat Ognia

Poziom 4: Niebiańska Gloria

Poziom 5: Boska Interwencja

Poziom 6: Boski Symbol lub Przywołanie Demona

Wtrącenia MG: Demon bada użytkowników boskiej magii. Rywalizujący kult ma problemy z naukami postaci.

\subsubsection{Pilotuje Statki Kosmiczne}\index{Specjalizacje!Lista!Pilotuje Statki Kosmiczne}

Jesteś pilotem statku kosmicznego

Poziom 1: Pilotaż

Poziom 1: Planetarna Wiedza

Poziom 2: Kryjówka w Kosmosie

Poziom 2: Umysłowa Odporność

Poziom 3: Expert-Pilot

Poziom 3: Obeznanie ze Statkiem Kosmicznym lub Kompan-Maszyna

Poziom 4: Sensory Statku Kosmicznego

Poziom 4: Ulepszona Szybkość

Poziom 5: Znam Ten Statek Jak Własną Dłoń)

Poziom 6: Wspaniały Pilot

Poziom 6: Kontrola Zdalna lub Umiejętny Atak

Wtrącenia MG: Statek się gubi, psuje, lub zostaje zaatakowany w kosmosie. Dokonujesz odkrycia obcego pasażera na gapę. 

\subsubsection{Podróżuje przez Czas}\index{Specjalizacje!Lista!Podróżuje przez Czas}

Widzisz poprzez strumienie czasu, próbujesz w nie sięgnąć i w końcu nawet przez nie podróżować. 

(Choć wszystkie wybory postaci są zależne od zgody MG, Podróżuje przez Czas jest specjalizacją, odnośnie której MG i gracz powinni odbyć długą konwersację, by gracz znał zasady gry odnośnie podróży w czasie, jeśli istnieją w settingu MG. Postać z tą 
specjalizacją może znacząco zmienić świat gry, jeśli zasady gry na to pozwalają.) 

Poziom 1: Przebłysk

Poziom 2: Historia Przedmiotu

Poziom 3: Przyspieszenie Czasoprzestrzenne lub Pętla Czasu

Poziom 4: Czasoprzestrzenne Przesunięcie

Poziom 5: Sobowtór Czasoprzestrzenny

Poziom 6: Wezwanie przez Czas lub Podróż w Czasie

Wtrącenia MG: Powstają paradoksy. Inni pamiętają przeszłe wydarzenia inaczej.

\subsubsection{Poluje}\index{Specjalizacje!Lista!Poluje}

Jesteś wytrwałym łowcą, który potrafi upolować to, co zechce.

Poziom 1: Estetyczny Atak

Poziom 1: Łowczy

Poziom 2: Cel

Poziom 2: Skradanie się

Poziom 3: Walczący z Hordą lub Bieg i Chwyt

Poziom 4: Atak z Zaskoczena

Poziom 5: Dążenie Łowcy

Poziom 6: Większa Umiejętność Ataku lub Wiele Celów

Wtrącenia MG: Ofiara dostrzega postać. Cel nie jest taki słaby, jak się wydawał.

\subsubsection{Pomaga Swoim Przyjaciołom}\index{Specjalizacje!Lista!Pomaga Swoim Przyjaciołom}

Kochasz swoich przyjaciół i pomagasz im w każdej trudności, niezależnie od wszystkiego.

Opcja do podmiany: Porada od Przyjaciela

Poziom 1: Pomoc Przyjaciela

Poziom 1: Odwaga

Poziom 2:  Ochrona Przed Zmiennym Losem

Poziom 3: Koka lub Umiejętny Atak

Poziom 4: Obrońca Przyjaciół

Poziom 4: Ulepszona Muskulatura

Poziom 5: Zainspirowanie Akcji

Poziom 6: Głębokie Przemyślenia lub Umiejętna Obrona

Wtrącenia MG: Inni czasami mają niecne motywacje. Służby porządkowe się Tobą interesują. Nawet, gdy wszystko idzie dobrze, będzie to miało swoje konsekwencje. 

\subsubsection{Porusza się jak Kot}\index{Specjalizacje!Lista!Porusza się jak Kot}

Lekki, zwinny i pełen gracji, poruszasz się szybko i łatwo, co pozwala Ci unikać niebezpieczeństw.

Poziom 1: Większa Ulepszona Szybkość

Poziom 1: Balansowanie

Poziom 2: Umiejętności Ruchu

Poziom 2: Bezpieczny Upadek

Poziom 3: Trudny to Trafienia

Poziom 3: Ulepszone Skupienie w Szybkości lub  Większa Ulepszona Szybkość

Poziom 4: Szybki Cios

Poziom 5: Śliski

Poziom 6: Perfekcyjna Szybkość or  Większa Ulepszona Szybkość

Wtrącenia MG: Nawet kot może być niezdarny. Skok nie jest tak łatwy jak wygląda. Ucieczka jest na tyle przesadzona, że umieszcza postać w bardzo niebezpiecznym położeniu.

\subsubsection{Porusza się jak Wiatr}\index{Specjalizacje!Lista!Porusza się jak Wiatr}

Możesz się poruszać tak szybko, że rozmywasz się w oczach.

Poziom 1: Większa Ulepszona Szybkość

Poziom 1: Szybkostopy

Poziom 2: Trudny do Trafienia

Poziom 3: Perfekcyjna Szybkość lub  Większa Ulepszona Szybkość

Poziom 4: W Mgnieniu Oka

Poziom 5: Rozmazany

Poziom 6: Perfekcyjna Szybkość or Niemożliwa Szybkość

Wtrącenia MG: Powierzchnie mogą być śliskie lub oferować ukryte przeszkody. Ruch innych istot może być trudny do przewidzenia, i postać może w nie wbiec.

\subsubsection{Posiada Licencję na Broń}\index{Specjalizacje!Lista!Posiada Licencję na Broń}

Posiadasz pistolet i wiesz, jak z niego skorzystać w walce. 

(Choć Posiada Licencjęna Broń zaprojektowaniu z myślą o współczesnej broni, może także dotyczyć futurystycznych blasterów lub innych broni dystansowych.)

Poziom 1: Rewolwerowiec

Poziom 1: Wyszkolony w Broni Palnej

Poziom 2: Ostrożny Strzał

Poziom 3: Wyszkolony Rewolwerowiec lub Dodatkowe Obrażenia

Poziom 4: Podwójny Wystrzał

Poziom 5: Potrójny Wystrzał

Poziom 6: Specjalny Strzał lub Śmiertelne Obrażenia

Wtrącenia MG: Chybiony strzał lub zacięcie się broni! Atak nie odnosi stutku i akcja jest stracona, plus potrzeba dodatkowej akcji, by zająć się problemem. 

\subsubsection{Posiada Magicznego Sprzymierzeńca}\index{Specjalizacje!Lista!Posiada Magicznego Sprzymierzeńca}

Sprzymierzona magiczna istota, przywiązana do przedmiotu (np.: pomniejszy dżin w lampie lub duch w fajce) to Twój przyjaciel, obrońca i broń.

Poziom 1: Związana Magiczna Istota

Poziom 2: Więź z Obiektem

Poziom 2: Szuflada-Skrytka

Poziom 3: Mniejsze Życzenie lub Rumak

Poziom 4: Ulepszona Więź z Przedmiotem

Poziom 5: Średnie Życzenie

Poziom 6: Mistrzostwo Więzi z Obiektem lub Zaufaj Swemu Szczęściu

Wtrącenia MG: Istota nagle znika w swoim przedmiocie. Związany obiekt zostaje uszkodzony. Istota nie zgadza się i nie czyni tego, o co się nią prosi. Istota twierdzi, że odchodzi, jeśli nie wykona się dla niej pewnego zadania.

\subsubsection{Pracuje w Ciemnych Uliczkach}\index{Specjalizacje!Lista!Pracuje w Ciemnych Uliczkach}

Działasz niedostrzeżenie, kradnąc od bogatych, by osiągnąć swoje cele.

Poziom 1: Umiejętności Złodzieja

Poziom 2: Kontakty w Półświatku

Poziom 3: Niezłe Oszustwo lub Trening Gildii

Poziom 4: Złodziejski Mistrz

Poziom 5: Nieczyste Zagrania

Poziom 6: Szczur Miejski lub Wysokie Skupienie

Wtrącenia MG: Złodzieje lądują w więzieniu. Dorabiasz się potężnych wrogów.

\subsubsection{Pracuje, by Żyć}\index{Specjalizacje!Lista!Pracuje, by Żyć}

Cieszysz się, gdy możesz dobrze wykonać swoją pracę, niezależnie, czy to programowanie, budowanie domów, czy górnictwo asteroidów.

Poziom 1: Zręczny Rzemieślnik

Poziom 2: Mięśnie z Żelaza

Poziom 3: Oko do Szczegółów lub Improwizacja

Poziom 4: Ulepszona Moc

Poziom 4: Stwardniały

Poziom 5: Umiejętność Eksperta

Poziom 6: Większy Ulepszony Potencjał lub  Ciężko Zapracowana Odporność

Wtrącenia MG: Naprawy czasami zawodzą. Kable mogą być trudne do odkodowania i ciągle bbyć pod napięciem. Czasami ludzie są niegrzeczni dla tych, co pracują, by żyć. 

\subsubsection{Przebudza Sny}\index{Specjalizacje!Lista!Przebudza Sny}

Możesz wyciągnąć obrazy ze snów i umieścić je w świecie jawy.

Poziom 1: Iluzja Snów

Poziom 1: Oneiro-alchemia

Poziom 2: Złodziej Snów

Poziom 3: Sen Staje się Prawdą lub Ulepszony Intelekt

Poziom 4: Sen na Jawie

Poziom 5: Koszmar

Poziom 6: Komnata Snów lub Pole Reakcyjne

Wtrącenia MG: Niespodziewany epizod lunatykowania stawia postać w niebezpiecznej sytuacji. Koszmar wyzwala się ze snu.

\subsubsection{Przeszukuje Ruiny}\index{Specjalizacje!Lista!Przeszukuje Ruiny}

Kiedy nie biegniesz lub się chowasz, przeszukujesz ruiny cywilizacji w celu znalezienia użytecznych pozostałości, co pozwala Ci przetrwać. 

Poziom 1: Ocalały z Apokalipsy

Poziom 1: Wiedza o Ruinach

Poziom 2: Rzemieślnik Rupieci

Poziom 3: Korzystanie z Okazji lub Ulepszone Zdrowie

Poziom 4: Wiesz, Gdzie Szukać

Poziom 5: Cyphery z Odzysku

Poziom 6: Artefakty z Odzysku lub Pole Reakcyjne

Wtrącenia MG: Przedmiot stworzony z zrecyklingowanych śmieci psuje się. Ktoś pojawia się i twierdzi, że użyteczne pozostałości nalezą do niego. Zrecyklingowany cypher eksploduje. 

\subsubsection{Przewodzi}\index{Specjalizacje!Lista!Przewodzi}

Twoje naturalne zdolności przywódcze pozwalają Ci na wydawanie rozkazów, wliczając w to Twoich oddanych podległych.

Poziom 1: Naturalna Charyzma

Poziom 1: Dobra Porada

Poziom 2: Potencjał

Poziom 2:  Podstawowy Kompan

Poziom 3: Zaawansowany Rozkaz or Kompan-Ekspert

Poziom 4: Zachwyt lub Inspiracja

Poziom 5: Większy Ulepszony Potencjał

Poziom 6: Drużyna Kompanów lub Umysł Lidera

Wtrącenia MG: Kompani odnoszą klęskę, zdradzają Cię, okłamują, przechodzą na złą stronę, dają się porwać lub umierają.

\subsubsection{Przyjmuje Zwierzęcy Kształt}\index{Specjalizacje!Lista!Przyjmuje Zwierzęcy Kształt}

Możesz się zmienić w zwierzę.

Poziom 1: Zwierzęcy Kształt

Poziom 2: Komunikacja

Poziom 2: Ukojenie Dzikiego

Poziom 3: Większy Zwierzęcy Kształt lub Większa Likantropia

Poziom 4: Zwierzęce Szpiegowanie

Poziom 5: Trudny do Zamordowania

Poziom 6: Rozmazana Prędkość lub  Rozszerzony Zwierzęcy Kształt

Wtrącenia MG: Postać niespodziewanie zmienia kształt. BN jest przerażony lub agresywny w stosunku do zmiennokształtnego. Transformacja zajmuje dłużej, niż się spodziewano.

Większa Likantropia stosuje się do używania Zwierzęcego Kształtu.

\subsubsection{Przywdziewa Połyskliwy Lód}\index{Specjalizacje!Lista!Przywdziewa Połyskliwy Lód}

Rozkazujesz zimowej mocy zimna i lodu.

Poziom 1: Lodowa Zbroja

Poziom 2: Lodowy Dotyk

Poziom 3: Mrozący Dotyk lub Lodowa Kreacja

Poziom 4: Twarda Lodowa Zbroja

Poziom 5: Emisja Zimna

Poziom 6: Śniegowa Zamieć lub Lodowe Rękawice

Wtrącenia MG: Lód czyni powierzchnie śliskimi. Ektremalne zimno sprawia, że przedmioty pękają i się psują. 

\subsubsection{Pływał z Piratami}\index{Specjalizacje!Lista!Pływał z Piratami}

Pływałeś razem ze straszliwymi piratami, ale zdecydowałeś się zerwać z piractwem i poświęcićsię innemu celowi. Powstaje pytanie: czy twoja przeszłość pozwoli o sobie zapomnieć?

Poziom 1: Zignorowanie Bólu

Poziom 1: Marynarz

Poziom 2: Korzystanie z Okazji

Poziom 2: Przerażająca Reputacja

Poziom 3: Umiejętny Atak lub Umiejętna Obrona

Poziom 4: Morskie Nogi

Poziom 4: Umiejętności Ruchu

Poziom 5: Zagubieni w Chaosie

Poziom 6: Pojedynek na Śmierć i Życie lub Następny Atak

Wtrącenia MG: Istnieje wiele niebezpieczeństw Siedmiu Mórz, wliczając sztormy i zarazy. Inni piraci czasem awansują poprzez zdradę. Pirat wyśledził dawnych kompanów, by odkryć ukryty skarb.

\subsubsection{Rozciąga się}\index{Specjalizacje!Lista!Rozciąga się}

Twoje ciało jest gumowe i elastyczne, zdolne rozciągać się na wielkie długości i kompresować z powrotem.

Poziom 1: Człowiek-Guma

Poziom 1: Daleki Krok

Poziom 2: Elastyczny Chwyt

Poziom 2: Bezpieczny Upadek

Poziom 3: Przeniknięcie Przez Barierę lub  Przekierowanie Ataku

Poziom 4: Odporność

Poziom 5: Mistrz Ruchu

Poziom 6: Ruch i Multiatak lub Jeszcze Żywy

Wtrącenia MG: Atak lub efekt wchodzi w interakcję z elastycznością BG. Rozciągnięta kończyna staje się przeciążona i słaba. 

\subsubsection{Rozdziera Ściany Świata}\index{Specjalizacje!Lista!Rozdziera Ściany Świata}

Szybkość i fazowanie daje Ci unikalną zdolność unikania zagrożeń i zadawania obrażeń jednocześnie. 

Poziom 1: Bieg Fazowy

Poziom 1: Przeszkadzający Dotyk

Poziom 2: Fazowe Zadrapanie

Poziom 3: Niewidzialne Fazowanie lub Przechodzenie Przez Ściany

Poziom 4: Detonacja Fazowa

Poziom 5: Bardzo Długi Bieg Fazowy

Poziom 6: Potężniejsze Fazowanie lub Nietykalny Podczas Ruchu

Wtrącenia MG: Poruszanie się tak szybko czasami prowadzi wprost na niespodziewane, egzotyczne przeszkody.

\subsubsection{Rozwiązuje Tajemnice}\index{Specjalizacje!Lista!Rozwiązuje Tajemnice}

Jesteś mistrzem dedukcji, używającym dowodów, by odnaleźć odpowiedzi.

Poziom 1: Śledczy

Poziom 1: Detektyw

Poziom 2: Z Dala od Niebezpieczeństwa

Poziom 3: Dobrze Wykształcony lub Umiejętny Atak

Poziom 4: Wyciągnięcie Wniosków

Poziom 5: Ogarnięcie Sytuacji

Poziom 6: Przejęcie Inicjatywy lub Większa Umiejętność Obrony

Wtrącenia MG: Dowody znikają, fałszywe tropy konfundują, a świadkowie kłamią. Początkowe wnioski mogą być błędne.

\subsubsection{Rośnie do Gigantycznych Rozmiarów}\index{Specjalizacje!Lista!Rośnie do Gigantycznych Rozmiarów}

Na krótkie okresy, rośniesz większy i, z odpowiednim doświadczeniem, do prawdziwie gigantycznych rozmiarów.

Poziom 1: Wzrost

Poziom 1: Przeogromny

Poziom 2: Większy

Poziom 2:  Zalety Bycia Dużym

Poziom 3: Wielki lub Rzut

Poziom 4: Chwyt

Poziom 5: Wielgachny

Poziom 6: Kolos lub Śmiertelne Obrażenia

Wtrącenia MG: Nagły wzrost przewraca meble lub sprawia, że przebijasz sufit. Powiększona postać przebija podłogę. 

\subsubsection{Rzeźbi Twardym Światłem}\index{Specjalizacje!Lista!Rzeźbi Twardym Światłem}

Tworzysz fizyczne przedmioty z twardego światła, które możesz wykorzystać do obrony lub ataku.

Poziom 1: Automatyczny Blask

Poziom 1: Chwilowe Światło

Poziom 2: Macki Mocy

Poziom 3: Twardsze Światło lub Rzeźbienie Światłem

Poziom 4: Większy Ulepszony Intelekt

Poziom 5: Ulepszone Rzeźbienie Światłem

Poziom 6: Pole Obronne lub Lot

Wtrącenia MG: Przedmiot z twardego światła przedwcześnie znika. Przedmiot z twardego światła nie może wpłynąć na daną istotę lub kolor.

\subsubsection{Rzuca ze Śmiertelną Dokładnością}\index{Specjalizacje!Lista!Rzuca ze Śmiertelną Dokładnością}

Wszystko co opuszcza Twoją dłoń idzie dokładnie gdzie sobie tego życzysz, z prędkością i w miejsce, gdzie osiągnie najlepszy efekt.

Poziom 1: Precyzja

Poziom 2: Ostrożny Rzut

Poziom 3: Szybki Rzut lub Umiejętna Obrona

Poziom 4: Wszystko Jest Bronią

Poziom 4: Wyspecjalizowany w Rzucaniu

Poziom 5: Wir Rzutek

Poziom 6: Śmiertelne Obrażenia lub Mistrzostwo Obrony

Wtrącenia MG: Chybione ataki trafiają w zły cel. Rykoszety mogą być niebezpieczne. Improwizowane bronie się psują. 

\subsubsection{Spaceruje w Dzikich Lasach}\index{Specjalizacje!Lista!Spaceruje w Dzikich Lasach}

Posługujesz się magią natury, która czerpie z potęgi drzew. 

Poziom 1: Życie w Dziczy

Poziom 1: Dodatkowy Odpoczynek

Poziom 2: Ciało z Drewna

Poziom 3: Drzewny Kompan lub Dzika Świadomość

Poziom 4: Podróż Przez Drzewa

Poziom 5: Wielkie Drzewo

Poziom 6: Straszny Las lub Odżywczy Wykwit

Wtrącenia MG: Drewniana postać zapala się. Dziki zamach konarem drzewa uderza w sprzymierzeńca. Pewne drzewa mają mroczne serca i nienawidzą wszystkich ludzi.

\subsubsection{Stawia Umysł Ponad Materią}\index{Specjalizacje!Lista!Stawia Umysł Ponad Materią}

Możesz poruszać telekinetycznie przedmioty bez fizycznego dotykania ich.

Poziom 1: Odbicie Ataków

Poziom 2: Telekineza

Poziom 3: Chmura Ochronna lub Ulepszenie Siły

Poziom 4: Aportacja

Poziom 5: Atak Psychikinetyczny

Poziom 6: Ulepszona Aportacja lub Przebudowa

Wtrącenia MG: Jeden mentalny błąd, a poruszanie obiekty upadają, a kruche obiekty niszczeją. Czasami zły przedmiot się porusza, upada lub niszczeje.

\subsubsection{Szuka Kłopotów}\index{Specjalizacje!Lista!Szuka Kłopotów}

Jesteś niebezpieczny i lubisz dobrą walkę.

Poziom 1: Pięści Furii

Poziom 1: Opatrywanie Ran

Poziom 2: Obrońca

Poziom 2: Bezpośredni

Poziom 3: Umiejętny Atak lub Większy Ulepsozny Potencjał

Poziom 4: Pozbawienie Przytomności

Poziom 5: Mistrzostwo Ataków

Poziom 6: Większa Ulepszona Moc lub Śmiertelne Obrażenia

Wtrącenia MG: Bronie psują się lub wypadają nawet z najsilniejszego uchwytu. Atakujący mogą się potknąć i upaść. Nawet pole bitwy może działać przeciwko Tobie, gdy przedmioty upadają.

\subsubsection{Szybko się Uczy}\index{Specjalizacje!Lista!Szybko się Uczy}

Radzisz sobie z trudnymi sytuacjami w miarę, jak się pojawiają, ucząc się czegoś nowego za każdym razem.

Poziom 1: Ulepszony Intelekt

Poziom 1: Oto Twój Problem

Poziom 2: Szybka Nauka

Poziom 3: Ciężki do Rozproszenia

Poziom 3: Ulepszone Skupienie w Inteligencji lub Skupienie na Umiejętności

Poziom 4: Dzielenie się Wiedzą

Poziom 5: Ulepszony Intelekt

Poziom 5: Parę Sztuczek w Zanadrzu

Poziom 6: Dwie Sprawy na Raz lub Umiejętna Obrona

Wtrącenia MG: Wypadki i pomyłki są świetnymi nauczycielami.

\subsubsection{Tańczy z Czarną Materią}\index{Specjalizacje!Lista!Tańczy z Czarną Materią}

Możesz manipulować ciemnością i "ciemną materią".

Poziom 1:  Wstęgi Mrocznej Materii

Poziom 2: Skrzydła Pustki

Poziom 3: Płaszcz Ciemnej Materii lub Cios Ciemnej Materii

Poziom 4: Powłoka Ciemnej Materii

Poziom 5: Podróżnik Niszczącego Wiatru

Poziom 6: Budowla Ciemnej Materii lub W Objęciach Nocy

Wtrącenia MG: Czarna Materia wycofuje się, zupełnie, jakby miała swoją własną wolę. 

\subsubsection{Tworzy Dziwną Naukę}\index{Specjalizacje!Lista!Tworzy Dziwną Naukę}

Twoja nadnaturalne wejrzenie w rzeczywistość czyni z Ciebie naukowca zdolnego do wielu rzeczy.

Poziom 1: Analiza Laboratoryjna

Poziom 1: Umiejętności Wiedzy

Poziom 2: Modyfikacja Urządzenia

Poziom 3: Lepsze Życie Dzięki Chemii lub Ulepszone Zdrowie

Poziom 4: Umiejętności Wiedzy

Poziom 4: Troszkę Szalony

Poziom 5: Przełom w Badaniach Dziwnej Nauki

Poziom 6: Niemożliwe Osiągnięcie Naukowe

Poziom 6: Wynalazca lub Pole Obronne

Wtrącenia MG: Twoje twory mogą się wymknąć spod kontroli. Czasami nie można przewidzieć efektów ubocznych. Dziwna nauka przeraża ludzi i przyciąga uwagę mediów. Kiedy przedmiot stworzony lub zmodyfikowany przez dziwną naukę się 
rozładowuje, wybucha. 

\subsubsection{Tworzy Iluzje}\index{Specjalizacje!Lista!Tworzy Iluzje}

Tworzysz obrazy ze światła tak perfekcyjne, że wydają się być realne. 

Poziom 1: Mniejsza Iluzja

Poziom 2:  Iluzoryczne Przebranie

Poziom 3: Rzuć Iluzję lub Większa Iluzja

Poziom 4: Iluzyjne Ja

Poziom 5: Przerażający Obraz

Poziom 6: Wielka Iluzja lub Permanentna Iluzja

Wtrącenia MG: Ciężko uwierzyć w iluzję. Iluzja zostaje przejrzana w najgorszym możliwym momencie.

\subsubsection{Tworzy Unikalne Obiekty}\index{Specjalizacje!Lista!Tworzy Unikalne Obiekty}

Jesteś wynalazcą dziwnych i użytecznych przedmiotów. 

Poziom 1: Rzemieślnik

Poziom 1: Mistrz Identyfikacji

Poziom 2: Mechanik Artefaktów

Poziom 2: Szybka Robota

Poziom 3: Mistrzowski Rzemieślnik lub Wbudowane Bronie

Poziom 4: Twórca Cypherów

Poziom 5: Innowator

Poziom 6: Wynalazca lub Zbroja Fuzyjna

Wtrącenia MG: Przedmiot zalicza awarię, niszczeje lub kończy swój byt w katastrofalny lub niespodziewany sposób.

\subsubsection{Ucieka Precz}\index{Specjalizacje!Lista!Ucieka Precz}

Twoim pierwszym instynktem jest ucieczka od niebezpieczeństwa, i jesteś w tym bardzo dobry. 

Poziom 1: W Defensywie

Poziom 2: Ulepszona Szybkość

Poziom 2: Szybka Ucieczka

Poziom 3: Niemożliwa Szybkość lub Większa Ulepszona Szybkość

Poziom 4: Determinacja

Poziom 4: Szybki Umysł

Poziom 5: Ponowne Ukrycie się

Poziom 6: Ucieczka lub Umiejętna Obrona

Wtrącenia MG: Szybkie ruchy czasami sprawiają, że upuszczasz przedmioty, poślizgujesz się lub przypadkowo kierujesz się w złą stronę

\subsubsection{Ujeżdża Błyskawicę}\index{Specjalizacje!Lista!Ujeżdża Błyskawicę}

Generujesz i wyzwalasz energię elektryczną.

Poziom 1: Szok

Poziom 1: Naładowanie

Poziom 2: Jeździec Błyskawicy

Poziom 3: Elektryczny Pancerz lub Wyssanie Ładunku

Poziom 4: Promienie Mocy

Poziom 5: Elektryczny Lot

Poziom 6: Szybkość Błyskawicy lub Ściana Błyskawic

Wtrącenia MG: Przypadkowi ludzie zostają zaatakowani prądem. Obiekty eksplodują. 

\subsubsection{Uzdrawia}\index{Specjalizacje!Lista!Uzdrawia}

Możesz leczyć innych dotykiem, wpływać na czas, by pomagać innym, i ogólnie jesteś kochany przez wszystkich.

Poziom 1: Leczący Dotyk

Poziom 2: Uzdrowienie

Poziom 3: Uzdrawiająca Fontanna lub Cudowne Zdrowie

Poziom 4: Zainspirowanie Akcji

Poziom 5: Cofnij

Poziom 6: Większy Leczący Dotyk lub Przywrócenie Życia

Wtrącenia MG: Próby uzdrowienia mogą zamiast tego spowodować krzywdę. Społeczność lub jednostka potrzebują uzdrowiciela tam bardzo, że przetrzymują go wbrew jego woli.

\subsubsection{Walczy Nieczysto}\index{Specjalizacje!Lista!Walczy Nieczysto}

Zrobisz wszystko, by wygrać walkę: będziesz gryzł, drapał, kopał, oszukiwał i czynił jeszcze gorsze rzeczy.

Poziom 1: Łowczy

Poziom 1: Stalker

Poziom 2: Skradanie się

Poziom 2: Cel

Poziom 3: Zdrada lub Atak z Zaskoczenia

Poziom 4: Gierki Umysłowe

Poziom 4: Zręczny Wojownik

Poziom 5: Korzyści z Otoczenia

Poziom 6: Obrócenie Noża lub Morderca

Wtrącenia MG: Ludzie nie cenią tych, którzy oszukują lub walczą bez honoru. Czasami brudna sztuczka uderza w Ciebie z powrotem. 

\subsubsection{Walczy z Robotami}\index{Specjalizacje!Lista!Walczy z Robotami}

Łatwo przychodzi Ci walka z robotami, automatonami i maszynami.

Poziom 1: Słabe Punkty Maszyn

Poziom 1: Umiejętności Technologiczne

Poziom 2: Ochrona Przed Robotami

Poziom 2: Polowanie na Maszyny

Poziom 3: Rozbrojeni Urządzenia lub Atak z Zaskoczenia

Poziom 4: Walczący z Robotami

Poziom 5: Wyssanie Mocy

Poziom 6: Deaktywacja Mechanizmu lub Śmiertelne Obrażenia

Wtrącenia MG: Robot eksploduje po pokonaniu. Inne roboty szukajązemsty na postaci. 

\subsubsection{Walcząc, Porywa Tłum}\index{Specjalizacje!Lista!Walcząc, Porywa Tłum}

Jesteś ryzykantem wywijającym mieczem, który posiada zachwycający styl walki, który przyjemnie jest oglądać. 

Poziom 1: Estetyczny Atak

Poziom 2: Szybki Blok

Poziom 3: Akrobatyczny Atak lub Czcze Przechwałki

Poziom 4: Chronienie Sprzymierzeńca

Poziom 4: Szybkie Zabójstwo

Poziom 5: Korzyśći z Otoczenia

Poziom 6: Szybki Umysł lub Kontratak

Wtrącenia MG: Pokaz okazuje się głupiutki, niezdarny lub nieatrakcyjny. 

\subsubsection{Wolałby Czytać}\index{Specjalizacje!Lista!Wolałby Czytać}

Książki to Twoi przyjaciele. Co jest ważniejszego od wiedzy? Nic. 

Poziom 1: Wiedza to Potęga

Poziom 2: Większy Ulepszony Intelekt

Poziom 3: Stosowanie Swojej Wiedzy lub Skupienie na Umiejętności

Poziom 4: Wiedza to Potęga

Poziom 4: Wiedza o Nieznanym

Poziom 5: Większy Ulepszony Intelekt

Poziom 6: Wiedza to Potęga

Poziom 6: Wieża Intelektu lub Czytając Znaki

Wtrącenia MG: Książki się palą, moczą lub gubią. Komputery się psują lub tracą zasilanie. Okulary się tłuką.

\subsubsection{Wpada w Furię}\index{Specjalizacje!Lista!Wpada w Furię}

Kiedy wpadasz w furię, wszyscy wpadają w popłoch.

Poziom 1: Szał

Poziom 2: Większa Ulepszona Moc

Poziom 2: Umiejętności Ruchu

Poziom 3: Potężne Uderzenie lub Wojownik Bez Zbroi

Poziom 4: Większy Szał

Poziom 5: Atak i Ponowny Atak

Poziom 6: Większy Ulepszony Potencjał lub Śmiertelne Obrażenia

Wtrącenia MG: To łatwe dla berserka, by utracić kontrolę i zaatakować zarówno przyjaciół jak i wrogów.

\subsubsection{Wspiera Społeczność}\index{Specjalizacje!Lista!Wspiera Społeczność}

Utrzymujesz miejsce gdzie żyjesz bezpieczne od wszelkich niebezpieczeństw.

Poziom 1: Wiedza o Społeczności

Poziom 1: Lokalny Aktywista

Poziom 2: Umiejętny Atak

Poziom 3: Furia Pasterza lub Umiejętna Obrona

Poziom 4: Większy Ulepszony Potencjał

Poziom 5: Unik

Poziom 6: Większa Umiejętność Ataku lub Mur Obronny

Wtrącenia MG: Ludzie w społeczności nie rozumieją motywów postaci. Rywele próbują pozby się postaci.

\subsubsection{Wyje do Księżyca}\index{Specjalizacje!Lista!Wyje do Księżyca}

Na krótki czas stajesz się przerażającą i potężną istotą, która ma problemy, żeby się kontrolować. 

Poziom 1: Likantropia

Poziom 2: Kontrolowana Przemiana

Poziom 3: Większy Likantrop lub Większa Likantropia

Poziom 4: Większa Kontrolowana Zmiana

Poziom 5: Ulepszona Likantropia

Poziom 6: Śmiertelne Obrażenia lub Perfekcyjna Kontrola

Wtrącenia MG: Przemiana przebiega w sposób niekontrolowany. Ludzie boją się potworów. 

\subsubsection{Wysysa Energię}\index{Specjalizacje!Lista!Wysysa Energię}

Wysysasz energię tak z maszyn, jak i z istot w celu wzmocnienia samego siebie.

(Roboty i inne żywe maszyny powinny być traktowane jak istoty, a nie maszyny, dla celów wysysania z nich energii.)

Poziom 1: Wyssanie Maszyny

Poziom 2: Wyssanie Istoty

Poziom 3: Wyssanie na Zasięg lub Wysuszająca Konsupcja

Poziom 4: Przechowanie Energii

Poziom 5: Dzielona Moc

Poziom 6: Wybuchowe Rozładowanie lub Pula Słoneczna

Wtrącenia MG: Wyssana moc nosi z sobą coś niechcianego – przymusy, choroby lub obce myśli. Wyssana moc może przeciążyć postać, powodując kłopoty.

\subsubsection{Wyszedł z Obelisku}\index{Specjalizacje!Lista!Wyszedł z Obelisku}

Twoje ciało, twarde jak kryształ, daje Ci unikalne zdolności, zyskane po wejściu w interakcję z lewitującym, kryształowym obeliskiem.

Poziom 1: Kryształowe Ciało

Poziom 2: Unoszenie się

Poziom 3: Zamieszkując Kryształ lub Nieruszony

Poziom 4: Kryształowe Soczewki

Poziom 5: Częstotliwość Rezonansowa

Poziom 6: Trzęsienie Rezonansowe lub Powrót do Obelisku

Wtrącenia MG: Cyphery i artefakty działają niespodziewanie w rękach postaci. 

\subsubsection{Włada Dziką Magią}\index{Specjalizacje!Lista!Włada Dziką Magią}

Jesteś użytkownikiem magii, który uczy się różnorodnych zaklęć zamiast skupiać na jednej szkole magii.

Poziom 1: Magiczne Zasoby

Poziom 1: Rzucanie Cypherów

Poziom 2: Zwiększenie Limitu Subtelnych Cypherów

Poziom 3: Przypływ Cyphera lub Szybsza Dzika Magia

Poziom 4: Zwiększenie Limitu Subtelnych Cypherów

Poziom 5: Wyuczone Zaklęcia

Poziom 6: Maksymalizacja Cyphera lub Dzikie Oświecenie

Wtrącenia MG: Zaklęcie działa losowo lub uderza w rzucającego. Coś przeszkadza w przygotowaniu zaklęć. Rzucanie zaklęć przyciąga uwagę potężnej istoty lub rywala.  Zaklęcie-cypher podczas rzucania zamienia się w przypadkowy cypher. 

\subsubsection{Włada Magnetyzmem}\index{Specjalizacje!Lista!Włada Magnetyzmem}

Rozkazujesz metalom i mocom magnetycznym.

Poziom 1: Poruszanie Metalu

Poziom 2: Odparcie Metalu

Poziom 3: Niszczenie Metalu lub Nakierowywany Pocisk

Poziom 4: Pole Magnetyczne

Poziom 5: Kontrola Metalu

Poziom 6: Diamagnetyzm lub Stalowy Cios

Wtrącenia MG: Metal się obraca, zgina i produkuje odpryski. Problem z koncentracją może sprawić, że coś Ci upadnie o złym czasie. 

\subsubsection{Włada Mocami Mentalnymi}\index{Specjalizacje!Lista!Włada Mocami Mentalnymi}

Wytrenowałeś swój umysł, by wykonywać zaskakujące psychiczne zadania. 

Poziom 1: Telepatia

Poziom 2: Czytanie Myśli

Poziom 3: Psioniczna Erupcja lub Psioniczna Sugestia

Poziom 4: Podłączony do Cudzych Zmysłów

Poziom 5: Wizja Przyszłości

Poziom 6: Kontrola Umysłu lub Sieć Telepatyczna

Wtrącenia MG: Coś podejrzanego w umyśle celu jest przerażające. Cel może odczytać myśli postaci.

\subsubsection{Włada Niewidzialną Mocą}\index{Specjalizacje!Lista!Włada Niewidzialną Mocą}

Naginasz światło i manipulujesz promieniami mocy dla ataku i obrony.

Poziom 1: Zniknięcie

Poziom 2: Macki Mocy

Poziom 2: Wyostrzone Zmysły

Poziom 3: Bariera Pola Siłowego lub Masowe Znikanie

Poziom 4: Niewidzialność

Poziom 5: Pole Obronne

Poziom 6: Wybuch lub Generacja Pola Siłowego

Wtrącenia MG: Niewidzialność częściowo zanika, ujawniając obecnośc postaci. Pole silowe jet przebite przez niecodzienny lub niespodziewany atak.

\subsubsection{Włada Rojem}\index{Specjalizacje!Lista!Włada Rojem}

Owady. Szczury. Nietoperze. Nawet ptaki. Władasz jednym rodzajem małych istot, które Tobie podlegają. 

Poziom 1: Wpływ na Rój

Poziom 2: Kontrola Roju

Poziom 3: Żywa Zbroja lub Umiejętny Atak

Poziom 4: Wezwanie Roju

Poziom 5: Pozyskanie Nietypowego Kompana

Poziom 6: Śmiercionośny Rój lub Umiejętna Obrona

Wtrącenia MG: Polecenie jest omylnie zinterpretowane. Kontrola jest chwilowa lub utracona. Ugryzienia i użądlenia nie są nietypowe dla władających rojami.

\subsubsection{Włada Zaklęciami}\index{Specjalizacje!Lista!Włada Zaklęciami}

Poprzez specjalizowanie się w zaklęciach i posiadanie księgi zaklęć, możesz szybko rzucać zaklęcie, takie jak błyskawicę, ogień, żywe cienie i przywoływanie. 

Poziom 1: Magiczny Błysk

Poziom 2: Promień Konfuzji

Poziom 3: Kwiat Ognia lub Przywołanie Wielkiego Pająka

Poziom 4: Przesłuchanie Duszy

Poziom 5: Ściana z Granitu

Poziom 6: Przywołanie Demona lub Słowo Śmierci

Wtrącenia MG: Zaklęcie działa źle. Przywołana istota rzuca się na czarownika. Mag-przeciwnik jest przyciągany przez magię użytkownika. 

\subsubsection{Zabawia}\index{Specjalizacje!Lista!Zabawia}

Występujesz, głównie dla innych ludzi.

Poziom 1: Beztroska

Poziom 2: Zainspirowanie Ułatwienia

Poziom 3: Umiejętności Wiedzy lub Większy Ulepszony Potencjał

Poziom 4: Uspokojenie

Poziom 5: Przydatna Pomoc

Poziom 6: Inspirujący Performer lub Okrutne Przedstawienie

Wtrącenia MG: Publiczność jest ziritowana lub obrażona. Muzyczne instrumenty się psują. Farby usychają w słoiczkach. Słowa wiersza lub piosenki wypadają Ci z pamięci.

\subsubsection{Zabija Potwory}\index{Specjalizacje!Lista!Zabija Potwory}

Zabijasz potwory.

(Choć noszenie miecza w settingu, w którym ludzie zazwyczaj nie noszą takich broni jest ok, możesz zmienić moce powiązane z mieczem Zabija Potwory na inną broń, taką jak pistolet ze srebrnymi pociskami.)

Poziom 1: Wyszkolony w Mieczach

Poziom 1: Sposób na Potwory

Poziom 1: Wiedza o Potworach

Poziom 2: Legendarna Wola

Poziom 3: Wyszkolony Morderca

Poziom 3: Ulepszony Sposób na Potwory lub Przekierowanie Ataku

Poziom 4: Niezłomny

Poziom 5: Większa Umiejętnosć Ataku (miecze)

Poziom 6: Morderca lub Heroiczny sposób na Potwory

Wtrącenia MG: Potwór stworzył pułapkę lub wziął Cię z zaskoczenia. Potwór ma zdolności, o których początkowo nie wiedziałeś. Matka potwora poprzysięga Ci zemstę. 

\subsubsection{Zadaje się z Martwymi}\index{Specjalizacje!Lista!Zadaje się z Martwym}

Martwi odpowiadają na Twoje pytania, a ich reanimowane ciała służą Tobie.

Poziom 1: Mówiący ze Zmarłymi

Poziom 2: Nekromancja

Poziom 3: Poznanie Lokacji lub Naprawa Ciała

Poziom 4: Większa Nekromancja

Poziom 5: Przerażające Spojrzenie

Poziom 6: Prawdziwa Nekromancja lub Słowo Śmierci

Wtrącenia MG: Reputacja postaci jako nekromanty ją wyprzedza. Zwłoki szukają zemsty za grzech zostania ożywionymi.

\subsubsection{Zaprowadza Sprawiedliwość}\index{Specjalizacje!Lista!Zaprowadza Sprawiedliwość}

Naprawiasz krzywdy, bronisz niewinnych i karasz winnych.

Poziom 1: Dokonanie Osądu

Poziom 1: Osąd

Poziom 2: Obrona Niewinnego 

Poziom 2: Ulepszony Osąd

Poziom 3: Chroń Wszystkich Niewinnych lub Ukaranie Winnego

Poziom 4: Odnalezienie Winnych

Poziom 4: Większy Osąd

Poziom 5: Ukaranie Wszystkich Winnych 

Poziom 6: Potępienie Winnych lub Zainspirowanie Niewinnych

Wtrącenia MG: Wina lub niewinność mogą być skomplikowane. Niektórzy ludzie gardzą samo-ustanowionymi sędziami. Dokonywanie osądów sprawia, iż zyskujemy sobie wrogów. 

\subsubsection{Zmniejsza się}\index{Specjalizacje!Lista!Zmniejsza się}

Możesz się zmniejszać do rozmiarów robaka, a z odpowiednią praktyką, być nawet mniejszym.

Poziom 1: Zmniejszenie się

Poziom 1: Niezauważalny

Poziom 2: Mniejszy

Poziom 2: Zalety Bycia Małym

Poziom 3: Wzrost lub Szybkie Skurczenie się

Poziom 4: Mały Lot

Poziom 5: Zmniejszenie Innych

Poziom 6: Większy lub Malutki

Wtrącenia MG: Istota myśli, że bohater to potencjalne pożywienie. Mała postać zostaje uwięziona w małej przestrzeni lub pod spadającym obiektem.

Postać, która Zmniejsza się, która wybiera zdolności takie jak Wzrost, nigdy nie będzie tak duża jak ktoś, kto Rośnie do Gigantycznych Rozmiarów, ale może cię cieszyć zaletami bycia dużym lub małym, w zależności od potrzeb.

\subsubsection{Został Przepowiedziany}\index{Specjalizacje!Lista!Został Przepowiedziany}

Jesteś "Wybrańcem" i przepowiednie, prognozy lub inne metody mówią, że dokonasz w pewnym momencie wielkich rzeczy.

Poziom 1: Umiejętności Międzyludzkie

Poziom 1: Wiedza

Poziom 2: Przeznaczenie Wielkości

Poziom 3:  Przezwyciężając Wszystkie Przeciwności lub Ciężko Zapracowana Odporność

Poziom 4: Centrum Uwagi

Poziom 5: Wskaż Im Drogę

Poziom 6: Jak Przepowiedziano or Większy Ulepszony Potencjał

Wtrącenia MG: Przeciwnik przepowiedziany w przepowiedni się pojawia. Niewierni grożą, że zrujnują plany postaci. Postać ma reputację w pewnych kręgach jako udawaniec. 

\subsubsection{Żyje w Dziczy}\index{Specjalizacje!Lista!Żyje w Dziczy}

Możesz przetrwać w dziczy, w której inni nie potrafią.

Poziom 1: Życie w Dziczy

Poziom 1: Ulepszona Moc

Poziom 2: Przetrwanie w Dziczy

Poziom 2: Badacz Dziczy

Poziom 3: Zwierzęce Zmysły lub Dzika Zachęta

Poziom 4: Dzika Świadomość

Poziom 5: Przyroda po Twojej Stronie

Poziom 6: Jedność z Dziczą lub Dziki Kamuflaż 

Wtrącenia MG: Ludzie w miastach i miasteczkach czasami są niechętni tym, którzy wyglądają (i pachną) jakby żyli w dziczy, uważając ich za ignorantów lub barbarzyńców.
\section{Tworzenie nowych Specjalizacji}\index{Specjalizacje!Tworzenie nowych Specjalizacji}

Ta sekcja zawiera wszystko, czego potrzebujesz, by tworzyć nowe specjalizacje. 

Każda specjalizacja ma swój własny styl, taki jak eksploracja, manipulowanie energią, lub po prostu zadawanie wielkich obrażeń w walce. Te ogólne klasyfikacje nazywamy kategoriami specjalizacji.

Każda kategoria specjalizacji ma swój motyw, razem z selekcją porad opisujących jak wybrać zdolności dla każdego poziomu z rozdziału Zdolności, od poziomu 1 do 6.

Nowo stworzona specjalizacja powinna być nazwana w formie czasownikowej, odpowiednio odmienionej, takiej jak Kontroluje Bestie lub Jest Stworzony z Kamienia. Dla przykładu, specjalizacja skupiona na ogniu jest tworzona na bazie porad odnośnie kategorie specjalizacji manipulacja energią i może być nazwana Nosi Halo Ognia (jedna z przykładowych specjalizacji w tym rozdziale). Alternatywnie, nowo stworzona specjalizacja może otrzymać nazwę Wznieca Ognie Apokalipsy lub Rozpala Ogień Samą Myślą. 

\subsection{Kategorie Specjalizacji}\index{Specjalizacje!Kategorie Specjalizacji}

 1. Korzystanie ze Sprzymierzeńców
   
2. Podstawowa
   
3. Manipulacja Energią
   
4. Eksploracja
   
5. Wpływ
   
6. Nieregularna
   
7. Ruch
   
8. Atak w walce
   
9. Wsparcie
   
10. Przyjmowanie Obrażeń

\subsubsection{Wybieranie zdolności, bazując na ich relatywnej mocy}\index{Specjalizacje!Wybieranie zdolności}

Porady odnośnie wybierania zdolności sugerują, by wybrać zdolności z jednego z trzech zbiorów – niski poziom, średni poziom i wysoki poziom. Te poziomy odpowiadają z "poziomami" danymi każdej zdolności. Te zdolności są dalej posegregowane w kategorie zdolności bazujące na tym, co robią – zdolności, które ulepszają fizyczne ataki są w kategorii umiejętności ataków, zdolności które wspierają sprzymierzeńców są w kategorii wsparcie itp. Patrz na poziomy i kategorie w sekcji Kategorie Zdolności i Relatywnej Mocy w rozdziale Zdolności.

Zdolności niskiego poziomu najlepiej się nadają na opcje na 1 i 2 poziomie. Zdolności średniego poziomu nadają się na opcje na 3 i 4 poziomie. Zdolności wysokiego poziomy pasują do poziomów 5 i 6

Powiedziawszy to – czasami uznasz za stosowne danie zdolności niskiego poziomu na poziomach 3 lub 4, lub może zdolność średniego poziomu na poziomie 1 i 2. Rób tak rzadko, ale bądź świadom takiej możliwości. To może być jedyny sposób na otrzymanie wszystkich zdolności, których chcesz, gdy tworzysz specjalizację. "Wyższe" zdolności zazwyczaj kosztują więcej punktów z Pul. Tak więc, jeśli zdolność średniego poziomu jest dostępna na poziomie 1 lub 2, lub zdolność wysokiego poziomu jest dostępna na poziomie 3 lub 4, wyższy koszt uczyni ją bardziej zbalansowaną.

\subsubsection{Balansowanie zdolności}

Porady odnośnie każdej kategorii mają za zadanie zapewnić, że zbudowane Specjalizacja będzie zbalansowana. Czasami jest stosowne dać zdolność niskiej mocy razem ze zwykłą zdolnością na danym poziomie, w zależności od potrzeb związanych ze Specjalnością. "Zdolność niskiej mocy" nie jest nigdzie zdefiniowana i pozostaje do interpretacji MG, ale ogólnie mówiąc, nie powinna być potężniejsza niż zdolność niskiego poziomu (co znaczy, poziomiu 1 lub 2).

Dla przykładu, ktoś kto ma lodowe moce może stworzyć małe rzeźby ze śniegu w dodatku do emisji promienia zimna. Ktoś, kto korzysta z elektryczności może naładować rozładowany artefakt lub mieć atut dla korzystania z elektrycznych systemów. I tak dalej.

Często, porady odnośnie Specjalizacji notują to jako możliwość. Jednakże, masz duża dowolność w zdecydowaniu, czy Specjalizacja potrzebuje dodatkowej zdolności, nawet jeśli porady odnośnie tego poziomu nie przewidują jej. Jeśli dodajesz zdolność, lub jest tam zdolność wysokiej mocy, której normalnie nie powinno być, może to oznaczać, że wybór dany na następnym poziomie lub na poprzednim nie był całkiem dobry. Zbalansowanie Specjalizacji to poniekąd sztuka. Musisz się oprzeć chęci przeładowania mocy Specjalizacji, ale także nie może ona być zbyt słaba.

\subsubsection{Porady odnośnie zdolności nie są wyryte w kamieniu}

Każda kategoria Specjalizacji zawiera porady, jaką zdolność powinno się wybrać na jakim poziomie. Ale nie patrz an te porady jak na coś, czego nie można przeskoczyć. Nie są one wyryte w kamieniu – po po prostu startowa pozycja. Możesz chcieć zmienić zdolność na danym poziomie na taką, której nie ma w poradach. Tak długo, jak wybrana zdolność znajduje się w odpowiedniej krzywej mocy dla tego poziomu, wszystko jest ok. Porady nie zostały zaprojektowane zm yślą o ograniczaniu graczy.

Dla przykładu, jeśli budujesz Specjalność skupioną na zimnie dla gry w świecie fantasy, możesz zadecydować, że zdolność, która wzywa demona, jest lepszym wyborem na danym poziomie niż zdolność, która zadaje obrażenia obszarowe, co jest poradą dla 5 poziomu manipulacji energią. Dokonanie takiej zmiany jest szczególnie wskazane, jeśli nowa Specjalność nazywa się Sięga do Dziewiątego Kręgu Piekieł.

\subsubsection{Zamienianie zdolności}

Jeśli tworzysz Specjalizację i myślisz, że powinna zawierać zbiór zdolności na pierwszym poziomie, które by ją mechanicznie przeciążyły, masz opcję dodać jedną jako zdolność "zamienianą". Łatwo to zrobić – postać może zamienić jedną ze swoich zdolności z typu na wskazaną zdolność niskiego poziomu z Specjalizacji. Tę zdolność zyskuje się zamiast jednej zwyczajowej wynikającej z typu postaci.

\subsubsection{Koncept i kategoria}

Wybór, by stworzyć Specjalizację, która korzysta z konkretnego konceptu – np.: tworzenia iluzji – nie oznacza, że musisz stworzyć focus określonej kategorii – w tym przypadku manipulacji środowiskiem. Specjalizację można stworzyć na wiele sposobów, korzystając z określonej energii, narzędzia lub pomysłu – każdy poskutkuje Specjalizacją odmiennej formy. Wszystko zależy od Ciebie. W tym przypadku, tworzenie iluzji może być wykorzystane, by zwodzić innych, co w takim wypadku powinno być kategorią wpływu.

W ten sam sposób, jeśli Specjalizacja daje postaci moc, by korzystać z jakiejś siły lub energii, nie oznacza to automatycznie, że powinna ona należeć do kategorii manipulacji energią (ale, oczywiście, może tam należeć, jeśli atakowanie i obrona są celem korzystania z tej energii). Ale może być zbudowana Specjalność, by dawała zdolności korzystające z energii lub siły, które są skupione na wytrzymałości, co sugeruje Specjalność Przyjmowanie Obrażeń (ktoś, kto może wytrzymać wiele ataków w walce); lub, jeśli główną cechą ma być maksymalizacja zadawanych obrażeń, to sugeruje to Specjalność Atak w Walce; lub tworzenie kompana stworzonego z tej energii lub siły, co sugeruje Specjalizację Korzystanie ze Sprzymierzeńców (czyli ktoś, kto korzysta z pomocnych istot, BN-ów lub nawet zduplikowanych wersji samego siebie by zyskać przewagę).

Oto inny przykład: Specjalizacja Kontroluje Grawitację może należeć do kategorii Manipulacja Środowiskiem lub Manipulacja Energią. To zależy od tego, czy ta Specjalność bardziej koncentruje się na miażdżeniu i trzymaniu przedmiotów (Manipulacja Środowiskiem) lub na uderzaniu w rzeczy i chronieniu siebie przy pomocy grawitacji (Manipulacja Energią).
Ta sama mnogość możliwych rozwiązań jest prawdziwa w innych przypadkach. Dla przykładu, jeśli ktoś może wzywać i kształtować ziemię, może on wykorzystać swoją moc, by zamienić się w istotę z kamienia (Przyjmowanie Obrażeń), by zaatakować wrogów (Atak w Walce) lub by tworzyć ściany, barykady i tarcze, by wspierać swoich sprzymierzeńców (Wsparcie).

Jeśli szukasz zdolności i nie możesz znaleźć odpowiedniej w długaśnym katalogu zdolności, możesz chcieć zmodyfikować jedną z nich, by mieć iluzję nowości (i uzyskać to, czego chcesz). Modyfikowanie oznacza skorzystanie z mechaniki zdolności, ale zmienieniu jej szczegółów w pewien sposób. Dla przykładu, może tworzysz nową Specjalność przemieszczania ziemi ale nie możesz znaleźć odpowiedniej ilości zdolności powiązanych z ziemią, by zaspokoić Twoje potrzeby. Łatwo jest zmienić inne zdolności, tak, by korzystały z ziemi zamiast z ognia, zimna lub magnetyzmu. Dla przykładu, Skrzydła Ognia mogą zostać zmienione na Skrzydła Ziemi, Lodowa Zbroja na Zbroję Ziemi itp. Te zmiany nie zmieniają niczego poza typem obrażeń i efektami odrzucenia (dla przykładu, Skrzydła Ziemi mogą generować chmury pyłu przy swoim przelocie).

\subsubsection{Zdolności, które odnoszą się do innych zdolności}

Pewne zdolności w rozdziale im poświęconym odnoszą się do innych zdolności. Jeśli wybierasz zdolność dla swojej Specjalizacji lub typu która odnosi siędo lub modyfikuje zdolnosć niskiego poziomu, umieść ową zdolność w swoim typie lub Specjalności, jako zdolność, którą gracz może wybrać na niższym poziomie.

\subsubsection{Tworzenie kompletnie nowych zdolności}

Możesz pójść dalej niż modyfikowanie i stworzyć jedną lub więcej kompletnie nowych zdolności. Kiedy to czynisz, spróbuj znaleźć coś podobnego do niej i skorzystaj z niej jak z szablonu. W każdym wypadku, zdecydowanie jak dużo powinno kosztować korzystanie z mocy (poprzez wydawanie punktów z Pul) jest jednym z ważniejszych aspektów stworzenia odpowiedniej zdolności.

Możesz zauważyć, że zdolności wyższego poziomu są droższe. Po części dzieje się tak dlatego, że czynią więcej, ale także dlatego, że postaci wyższego poziomu mają wyższe Skupienie niż postaci mniejszego poziomu, co oznacza, że płacą mniej punktów z odpowiednich Pul. Trzeciopoziomowa postać ze Skupieniem 3 w jednej z Pul nie płaci kosztu za zdolności kosztujące 3 lub mniej punktów. To świetne odnośnie zdolności mniejszego poziomu, ale zazwyczaj będziesz chciał, by gracz pomyślał trochę o tym, jak często chce skorzystać ze swoich najpotężniejszych zdolności. To oznacza, że powinny one kosztować przynajmniej 1 punkt więcej niż prawdopodobne Skupienie postaci na danym poziomie. (Bardzo często, postać będzie miała Skupienie w odpowiedniej Puli odpowiadające jej poziomowi.)

Jako ogólna zasada, typowa zdolność powinna kosztować tyle punktów, na jakim poziomie się ona znajduje. 

\subsubsection{Wybierz wtrącenia MG}

Pomysł o tym, jakie rodzaje wydarzeń mogą zaskoczyć, zaalarmować BG lub być dla niego katastrofą, gdy tworzysz nową Specjalizację, i przypisz te wtrącenia MG do niej. Zazwyczaj wydarza się to "w biegu" na sesji. Ale pomyślenie o tym zawczasu, gdy Specjalizacja jest tworzona i masz świeże pomysły w głowie, na pewno da Ci szczególnie piekielne opcje.

\subsection{Kategorie Specjalizacji}\index{Specjalizacje!Kategorie Specjalizacji}

\subsubsection{Korzystanie ze sprzymierzeńców}\index{Specjalizacje!Kategorie Specjalizacji!Korzystanie ze Sprzymierzeńców}

Te Specjalności przede wszystkim zapewniają BN-ów (kompanów). Ci kompani zapewniają pomoc BG na różne sposoby, ale zazwyczaj w formie atutu do akcji postaci.
Istnieje wiele potencjalnych motywów w tej kategorii, od zdolności, które pozwalają postaci na przywoływanie lub tworzenie sprzymierzeńców do takich, które pozwalają im na przyciąganie sprzymierzeńców poprzez sławę, magię, lub autorytet i charyzmę. 

\textbf{Połączenia z innymi BG}: Wybierz 4 odpowiednie połączenia z powyższej listy.

\textbf{Dodatkowy Ekwipunek}: Każdy przedmiot koniczny, by bohater mógł zachować przy sobie sprzymierzeńca. Dla przykładu, ktoś ze Specjalizacją, która wymaga super-nauki do tworzenia robotów-kompanów może mieć narzędzia wymagane, by budować i naprawiać owych sprzymierzeńców. Pewne Specjalności w tej kategorii nie wymagają niczego, by zyskać lub utrzymać korzyści.

\textbf{Sugestia Mniejszego Efektu}: BN-Sprzymierzeniec otrzymuje Ułatwienie w swojej następnej turze.

\textbf{Sugestia Większego Efektu}: BN-Sprzymierzeniec zyskuje natychmiastową dodatkową akcję.

Poniższe to przykłady i nie są kompletną listą wszystkich możliwych Specjalizacji w tej kategorii.

\begin{itemize}
\item  Buduje Roboty
\item  Zadaje się z Martwymi
\item  Kontroluje Bestie
\item  Istnieje w Dwóch Miejscach na Raz
\item  Przewodzi
\item  Włada Rojem
\item Mówi do Duchów
\end{itemize}    

\textbf{Porady odnośnie wyboru zdolności}

\textbf{Poziom 1}: Wybierz zdolność niskiego poziomu, która daje kompan 2 poziomu postaci, lub daje podobną korzyść zapewnioną przez BN-a. Alternatywnie, zapewnij podstawy pod pozyskiwanie takich kompanów na wyższych poziomach przez wybór zdolności, która daje postaci wpływ na innych.

Czasami dodatkowa zdolność niskiej mocy jest wskazana, w zależności od Specjalności. Często, jest to umiejętność, która przyznaje wyszkolenie w odpowiedniej dziedzinie wiedzy lub powiązanej umiejętności. Dla przykładu, wyszkolenie w umiejętności powiązanej z rodzajem kompana, który zostanie pozyskany przez BG, byłoby wskazane. 

\textbf{Poziom 2}: Wybierz zdolność niskiego poziomu która zapewnia wpływ na zbliżone rodzaje BN-ów jak kompan pozyskany na poprzednim poziomie. Jeśli żaden kompan nie był pozyskany na poprzednim poziomie, ta zdolność powinna teraz zapewnić ów benefit. Czasami drugorzędna zdolność może być wskazana w dodatku do mocy zapewnionej powyżej, może zdolność niskiego poziomu, która dodaje 2 lub 3 punkty do Puli.

\textbf{Poziom 3}: Wybierz dwie zdolności średniego poziomu. Daj je obydwie jako opcje dla Specjalności: BG wybiera jedną lub drugą.

Jedna opcja powinna dawać zdolność średniego poziomu, która ulepsza kompana, którego już ma BG (zazwyczaj awans z poziomu 2 do 3) lub daje dodatkowego kompana.

Druga opcja powinna dawać coś korzystnego postaci – może moc ofensywną lub defensywną, lub coś, co poszerzy wpływ jaki BG ma na swoich kompanów (lub potencjalnych kompanów).

\textbf{Poziom 4}: Wybierz zdolność średniego poziomu, która daje postaci moc ofensywną lub defensywną, jeśli BG jeszcze jej nie posiada, najlepiej powiązaną tematycznie ze Specjalizacją. Dla przykładu, jeśli BG zyskuje kompanów ze względu na swoją charyzmę, ta zdolność może mu pozwolić rozkazywać wrogom na krótki czas. Jeśli postać zyskuje kompanów poprzez udowanie ich lub przywoływanie, ta zdolność może im pozwolić wpływań na byty tego samego typu, które nie są jeszcze jej kompanami.

Alternatywnie, ta zdolność może dalej ulepszyć poprzednio uzyskanego kompana z poziomu 3 na poziom 4, lub pozwolić postaci pozyskać dodatkowego kompana.

\textbf{Poziom 5}: Wybierz zdolność, która ulepsza postać poprzez zapewniania obrony, zwiększenie jednej z Pul, lub inną moc natury defensywnej.

Alternatywnie, taa zdolność może umożliwić nowy sposób wpływania na i przywoływania BN-ów, w sposób odpowiedni do tematu przewodniego Specjalizacji. Dla przykładu, ktoś kto trzyma bestie-sprzymierzeńców może zyskać zdolność wezwania hordy mniejszych bestii. Ktoś, kto buduje roboty może zyskać zdolność by zbudować kilku mniejszych robotów-pomocników. I tak dalej.

W końcu, ta zdolność może ulepszyć poprzednio zyskanego kompana do 5 poziomu.

\textbf{Poziom 6}: Wybierz dwie zdolności wysokiego poziomu. Obydwie są opcjami w tej Specjalności – BG wybiera jedną z nich.

Jedna z tych zdolności powinna ulepszyć poprzednio pozyskanego kompana do 5 poziomu, jeśli nie wydarzyło się to na poziomie 5. W takim wypadku, ta zdolność powinna być zapewniona w dodatku do dwóch innych powiązanych zdolności.

Inna opcja wysokiego poziomu może zapewnić zbiór kompanów 3 poziomu dla postaci.

Ostatnia zdolność wysokiego poziomu może zapewnić nowe sposoby wpływania na i przywoływania BN-ów w sposób odpowiedni do motywu przewodniego Specjalności. Dla przykładu, ktoś, kto zyskuje kompanów poprzez wysoką charyzmę i trening może zyskać zdolność pozyskiwania informacji, które inaczej nie byłyby możliwe do poznania.

\subsubsection{Podstawowa}\index{Specjalizacje!Kategorie Specjalizacji!Podstawowa}

Kategoria Specjalizacji zapewniająca głównie wyszkolenie w umiejętnościach, atuty do zadań i zwiększenie Pul oraz Skupień w celu ulepszenia postaci. Każda Specjalność ma także swój motyw tematyczny, jak z innymi kategoriami, co nadaje wszystkim zdolnościom sens zamiast poczucia bycia przypadkowymi.

Dodatkowo, ponieważ korzyści zapewniane przez takie Specjalizacje sązazwyczaj bezpośrednie (z nielicznymi wyjątkami), większość Podstawowych Specjalizacji byłaby w porządku w niefantastycznej kampanii gdzie magia, super-nauka i zdolności psioniczne zazwyczaj nie istnieją. Powiedziawszy to, ponieważ zdolności zapewniane przez podstawowe Specjalizacje są bezpośrednie, nie oznacza to, że nie są potężne gdy wymiesza się je ze zdolnościami zapewnianymi przez typ, deskryptor, cyphery i inne cechy postaci.

\textbf{Połączenia z innymi BG}: Wybierz 4 stosowne połączenia z powyższej listy.

\textbf{Dodatkowy Ekwipunek}: Każdy obiekt konieczny, by wypełniać motyw przewodni Specjalności. Dla przykładu, Specjalizacja Wolałby Czytać może dać postaci parę książek. Specjalizacja Pracuje, by Żyć powinna zapewniać zestaw narzędzi.

\textbf{Sugestie Mniejszego Efektu}: Następna akcja jest Ułatwiona.

\textbf{Sugestie Większego Efektu}: Wykonaj darmowy, niezajmujący akcji test na odzyskanie zdrowia który nie liczy się do dziennego limitu.

Poniższa lista to przykłady i nie jest kompletną listą wszystkich możliwych Specjalności w tej kategorii.

\begin{itemize}
\item Nie Robi Zbyt Dużo
\item Interpretuje Prawo
\item Szybko się Uczy
\item Pracuje, by Żyć
\item Wolałby Czytać
\end{itemize}

\textbf{Porady odnośnie wyboru zdolności}

\textbf{Poziom 1}: Wybierz zdolność która daje wyszkolenie lub atut do umiejętności powiązanej z motywem przewodnim Specjalizacji, lub który daje 5 lub 6 punktów do jednej Puli.

Alternatywnie, wybierz zdolność która daje tylko 2 lub 3 punkty do jednej Puli i zdolność, która zapewnia trening lub atut do jednego zadania.

\textbf{Poziom 2}: Wybierz zdolność, której nie wybrałeś na 1 poziomie.

\textbf{Poziom 3}: Wybierz dwie zdolności średniego poziomu. Obydwie są opcjami tej Specjalności – BG wybierze jedną z nich.

Jedna opcja powinna być nie-fantastyczną zdolnością, która zwiększa umiejętności postaci i jest utrzymana w motywie przewodnim. Dla przykładu, jeśli motywem jest zwracanie uwagi, zdolność dająca informacje może być odpowiednia.

Inną opcją powinno być coś, co albo zwiększa Skupienie postaci w odpowiedniej statystyce, albo zapewnia postaci jakiś rodzaj obrony.

\textbf{Poziom 4}: Wybierz kolejną zdolność, która zapewnia wyszkolenie lub atut w umiejętnościach stosownych do motywu przewodniego, lub któa zapewnia 5 lub 6 punktów do jednej Puli najlepiej pasującej do Specjalności. Lub wybierz dwie zdolności, które zapewniają tylko 2 lub 3 punkty plus kolejną zdolność 4 poziomu, która polepsza jedno zadanie lub umiejętność. 

Alternatywnie, zapewnij umiejętność rozszerzającą repertuar postaci sugerowaną na 5 poziomie.

W końcu, jeśli Specjalność jeszcze nie dała żadnej mocy obronnej, zdolność obronna może być teraz zapewniona.

\textbf{Poziom 5}: Wybierz zdolność nieco rozszerzającą repertuar postaci – może coś jak Umiejętność Ekspercka, która może dać Ci automatyczny sukces na zadaniu, w którym jesteś wyszkolony.

Alternatywnie, jeśli na poziomie 4 był zapewniony niestandardowy benefit, daj tutaj opcje zasugerowane na 4 poziomie.

\textbf{Poziom 6}: Wybierz dwie zdolności wysokiego poziomu. Obydwie są opcjami Specjalności – BG wybiera jedną z nich.

Jedna opcja powinna być zdolnością, która zapewnie dodatkowe 5 lub 6 punktów w Puli powiązanej z Specjalizacją lub które postać może rozdzielić zgodnie z własnym życzeniem. Alternatywnie, trening w obronie lub ataku także mógłby tutaj wystąpić.

Druga opcja na 6 poziomie powinna dać postaci zupełnie nową zdolność powiązanąz motywem przewodnim, ale taką, która nie jest fantastyczna. Dla przykładu, zdolność która pozwala postaci na wzięcie dwóch akcji zamiast jednej byłaby ok. Dawanie dodatkowych wyszkoleń, atutów lub Skupienia także jest możliwe.

\subsubsection{Manipulacja Energią}\index{Specjalizacje!Kategorie Specjalizacji!Manipulacja Energią}

Manipulacja energią oferuje zdolności które kontrolują ogień, elektryczność, siły, magnetyzm i niestandardowe formy energii takie jak zimno, kamień lub coś dziwniejszego, jak "pustka" lub "cień". Te zdolności zazwyczaj dają postaci sposoby na osiągnięcie równowagi między atakowaniem przeciwników a uzyskiwaniem dodatkowej ochrony dla siebie i sprzymierzeńców. Specjalizacja zazwyczaj zapewnia też zdolności, które umożliwiają inne użytkowanie konkretnej energii, takie jak transport, tworzenie wielkich zbiorowisk energii, które atakująwiele celów, lub tworzenie chwilowych obiektów lub barier z energii. 

\textbf{Połączenia z innymi BG}: Wybierz 4 odpowiednie połączenia z listy powyżej.

\textbf{Dodatkowy Ekwipunek}: Jeden lub więcej przedmiotów w ekwipunku, które są odporne na manipulowanie daną energią – może to być zestaw ubrań. Alternatywnie, coś powiązanego z energią, którą się generuje. Pewne Specjalizacje w tej kategorii nie oferują dodatkowego ekwipunku.

\textbf{Zdolności Energetyczne}: Jeśli typ postaci zapewnia specjalne zdolności, które normalnie korzystają z innego rodzaju energii, teraz działają one w oparciu o energię tej Specjalności. Dla przykładu, jeśli postać ma Specjalizację w manipulowaniu elektrycznością, jej Wystrzał Mocy teraz działa na bazie elektryczności. Te zmiany nie dotyczą niczego poza typem obrażeń i efektami specjalnymi (dla przykładu, elektryczność może chwilowo zaburzyć systemy elektroniczne).

\textbf{Sugestie Mniejszego Efektu}: Cel lub coś w jego pobliżu jest Utrudnione z powodu pozostałości energetycznych.

\textbf{Sugestie Większego Efektu}: Ważny przedmiot we władaniu celu jest zniszczony.

Poniższa lista to przykłady i nie jest kompletną listą wszystkich możliwych Specjalności w tej kategorii.

\begin{itemize}
\item Absorbuje Energię
\item Nosi Halo Ognia
\item Tańczy z Czarną Materią
\item Ujeżdża Błyskawicę
\item Grzmi
\item Przywdziewa Połyskliwy Lód
\end{itemize}    
    
\textbf{Porady odnośnie wyboru zdolności}

\textbf{Poziom 1}: Wybierz zdolność niskiego poziomu, która albo zadaje obrażenia, albo zapewnia ochronę korzystając z odpowiedniego rodzaju energii.
Czasami, stosowna jest zdolność niskiej mocy, w zależności od typu energii. Dla przykładu, specjalizacja, która manipuluje zimnem może dać zdolność tworzenia lodowych rzeźb. Specjalność, która manipuluje elektrycznością może dać zdolność ładowania zużytych artefaktów lub atut przy korzystaniu z systemów elektrycznych. Specjalizacja, która pochłania energię może dać zdolność jej wyzwalania jako formę bazowego ataku. I tak dalej.

\textbf{Poziom 2}: Wybierz zdolność, której nie wybrałeś na 1-szym poziomie.

\textbf{Poziom 3}: Wybierz dwie zdolności średniego poziomu. Obydwie są opcjami tej specjalizacji  BG wybiera jedną z nich.

Pierwsza opcja powinna być zdolnością, która zadaje obrażenia, korzystając z odpowiedniej energii (może tu być też dodatkowy efekt).

Druga opcja powinna ulepszać zdolność ruchu korzystając z odpowiedniej energii, dawać dodatkową obronę, lub korzystać z energii w kompletnie nowy sposób, np.: wysysając energię z maszyn (elektryczność), wiążąc ofiarę w warstwach lodu (zimno), tworzyć absolutną ciszę (dźwięk), tworzyć oślepiający pokaz świateł (światło) itp.

\textbf{Poziom 4}: Wybierz zdolność, której nie wybrano na 3 poziomie.

\textbf{Poziom 5}: Wybierz umiejętność wysokiego poziomu, która zadaje obrażenia (i może ma dodatkowy efekt) więcej niż jednemu celowi korzystając z odpowiedniej energii, lub zdolność, która korzysta z energii w sposób, który nie został wykorzystany na poziomach 3 i 6.

\textbf{Poziom 6}: Wybierz dwie zdolności wysokiego poziomu. Obydwie są opcjami tej specjalności – BG wybierze jedną z nich.

Pierwsza opcja powinna korzystać z energii, by zadać dużo obrażeń jednemu celowi lub kilku.

Druga opcja powinna korzystać z energii, by osiągnąć jakiś cel, który nie był możliwy na niższych poziomach. Przykłady to przywołanie kompana stworzonego z ognia, teleportacja na długi zasieg korzystając z błyskawicy, tworzenie przedmiotów z utwardzonej energii itp.

\subsubsection{Manipulacja Środowiskiem}\index{Specjalizacje!Kategorie Specjalizacji!Manipulacja Środowiskiem}

Specjalności, które pozwalają postaci na przemieszczanie przedmiotów, wpływanie na grawitację, tworzenie obiektów (lub iluzji obiektów) itp. należą do kategorii manipulacji środowiskiem. Zważywszy na to, że w wielu przypadków energia jest używana jako część procesu, ta kategoria i manipulacja energią do pewnego stopnia na siebie nachodzą. Manipulacja środowiskiem priorytyzuje zdolności, które niebezpośrednio wpływają na wrogów i sprzymierzeńców poprzez obiekty, moce i zmiany w otoczeniu; specjalizacje manipulacji energią są skupione na bezpośrednim uszkadzaniu obiektów wybraną energią lub siłą.

Dla przykładu, zamiast atakować wroga pulsem grawitacyjnym, który zadaje obrażenia, postać korzystająca z manipulacji środowiskiem bazującą na grawitacji będzie raczej korzystała z mocy trzymających cel w miejscu, korzystała z grawitacji, by rzucać ciężkie obiekty jako atak, lub obniżała grawitację na danym obszarze lub nawet dla konkretnego obiektu.

\textbf{Połączenia z innymi BG}: Wybierz 4 odpowiednie połączenia z powyższej listy.

\textbf{Dodatkowy Ekwipunek}: Każdy obiekt konieczny, by manipulować środowiskiem. Dla przykładu, ktoś z specjalizacją dającą możliwość tworzenia obiektów potrzebuje podstawowych narzędzi. Pewne specjalizacje w tej kategorii nie wymagają niczego do zyskania jej korzyści.

\textbf{Zdolności manipulacji środowiskiem}: Motywy specjalizacji wliczają obrazy lub widoczne energie, które wpływają na wygląd Twoich zdolności. Takie zmiany, jeśli jakiekolwiek, tylko zmieniają wygląd efektu. Jeśli manipuluje się grawitacją, może błękitna poświata towarzyszy korzystaniu ze zdolności, wliczając w to zdolności typu. Jeśli tworzy się iluzje, może okazałe efekty wizualne towarzyszą zdolnościom typu, np.: mackowata bestia trzymająca cele, jeśli isę korzysta z umiejętności Zastój. I tak dalej.

\textbf{Sugestie Mniejszego Efektu}: Cel się wywraca i jego następny atak jest utrudniony.

\textbf{Sugestia Większego Efektu}: Postać się regeneruje i odzyskuje 4 punkty w jednej Puli.

Poniższe to przykłady i nie jest to kompletna lista wszystkich możliwych specjalizacji w tej kategorii:

\begin{itemize}
\item Przebudza Sny
\item Jaśnieje Światłością
\item Oblicza Nieobliczalne
\item Kontroluje Grawitację
\item Tworzy Iluzje
\item Tworzy Unikalne Obiekty
\item Włada Magnetyzmem
\item Stawia Umysł Ponad Materią
\end{itemize}    
    
\textbf{Porady odnośnie wyboru zdolności}

\textbf{Poziom 1}: Wybierz zdolność niskiego poziomu, która daje podstawową zdolność, która zmienia środowisko (lub jest w stanie je przewidzieć) korzystając z motywu specjalności. Dla przykładu, specjalizacja wpływająca na grawitację może dać zdolność czyniącą cel lżejszym lub cięższym. Specjalizacja tworząca iluzje może dać zdolność tworzenia obrazu. Specjalność tworząca przedmioty może dać podstawową  zdolność w tworzeniu określonych przedmiotów. Specjalność przewidująca przyszłość może wyliczyć wynik i zapewnić postaci korzyść z owej wiedzy. I tak dalej.

Czasami, dodatkowa zdolność niskiego poziomu jest stosowna, w zależności od specjalizacji. Często, jest to zdolność, która zapewnia wyszkolenie w określonej dziedzinie wiedzy.

\textbf{Poziom 2}: Wybierz zdolność niskiego poziomu, która zapewnia nową zdolność obronną ;lu ataku, zgodną z motywem przewodnim specjalności. 

Alternatywnie, wybierz zdolność, która zapewnia dodatkową lub zupełnien ową zdolność manipulowania środowiskiem, co jest powiązane z motywem przewodnim specjalizacji.

\textbf{Poziom 3}: Wybierz dwie zdolności średniego poziomy. Obydwie są opcjami specjalizacji – BG wybierze jedną lub drugą.

Pierwsza opcja powinna być zdolnością średniego poziomu powiązaną ze specjalnością, która zapewnia dodatkową zdolność wpływania na środowisko lub ulepsza podstawową zdolność wpływania na środowisko, którą BG już pozyskał. Ta zdolność nie jest bezpośrednio ofensywna lub defensywna, ale zapewnia albo nową zdolność powiązaną z podstawową zdolnością, lub polepsza jej siłę, zasięg lub w inny sposób ją rozszerza. 

Druga opcja powinna byś średniego poziomu i zapewniać zdolność obronną lub ataku, powiązaną ze specyficzną formą ruchu która zapewnia specjalność, jeśli tylko to możliwe.

\textbf{Poziom 4}: Wybierz zdolność średniego poziomu, która albo zapewnia ochronne lub atakujące wykorzystanie zdolności, w zależności od tego, która opcja nie została wybrana na poprzednim poziomie. 

\textbf{Poziom 5}: Wybierz zdolność wysokiego poziomu, która jest prawie ostatecznym wpływem na środowisko. Dla przykładu, jeśli manipulacja jest iluzją, ta zdolność może prześladować cel przerażającymi obrazami. Jeśli specjalność polega na grawitacji, może odblokować lot. Jeśli jest magnetyczna, może pozwolić użytkownikowi na przekształcenia metali. Jeśli specjalność zapewnia moce telekinetyczne, ta zdolność może pozwolić postaci rzucac ciężkie przedmioty na swoich wrogów. I tak dalej.

\textbf{Poziom 6}: Wybierz dwie zdolności wysokiego poziomu. Są one opcjami tej specjalności – BG wybierze jedną z nich.

Pierwsza z tych zdolności powinna zapewnić zdolność obronną lub ataku, (odmienna kategoria niż zdolność z 4 poziomu, ale oczywiście, wysokiego, a nie średniego poziomu).

Druga opcja pwinna być czymś, co dalej rozszerza podstawową zdolność manipulacji środowiskiem. Na 5 poziomie wybierano zdolność prawie ostateczną, tutaj można wybrać zdolność ostateczną, powiązaną z oferowanym rodzajem manipulacji, lub inny sposób wykorzystania zdolności, który korzysta nieznanego wcześniej aspektu tej specjalności.

\subsubsection{Eksploracja}\index{Specjalizacje!Kategorie Specjalizacji!Eksploracja}

Specjalności, które pozwalają postaci na pozyskiwanie informacji, przetrwanie w nieznanych środowiskach i znalezienie drogi do nowych lokacji lub wyśledzenie konkretnych istot są specjalnościami eksploracji. Przetrwanie w nieznanym środowisku wymaga odpowiedniej selekcji zdolności defensywnych; jednakże zdolności, które pozwalają na znajdowanie rzeczy i naukę są na pierwszym miejscu.

Specjalności eksploracji polegają na różnorodnych metodach, lecz wyszkolenie i doświadczenie są najważniejsze. Niektóre metody wymagają specyficznych narzędzi (takich jak pojazdy) by skorzystać z zapewnianych korzyści, a inne polegają na nadnaturalnych lub naukowych zdolnościach, by uczyć się nowych rzeczy i badać dziwne miejsca z daleka.

\textbf{Połączenia z innymi BG}: Wybierz 4 odpowiednie połączenia z powyższej listy.

\textbf{Dodatkowy Ekwipunek}: Przedmiot potrzebny do eksploracji. Dla przykładu, mapy i/lub kompass, a ktoś, kto korzysta z mocy psionicznych mógłby otrzymać lustro lub kryształową kulę. W ekwipunek może się także wliczać dostęp do pojazdu wymaganego do eksploracji, jak wcześniej zanotowano. 

\textbf{Sugestia Mniejszego Efektu}: Masz atut na każdej akcji, która korzysta z Twoich zmysłów, takiej jak percepcja lub atak, do końca następnej tury.

\textbf{Sugestia Większego Efektu}: Twoje Skupienie w Intelekcie zwiększa się o 1 do końca następnej tury.

Poniższe to przykłady i nie jest to kompletna lista wszystkich możliwych specjalizacji w tej kategorii:

\begin{itemize}
\item Bada Ciemne Miejsca
\item Infiltruje
\item Działa pod Przykrywką
\item Pilotuje Statki Kosmiczne
\item Jest Jasnowidzem
\item Izoluje Umysł od Ciała
\end{itemize}

\textbf{Porady odnośnie wyboru zdolności}

\textbf{Poziom 1}: Wybierz zdolność niskiego poziomu, która zapewnia podstawową zdolność eksploracji, przetrwania lub zbierania informacji powiązaną z motywem specjalizacji.
Czasami, dodatkowa zdolność niskiej mocy jest stosowna, w zależności od specjalności. Często jest to zdolność, która zapewnia wyszkolenie w powiązanej dziedzinie wiedzy lub powiązanej umiejętności (choć za to może już odpowiadać główna zdolność). Alternatywnie, może to być prosty bonus 2 lub 3 punktów do Puli Mocy.

\textbf{Poziom 2}: Wybierz kolejną moc niskiego poziomu, która zapewnia dodatkowe możliwości powiązane  eksploracją, przetrwaniem lub zbieraniem informacji. 

Dla przykładu, specjalność poświęcona przetrwaniu w ciężkich warunkach może zaoferować zdolność (lub dwie) które czynią łatwiejszym unikanie naturalnych niebezpieczeństw, trucizn, trudnego terenu itp. Specjalność poświęcona eksploracji konkretnego terenu może dawać zdolności dostępu do niego, lub możliwość, której inni nie mają (np.: zdolność widzenia w ciemnościach).

\textbf{Poziom 3}: Wybierz dwie zdolności średniego poziomu. Obydwie są opcjami tej specjalności – BG wybierze jedną lub drugą.

Jedna opcja powinna polepszać bazową zdolność eksploracji lub dawać nową zdolność eksploracji, przetrwania lub zbierania informacji.

Druga opcja powinna być jakąś korzyścią dla postaci, atakiem lub obroną (zwłaszcza, jeśli specjalność jeszcze tego nie zapewniła) lub czymś, co poszerza zdolność postaci do eksploracji, zgodnie z motywem specjalności. 

\textbf{Poziom 4}: Wybierz zdolność średniego poziomu (atak lub obronę, w zależności od tego, czego nie zapewniono na 3 poziomie), która jest korzyścią dla postaci. Alternatywnie, jeśli zdolności ofensywne lub defensywne postaci są już dobrze rozwinięte, wybierz inną zdolność średniego poziomu, która poszerza zdolność postaci do eksploracji, przetrwania lub zbierania informacji. 

\textbf{Poziom 5}: Wybierz zdolność wysokiego poziomu, która wpływa na pewne kary związane z eksploracją, przetrwaniem lub zbieraniem informacji w normlanie niedostępnym miejscu.

\textbf{Poziom 6}: Wybierz dwie zdolności wysokiego poziomu. Obydwie są opcjami tej specjalności – BG wybierze jedną z nich.

Pierwsza opcja powinna znowu polepszać bazową zdolność eksploracji, którą już otrzymano, lub dać kompletnie nową zdolność eksploracji, przetrwania lub zbierania informacji.

Druga opcja powinna być czymś, z czego postać będzie czerpała korzyść, zdolnością ataku lub obrony, lub kolejną zdolnością, która rozszerza możliwość eksploracji.

\subsubsection{Wpływ}\index{Specjalizacje!Kategorie Specjalizacji!Wpływ}

Specjalność która stawia na pierwszym miejscu autorytet i wpływ – na ludzi lub maszyny zgodnie z wolą posiadacza, by pomagać innym, lub by posiąść jakąś inną prestiżową i ważną pozycję. Te specjalności zapewniają wpływ poprzez trening i perswazję, bezpośrednią manipulację mentalną, użycie sławy, by skupić na sobie uwagę i wpłynąć na akcje innych, lub po prostu wiedząc i zdobywając wiedzę, która wpływa na późniejsze decyzje. W ten sposób, koncept wpływu jest bardzo szeroki.

\textbf{Połączenia z innymi BG}: Wybierz 4 odpowiednie połączenia z powyższej listy.

\textbf{Dodatkowy Ekwipunek}: Każdy obiekt konieczny, by wpływać na innych. Niektóre specjalności wpływu nie wymagają niczego by korzystać ze swoich benefitów.

\textbf{Sugestie Mniejszego Efektu}: Zasięg lub czas trwania zdolności wywierającej wpływ się podwaja.

\textbf{Sugestie Większego Efektu}: Sprzymierzeniec lub wskazany cel może wykonać dodatkową akcję.

Poniższe to przykłady i nie jest to kompletna lista wszystkich możliwych specjalizacji w tej kategorii:

\begin{itemize}
\item Włada Mocami Mentalnymi
\item Tworzy Dziwną Naukę
\item Łączy Umysł i Maszynę
\item Jest Idolem Milionów
\item Rozwiązuje Tajemnice
\item Mówi z Maszynami
\item Łamie Systemy
\end{itemize}

\textbf{Porady odnośnie wyboru zdolności}

\textbf{Poziom 1}: Wybierz zdolność niskiego poziomu, która pozwala postaci na zdobycie wanej informacji (na tyle, by dokonać mądrego wyboru – alternatywnie, użyć tej wiedzy do perswazji lub zastraszania). Jak postać zdobywa tę informację zależy od specjalności. Jedna postać może wykonać eksperymenty by poznać odpowiedzi, inna może utworzyć link telepatyczny z innymi by sekretnie wymieniać dane, a inna może po prostu być wyszkolona w interakcjach społecznych. 

Czasami, dodatkowa zdolność niskiego poziomu jest stosowna, w zależności od specjalności. Często ta zdolność zapewnia trening w powiązanej dziedzinie wiedzy.

\textbf{Poziom 2}: Wybierz zdolność niskiego poziomu, która polepsza zdolność postaci do wywierania presji. Może to otworzyć nowe sposoby na skorzystanie ze specjalizacji lub po prostu ulepszyć podstawową zdolność, którą BG już posiada. Dla przykładu, ta zdolność 2 poziomu może ułatwić zadania powiązane z wpływem o parę stopni, pozwolić telepacie na czytanie umysłów innych (ujawniając sekrety, które inaczej pozostały tajne) lub zapewnić wpływ na fizycznie obiekty (by je ulepszyć lub dowiedzieć się czegoś o nich). I tak dalej.

\textbf{Poziom 3}: Wybierz dwie zdolności średniego poziomu. Obydwie są opcjami tej specjalizacji – BG wybierze jedną lub drugą.

Pierwsza opcja powinna zapewniać zdolność ataku lub obrony powiązaną z wpływem specjalności, jeśli to możliwe. Dla przykładu, wynalazca może stworzyć serum, które daje im zwiększone możliwości (których można użyć do ataku lub obrony), telepata może mieć zdolność atakowania wrogów mentalną energią, a ktoś z podstawowymi zdolnościami debaty i wpływu poprzez sławę może polegać na treningu w używaniu broni lub na swojej świcie.

Druga opcja średniego poziomu powinna zapewniać dodatkową zdolność wpływania na świat, lub rozszerzyć ;podstawowy wpływ zapewniany przez podstawową zdolność, którą postać już posiada. Ta opcja nie jest bezpośrednio ofensywna lub defensywna, ale zapewnia albo zupełnie nową zdolność powiązaną z podstawową zdolnością, albo zwiększa siłę, zasięg, lub inną właściwość poprzednio przyznanej podstawowej zdolności. Dla przykładu, telepata może mieć zdolność psionicznej sugestii.

\textbf{Poziom 4}: Wybierz zdolność średniego poziomu, która jest albo ofensywnym, albo defensywnym użyciem wpływu, w zależności od tego, której opcji nie wybrałeś na poprzednim poziomie.

Alternatywnie, ta zdolność może zapewnić dodatkowe możliwości powiązane z wpływem, który wywiera specjalizacja. 

\textbf{Poziom 5}: Wybierz prawie-ostateczną zdolność wysokiego poziomu powiązaną ze zdolnością zapewnianą na niższym poziomie. Alternatywnie, wybierz zdolność, której nie otrzymałeś na niższym poziomie, która pozwala Ci wywierać wpływ na nowy sposób. Dla przykładu, jeśli zdolnością specjalności jest telepatia, zdolność poziomu 5-tego może pozwolić postaci na spojrzenie w przyszłość by zyskać atuty podczas działań przeciwko przeciwnikom (i wspierania sprzymierzeńców).

\textbf{Poziom 6}: Wybierz dwie zdolności 6 poziomu. Obywie są opcjami tej specjalności – BG wybierze jedną lub drugą.

Jedna z tych opcji powinna zapewnić ofensywną lub defensywną zdolność, przeciweństwo tej, którą zapewniono na 4 poziomie (choć wysokiego poziomu zamiast średniego). 

Druga opcja powinna być czymś, co dalej posuwa motyw podstawowego wpływu zapewnianego przez specjalizację. Jeśli zdolność poziomu 5-tego była zdolnością prawie-ostateczną, tutaj może pojawić się jeszcze lepsze skorzystanie z wywieranego wpływu, lub odmienny sposób korzystania z tej zdolności jako wcześniej niezbadane wykorzystanie tej zdolności. 

\subsubsection{Nieregularna}\index{Specjalizacje!Kategorie Specjalizacji!Nieregularna}

Większość specjalizacji posiada bazowy motyw, pewną "historię postaci", z której logicznie wynikają powiązane zdolności. Jednakże, pewne specjalizacje mają tak szerokie tematy przewodnie, że nie wliczają się do żadnej innej kategorii z wyjątkiem nieregularnej. Nieregularne specjalizacje zapewniają zbiór różnych zdolności. Zazwyczaj dzieje się tak, gdyż nadrzędny motyw jest takim, który wymaga różnorodności i dostępu do paru różnych rodzajów zdolności. Często, te specjalizacje można znaleźć w konwencjach, które sugerują dodatkowe zmiany zasad, takie jak punkty mocy w konwencji superbohaterskiej lub rzucanie zaklęć w fantasy. Jednakże, inne specjalności nieregularne są możliwe.

\textbf{Połączenia z innymi BG}: Wybierz 4 pasujące połączenia z powyższej listy.

\textbf{Dodatkowy Ekwipunek}: Każdy przedmiot konieczny z uwzględnieniem motywu specjalizacji. Przykładowo, motyw superbohaterski może dać strój superbohatera.

\textbf{Sugestia Mniejszego Efektu}: Cel jest także oszołomiony na jedną rundę, podczas której wszystkie jego zadania są utrudnione.

\textbf{Sugestia Większego Efektu}: Cel jest wstrząśnięty i traci następną turę. 

Poniższe to przykładowa lista i nie wyczerpuje ona listy specjalności w tej kategorii:

\begin{itemize}
\item Otrzymuje Boskie Błogosławieństwa
\item Ma Szlachetną Krew
\item Wyszedł z Obelisku
\item Fruwa Szybciej niż Pocisk
\item Włada Zaklęciami
\item Mówi Głosem Ziemi
\end{itemize}

\textbf{Porady odnośnie wyboru zdolności}

\textbf{Poziom 1}: Wybierz zdolność niskiego poziomu, która zapewnie jedną z korzyści tematycznych specjalizacji, taką, którą mogłaby mieć postać 1-szego poziomu.

Czasami, dodatkowa zdolność niskiej mocy jest wskazana, w zależności od specjalizacji. Często, ta zdolność zapewnia trening w odpowiedniej dziedzinie wiedzy lub powiązanej umiejętności. Alternatywnie, może ona oferować prosty bonus od 2 do 3 punktów Puli.

\textbf{Poziom 2}: Wybierz zdolność niskiego poziomu, która jest powiązana z motywem przewodnim specjalizacji. Powinna ona nie być powiązana ze zdolnością 1-szego poziomu. Warto zaznaczyć, że jeśli nie zapewniono zdolności defensywnej na poziomie 1-zym, poziom 2 to dobre miejsce, by go przyznać.

\textbf{Poziom 3}: Wybierz dwie zdolności średniego poziomu. Obydwie są opcjami tej specjalizacji – BG wybiera jedną lub drugą.

Pierwsza opcja powinna zapewniać jedną z korzyści powiązanych ze specjalnością, która nie musi być bezpośrednio powiązana z tymi na wcześniejszych poziomach.

Druga opcja powinna zapewniać jakąś metodę ataku, jeśli wcześniej nie zapewniono żadnej. Alternatywnie, jeśli zdolności niższego poziomu nie zapewniły postaci tego, co powinna mieć, ta zdolność może ulepszyć zdolność zapewnianą na niższym poziomie.

\textbf{Poziom 4}: wybierz zdolność średniego poziomu, która zapewnia jedną z korzyści tematycznych specjalizacji, która nie musi być powiązana ze zdolnościami niższego poziomu.

\textbf{Poziom 5}: Wybierz zdolność wysokiego poziomu, która zapewnia jedną z korzyści tematycznych specjalizacji – nie musi ona być powiązana ze zdolnościami wcześniejszych poziomów.

\textbf{Poziom 6}: Wybierz 2 zdolności wysokiego poziomu. Obydwie są opcjami tej specjalizacji – BG wybierze jedną lub drugą. 

Pierwsza opcja powinna zapewniać korzyść tematyczną specjalizacji, która nie musi być powiązana z tymi zapewnianymi wcześniej. Jednakże, ta zdolność może także zapewniać ostateczną wersję zdolności niższego poziomu, jeśli wersja z średniego lub niskiego poziomu nie była wystarczająca.

Druga opcja powinna zapewniać alternatywną metodę uzupełnienia postaci, w sposób, który jest odmienny od opcji numer 1. Dla przykładu, jeśli pierwsza opcja zapewnia jakiś atak, to druga może być interakcją, zbieraniem informacji lub zdolnością leczącą, w zależności od motywu przewodniego specjalizacji. 

\subsubsection{Ruch}\index{Specjalizacje!Kategorie Specjalizacji!Ruch}

Specjalizacje, która zapewniają nowe formy ruchu - w celu przodowania w walce, uciekania od sytuacji, od których inni niem ogą uciec, poruszania się cicho w celu złodziejstwa lub ucieczki, lub poruszania się w lokacjach normalnie niedostępnych – znajdują się w tej kategorii. Te specjalizacje zazwyczaj mają metody zapewniające atak lub obronę poprzez ruch, choć mogą też zapewnić inne sposoby na osiągnięcie tych celów. 

Klasyczna specjalność ruchu jest taką, która polega na szybkości by wykonywać więcej ataków i uniknąć zranienia, choć ogólne zdolności mogą zapewnić te same korzyści. Inne specjalności w tej kategorii mogą zapewnić postaci zdolność stanie się niematerialną, zmienić jej formę w np.: wodę lub powietrze, lub natychmiast się poruszyć za pośrednictwem teleportacji.

\textbf{Połączenia z innymi BG}: Wybierz 4 stosowne połączenia z powyższej listy.

\textbf{Dodatkowy Ekwipunek}: Każdy obiekt konieczny do osiągnięcia wielkich prędkości, zmieny stanu skupienia, lub innego pozyskania korzyści tej specjalizacji powinien być zapewniony jako dodatkowy ekwipunek. Pewne specjalizacje nie wymagają niczego by pozyskać ich benefity.

\textbf{Sugestia Mniejszego Efektu}: Cel jest oszołomiony i jego następna akcja jest utrudniona.

\textbf{Sugestia Większego Efekt}u: Cel jest wstrząśnięty i traci następną akcję.

Poniższa lista to przykłady i nie jest wyczerpująca względem wszystkich specjalizacji w tej kategorii:

\begin{itemize}
\item Istnieje Częściowo Poza Fazą
\item Porusza się jak Kot
\item Porusza się jak Wiatr
\item Ucieka Precz
\item Rozdziera Ściany Świata
\item Podróżuje przez Czas
\item Pracuje w Ciemnych Uliczkach
\end{itemize}

\textbf{Porady odnośnie wyboru zdolności}

\textbf{Poziom 1}: Wybierz zdolność niskiego poziomu, która zapewnia podstawową korzyść specyficznego stylu ruchu – np.: zwiększoną szybkość, zręczność, niematerialność itp.

Czasami, dodatkowa zdolność niskiej mocy jest stosowna, w zależności od specjalizacji. Jeśli podstawowa korzyść ruchu wymaga pewnego rodzaju dodatkowego zrozumienia lub wyszkolenia, ta zdolność może ją zapewniać. Alternatywnie, jeśli zapewniony ruch wydaje się odblokowywać podstawowy atak lub obronę, (polegający na używaniu pierwszej zdolności), dodaj to.

\textbf{Poziom 2}: Wybierz zdolność niskiego poziomu, która zapewnia nową zdolność ataku lub obrony powiązaną z motywem przewodnim specjalności. 

Alternatywnie, ta zdolność moża zapewnić dodatkową możliwość związaną z formą ruchu, która zapewnia użyteczną informację, która normalnie byłaby niemożliwa do zdobycia przez kogoś bez tej specjalizacji. 

\textbf{Poziom 3}: Wybierz dwie umiejętności średniego poziomu. Obydwie są opcjami tej specjalizacji – BG wybierze jedną lub drugą.

Pierwsza opcja powinna zapewniać dodatkową możliwość ruchu lub ulepszyć podstawową możliwość ruchu w nawiązaniu do motywu przewodniego specjalizacji. Nie jest to bezpośrednio ofensywne lub defensywne, ale zapewnia postaci nowy poziom zdolności lub kompletnie nową zdolność powiązaną z jej podstawową zdolnością ruchu.

Druga zdolność powinna zapewniać ofensywną lub defensywną zdolność powiązaną ze specyficzną formą ruchu, którą zapewnia specjalizacja.

\textbf{Poziom 4}: Wybierz zdolność średniego poziomu, która ulepsza przewagi zapewniane przez paradygmat ruchu specjalności. Może to zapewnić nową lub lepszą formę obrony (bezpośrednio lub pośrednio, jeśli przemieszczamy się do lokacji lub czasu, w których niebezpieczeństwo nie istnieje) lub nową/lepszą formę ataku.

\textbf{Poziom 5}: Wybierz prawie-ostateczną zdolność wysokiego poziomu, która korzysta z ruchu. Dla przykładu, jeśli specjalizacja zapewnia przemieszczanie się w czasie, ta zdolność może zapewnić faktyczne (lecz chwilowe) przemieszczenie siew czasie. Jeśli specjalizacja ulepsza szybkość, ta zdolność może pozwolić postaci poruszać się na bardzo długi dystans w ciągu akcji. I tak dalej.

Alternatywnie, odblokuj jeszcze nie używane zdolności, które wynikają z bazowej formy ruchu zapewnianej przez tą specjalizację. 

\textbf{Poziom 6}: Wybierz dwie zdolności wysokiego poziomu. Obydwie są opcjami tej specjalności – BG wybierze jedną lub drugą.

Pierwsza z opcji powinna zapewniać ofensywną lub defensywną zdolność, odmienną od tej zapewnianej na 4 poziomie (lecz wysokiego poziomu zamiast średniego).

Druga opcja powinna dalej poszerzać możliwości ruchu podstawowej zdolności. Jeśli wybór 5-tego poziomu był zdolnością prawie-ostateczną, tutaj powinna pojawić się opcja ostateczna powiązana z ruchem.

\subsubsection{Atak w walce}\index{Specjalizacje!Kategorie Specjalizacji!Atak w walce}

Specjalności ataku w walce zajmują się zadawaniem obrażeń. Specjalności w tej kategorii oferują także defensywne zdolności, ale kładą nacisk przede wszystkim na obrażenia, których nie osiągają inne specjalizacje.

Aby osiągnąć ten cel, specjalność ataku w walce może oferować mistrzostwo w danym stylu walki, co może być treningiem w konkretnej broni lub sztuce walki, lub użyciem konkretnego narzędzia (lub nawet rodzaju energii). Styl może być czymś tak prostym jak bycie najlepszym przeciwko konkretnemu rodzajowi przeciwnika, lub czymś znacznie szerszym, jak zastosowanie szczególnie wrednego lub nieuczciwego stylu. Walczący atakiem w walce może użyć ognia, siły lub magnetyzmu jako swoją preferowaną metodę zadawania obrażeń.

\textbf{Połączenia z innymi BG}: Wybierz 4 stosowne połączenia z powyższej listy.

\textbf{Dodatkowy Ekwipunek}: Broń, narzędzie lub inny specjalny przedmiot lub substancja (jeśli jakakolwiek) wymagana, by stosować dany styl walki. Dla przykładu, porcja 5-poziomowej trucizny dla Walczy Nieczysto lub Morduje, trofeum po poprzednio pokonanym wrogu dla Walczy z Robotami, lub stylowe ubrania dla Walcząc, Porywa Tłum.

\textbf{Sugestia Mniejszego Efektu}: Cel jest tak oszołomiony przez Twoje manewry, że na jedną turę wszystkie jego zadania są utrudnione. 

\textbf{Sugestia Większego Efektu}: Wykonaj natychmiastowo dodatkowy atak, korzystając z ataku zapewnianego przez specjalizację w swojej turze.

Poniższe to przykłady i ta lista nie wyczerpuje wszystkich możliwych specjalizacji w tej kategorii:

\begin{itemize}
\item Walczy z Robotami
\item Walczy Nieczysto
\item Walcząc, Porywa Tłum 
\item Poluje
\item Posiada Licencję na Broń
\item Szuka Kłopotów
\item Mistrzowsko Posługuje się Bronią
\item Morduje
\item Nie Potrzebuje Broni
\item Jest Bardzo Silny
\item Wpada w Furię
\item Zabija Potwory
\item Rzuca ze Śmiertelną Dokładnością
\item Dzierży Dwie Bronie Naraz
\end{itemize}

\textbf{Porady odnośnie wyboru zdolności}

\textbf{Poziom 1}: Wybierz zdolność niskiego poziomu, która zadaje dodatkowe obrażenia, korzystając z stylu walki, energii lub podejścia specjalności, lub gdy jest używane przeciwko wybranemu wrogowi.

Czasami, dodatkowa zdolność niskiego poziomu jest wskazana, w zależności od specjalizacji. Dla przykładu, specjalizacja zapewniająca zdolność walki specjalną bronią może oferować trening w zadaniach tworzenia tej broni. Specjalność, która zapewnia zwiększone obrażenia przeciwko danemu typowi wrogów może zapewnić trening w rozpoznawaniu, lokalizowaniu lub po prostu ogólnej wiedzy o tym wrogu. Styl walki, który jest szczególnie nieczysty, może zapewniać trening w zastraszaniu. I tak dalej.

Jeśli specjalność dotyczy walki z specyficznym wrogiem, dodatkowe moce (więcej niż normalnie byłyby oferowane) mogą być odpowiednie. Albo zwiększają one efektywność przeciwko wybranemu wrogowi, albo oferują szersze, lecz powiązanie zdolności, które dają specjalizacji pewne funkcjonalności, nawet, gdy nie walczy się z wrogiem.

\textbf{Poziom 2}: Wybierz zdolność niskiego poziomu która zapewnia jakąś formęobrony korzystając z broni, stylu walki lub wybranej energii. Jeśli styl walki jest szczególnie dobry w walczeniu z wybranymi rodzajami przeciwników, zdolność powinna zapewniać obronę przed tym rodzajem wrogów. Alternatywnie, specjalizacja może oferować kolejną metodę zwiększania obrażeń korzystając z wybranego paradygmatu.
Czasami, dodatkowa zdolność niskiego poziomu jest stosowna na poziomie 2. Jeśli tak jest, wybierz rodzaj zdolności niskiego poziomu, której nie wybrano na poziomie 1.

\textbf{Poziom 3}: Wybierz dwie zdolności średniego poziomu. Obydwie sąopcjami tej specjalizacji: BG wybiera jedną z nich.

Jedna opcja powinna zadawać dodatkowe obrażenia korzystając z stylu walki, energii lub podejścia specjalności, lub przeciwko konkretnemu wrogowi. Może być to coś tak prostego jak dodatkowy atak danego rodzaju.

Druga opcja powinna zapewniać metodę chwilowej neutralizacji przeciwnika poprzez rozbrajanie go, oszołomienie lub wstrząsanie nim, spowalnianie go lub utrzymywanie w miejscu, lub w inny sposób ograniczanie jego możliwości poprzez korzystanie ze stylu walki, energii lub podejścia specjalności lub przeciwko wybranemu wrogowi.

\textbf{Poziom 4}: Wybierz zdolność średniego poziomu, która poszerza paradygmat specjalności.  Często chodzi o wytrenowanie w konkretnym rodzaju ataku. Alternatywnie, ta zdolność może zwiększyć przewagę, uzyskując konkretny status w walce, taki jak zaskoczenie.

\textbf{Poziom 5}: Wybierz zdolność wysokiego poziomu, która zadaje obrażenia. Alternatywnie, jeśli specjalizujesz się w walce z danym typem wroga, ta zdolność może zapewnić postaci szansę na kompletną jego neutralizację, destrukcję, oślepienie lub zabicie jednego celu do poziomu 3 (lub wyższego, jeśli walczysz z danym typem wroga).

\textbf{Poziom 6}: Wybierz dwie zdolności wysokiego poziomu. Obydwie są opcjami tej specjalizacji: BG wybierze jedną z nich.

Jedna z opcji powinna korzystać z paradygmatu specjalności, by zadawać wyjątkowo dużą ilość obrażeń. Druga opcja powinna być odmiennym sposobem na zadawanie obrażeń, korzystając z paradygmatu specjalności lub po prostu zadając dużo obrażeń w ogólności (i polegając na poprzednich zdolnościach specjalizacji, by polepszyć celowanie). To może być atak przeciwko wielu celom, jeśli pierwsza opcja dotyczyła pojedynczego celu, zdolność natychmiastowego zabicia lub zneutralizowania wroga (zaczynając od poziomu 4, lecz z możliwością korzystania z Wysiłłku, by zwiększyć poziom celu) lub zdolność wybrania kolejnego wroga, wykonania kolejnego ataku lub zdolność ucieczki, by móc walczyć w przyszłości.

\subsubsection{Wsparcie}\index{Specjalizacje!Kategorie Specjalizacji!Wsparcie}

Specjalizacje, które pozwalają na pomaganie innym w osiągnięciu sukcesu, obrony innych, leczeniu ich i tak dalej są specjalizacjami wsparcia. Oczywiście, większość innych zdolności można wykorzystać, wspomagając innych, ale specjalizacje wsparcia (takie jak Wysysa Energię) są nastawione głównie na pomaganie, leczenie i ulepszania postaci, która wybiera tę specjalność. Specjalności wsparcia polegają na rożnych metodach zapewniania pomocy, wliczając trening bitewny wykorzystywany do bronienia, zdolności nadnaturalne lub wysoko-technologiczne umożliwiające leczenie, lub po prostu pozwalanie innym, by się zapomnieli poprzez zapewnianie rozrywki.

\textbf{Połączenia z innymi BG}: Wybierz 4 odpowiednie Połączenia z listy powyżej.

\textbf{Dodatkowy Ekwipunek}: Każdy obiekt odpowiedni, by zapewniać wsparcie. Dla przykładu, ktoś ze specjalizacją, która zabawia innych mógłby otrzymać muzyczny instrument lub podobny obiekt swojej specjalności. Niektóre specjalności z tej kategorii nie wymagają niczego, by zyskać swoje korzyści.

\textbf{Sugestia Mniejszego Efektu}: Do końca następnej rundy możesz ściągnąć na siebie atak bez konieczności korzystania z akcji.

\textbf{Sugestia Większego Efektu}: Możesz wziąć dodatkową darmowąakcję, by wesprzeć swojego sprzymierzeńca. 

Poniższe specjalności to przykłady i nie są kompletną listą możliwych specjalności w tej kategorii.

\begin{itemize}
\item Chroni Słabszych
\item Zabawia
\item Pomaga Swoim Przyjaciołom
\item Zaprowadza Sprawiedliwość
\item Wspiera Społeczność
\item Wysysa Energię
\item Uzdrawia
\end{itemize}

\textbf{Porady odnośnie wyboru zdolności}

\textbf{Poziom 1}: Wybierz zdolność niskiego poziomu, która zapewnia jakąś formę obrony, wsparcia lub rozrywki, korzyść w procesie leczenia lub obronę. Ta obrona może dotyczyć BG, an ie jego sprzymierzeńca, gdyż postać nie może chronić innych, najpierw nie broniąc samej siebie (a czasami bronienie samego siebie to ta rzecz, o którą chodzi).

Czasami, dodatkowa zdolność niskiego poziomu jest wskazana, w zależności od specjalizacji. Często, jest to zdolność zapewniająca trening w powiązanej dziedzinie wiedzy lub umiejętności, ale może to być coś, co łączy się z zapewnianą zdolnością, ale samo w sobie nie robiłoby zbyt wiele.

\textbf{Poziom 2}: Wybierz zdolność niskiego poziomu która dalej wspiera styl poprzedniego poziomu. Jeśli zdolność poprzedniego poziomu zapewniała sposób na obronę, który dotyczył tylko posiadacza umiejętności, poziom 2 powinien zapewniać zdolność chronienia innych. Jeśli taka zdolność już została zapewniona, zdolność 2 poziomu powinna chronić posiadacza zdolności lub zapewniać zdolność ataku, która, jeśli to możliwe, powinna być powiązana z motywem specjalizacji.

\textbf{Poziom 3}: Wybierz dwie zdolności średniego poziomu. Obydwie są opcjami tej specjalności: BG wybierze jedną z nich.

Pierwsza opcja powinna działać zgodnie z motywem przewodnim leczenia, wspierania lub bronienia, lub w inny sposób pomagać drugiej osobie. Druga opcja powinna w jakiś sposób wspierać postać, ofensywnie lub defensywnie lub w sposób, który rozszerzy jej zdolności. Alternatywnie, może to być kolejna, zupełnie odmienna metoda pomocy komuś innemu.

\textbf{Poziom 4}: Wybierz zdolność średniego poziomu, która daje sprzymierzeńcowi bezpośredni bonus lub daje postaci możliwość pomagania innym. Może to być także coś, co rani lub obezwładnia wroga, gdyż eliminacja przeciwników z pewnością pomaga sojusznikom.

\textbf{Poziom 5}: Wybierz zdolność wysokiego poziomu, która zapewnia opcję defensywną lub ofensywną, jeśli jeszcze nie została zapewniona. Jeśli już siętym zajęto lub jest to uważane za zbędne, wybierz zdolnosć wysokiego poziomu, która zapewnia jakąś formę obrony, pomocy lub rozrywki, korzyść w leczeniu lub ochronę innej postaci. Dla przykładu, zdolność 5 poziomu może dać sprzymierzeńcowi dodatkową darmową akcję lub pomóc mu w powtórce akcji, jeśli wcześniej ją oblał.

\textbf{Poziom 6}: Wybierz dwie zdolności wysokiego poziomu. Obydwie są opcjami tej specjalności: BG wybierze jedną z nich.

Pierwsza opcja powinna zapewniać ostateczną metodą pomocy innej postaci, zgodną z motywem przewodnim specjalności. Druga opcja może zapewnić alternatywną metodą pomagania innym, wiele specjalności z tej grupy to robi. Jednakże, opcja zapewniająca wysoko-poziomowy atak lub obronę także może wystąpić. 

\subsubsection{Przyjmowanie Obrażeń}\index{Specjalizacje!Kategorie Specjalizacji!Przyjmowanie Obrażeń}

Specjalności, które skupiają się na przyjmowaniu wielu obrażeń znajdują się w tej kategorii. Te specjalizacje zapewniają także zdolności ofensywne, jak i dodatkowe zdolności powiązane z ich motywem przewodnim, ale najważniejsze są tutaj zdolności obronne.

Pewne specjalności przyjmowania obrażeń polegają na fizycznej transformacji, która zapewnia dodatkową obronę, a inne polegają na wyspecjalizowanym treningu, korzystaniu z narzędzi takich jak tarcze lub ciężki pancerz, lub zapewniają zdolność naprawdę szybkiego leczenia się. Rodzaj fizycznej transformacji, który zapewnia ta specjalizacja, jeśli występuje, jest bardzo różnorodny. Specjalność może zamienić skórę postaci w kamień, wzmocnić ciało metalem, zamienić ją w potwora, powiększyć ją tak, by było trudniej ją zranić, i tak dalej.

\textbf{Połączenia z innymi BG}: Wybierz 4 odpowiednie połączenia z listy powyżej.

\textbf{Dodatkowy Ekwipunek}: Każdy obiekt konieczny, by utrzymywać fizyczną transformację (taki jak narzędzia do naprawy robotycznych części, tarcza lub inne narzędzie defensywne do umiejętnej obrony, lub może jakiś rodzaj amuletu bądź serum)/ Niektóre specjalności przyjmowania obrażeń nie wymagająniczego, by uzyskać ich korzyści.

\textbf{Sugestia Mniejszego Efektu}: +2 do Pancerza na kilka tur.

\textbf{Sugestia Większego Efektu}: Odzyskaj 2 punkty w Puli Mocy.

Poniższe są tylko przykładami i nie są kompletną listą wszystkich możliwych specjalności w tej kategorii:

\begin{itemize}
\item Jest Stworzony z Kamienia
\item Nosi Egzotyczną Tarczę
\item Chroni Wrota
\item Łączy Ciało i Stal
\item Rośnie do Gigantycznych Rozmiarów
\item Wyje do Księżyca
\item Żyje w Dziczy
\item Jest Mistrzem Obrony
\item Nigdy się nie Poddaje
\item Jest Jednoosobowym Bastionem
\end{itemize}

\textbf{Porady odnośnie wyboru zdolności}

\textbf{Poziom 1}: Wybierz zdolność niskiego poziomu, która zapewnia obronę zogdną z motywem przewodnim specjalności. Jeśli motyw to po prostu intensywny trening lub użytkownik ma defensywne narzędzie, ta zdolność może być tak prosta jak bonus do Pancerza. Jeśli obrona pochodzi od fizycznej transformacji, ta zdolność zapewnia bazową formę obrony wraz z efektami, korzyściami i czasami problemami wynikającymi z transformacji. Nisko-poziomowa zdolność lecząca także byłaby stosowna na pierwszym poziomie.

Czasami, dodatkowa zdolność niskiego poziomu jest potrzebna, w zależności od specjalności. Jeśli postać przechodzi przemianę, ta zdolność może zapewnić dodatkowy efekt, choć w stosunku do pewnych transformacji, może to być po prostu opis jak ktoś z anormalną fizjologią w pełni się leczy. W innych przypadkach, dodatkowy efekt może być treningiem w powiązanej umiejętności, lub może odblokować zdolność korzystania z konkretnej zbroi lub tarczy bez kary.

\textbf{Poziom 2}: Jeśli motyw przewodni specjalizacji nie jest fizyczną transformacją, wybierz nisko-poziomową zdolność, która zapewnia dodatkową metodę obronną, leczenie obrażeń, lub unikanie ataków.

Jeśli motyw przewodni specjalności to fizyczna transformacja, wybierz nisko-poziomową zdolność, która odblokowuje nowe możliwości związane z formą postaci. Może to być lepsza kontrola nad transformacją, odblokowywanie robotycznego interfejsu, lub w inny sposób w pełni odblokowanie tej formy. Ta zdolność nie jest koniecznie defensywna, choć może być.

\textbf{Poziom 3}: Wybierz dwie zdolności średniego poziomu. Obydwie są opcjami tej specjalizacji: BG wybiera jedną z nich.

Pierwsza opcja powinna zapewniać dodatkową obronę związaną z motywem przewodnim specjalizacji, taką jak zdolności obronne odblokowane poprzez transformację (co może także zwiększyć zdolności ataku) lub proste fizyczne ulepszenie jeśli obrona jest zapewniana przez umiejętności lub ulepszone uzdrawianie.

Druga opcja powinna zapewnić ofensywną zdolność, zwłaszcza jeśli tworzysz specjalizację nie skupioną na transformacji, która jeszcze nie ma ofensywnych korzyści. Ta zdolność może być ulepszonym atakiem lub zapewnić jakąś inną korzyść użyteczną w walce, taką jak szybkie uniku lub (na innym końcu kontinuum) bycie nieporuszalnym.

\textbf{Poziom 4}: Wybierz zdolność średniego poziomu dodatkowo ulepsza zalety zapewniane przez paradygmat unikania obrażeń specjalizacji. Często, wlicza się w to trening w konkretnym rodzaju obrony. Alternatywnie, może to zapewniać zalety zapewnione poprzednio, niezależnie, czy to oznacza większą kontrolę nad transformacją, zyskanie dodatkowych szans by uniknąć obrażeń lub powtórzenie zadania związanego z ulepszoną determinacją, itp. Jeśli specjalizacji brakuje opcji ofensywnej, jest to dobre miejsce, by ją zyskała.

\textbf{Poziom 5}: Wybierz zdolność wysokiego poziomu, która zapewnia obronę, możliwe, że w formie odrzucenia jakiegoś trudnego stanu (wliczając śmierć). Jeśli specjalizacja oferuje fizyczną transformację, ta zdolność może odblokować dalszą dodatkową, powiązaną zdolność, ofensywną, defensywną lub coś powiązanego z eksploracją i interakcją (np.: lot jeśli forma posiada skrzydła, zastraszanie jeśli jest straszna itp.).

\textbf{Poziom 6}: Wybierz dwie zdolności wysokiego poziomu. Obydwie są opcjami tej specjalności: BG wybierze jedną z nich.

Pierwsza opcja powinna używać paradygmatu specjalności, by zwiększać obronę lub zdolność poradzenia sobie z obrażeniami.

Druga opcja powinna oferować odmienny sposób na bycie defensywnym. W pewnych wypadkach, najlpeszą obroną jest dobry atak, więc ta opcja może zapewnić wysoko-poziomową zdolność ataku zgodnąz motywem przewodnim specjalizacji, albo jako po prostu wzrost obrażeń przy atakowaniu, albo jako lepszą kontrolę niestabilnej, fizycznej transformacji.

\subsubsection{Customizacja Specjalności}\index{Specjalizacjei!Customizacja Specjalności}

Czasami nie wszystkie opcje specjalizacji pasują do konceptu postaci, lub może MG potrzebuje dodatkowych porad odnośnie tworzenia nowej specjalności. Niezależnie od tego, odpowiedź leży w patrzeniu na zdolności specjalności na najbardziej fundamentalnym poziomie.

Na każdym poziomie, gracz może wybrać jedną z poniższych zdolności zamiast zdolności zapewnianej na danym poziomie. Wiele z nich to zdolności zastępcze, zwłaszcza na wysokim poziomie, które polegają na modyfikacji ciała, integracji z wysoko-technologicznym ekwipunkiem, uczeniem się potężnych zaklęć, odkrywaniem zapomnianych sekretów lub czymś podobnym, jak dyktowane przez konwencję gry.

\textbf{Poziom 1}

\begin{itemize}
\item Zdolności Bojowe
\item Potencjał
\end{itemize}

\textbf{Poziom 2}

\begin{itemize}
\item Zdolność niższego poziomu – wybierz zdolność 1-szego poziomu, powyżej.
\item Umiejętna Obrona
\item Wyszkolony We Wszystkich Broniach
\item Umiejętny Atak
\end{itemize}

\textbf{Poziom 3}

\begin{itemize}
\item Zdolność niższego poziomu – wybierz dowolną zdolność 1 lub 2 poziomu, powyżej.
\item Ulepszone Zdrowie
\item Zbroja Fuzyjna
\end{itemize}

\textbf{Poziom 4}

\begin{itemize}
\item Zdolność niższego poziomu – wybierz dowolną zdolność 1, 2 lub 3 poziomu, powyżej.
\item Odporność na Trucizny
\item Wbudowanie Bronie
\end{itemize}

\textbf{Poziom 5}

\begin{itemize}
\item Zdolność niższego poziomu – wybierz dowolną zdolność 1, 2, 3 lub 4 poziomu, powyżej.
\item Adaptacja
\item Pole Obronne
\end{itemize}

\textbf{Poziom 6}

\begin{itemize}
\item Zdolność niższego poziomu – wybierz dowolną zdolność 1, 2, 3, 4 lub 5 poziomu, powyżej.
\item Pole Reakcyjne
 \end{itemize}

\input{src/Zdolności.tex}

% below go the alphabetic abilities lists

% hyperref !!! https://tex.stackexchange.com/questions/180571/making-clickable-links-to-sections-with-hyperref

\chapter{Zdolności w kolejności alfabetycznej}

\section{A}

\textbf{Uśmiech i Słowo}\index{Zdolności!Alfabetycznie!Uśmiech i Słowo}\label{sec:Uśmiech i Słowo} - kiedy korzystasz z Wysiłku do dowolnej akcji interakcji społecznej - nawet takiej która polega na uspokajaniu zwierząt lub komunikowania się z kimś, czyim językiem nie mówisz - uzyskujesz darmowy poziom Wysiłku na tym zadaniu. Akcja.

\textbf{Przydatna Pomoc}\index{Zdolności!Alfabetycznie!Przydatna Pomoc}\label{sec:Przydatna Pomoc} - kiedy pomagasz komuś z zadaniem i stosuje on poziom wysiłku, zyskuje on darmowy poziom Wysiłku na tym zadaniu. Umożliwienie. 

\textbf{Absorpcja Energii}\index{Zdolności!Alfabetycznie!Absorpcja Energii}\label{sec:Absorpcja Energii} (7 punktów Intelektu) - dotykasz obiektu i absorbujesz jego energię. Jeśli dotykasz zamanifestowanego Cyphera, czynisz go bezużytecznym. Jeśli dotykasz artefaktu, rzuć na jego wyczerpanie. Jeśli dotykasz innego rodzaju zasilanego urządzenia lub maszyny, GM określa, czy jego moc jest w pełni wyssana. W każdym razie, absorbujesz energię z obiektu i odzyskujesz 1k10 punktów Intelektu. Jeśli to dałoby Ci więcej punktów Intelektu niż maksimum Twojej Puli, dodatkowe punkty są utracone, i musisz wykonać rzut na Obronę Mocy. Trudność tego rzutu to numer punktów powyżej Twojego maksimum, które zaabsorbowałeś. Jeśli oblejesz ten rzut, otrzymujesz 5 punktów obrażeń i nie możesz podejmować działań przez jedną rundę. Możesz wykorzystać tę zdolność jako akcję obronną kiedy jesteś celem ataku zdolnością. Taka akcja niweluje atak zdolnością, a ty absorbujesz energię, jakby pochodziła z urządzenia. Akcja.

\textbf{Absorpcja Energii Kinetycznej}\index{Zdolności!Alfabetycznie!Absorpcja Energii Kinetycznej}\label{sec:Absorpcja Energii Kinetycznej} - absorbujesz porcję energii ataku fizycznego lub uderzenia. Negujesz 1 punkt obrażeń, które normalnie byś poniósł i przechowujesz tę energię. Po tym, jak zaabsorbujesz 1 punkt energii, kontynuujesz obniżać obrażenia o 1 punkt z nadchodzących ataków, ale pozostała energia wycieka z Ciebie w formie błysku nieszkodliwego światła (nie możesz przechowywać na raz więcej niż 1 punktu energii w tym samym czasie). Umożliwienie. 

\textbf{Absorpcja Czystej Energii}\index{Zdolności!Alfabetycznie!Absorpcja Czystej Energii}\label{sec:Absorpcja Czystej Energii} - kiedy korzystasz z Absorpcji Energii Kinetycznej, możesz także absorbować i przechowywać energię ataków bazujących na czystej energii (światło, promieniowanie, energie międzywymiarowe, psioniczne itp.) lub z przekaźników owej energii, gdy masz z nimi bezpośredni kontakt. Ta zdolność nie zmienia tego, ile punktów energii możesz przechowywać. Jeśli masz również Ulepszoną Absorpcję Energii Kinetycznej, możesz również absorbować do 2 punktów obrażeń ze źródeł czystej energii. Umożliwienie. 

\textbf{Akceleracja}\index{Zdolności!Alfabetycznie!Akceleracja}\label{sec:Akceleracja} (4+ punkty Intelektu) - Twoje słowa umacniają ducha postaci w bliskim zasięgu, która jest w stanie zrozumieć Cię, przyspieszając ją, tak, że zyskuje ona atut na testach inicjatywy i rzutach na Obronę Szybkości przez 10 minut. Dodatkowo, poza zwykłymi opcjami korzystania z Wysiłku, możesz z niego skorzystać, by objąć celem tej zdolności więcej postaci - każdy poziom Wysiłku obejmuje dodatkowy cel. Musisz przemówić do dodatkowych celów, by je przyspieszyć, jeden cel na rundę. Jednak akcja na jeden cel by rozpocząć. 

\textbf{Akrobatyczny Atak}\index{Zdolności!Alfabetycznie!Akrobatyczny Atak}\label{sec:Akrobatyczny Atak} (1+ punktów Szybkości) - wyskakujesz w ataku, przesuwając się przez powietrze. Jeśli wyrzucasz naturalne 17 lub 18, możesz wybrać mniejszy efekt zamiast dodatkowych obrażeń. Jeśli zastosujesz Wysiłek do tego ataku, uzyskujesz darmowy poziom Wysiłku na zadaniu. Nie możesz skorzystać z tej zdolności, jeśli Twój Wysiłek Szybkości jest zredukowany wskutek noszenia zbroi. Umożliwienie. 

\textbf{Procesor Akcji}\index{Zdolności!Alfabetycznie!Procesor Akcji}\label{sec:Procesor Akcji} (4 punkty Intelektu) - korzystając z przechowywanych informacji i zdolności analizowania nadchodzących danych z wielką szybkością, jesteś wyszkolony w jednym fizycznym zadaniu Twojego wyboru na 10 minut. Dla przykładu, możesz wybrać bieg, wspinaczkę, pływanie, Obronę Szybkości lub atak specyficzną bronią. Akcja by rozpocząć.

\textbf{Adaptacja}\index{Zdolności!Alfabetycznie!Adaptacja}\label{sec:Adaptacja} - dzięki ukrytej mutacji, urządzeniu wbudowanemu w Twój kręgosłup, rytuałowi krwii smoka, lub jakiemuś innemu darowi, jesteś teraz w komfortowej temperaturze; nie musisz sie nigdy martwić o niebezpieczne promieniowanie, choroby lub gazy; i możesz zawsze oddychać w dowolnym środowisku (nawet w próżni kosmosu). Umożliwienie.

\textbf{Zaawansowany Użytkownik Cypherów}\index{Zdolności!Alfabetycznie!Zaawansowany Użytkownik Cypherów}\label{sec:Zaawansowany Użytkownik Cypherów} - możesz mieć przy sobie 4 Cyphery w danym czasie. Umożliwienie. 

\textbf{Zaawansowany Rozkaz}\index{Zdolności!Alfabetycznie!Zaawansowany Rozkaz}\label{sec:Zaawansowany Rozkaz} (7 punktów Intelektu) - cel w średnim zasięgu słucha każdej komendy, którą mu wydasz, tak długo, jak słyszy Cię i rozumie. Co więcej, tak długo, jak nie robisz nic innego niż wydawanie komend (nie wolno Ci wziąć żadnej innej akcji) możesz dać temu samemu celowi nową komendę. Ten efekt kończy się, gdy kończysz wydawać komendy lub gdy cel opuszcza średni zasięg względem Ciebie. Akcja by rozpocząć. 

\textbf{Atak z Rozbrojeniem}\index{Zdolności!Alfabetycznie!Atak z Rozbrojeniem}\label{sec:Atak z Rozbrojeniem} (3 punkty Szybkości) - za pomocą serii szybkich ruchów, wykonujesz atak przeciwko uzbrojonemu przeciwnikowi, zadając mu obrażenia i rozbrajając go, tak, że jego broń jest teraz w Twoich rękach lub 3 metry od niego na ziemi - Ty wybierasz. Ten atak rozbrajający jest utrudniony. Akcja.

\textbf{Zalety Bycia Dużym}\index{Zdolności!Alfabetycznie!Zalety Bycia Dużym}\label{sec:Zalety Bycia Dużym} - kiedy korzystasz ze Wzrostu, jesteś tak duży, że możesz łatwiej przenosić duże obiekty, wspinać sie na budynki korzystając z uchwytów niedostępnych dla zwykłych ludzi i skakać znacznie dalej. Kiedy korzystasz ze Wzrostu, wszystkie zadania wspinaczki, podnoszenia ciężarów i skakania są dla Ciebie ułatwione. Umożliwienie.

\textbf{Zalety Bycia Małym}\index{Zdolności!Alfabetycznie!Zalety Bycia Małym}\label{sec:Zalety Bycia Małym} - nauczyłeś się, jak wykorzystać swój rozmiar, siłę i dokładność. Twoje obrażenia już się nie dzielą na pół gdy korzystasz ze Zmniejszenia się, a zadania wspinaczki i skakania są ułatwione. Umożliwienie.

\textbf{Porada od Przyjaciela}\index{Zdolności!Alfabetycznie!Porada od Przyjaciela}\label{sec:Porada od Przyjaciela} (1 punkt Intelektu) - znasz słabe i mocne strony swojego przyjaciela, i wiesz jak go zmotywować, by osiągnął sukces. Kiedy dajesz przyjacielowi sugestię powiązaną z jego następną akcję, postać ta jest wyszkolona w tej akcji na jedną rundę. Akcja. 

\textbf{Znowu i Znowu}\index{Zdolności!Alfabetycznie!Znowu i Znowu}\label{sec:Znowu i Znowu} (8 punktów Szybkości) - możesz wziąć kolejną akcję w rundzie, w której już podjąłeś akcję. Umożliwienie.

\textbf{Nieśmiertelny}\index{Zdolności!Alfabetycznie!Nieśmiertelny}\label{sec:Nieśmiertelny} - Twoje ciało i umysł się nie starzeją. Jeśli nie zostaniesz zabity przez akt przemocy (lub jakąś zewnętrzną siłę jak trucizna lub infekcja), nigdy nie umrzesz. Umożliwienie.  

\textbf{Agent-Prowokator}\index{Zdolności!Alfabetycznie!Agent-Prowokator}\label{sec:Agent-Prowokator} - wybierz jedna z poniższych, by być wytrenowanym w: atakowanie bronią swojego wyboru, ładunki wybuchowe, lub skradanie się i otwieranie zamków (jeśli wybierzesz ostatnią opcję, posiadasz trening w dwóch umiejętnościach). Umożliwienie.

\textbf{Agresja}\index{Zdolności!Alfabetycznie!Agresja}\label{sec:Agresja} (2 punkty Mocy) - skupiasz się na atakowaniu w tak wielki sposób, że zostawiasz siebie wysuniętego na ataki wrogów. Kiedy ta zdolność jest aktywna, zyskujesz atut na atakach wręcz i Twoje rzuty na Obronę Szybkości przeciwko atakom wręcz i dystansowym są utrudnione. Ten efekt trwa tak długo, jak sobie życzysz ale kończy się, jeśli walka nie ma miejsca w zasięgu Twoich zmysłów. Umożliwienie.

\textbf{Szybki Umysł}\index{Zdolności!Alfabetycznie!Szybki Umysł}\label{sec:Szybki Umysł} - kiedy próbujesz wykonać zadanie Szybkości, możesz zamiast tego rzucić (i wydać punkty z puli) jakby to była akcja Intelektu. Jeśli stosujesz Wysiłek do tego zadania, możesz wydać punkty z Puli Intelektu zamiast Puli Szybkości (wtedy stosujesz też Skupienie w Intelekcie zamiast w Szybkości). Umożliwienie. 

\textbf{Wysokie Skupienie}\index{Zdolności!Alfabetycznie!Wysokie Skupienie}\label{sec:Wysokie Skupienie} (7 punktów Intelektu) - wkładasz w swoje zadanie wszystko. Dodajesz trzy darmowe poziomy Wysiłku to następnego zadania, które podejmujesz. Nie możesz wykorzystać tej zdolności znowu, dopóki nie zakończysz 10-godzinnego odpoczynku. Akcja.

\textbf{Uzdrowienie}\index{Zdolności!Alfabetycznie!Uzdrowienie}\label{sec:Uzdrowienie} (3 punkty Intelektu) - możesz spróbować uzdrowić jedno schorzenie (np: chorobę lub truciznę) dotyczące jednej istoty. Akcja.

\textbf{Szczur Miejski}\index{Zdolności!Alfabetycznie!Szczur Miejski}\label{sec:Szczur Miejski} (6 punktów Intelektu) - kiedy jesteś w mieście, odnajdujesz lub tworzysz znaczące skróty, sekretne wejścia lub ostateczne trasy ucieczki tam, gdzie wcześniej ich nie było. Aby to zrobićm musisz uzyskać sukces na kacji Intelektu, której trudność określa MG bazując na danej sytuacji. Powinieneś ustalić detale wraz ze swoim MG. Akcja.

\textbf{Zawsze Majsterkując}\index{Zdolności!Alfabetycznie!Zawsze Majsterkując}\label{sec:Zawsze Majsterkując} - jeśli masz narzędzia i materiały i nosisz mniej cypherów niż Twój limit, możesz stworzyć zamanifestowany cypher, jeśli poświęcisz na to godzinę. Nowy cypher jest wybierany przypadkowo i zawsze o 2 poziomy mniej niż normalnie (minimum to 1-szy poziom). Jest on także chwilowy i wrażliwy na uszkodzenia. Nazywa się go chwilowym cypherem. Jeśli dasz go komuś, by z niego korzystał, rozpada się on natychmiast w bezużyteczne śmieci. Akcja by rozpocząć; 1 godzina by ukończyć.

\textbf{Cudowne Kopiowanie}\index{Zdolności!Alfabetycznie!Cudowne Kopiowanie}\label{sec:Cudowne Kopiowanie} - możesz skorzystać ze zdolności Skopiuj Moc, aby skopiować potężniejsze zdolności. W dodatku do normalnych opcji korzystania z Wysiłku przy użyciu Skopiuj Moc, jeśli zaaplikujesz 2 poziomy Wysiłku, MG wybiera moc wysokiego poziomu, która najbardziej przypomina moc, którą pragniesz skopiować (zamiast zdolności niskiego poziomu). Umożliwienie.

\textbf{Dodatkowy Wysiłek}\index{Zdolności!Alfabetycznie!Dodatkowy Wysiłek}\label{sec:Dodatkowy Wysiłek} - kiedy stosujesz przynajmniej jeden poziom Wysiłku do akcji niebojowej, otrzymujesz darmowy, dodatkowy poziom Wysiłku na tym zadaniu. Kiedy wybierasz tę zdolność, musisz zdecydować, czy dotyczy ona Wysiłku Mocy, czy też Wysiłku Szybkości. Umożliwienie.

\textbf{Wielki Skok}\index{Zdolności!Alfabetycznie!Wielki Skok}\label{sec:Wielki Skok} (2 punkty Mocy) - skaczesz w powietrze i lądujesz bezpiecznie w pewnej odległości. Możesz skoczyć wzwyż, w dół lub w poziomie gdziekolwiek w dalekim zasięgu  jeśli masz czystą trasę do tego miejsce, bez żadnych przeszkód. Jeśli masz 3 lub więcej punktów mocy zainwestowanych w siłę, Twój zasięg się ulepsza do bardzo dalekiego. Jeśli masz 5 lub więcej punktów mocy zainwestowanych w siłę, Twój zasięg skoku zostaje ulepszony do 300 metrów. Akcja.

\textbf{Czatownik}\index{Zdolności!Alfabetycznie!Czatownik}\label{sec:Czatownik} - kiedy atakujesz istotę, która jeszcze nie wzięła swojej pierwszej rundy w walce, Twój atak jest ułatwiony. Umożliwienie.

\textbf{Wzmocnienie Dźwięku}\index{Zdolności!Alfabetycznie!Wzmocnienie Dźwięku}\label{sec:Wzmocnienie Dźwięku} (2 punkty Mocy) - na jedną minutę, możesz wzmocnić dalekie lub ciche dźwięki, tak, byś mógł je słyszeć wyraźnie, nawet jeśli jest to rozmowa lub dźwięk małego zwierzęcia poruszającego się w podziemnej norze w bardzo dalekim zasięgu. Możesz spróbować usłyszeć dźwięk, nawet jeśli istnieją bariery blokujące dźwięk lub jest on bardzo cichy, choć to wymaga paru dodatkowych rund koncentracji. Aby odróżnić dźwięk, którego poszukujesz, od głośnego środowiska, także powinieneś poświęcić parę rund na skupienie, gdy przeszukujesz słuchem swoją okolicę. Mając odpowiednio dużo czasu, możesz wyśledzić każdą konwersację, oddychającą istotę i każde urządzenie wydające dźwięk w zasięgu. Akcja by rozpocząć, do paru rund by ją zakończyć, w zależności od trudności zadania.

\textbf{Anegdota}\index{Zdolności!Alfabetycznie!Anegdota}\label{sec:Anegdota} (2 punkty Intelektu) - możesz polepszyć morale grupy istot i pomóc im w nawiązaniu więzi, poprzez zabawianie ich podnoszącą na duchu anegdotą. Przez następną godzinę, ci którzy słuchali Twojej historii są wyszkoleni w jednym zadaniu Twojego wyboru, które jest powiązane z anegdotą, tak długo, jak nie jest to atak lub obrona. Akcja by rozpocząć, jedna minuta by zakończyć.

\textbf{Zwierzęce Szpiegowanie}\index{Zdolności!Alfabetycznie!Zwierzęce Szpiegowanie}\label{sec:Zwierzęce Szpiegowanie} (4+ punkty Intelektu) - jeśli znasz ogólną lokalizację zwierzęcia, które jest przyjazne względem Ciebie i w zasięgu 1.5 km od Ciebie, możesz postrzegać świat jego zmysłami do 10 minut. Jeśli nie jesteś w formie zwierzęcej lub w formie podobnej do tego zwierzęcia, musisz zastosować poziom Wysiłku do korzystania z tej umiejętności. Akcja by rozpocząć. 

\textbf{Zwierzęcy Kształt}\index{Zdolności!Alfabetycznie!Zwierzęcy Kształt}\label{sec:Zwierzęcy Kształt} (3+ punkty Intelektu) - zmieniasz się w zwierzę tam małe jak szczur lub tak duże jak ty (np: duży pies lub mały niedźwiedź) na 10 minut. Za każdym razem, gdy zmieniasz kształt, możesz wybrać inne zwierzę. Twój ekwipunek staje się częścią owej transformacji, co czyni go nieużytecznym, o ile nie ma pasywnego efektu, takiego jak zbroja. W tej formie Twoje Statystyki pozostają takie same jak w Twojej normalnej formie, ale możesz się ruszać i atakować zgodnie z Twoim zwierzęcym kształtem (ataki większości zwierząt tego rozmiaru to bronie średnie, z których możesz korzystać bez żadnej kary). Zadania wymagające rąk - takie jak naciskanie klamek lub przycisków są utrudnione kiedy jesteś w formie zwierzęcej. Nie możesz mówić, ale dalej możesz korzystać ze zdolności, które nie polegają na ludzkiej mowie. Uzyskujesz dwie pomniejsze zdolności powiązane z istotą, w którą sie zmieniłeś (patrz tabela Mniejsze Zdolności Zwierzęcego Kształtu). Dla przykładu, jeśli zamieniasz się w nietoperza, jesteś wyszkolony w percepcji i możesz latać na daleki zasięg w każdej rundzie. Jeśli zamienisz się w ośmiornicę, jesteś wyszkolony w skradaniu się i oddychasz pod wodą. Jeśli zastosujesz poziom Wysiłku do stosowania tej zdolności, możesz albo przybrać kształt mówiącego zwierzęcia, albo hybrydowy. Kształt mówiącego zwierzęcia wygląda dokładnie jak zwykłe zwierzę, ale możesz dalej mówić i korzystać ze zdolności bazujących na ludzkiej mowie. Kształt hybrydowy wygląda jak Twoja normalna forma, ale z cechami zwierzęcia, nawet jeśli to konkretne zwierzę jest znacznie mniejsze od Ciebie (jak nietoperz lub szczur). W formie hybrydowej możesz mówić, korzystać ze swoich wszystkich zdolności, atakować jak zwierzę i wykonywać zadania przy użyciu rąk bez utrudnienia. Każdy kto dobrze się przypatrzy Tobie w formie hybrydowej nigdy nie pomyliłby Cię ze zwierzęciem. Akcja by się przemienić lub odwrócić transformację. 

``Podobieństwo'' to termin ogólnikowy. Lwy są podobne do tygrysów i leopardów, orły są podobne do kruków i łabędzi, psy są podobne do wilków i lisów itp.

Nawet jeśli Twój zwierzęcy kształt ma wiele typów ataku (np: zębami i pazurami), możesz zaatakować tylko raz w rundzie, chyba że masz jakąś zdolność, która pozwala CI na dokonywanie dodatkowych ataków w swojej turze.

Wariant Zwierzęcego Kształtu: Jeśli Twój koncept postaci sprawia, że zawsze zmienia się ona w ten sam zwierzęcy kształt zamiast wybierać z wielu, podwój czas trwania Zwierzęcego Kształtu (20 minut na jedno wykorzystanie). MG może pozwolić postaci z tym ograniczeniem na uczenie się dodatkowych zwierzęcych form poprzez wydanie 4 PD jako długotrwałą korzyść. 

\begin{table*}[t]

\centering
\caption{Tabela Mniejszych Zdolności Zwierzęcej Formy}
\label{Tabela Mniejszych Zdolności Zwierzęcej Formy}

\begin{tabularx}{\textwidth}{| X | X | X |}
\hline
 
 \textbf{Zwierzę} & \textbf{Umiejętność} & \textbf{Inne zdolności} \\ \hline

 Małpa & Wspinaczka & Ręce \\ \hline
 Borsuk & Wspinaczka & Czuły węch \\ \hline
 Nietoperz & Percepcja & Latanie \\ \hline
 Niedźwiedź & Wspinaczka & Czuły węch \\ \hline
 Ptak & Percepcja & Latanie \\ \hline
 Dzik & Obrona Mocy & Czuły węch \\ \hline
 Kot & Wspinaczka lub skradanie się & Mały \\  \hline
 Wąż dusiciel & Wspinaczka & Duszenie \\  \hline
 Krokodyl & Skradanie się lub pływanie & Duszenie \\  \hline
 Deinonych & Percepcja & Szybki \\  \hline
 Delfin & Percepcja lub pływanie & Szybki \\  \hline
 Ryba & Skradanie się lub pływanie & Wodny \\ \hline
 Żaba & Skakanie lub skradanie się & Wodny \\ \hline
 Koń & Percepcja & Szybki \\ \hline
 Leopard & Wspinaczka lub skradanie się & Szybki \\ \hline
 Jaszczurka & Wspinaczka lub skradanie się & Mały \\ \hline
 Ośmiornica & Skradanie się & Wodny \\ \hline
 Rekin & Pływanie & Wodny \\ \hline
 Żółw & Obrona Mocy & Pancerz \\ \hline
 Jadowity wąż & Wspinaczka & Trucizna \\ \hline
 Wilk & Percepcja & Czuły węch \\ \hline
 
 \end{tabularx}
 \end{table*}

\textbf{Wodny}: Zwierzę albo oddycha pod wodą zamiast powietrzem, albo jest w stanie oddychać wodą w dodatku do powietrza.

\textbf{Pancerz}: Zwierzę ma twardą skorupę lub skórę, co daje mu +1 do Pancerza.

\textbf{Duszenie}: 

% below go licenses

\chapter{Licencje}

\section{Licencja polska}

Otwarta Licencja Cypher System

Ta licencja („Zgoda”) obowiązuje między wydawcą lub autorem („Tobą”) i Monte Cook Games, LLC („MCG”) i nadaje Ci stałą, nie-wyłączną, darmową, obowiązującą na całym świecie licencję, by publikować i dystrybuować materiały gier fabularnych („Dzieła”) bazujące na i wykorzystujące Cypher System Reference Document („CSRD”) oraz pozwalającą na deklarowanie kompatybilności z Cypher System. Poprzez załączenie słów „kompatybilne z Cypher System” na okładce Dzieła, lub poprzez załączenie loga „kompatybilne z Cypher System” na okładce Dzieła, lub poprzez załączenie owych rzeczy w jakichkolwiek dokumentach reklamowych, promocyjnych, informacjach prasowych lub innych dokumentach powiązanych z Dziełem, wskazujesz swoje zaakceptowanie warunków Zgody.

Dzieło może w sobie zawierać dowolne lub wszystkie treści wliczone w CSRD. Nie może ono zawierać tekstu, obrazków lub innej zawartości z innych publikacji MCG. MCG może opublikować unowocześnione wersje CSRD. Możesz skorzystać z dowolnej autoryzowanej wersji CSRD w swoim Dziele.

Dzieło Musi posiadać frazę „kompatybilne z Cypher System” lub logo „kompatybilne z Cypher System” na swojej okładce. Także, musi ono zawierać w Dziele, gdziekolwiek Dzieło wymienia informacje prawne i licencyjne, następujący tekst:

\begin{displayquote}
Ten produkt jets niezależną publikacją i nie jest powiązany z Monte Cook Games, LLC. Opublikowano go zgodnie z Otwartą Licencją Cypher System, którą można znaleźć pod adresem https://csol.montecookgames.com/ 

CYPHER SYSTEM i jego logo są znakiem handlowym Monte Cook Games, LLC w USA i innych państwach. Wszystkie postaci Monte Cook Games i ich nazwy oraz ich cechy charakterystyczne, są znakami handlowymi Monte Cook Games, LLC.
\end{displayquote}

Dzieło nie może zawierać loga Cypher System, loga MCG, lub innego znaku handlowego MCG, z wyjątkiem loga „kompatybilne z Cypher System”.

Możesz skorzystać z każdej autoryzowanej wersji loga „kompatybilne z Cypher System”. Nie możesz zmienić loga w żaden sposób, z wyjątkiem zmiany jego rozmiaru, zachowując proporcje. 

Nie możesz sprzedawać lub reklamować Dzieła używając imienia swoich współpracownika/ów, chyba, że masz pisemną zgodę od niego/nich, by tak uczynić.

Poza uznaniem, że Dzieło jest produkowane i dystrybuowane na bazie Zgody, ani Dzieło, ani żadne dokumenty promocyjne, reklamowe, informacje prasowe lub inne dokumenty związane z Dziełem nie mogą zawierać żadnego twierdzenia, jakoby Ty lub Dzieło mielibyście zgodę lub pozwolenie MCG na publikację, lub że Ty lub Dzieło jesteście stowarzyszeni z MCG w jakikolwiek sposób.

Ani Dzieło, ani żadne materiały promocyjne, reklamowe, informacje prasowe lub inne dokumenty związane z Dziełem nie mogą zawierać poglądów rasistowskich, homofobicznych, dyskryminujących lub w inny sposób odrażających; otwarcie politycznych celów lub poglądów; opisów kryminalnej przemocy względem dzieci; gwałtu lub innych aktów perwersji kryminalnej; lub innych obscenicznych materiałów.

Dzieło nie może naruszać, nadużywać lub wykorzystywać w szkodliwy sposób praw własności intelektualnej żadnej trzeciej strony. Niniejszym zobowiązujesz się chronić MCG przeciwko wszelkim roszczeniom, pozwom, stratom i szkodom wynikającym z postulowanego naruszenia praw intelektualnych, szkodliwego wykorzystania lub nadużycia intelektualnych praw własności jakiejkolwiek trzeciej strony. Trwa to nawet po wycofaniu tej Zgody.

MCG nie bierze żadnej odpowiedzialności za Dzieło. Zgadzasz się przypilnować, by MCG i jego pracownicy, partnerzy i reprezentanci pozostali wolni od szkody w przypadku, gdy Twoja publikacja Dzieła poskutkuje pozwem sądowym.
CSRD, logo „kompatybilne z Cypher System”, logo Cypher System, logo MCG i wszystkie inne znaki handlowe MCG należą wyłącznie do MCG.

Jeśli złamiesz jakiekolwiek warunki tej Zgody, zostaje ona automatycznie Tobie wycofana. Jeśli naruszenie warunków Zgody nie zostanie wyjaśnione z MCG i udokumentowane pomiędzy Tobą a MCG w ciągu 15 dni od złamania warunków, musisz natychmiast wycofać i zniszczyć wszystkie istniejące kopie Dzieła. Dodatkowo, możesz zostać pozwany za szkody wynikłe wskutek złamania warunków Zgody.

Te Zgoda jest zarządzana przez prawa Stanu Waszyngton i prawa Stanów Zjednoczonych Ameryki. W związku z jakimikolwiek pozwami wynikłymi na bazie tej Zgody, strony sporu poddają się jurysdykcji sądów Stanu Waszyngton i Stanów Zjednoczonych Ameryki.

KONIEC DOKUMENTU

{ \color{red} \textbf{Uwaga}}: Legalnie obowiązująca wersja licencji to licencja angielska. W razie sporów prawnych, to do niej należy się odwoływać, nie zaś do powyższego tłumaczenia. Zamieszczono je tutaj w celach czysto poglądowych.

\section{Licencja angielska}

Cypher System : : Open License

This license (the “Agreement”) is an agreement between a publisher or author (“You”) and Monte Cook Games, LLC (“MCG”), that grants You a perpetual, non-exclusive, royalty-free, worldwide license to publish and distribute tabletop roleplaying game materials (the “Work”) based on and incorporating the Cypher System Reference Document (“CSRD”) and declaring compatibility with the Cypher System. By including the words “Compatible with the Cypher System” on the cover of the Work, or by including the “Compatible with the Cypher System” logo on the cover of the work, or by including these items on or in any advertising, promotions, press releases, or other documents affiliated with the Work, You indicate Your acceptance of the terms of this Agreement.

The Work may include any or all text included in the CSRD. It may not include text, art, or other content from other MCG publications. MCG may publish updated versions of the CSRD. You may use any authorized version of the CSRD in the Work.
The Work must include the phrase “Compatible with the Cypher System” or the “Compatible with the Cypher System” logo on the cover of the Work. And it must include within the Work, wherever the Work otherwise lists legal and copyright information, the following text:

\begin{displayquote}
This product is an independent production and is not affiliated with Monte Cook Games, LLC. It is published under the Cypher System Open License, found at http://csol.montecookgames.com.

CYPHER SYSTEM and its logo are trademarks of Monte Cook Games, LLC in the U.S.A. and other countries. All Monte Cook Games characters and character names, and the distinctive likenesses thereof, are trademarks of Monte Cook Games, LLC.
The Work may not use or incorporate the Cypher System logo, the MCG logo, or any other trademark of MCG, except the “Compatible with the Cypher System” logo.
\end{displayquote}

You may use any authorized version of the “Compatible with the Cypher System” logo. You may not crop or alter the logo in any way, except to resize it proportionally.

You may not market or advertise the Work using the name of any contributor unless You have written permission from the contributor to do so.

Other than to acknowledge that the Work is produced and distributed under this Agreement, neither the Work nor any advertising, promotions, press releases, or other documents affiliated with the Work may contain any claim that You or the Work has been sanctioned or approved by MCG, or is affiliated with MCG in any way.

Neither the Work nor any advertising, promotions, press releases, or other documents affiliated with the Work may contain racist, homophobic, discriminatory, or other repugnant views; overt political agendas or views; depictions or descriptions of criminal violence against children; rape or other acts of criminal perversion; or other obscene material.

The Work may not infringe, wrongfully use, or misappropriate the intellectual property rights of any third party. You hereby indemnify MCG and undertake to defend MCG against and hold MCG harmless from any claims, suits, loss, and damages arising out of alleged infringement, wrongful use, or misappropriation of any third party’s intellectual property by the Work. The indemnification obligations shall survive the termination of this Agreement.

MCG takes no responsibility for the Work. You agree to hold MCG and its officers, partners, and employees harmless in the event that Your publication of the Work results in legal action.

The CSRD, the “Compatible with the Cypher System” logo, the Cypher System logo, the MCG logo, and all other trademarks of MCG belong solely and exclusively to MCG.

If You breach any of the terms of this Agreement, it results in automatic termination of the Agreement. Unless the breach is cured to MCG’s sole satisfaction and such cure is documented by a written agreement between You and MCG within 15 days of breach, You must immediately recall and destroy all existing copies of the Work. You may additionally be subject to damages as a result of breach.

This Agreement shall be construed and governed by the laws of the State of Washington and the laws of the United States. With regard to any disputes arising under this Agreement, the parties hereby submit to the jurisdiction of the courts of the State of Washington and the United States.

END OF AGREEMENT

\printindex

\listoftables

\end{document}