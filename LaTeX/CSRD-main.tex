\documentclass[12pt, a4paper, twocolumn, openright]{book}
\usepackage[pdftex, breaklinks=true]{hyperref}
\usepackage{polski}
\usepackage[utf8]{inputenc}
\usepackage{hyperref}
\usepackage{makeidx}
\usepackage{tabularx}
\usepackage{xcolor }

\definecolor{purple}{HTML}{92268F}

\makeindex

\newcommand{\mytext }[1] {{\color{purple}  \texttt {#1}}}

% \title{Cypher System Reference Document 2024-07-02 (Edycja Polska)}
% \author{Zespół Monte Cook Games\thanks{Strona projektu: \url{https://www.montecookgames.com/cypher-system-open-license/}} \and Szymon ``Kaworu'' Brycki\thanks{\href{mailto:szymon.brycki@gmail.com}{\tt szymon.brycki@gmail.com}}}

\begin{document}

\begin{titlepage}
	\centering
	{\Huge\bfseries\title  CCypher System Reference Document \par}
	\vspace{1cm}
	{\large\itshape 2024-07-02 \par}
	{\large\itshape Edycja polska \par}
	\vspace{1cm}
	{\normalsize Zespół Monte Cook Games,   Szymon ``Kaworu'' Brycki \par}
	\vspace{1cm}
	{\normalsize Licencja: \bfseries Cypher System Open License\par}
	\vspace{1cm}
	{\normalsize Stworzono w technologii \LaTeX \par}
	\vspace{1cm}
	{\large \today \par}
\end{titlepage}

% \maketitle

\tableofcontents

% here go all the chapters

\chapter {Jak grać w Cypher System}

Zasady Cypher System są całkiem proste i cała rozgrywka bazuje na ledwie kilku podstawowych konceptach.

Ten rozdział zapewnia krótkie wyjaśnienie jak grać w tę grę, i jest przydatny dla dopiero uczących się rozgrywki. Kiedy zrozumiesz już podstawowe koncepty, będziesz pewnie chcieć przeczytać \mytext{Zasady Gry} po więcej szczegółów. 
Cypher System korzysta z kości dwudziestościennej (k20) by określić wynik większości akcji. Za każdym razem, gdy wymagany jest rzut, a nie podano kości, rzuć k20.

Mistrz Gry określa stopień trudności danego zadania. Istnieje 10 stopni trudności. Tak więc, trudność można określić na skali od 1 do 10.

Każda trudność ma minimalny wynik powiązany z sobą. Minimalny wynik (inaczej zwany stopniem trudności) to zawsze 3x poziom trudności, więc stopień trudności 1 ma minimalny wynik 3, a stopień trudności 4 ma minimalny wynik 12. By odnieść sukces, należy wyrzucić minimalny wynik lub więcej danego ST. Patrz Tabela Stopnie Trudności po więcej danych.
Umiejętności postaci, przydatne okoliczności lub doskonały ekwipunek mogą zmniejszyć trudność zadania. Dla przykładu, postać wytrenowana we wspinaczce może zamienić trudność 6 testu wspinaczki na trudność 5. Nazywa się to  Ułatwianiem albo Obniżaniem trudności o jeden stopień (albo po prostu Obniżaniem trudności, gdzie przyjmuje się domyślnie, że dotyczy ona jednego stopnia). Jeśli postać jest wyspecjalizowana we wspinaczce, zamienia ona trudność 6 na trudność 4. Nazywa się to Obniżaniem trudności o dwa stopnie. Obniżanie poziomu trudności może także być nazywane ułatwieniem zadania. Niektóre sytuacje zwiększają, lub Utrudniają, trudność zadania. Jeśli zadanie jest utrudnione, należy zwiększyć jego trudność o jeden poziom.

Umiejętność to kategoria wiedzy, zdolności lub aktywności powiązania z zadaniem, np.: wspinaczka, geografia lub perswazja. Postać, która posiada umiejętność, jest lepsza w powiązanych z nią zadaniach niż postać, która nie posiada danej umiejętności. Posta posiada albo wytrenowaną (do pewnego stopnia) umiejętność, albo wyspecjalizowaną (bardzo dużą).
Jeśli jesteś wytrenowany w umiejętności powiązanej z danym zadaniem, ułatwiasz rzut o stopień. Jeśli jesteś wyspecjalizowany, obniżasz poziom trudności o dwa stopnie. Umiejętność nigdy nie może obniżyć trudności testu o więcej niż dwa stopnie. 

Wszystko inne co obniża trudność danego zadania nazywa się Wysiłkiem. (Wysiłek opisano szczegółowo w rozdziale Zasady Gry).

Podsumowując, trzy rzeczy mogą obniżyć trudność zadania: umiejętności, atuty i Wysiłek. 

Jeśli ułatwisz rzut tak mocno, że jego trudność wynosi 0, wtedy automatycznie uzyskujesz sukces i nie musisz rzucać kośćmi. 

\section {Kiedy rzucać kośćmi?}

Za każdym razem, gdy Twoja postać chce wykonać jakieś zadanie, MG daje mu Poziom Trudności i rzucasz k20 przeciwko Stopniowi Trudności powiązanemu z danym Poziomem Trudności.

Kiedy wyskakujesz z płonącego pojazdu, zamachujesz się toporem na zmutowanąbestię, płyniesz poprzez rwącą rzekę, identyfikujesz dziwne urządzenie, przekonujesz handlarza, by dał Ci niższą cenę, tworzysz obiekt, korzystasz z mocy, by kontrolować umysł przeciwnika lub korzystasz z laserowego działka, by zrobić dziurę w ścianie, wykonujesz rzut k20.
Jednakże, jeśli twój Poziom Trudności ma wartość 0, rzut nie jest konieczny – automatycznie uzyskujesz sukces. Wiele akcji ma trudność 0. Przykłady to przejście przez pokój i otworzenie drzwi, skorzystanie ze specjalnej zdolności lotu, korzystanie z mocy, by ochronić swojego przyjaciela przed promieniowaniem, lub aktywowanie urządzenia (które się rozumie) by stworzyć pole siłowe. To wszystko to są rutynowe akcje i nie wymagają one rzutów.

Korzystając z umiejętności, atutów i Wysiłku, można teoretycznie obniżyć trudność dowolnej akcji do 0 i zlikwidować konieczność rzucania kostką. Przejście po wąskim drewnie jest trudne dla większości ludzi, ale nie dla doświadczonego gimnastyka. Możesz nawet obniżyć trudność ataku na swojego wroga do 0 i odnieść sukces bez rzucania.

Jeśli nie ma rzutu, nie ma szansy, by odnieść porażkę. Jednakże, nie ma także szansy na wyjątkowy sukces (w Cypher System zazwyczaj oznacza to wyrzucenie 19 lub 20, co jest znane jako specjalne rzuty; rozdział Zasady Gry omawia je w szczegółach).

\begin{table*}[t]
 \centering
 \begin{tabularx}{\textwidth}{ | X | X | X | X |}
  \hline
   Poziom trudności & Opis & Stopień trudności & Szczegóły  \\ \hline
    0 & Rutyna & 0 & Każdy może to zrobić zawsze \\ \hline
    1 & Proste & 3 & Większość ludzi może to zrobić przez większość czasu  \\ \hline
 \end{tabularx}
  \caption {Tabela: Trudność zadań}
  \label {Tabela: Trudność zadań}
 \end{table*}
 
\section {Walka}\index{Walka!Wstęp}
Wykonywanie ataków w walce działa tak samo jak inne rzuty – MG określa trudność zadania, a następnie należy rzucić k20 przeciwko Stopniowi Trudności.

Trudność Twojego testu ataku zależy od tego, jak bardzo potężny jest przeciwnik. Istoty mają poziomy od 0 do 10, tak jak i zadania, które może wykonać postać. Zazwyczaj trudność rzutu to ST powiązanie z poziomem istoty. Dla przykładu, atak na bandytę 2 poziomu to zadanie o Poziomie Trudności 2, więc Stopień Trudności wynosi 6. 

Trzeba zaznaczyć, że gracze wykonują wszystkie rzuty w Cypher System. Jeśli gracz atakuje istotę, ten gracz wykonuje rzut na atak. Jeśli istota atakuje gracza, to on wykonuje rzut obronny. 

Obrażenia zadawane przez atak nie są definiowane przez rzut kością – jest to stała wartość bazująca na broni lub ataku. Dla przykładu, włócznia zawsze zadaje 4 punkty obrażeń.

Twój Pancerz redukuje obrażenia które otrzymujesz. Otrzymujesz Pancerz za noszenie fizycznej zbroi (takiej jak skórzana kurtka e współczesnym świecie lub pancerz w świecie fantasy) lub ze specjalnych zdolności. Tak jak wartość obrażeń, Pancerz to stała wartość, nie wynik rzutu. Jeśli jesteś zaatakowany, odejmij swój Pancerz od otrzymanych obrażeń. Dla przykładu, skórzana kurtka daje Ci +1 do Pancerza, co oznacza, że otrzymujesz o 1 mniejsze obrażenia z ataków. Jeśli ktoś trafi Cię atakiem nożem za 2 punkty obrażeń, kiedy ją nosisz, otrzymasz tylko 1 punkt obrażeń. Jeśli Pancerz redukuje obrażenia do 0, wtedy nie otrzymujesz w ogóle żadnych obrażeń. 

Kiedy widzisz w zasadach gry słowo „Pancerz” pisane wielką literą, odnosi się do to statystyki Pancerz – do liczby, o którą obniżasz obrażenia. Kiedy widzisz „pancerz” pisany małą literą, dotyczy to dowolnego fizycznego pancerza, który postać może nosić. 

Fizyczne bronie posiadają 3 kategorie: lekkie, średnie i ciężkie.  

Lekkie bronie zadają tylko 2 punkty obrażeń, ale ułatwiają rzuty na atak, ponieważ są szybkie i łatwe w użyciu. Lekkie bronie to ciosy pięścią, kopnięcia, maczugi, noży, toporki ręczne, rapiery, małe pistolety itp. Bronie, które są małe, są broniami lekkimi.
Średnie bronie zadają 4 punkty obrażeń. Średnie bronie to między innymi miecze, topory bojowe, większe maczugi, kusze, włócznie, pistolety, blastery itp. Większość broni to bronie średnie. Wszystko, co może być użyte w jednej dłoni (nawet, jeśli często korzysta się z dwóch, jak w przypadku kostura i włóczni) jest średnią bronią. 

Ciężkie bronie zadają 6 punktów obrażeń, i trzeba korzystać z obydwu dłoni, by z nich korzystać. Ciężkie bronie to wielkie miecze, młoty bojowe, potężne topory, halabardy, ciężkie kusze, karabinki laserowe itp. Wszystko, z czego trzeba korzystać z obydwu dłoni, to ciężkie bronie.

\section {Specjalne wyniki rzutów}\index{Specjalne wyniki rzutów!Wstęp}

Kiedy wyrzucasz naturalne 19 (k20 pokazuje „19”) i test jest sukcesem, uzyskujesz mniejszy efekt. W walce, mniejszy efekt zadaje dodatkowe 3 do obrażeń, lub, jeśli wolisz efekt specjalny, możesz odrzucić wroga do tyłu, rozproszyć jego uwagę lub coś podobnego. Kiedy nie walczysz, mniejszy efekt może oznaczać, że wykonałeś akcję ze stylem. Przykładowo, gdy przeskakujesz przez płot, lądujesz z gracją na własnych stopach, lub gdy przekonujesz kogoś, wierzy on, że jesteś mądrzejszy, niż jesteś naprawdę. W innych słowach, nie tylko osiągasz zwykły sukces, ale także uzyskujesz pomniejszy bonus. 

Kiedy wyrzucasz naturalne 20 (k20 pokazuje „20”) i rzut się powiódł, uzyskujesz dodatkowo większy efekt. Jest to podobne do mniejszego efektu, ale na większą skalę. W walce, zadajesz dodatkowe 4 punkty obrażeń, ale znowu, można zamiast tego wybrać jakiś efekt dodatkowy, taki jak przewrócenie wroga, ogłuszenie go, lub wzięcie akcji dodatkowej. Poza walką, większy efekt oznacza, że dzieje się coś korzystnego, w zależności od okoliczności. Dla przykładu, kiedy wspinasz się na ścianę, robisz to dwa razy szybciej. Kiedy rzut daje Ci większy efekt, możesz zamiast tego skorzystać z mniejszego efektu, jeśli taka jest Twoja wola.

W walce (i tylko wtedy) jeśli rzucisz naturalne 17 lub 18 na rzucie na atak, zadajesz – odpowiednio - dodatkowe 1 lub 2 punkty obrażeń. Te rzuty nie dają żadnych innych specjalnych efektów – tylko zwiększają obrażenia.

(Po więcej informacji o specjalnych wynikach rzutów i tym, jak wpływają na walkę i inne akcje, patrz Zasady Gry).

Wyrzucenie naturalnej 1 jest zawsze złe. To oznacza, że MG wprowadza nowe utrudnienie do sceny. 

\section {Słowniczek}\index{Słowniczek}

{\bfseries Mistrz Gry (MG)}: Gracz, który nie ma własnej postaci, a który zamiast tego kieruje całą fabułą i wszystkimi BN-ami.

{\bfseries Bohater Niezależny (BN)}: Postać kierowana przez MG. Myśl o niej jak o pomniejszej postaci w historii, lub jak o złoczyńcy lub oponencie. Wlicza się w to każde każda istota lub potwór.

{\bfseries Drużyna}: Grupa BG (i może jacyś BN-i sojusznicy).

{\bfseries Bohater Gracza (BG)}: Postać odgrywana przez gracza zamiast przez MG. Myśl o BG jak o głównych bohaterach historii.

{\bfseries Gracz}: Gracz, który kieruje BG.

{\bfseries Sesja}: Pojedyncza doświadczenie roleplayowe. Zazwyczaj trwa kilka godzin. Czasami jedną przygodę można ukończyć w czasie jednej sesji. Częściej, jedna przygoda zajmuje kilka sesji.

{\bfseries Przygoda}: Pojedyncza część kampanii z początkiem i końcem. Zazwyczaj zdefiniowana na początku przez wspólny cel BG i na końcu przez to, czy go osiągnęli, czy też nie. 

{\bfseries Kampania}: Seria sesji połączona wspólną historią (lub połączonymi historiami) z tymi samymi BG. Często, lecz nie zawsze, kampania to zbiór przygód.

{\bfseries Postać}: Cokolwiek, co podejmuje akcje w grze. Choć wliczają się w to BG i ludzcy BN-i, technicznie wliczają się w to potwory, kosmici, mutanci, automatony, ruchome rośliny itp. Synonimem jest „istota” bądź „potwór”.

\section {Zasięg i szybkość}\index{Zasięg i szybkość}

Zasięg dzieli się na 4 ogólnikowe kategorie: bliski, średni, daleki i bardzo daleki.

Bliski zasięg to odległość ręki lub paru kroków. Jeśli postać stoi w małym pokoju, wszystko wokół jest w jej bliskim zasięgu. Górna granica bliskiego zasięgu to 3 metry.

Średni zasięg to wszystko, co jest większe od bliskiego, ale mniejsze niż 15 metrów.

Długi dystans to wszystko większe od średniego zasięgu, ale mniejsze niż 30 metrów.

Bardzo długi zasięg to wszystko większe od długiego dystansu, ale nie większe niż 150 metrów. Poza tym zasięgiem, odległości zawsze się ściśle określone – 300 metrów, 1,5 kilometra itp.

Ogólną ideą tego systemu jest to, że nie trzeba dokładnie mierzyć i określać odległości. Bliski zasięg to tutaj, obok postaci. Średni zasięg to blisko postaci. Długi dystans jest dalej, a bardzo długi – znacznie dalej.
Wszystkie bronie i specjalne zdolności korzystają z tych terminów. Dla przykładu, wszystkie bronie do walki wręcz mają bliski zasięg – służą przecież do walki wręcz i można je użyć tylko na osobach, które stoją obok nas. Nóż do rzucania (i większość innych broni rzucanych) mają średni zasięg. Łuk ma długi zasięg. Pocisk Adepta także ma średni zasięg.

Postać może się przemieścić o bliski zasięg jako część innej akcji. Innymi słowy, może ona podejść do panelu kontrolnego i z niego skorzystać. Może przejść przez mały pokój i zaatakować wroga. Mogę otworzyć drzwi i przejść przez nie.

Postać może się przemieścić o średni zasięg jeśli poświęci na to całą akcję w turze. Może także spróbować się przemieścić o długi zasięg w jednej akcji, ale trzeba wykonać rzut, by stwierdzić, czy postać się nie poślizgnęła lub  przewróciła w efekcie tak szybkiego ruchu.

Dla przykładu, jeśli BG walczą z grupą kultystów, każda postać może zaatakować, ogólnie rzecz ujmując, dowolnego kultystę wręcz – wszyscy są w zasięgu. Dokładne pozycje nie są tak ważne. Istoty w walce zawsze się zresztą poruszają. Jednakże, jeden z kultystów został z tyłu by wystrzelić z pistoletu i BG może musieć poświęcić całą akcję, by się do niego dostać. Nie ma większego znaczenia, czy ten kultysta jest 6 metrów od postaci graczy, czy może 12 – po prostu jest w średnim zasięgu. Ma znaczenie, czy kultysta stoi o więcej niż 15 metrów od BG, ponieważ wtedy zasięg by się zwiększył do dalekiego.

(Wiele zasad w tej grze unika konieczności nadmiernej precyzji. Czy naprawdę się liczy to, czy duch jest o 13, czy o 18 stóp od Ciebie? Najpewniej nie. Taki rodzaj niepotrzebnej ścisłości tylko spowalnia rozgrywkę i odciąga uwagę od akcji i fabuły, zamiast być miłym dodatkiem do opowiadanej historii.)

\section {Punkty doświadczenia}\index{Punkty doświadczenia!Wstęp}

Punkty doświadczenia (PD-ki) są nagrodą dawaną graczom, gdy GM wtrąca się narrację (nazywamy to Wtrąceniem MG) z nowym i niespodziewanym wyzwaniem. Dla przykładu, w środku walki, MG może poinformować graczy, że upuszczają oni swoje bronie. Jednakże, aby się wtrącić w taki sposób, MG musi dać graczowi 2 PD-ki. Nagrodzony gracz, z kolei, musi natychmiast dać jednego z owych PD-ków innemu graczowi, uzasadniając to (może ten gracz miał dobry pomysł, powiedział zabawny żart, wykonał akcję, która ocaliła życie jakiegoś BN/BG itp.).

Alternatywnie, gracz może odrzucić Wtrącenie MG. Jeśli on tak uczyni, nie otrzymuje on 2 PD-ków od GM, i musi wydać i PD z posiadanych przez siebie. Jeśli gracz nie ma PD-ków, nie może odrzucić Wtrącenia MG.

MG może także dać graczom PD-ki pomiędzy sesjami, jako nagrody za dokonywanie odkryć podczas gry. Odkrycia to ciekawe fakty, cudowne sekrety, potężne artefakty, odpowiedzi na pytania lub rozwiązania problemów (np.: gdzie porywacze przetrzymują swoje ofiary lub jak gracze naprawią statek kosmiczny). Nie otrzymujesz PD-ków za zabijanie potworów lub przezwyciężanie zwykłych trudności podczas gry. Odkrycia są duszą Cypher System.

Punkty Doświadczenia głównie służą awansowaniu postaci na poziomy (po detale, patrz: rozdział Tworzenie Własnej Postaci), ale gracz może także wydać 1 PD-ek, by przerzucić kość i wybrać leszy z dwóch wyników. 

\section  {Cyphery}\index{Cyphery!Wstęp}

Cyphery to zdolności, z których można skorzystać tylko raz. W wielu kampaniach, cyphery nie są fizycznymi obiektami – mogą być zaklęciem rzuconym na postać, błogosławieństwem od boga, lub po prostu zrządzeniem losu, które daje chwilową przewagę. W pewnych kampaniach, cyphery to obiekty fizyczne które postaci mogą z sobą nosić. Niezależnie od tego, czy cyphery to przedmioty, czy też nie, są częścią postaci (tak jak ekwipunek lub specjalna zdolność). I są czymś, z czego postać może skorzystać podczas gry. Forma, którą przyjmują fizyczne cyphery, zależy od settingu. W świecie fantasy mogą być różdżkami lub eliksirami, ale w grze science fiction mogą być obcymi kryształami lub prototypowymi technologiami.

Postaci często będą znajdowały nowe cyphery, więc gracze powinni równie często z nich korzystać. Ponieważ cyphery zawsze będą odmienne od innych cypherów, postać zawsze będzie miała nowe specjalne zdolności do wypróbowania. 

\section {Inne kości}

W dodatku do k20, potrzebujesz jeszcze k6 (sześciościennej kostki). Czasami będziesz potrzebował k100 (do losowania numerów od 1 do 100), co można osiągnąć, rzucając k20 dwa razy – ostatnia liczba pierwszego rzutu to “dziesiątki” a ostatnia liczba drugiego rzutu to “jedności”. Dla przykładu, rzut 17 i 9 daje nam 79, a 3 i 18 daje nam 38, a rzucenie 20 i 10 daje nam 00 (także znane jako 100). Jeśli masz k10 (dziesięciościenną kostkę) możesz skorzystać z niej zamiast z k20 by losować liczby od 1 do 100.

(k6 jest najczęściej wykorzystywana do rzutów na odzyskiwanie zdrowia i do określania poziomu cypherów).
\input{src/Tworzenie własnej postaci.tex}

\printindex

\listoftables

\end{document}