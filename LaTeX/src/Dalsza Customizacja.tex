\subsubsection{Dalsza Customizacja}\index{Dalsza Customizacja}

Zasady w tej sekcji są bardziej zaawansowane i zawsze zależą od MG. Mogą być użyte, by MG dostosował typ do konwencji lub settingu, lub przez gracza i MG, by dostosować koncept postaci.

\paragraph{Modyfikacja Aspektów Typu}

Poniższe aspekty czterech typów postaci mogą zostać zmodyfikowane podczas tworzenia postaci. Inne zdolności nie powinny być zmienione.

Pule Statystyk: Każda Pula postaci ma wartość startową. Gracz może zamieniać punkty w Pulach kosztem 1-na-1. Dla przykładu, może on przesunąć 2 punkty z Mocy na 2 punkty w Szybkości. Jednakże, żadna początkowa Statystyka nie może być wyższa niż 20.

Skupienie: Gracz może zacząć grę ze Skupieniem w dowolnej Statystyce na 1.

Korzystanie z Cypherów: Jeśli gracz odda zdolność noszenie jednego cyphera, uzyskuje on dodatkową umiejętność swojego wyboru.

Bronie: Pewne typy mają statyczne zdolności pierwszego poziomu które pozwalają im korzystać z lekkich, średnich i/lub ciężkich broni bez kary. Wojownicy mogą korzystać z wszystkich broni, Odkrywcy z lekkich i średnich, a Adepci i Mówcy mogą korzystać tylko z lekkich broni. Każda z tych zdolności może zostać poświęcona, by zyskać trening w odmiennej umiejętności, którą wybierze gracz.

\paragraph{Wady i Kary}

W dodatku do innych opcji customizacji, gracz może wybrać wzięcie kar lub wad, by zyskać dalsze bonusy.

Słabość: Słabość to, esencjalnie, przeciwieństwo Skupienia. Jeśli masz Słabość 1 w Szybkości, wszystkie akcje Szybkości wymagają od Ciebie dodatkowego 1 punktu z Twojej Puli. W każdym momencie, gracz może dać swojej postaci słabość w jednej statystycei otrzymać +1 do Skupienia w jednej z pozostałych dwóch. Tak więc gracz może wziąć słabość 1 w Szybkości i otrzymać +1 do swojego Skupienia w Mocy.

Normalnie, możesz mieć słabość tylko w statystyce, w której Twoje Skupienie wynosi 0. Co więcej, nie możesz mieć więcej niż jednej słabości, i niem ożesz mieć słabości większej niż 1, chyba, że dodatkowa słabość pochodzi z innego źródła (takiego jak zaraza lub niepełnosprawność wynikająca z akcji lub kondycji w grze).

Nieumiejętność: Nieumiejętności są jak negatywne umiejętności. Czynią jeden rodzaj akcji trudniejszym. Jeśli postać wybiera nieumiejętność, zyskuje ona umiejętność swojego wyboru. Normalnie, postać może mieć tylko jedną nieumiejętność, chyba, że pozostałe pochodzą z innego źródła (takiego jak deskryptor, choroba lub niepełnosprawność wynikająca z akcji lub kondycji w grze).

\paragraph{Posmaki}\index{Posmaki}

Posmaki to grupa specjalnych zdolności które MG i gracze mogą wykorzystać, aby zmienić typ postaci – np.: w zgodzie z settingiem lub konwencją. Dla przykładu, jeśli gracz chce stworzyć czarodzieja, który jest też złodziejem, może zagrać Adeptem z Posmakiem w skradaniu się. W settingu science fiction, Wojownik może mieć także wiedzę o maszynach, więc postać może mieć posmak w technologii. 

Na danym poziomie, zdolności z standardowego typu są wymieniane na zdolności z posmaku. Tak więc, aby dodać Zmysł Niebezpieczeństwa z posmaku skradanie się do Wojownika, trzeba poświęcić coś innego – być może Ogłuszenie. Teraz postać może wybrać Zmysł Niebezpieczeństwa, tak jak każdą inną zdolność pierwszego poziomu, ale nigdy nie może wziąć Ogłuszenia.

MG zawsze powinien brać udział w modyfikacji typu za pośrednictwem posmaku. Dla przykładu, może on określić, że w grze science fiction chce stworzyć type zwany “Glam”, czyli Mówcę z pewnymi zdolnościami technologicznymi – konkretniej tymi, które czynią z niego ekstrawaganckiego pilota statków kosmicznych. Tak więc, zamienia on pierwszopoziomowe zdolności Fałszywa Tożsamość i Zainspirowanie Agresji na Implant Wizualnej Identyfikacji i Umiejętności Technologiczne, tak, że postać może połączyć siębezpośrednio ze statkiem i mieć umiejętności komputerowe i pilotażu.

Ostatecznie, posmak to głównie narzędzia dla MG do łatwego tworzenia typów zrobionych pod kampanie, poprzez parę lekkich zmian tu i ówdzie. Choć gracze mogą chcieć skorzystać z posmaków, by stworzyć postać, jakiej pragną, pamiętaj, że mogą oni także dookreślić swojego BG przy pomocy deskryptorów i specjalizacji. 

Dostępne posmaki to: skradanie się, technologia, magia, walka oraz umiejętności i wiedza.
Pełen opis wszystkich zdolności można znaleźć w rozdziale Zdolności, który zawiera także opisy zdolności typów i specjalności w jednym pokaźnym katalogu.

\subparagraph{Posmak - Skradanie się}\index{Posmaki!Skradanie się}

Postaci z posmakiem skradanie się są dobre w skradaniu, infiltracji rożnych miejsc, gdzie nie powinny być i oszukiwaniu innych. Korzystają z tych zdolności na rożne sposoby, wliczając walkę. Odkrywca z posmakiem skradanie się może być złodziejem, a Wojownik zabójcą. Odkrywca z posmakiem skradanie się w settingu superbohaterskim może być pogromcą przestępców, który chodzi po ulicach nocami.

\textbf{Poziom pierwszy}

\begin{itemize}
\item Zmysł Niebezpieczeństwa
\item Prowokacja
\item Zwinne Dłonie
\item Oportunista
\item Umiejętności Złodzieja
\end{itemize}

\textbf{Poziom drugi}

\begin{itemize}
\item Człowiek-Guma
\item Wyczulenie na Okazję
\item Ucieknij
\item Niemożliwy do Zaskoczenia
\item Atak z Zaskoczenia
\end{itemize}

\textbf{Poziom trzeci}
\begin{itemize}
\item Zniknięcie
\item Z Cieni
\item Ryzykant
\item Wewnętrzna Obrona
\item Przekierowanie Ataku
\item Bieg i Walka
\item Chwytaj Moment
\end{itemize}

\textbf{Poziom czwarty}
\begin{itemize}
\item Czatownik
\item Bolesne Uderzenie
\item Przechytrzenie
\item Wyczulone Zmysły
\item Błyskotliwe Ruchy
\end{itemize}

\textbf{Poziom piąty}
\begin{itemize}
\item Cios Strytobójcy
\item Maska
\item Kontratak
\item Niezwykłe Szczęście
\end{itemize}

\textbf{Poziom szósty}
\begin{itemize}
\item Wykorzystanie Przewagi
\item Odsunięcie się
\item Szczęście Złodzieja
\item Zmiana Przeznaczenia
\end{itemize}

\subparagraph{Posmak - Technologia}\index{Posmaki!Technologia}

Postaci z posmakiem technologii zazwyczaj występują w settingach sci-fi lub przynajmniej współczesnych (choć wszystko jest możliwe). Dobrze im idzie używanie, radzenie sobie z i tworzenia maszyn. Odkrywca z posmakiem technologii może być pilotem statku kosmicznego, a Mówca nano-kapłanem.

Pewnie z mniej zorientowanych na komputery zdolności tego posmaku mogą pasować do steampunka, a postać żyjąca współcześnie może wykorzystać te zdolności, które nie są powiązane ze statkami kosmicznymi i ultra-technologią.

\textbf{Poziom pierwszy}
\begin{itemize}
\item Implant Wizualnej Identyfikacji
\item Haker
\item Interfejs Maszyn
\item Popsucie Maszyny
\item Umiejętności Technologiczne
\item Majsterkowanie
\end{itemize}

\textbf{Poziom drugi}
\begin{itemize}
\item Interfejs Zasięgowy
\item Wydajność Maszyny
\item Przeciążenie Maszyny
\item Serv-0
\item Serv-0 Obrońca
\item Naprawa Serv-0
\item Mistrzostwo Narzędzi
\end{itemize}

\textbf{Poziom trzeci}
\begin{itemize}
\item Mechaniczna Telepatia
\item Skany Serv-0
\item Obeznanie ze Statkiem Kosmicznym
\item Mowa Statku Kosmicznego
\item Ostrzał Ciągły
\end{itemize}

\textbf{Poziom czwarty}
\begin{itemize}
\item Więź z Maszyną
\item Walczący z Robotami
\item Celowanie Serv-0
\item Serv-0 Wojownik
\item Serv-0 Szpieg 
\end{itemize}

\textbf{Poziom piąty}
\begin{itemize}
\item Kontrola Maszyny
\item Naprawa na Oko
\item Kompan-Maszyna
\end{itemize}

\textbf{Poziom szósty}
\begin{itemize}
\item Zbieranie Informacji
\item Mistrz Maszyn 
\end{itemize}

\subparagraph{Posmak - Magia}\index{Posmaki!Magia}

Znasz się trochę na magii. Możesz nie być czarodziejem, ale znasz podstawy – jak to działa, i jak zrobić parę wspaniałych rzeczy. Oczywiście, w Twoim settingu, “magia” może oznaczać moce psioniczne, moce mutantów, dziwną technologięobcych lub cokolwiek innego. Odkrywca z posmakiem magii może być czarodziejem-łowcą, a Mówca z posmakiem magii może być zaklinaczem-bardem. Choć Adept z posmakiem magii jest dalej Adeptem, możesz odkryć, że zamiana paru jego zdolności na poniższe dopieszcza Twoją postać tak, jak sobie tego wymarzyłeś. 

\textbf{Poziom pierwszy}
\begin{itemize}
\item Błogosławieństwo Bóstw
\item Umysł-Twierdza
\item Macki Mocy
\item Sztuczki Magiczne
\item Trening Magiczny
\item Link Mentalny
\end{itemize}

\textbf{Poziom drugi}
\begin{itemize}
\item Promień Odrzucający
\item Przywołanie
\item Pole Siłowe
\item Kłódka
\item Naprawa Ciała
\end{itemize}

\textbf{Poziom trzeci}
\begin{itemize}
\item Dalekie Spojrzenie
\item Kwiat Ognia
\item Rzut
\item Moc na Odległość
\item Przywołanie Wielkiego Pająka
\end{itemize}

\textbf{Poziom czwarty}
\begin{itemize}
\item Obrona Przed Żywiołami
\item Zapłon
\item Przełamanie Obrony
\end{itemize}

\textbf{Poziom piąty}
\begin{itemize}
\item Stworzenie
\item Boska Interwencja
\item Szczęki Smoka
\item Szybka Podróż
\item Prawdziwe Zmysły
\end{itemize}

\textbf{Poziom szósty}
\begin{itemize}
\item Relokacja
\item Przywołanie Demona
\item Podróż Między Światami
\item Słowo Śmierci
\end{itemize}

\subparagraph{Posmak - Walka}\index{Posmaki!Walka}

Posmak walka czyni postać bardziej śmiercionośną. Mówca z tym posmakiem w settingu fantasy może być wojennym bardem. Badacz z posmakiem walka w historycznym świecie może być piratem. Adept z tym posmakiem może w świecie science fiction być weteranem tysiąca psionicznych wojen.


\textbf{Poziom pierwszy}
\begin{itemize}
\item Zmysł Niebezpieczeństwa
\item Przywykły do Noszenia Zbroi
\item Wyszkolony w Średnich Broniach
\end{itemize}

\textbf{Poziom drugi}
\begin{itemize}
\item Zew Krwi
\item Zdolności Bojowe
\item Wyszkolony Bez Zbroi
\end{itemize}

\textbf{Poziom trzeci}
\begin{itemize}
\item Wyszkolony we Wszystkich Broniach
\item Umiejętny Atak
\item Umiejętna Obrona
\item Następny Atak
\end{itemize}

\textbf{Poziom czwarty}
\begin{itemize}
\item Zręczny Wojownik
\item Śmiertelna Salwa
\item Furia
\item Przekierowanie Ataku
\item Ostrzał Ciągły
\end{itemize}

\textbf{Poziom piąty}
\begin{itemize}
\item Doświadczony Obrońca
\item Trudny Cel
\item Blokowanie
\end{itemize}

\textbf{Poziom szósty}
\begin{itemize}
\item Większa Umiejętność Ataku
\item Mistrzowska Biegłość w Pancerzach
\item Mistrzostwo Obrony
\end{itemize}

\subparagraph{Posmak - Umiejętności i Wiedza}\index{Posmaki!Umiejętności i Wiedza}

Ten posmak jest dla postaci, które posiadają wiedzę i bardziej realistyczne aplikacje swoich talentów. Jest mniej kinematyczny i dramatyczny niż nadnaturalne zdolności lub moc zaatakowania naraz kilku wrogów, ale czasami doświadczenie lub know-how jest dobrym rozwiązaniem problemów. Wojownik z posmakiem umiejętności i wiedza może być wojskowym inżynierem. Badacz może być polowym naukowcem. Mówca z tym posmakiem może być nauczycielem.

\textbf{Poziom pierwszy}
\begin{itemize}
\item Umiejętności Międzyludzkie
\item Umiejętności Śledcze
\item Umiejętności Wiedzy
\item Umiejętności Fizyczne
\item Umiejętności Podróżnicze
\end{itemize}


\textbf{Poziom drugi}
\begin{itemize}
\item Dodatkowa Umiejętność
\item Mistrzostwo Narzędzi
\item Zrozumienie
\end{itemize}

\textbf{Poziom trzeci}
\begin{itemize}
\item Skupienie na Umiejętności
\item Improwizacja
\end{itemize}

\textbf{Poziom czwarty}
\begin{itemize}
\item Wiele Umiejętności
\item Szybki Umysł
\item Specjalizacja w Zadaniu
\end{itemize}

\textbf{Poziom piąty}
\begin{itemize}
\item Wyszkolony w Średnich Broniach
\item Czytając Znaki
\end{itemize}

\textbf{Poziom szósty}
\begin{itemize}
\item Umiejętny Atak
\item Umiejętna Obrona
\end{itemize}

