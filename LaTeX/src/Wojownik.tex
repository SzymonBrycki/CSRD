\subsection{Wojownik}\index{Typ!Wojownik}

Fantasy/Baśń: Wojownik, rycerz, barbarzyńca, żołnierz, walkyria.

Współczesność/Horror/Romans: policjant, żołnierz, strażnik, detektyw, ochroniarz, atleta.

Science fiction: oficer bezpieczeństwa, wojownik, żołnierz, najemnik.

Superbohaterowie/Post-apokalipsa: bohater. 

Jesteś dobrym sprzymierzeńcem w potyczce. Wiesz, jak korzystać z broni i chronić siebie. W zależności od konwencji i settingu, może to znaczyć, że nosisz miecz i tarczę na arenie gladiatorów, posiadasz karabin maszynowy i zestaw granatów przydatne w wymianie ognia, lub posiadasz blastera i zasilany pancerz, z których korzystasz na obcej planeci

Rola w grze: Wojownicy są fizyczni i zorientowani na akcję. Cześciej rozwiązują problemy, korzystając z siły, niż na inne sposoby, i często wybierają najprostszą drogę do osiągnięcia swoich celów. 

Rola w drużynie: Wojownicy najczęściej zadają i biorą na klatę najwięcej obrażeń w bitwie. To od nich zależy obrona reszty członków drużyny przed atakami. To czasami oznacza, że wojownicy bywają liderami, przynajmniej w walce i w obliczu innych niebezpieczeństw.

Rola społeczna: Wojownicy nie zawsze są żołnierzami lub najemnikami. Każdy, kto jest gotów do odrobinę przemocy w swoim życiu, lub choćby jej potencjał, może być Wojownikiem, mówiąc ogólnikowo. Wliczają się w to strażnicy, policjanci, marynarze lub ludzie innych profesji, którzy wiedzą, jak się bronić.

Zaawansowani Wojownicy: W miarę, jak wojownicy awansują na poziomy, ich umiejętności bitewne – zarówno obrony, jak i ataku – zwiększają się do niemożliwych poziomów. Na wyższych poziomach, mogą oni często przeciwstawić się grupom wrogów lub stanąć 1-na-1 przeciwko dowolnemu przeciwnikowi.

\subsubsection{Historia Wojownika}

Twój typ pomaga Ci określić połączenie Twojej postaci z settingiem. Rzuć k20 lub wybierz z poniższej listy, by określić pewien fakt odnośnie Twojej historii, który łączy Twoją postać ze światem. Możesz także stworzyć swój własny fakt historyczny.

\begin{table*}[t]
 \centering
 \begin{tabularx}{\textwidth}{| p{0.10\textwidth} | X |}
  \hline
  \textbf{d20} & \textbf{Historia Wojownika}  \\ \hline
    1 & Byłeś w armii i dalej masz przyjaciół, którzy tam są. Twój były dowódca dobrze Cię pamięta. \\ \hline
    2 & Byłeś ochroniarzem bogatej kobiety, która oskarżyła Cię o kradzież. Opuściłeś jej służbę w cieniu podejrzeń. \\ \hline
    3 & Byłeś ochroniarzem lokalnego baru, i właściciele pamiętają Cię.  \\ \hline
    4 & Trenowałeś z szanowanym mentorem. Trzyma on Cię w estymie, ale ma wielu wrogów. \\ \hline
    5 & Trenowałeś w odosobnionym zakonie. Mnisi myślą o Tobie jak o bracie, ale dla wszystkich innych jesteś obcym. \\ \hline
    6 & Nie masz formalnego wyszkolenia. Twoje zdolności po prostu są (naturalnie bądź nie). \\ \hline
    7 & Spędziłeś czas na ulicach i byłeś przez pewien czas w więzieniu. \\ \hline
    8 & Zapisano Cię do służby wojskowej, ale uciekłeś po niedługim czasie. \\ \hline
    9 & Służyłeś jako ochroniarz dla potężnego kryminalisty, który teraz jest Ci winny życie. \\ \hline
    10 & Pracowałeś jako oficer policji lub detektyw. Każdy Cię zna, ale opinie o Tobie są różne. \\ \hline
    11 & Twoje starsze rodzeństwo to niesłynna postać, która żyje w hańbie.  \\ \hline
    12 & Służyłeś jako strażnik komuś, kto dużo podróżował. Znasz ludzi w wielu miejscach. \\ \hline
    13 & Twój najlepszy przyjaciel to nauczyciel lub badacz. Jest on świetnym źródłem wiedzy. \\ \hline
    14 & Ty i Twój przyjaciel palicie ten sam rodzaj rzadkiego, drogiego tytoniu. Spotykacie się co tydzień, by porozmawiać i zapalić. \\ \hline
    15 & Twój wuj prowadzi teatr w mieście. Znasz wszystkich aktorów i masz wolny wstęp na występy. \\ \hline
    16 & Twój przyjaciel-rzemieślnik czasami prosi Cię o pomoc. Jednakże, płaci on dobrze. \\ \hline
    17 & Twój mentor napisał książkę o sztukach walki. Czasami ludzie pragną Cię odszukać i zapytać o jej dziwne zapisy. \\ \hline
    18 & Ktoś, z kim walczyłeś ramię w ramię w armii, teraz jest burmistrzem lokalnego miasteczka. \\ \hline
    19 & Ocaliłeś życie rodziny, gdy jej dom płonął. Mają oni u Ciebie dług, a ich sąsiedzi traktują Ciebie jak bohatera. \\ \hline
    20 & Twój stary trener dalej spodziewa się, że wrócisz i sprzątniesz po jego zajęciach; gdy to robisz, dzieli się on z Tobą okazjonalnie ciekawymi plotkami. \\ \hline
 \end{tabularx}
  \caption {Historia Wojownika}
  \label {Historia Wojownika}
 \end{table*}
 
\subsubsection{Wojownik - Wtrącenia Gracza}

Możesz wydać 1 PD by skorzystać z poniższych wtrąceń gracza, jeśli jest to stosowne do sytuacji, a MG się zgodzi.

Perfekcyjna pozycja: Walczysz przynajmniej z trzema wrogami i każdy z nich stoi w odpowiednim miejscu, możesz więc wykorzystać ruch, który ćwiczyłeś dawno temu, co pozwala Ci zaatakować wszystkich trzech w jednej akcji. Wykonaj odrębne rzuty na atak dla kazego z wrogów. Jesteś ograniczony Wysiłkiem, który możesz wykorzystać w jednej akcji.

Stary Przyjaciel: Towarzysz borni z przeszłości pojawia sięnagle i pomaga w tym, co teraz robisz. Jest on na własnej misji i nie może zostać dłużej niż czas potrzebny na udzielenie pomocy, porozmawianie przez chwilę i być może na wspólne zjedzenie szybkiego posiłku.

Słabość Broni: Broń Twojego przeciwnika ma słaby punkt. Podczas walki, szybko się ona psuje i spada o dwa stopnie w dół na \mytext{liczniku obrażeń przedmiotu}.

\begin{table*}[t]
 \centering
 \begin{tabularx}{\textwidth}{ | X | X |}
  \hline
   \textbf{Statystyka} & \textbf{Początkowa Wartość Puli}  \\ \hline
    Moc & 10  \\ \hline
    Szybkość & 10  \\ \hline
    Intelekt & 8  \\ \hline
 \end{tabularx}
  \caption {Pule Statystyk Wojownika}
  \label {Pule Statystyk Wojownika}
 \end{table*}
 
 Otrzymujesz dodatkowe 6 punktów do podziału pomiędzy Pule, jakkolwiek sobie życzysz.
 
\subsubsection{Wojownik pierwszego poziomu}

Pierwszo-poziomowi wojownicy mają następujące zdolności:

Wysiłek: Twój Wysiłek to 1.

Fizyczna Natura: Masz Skupienie w Mocy 1 i Skupienie w Szybkości 0 lub Skupienie w Mocy 0 i Skupienie w Szybkości 1. Niezależnie od tego, Twoje Skupienie w Intelekcie to 0.

Korzystanie z Cypherów: Możesz nosić dwa Cyphery w danym czasie.

Bronie: Jesteś wytrenowany w lekkich, średnich i ciężkich broniach i nie stosuje siędo Ciebie kara za używanie jakiegokolwiek rodzaju broni. Umożliwienie.

Początkowy Ekwipunek: Odpowiednie ubranie i dwie bronie Twojego wyboru, plus jeden drogi przedmiot, dwa przedmioty średniej ceny i cztery niedrogie.

Specialne Zdolności: Wybierz cztery zdolności z poniższej listy. Nie możesz wybrać tej samej zdolności więcej niż raz, chyba, że jej opis mówi inaczej. Pełny opis wszystkich zdolności znajduje się w rozdziale \mytext{Zdolności}, który także zawiera opis Posmaków i zdolności Specjalizacji w pojedynczym, sporym katalogu.

\begin{itemize}
\item Broń Niepotrzebna
\item Kontrola Bitewna
\item Na Straży
\item Ogłuszenie
\item Szybki Rzut
\item Ulepszone Skupienie
\item Umiejętności Fizyczne
\item Wyszkolony Bez Zbroi
\item Wyszkolony w Zbroi
\item Zamach
\item Zdolności Bojowe
\end{itemize}

\subsubsection{Wojownik Drugiego Poziomu}

Wybierz dwie zdolności z poniższej lisy (lub z niższego poziomu) i dodaj je do swoich zdolności. Dodatkowo, możesz zamienić jedną ze zdolności niższego poziomu na inną z niższego poziomu.

\begin{itemize}
\item Krwawienie
\item Miażdżący Cios
\item Następny Atak
\item Przeładowanie
\item Umiejętny Atak
\item Umiejętna Obrona
\end{itemize}

\subsubsection{Wojownik Trzeciego Poziomu}

Wybierz trzy zdolności z poniższej listy (lub z niższego poziomu) i dodaj je do swoich zdolności. Dodatkowo, możesz zamienić jedną ze zdolności niższego poziomu na inną z niższego poziomu.

\begin{itemize}
\item Chwytaj Moment
\item Cięcie
\item Cios z Wyciągnięciem
\item Czujność
\item Ekspercki Użytkownik Cypherów
\item Furia
\item Odporność na Energię
\item Ostrzał Ciągły
\item Podwójny Strzał
\item Przywykły do Noszenia Zbroi
\item Reakcja
\item Śmiertelna Salwa
\end{itemize}

\subsubsection{Wojownik Czwartego Poziomu}

Wybierz dwie z poniższych zdolności (lub z niższego poziomu) i dodaj je do swoich zdolności. Dodatkowo, możesz zamienić jedną z zdolności niższego poziomu na inną z niższego poziomu.

\begin{itemize}
\item Dodatkowy Wysiłek
\item Doświadczony Obrońca
\item Finta
\item Pęd
\item Przełamanie Obrony
\item Wycelowanie
\item Wyjątkowo Wytrzymały
\item Zręczny Wojownik
\item Zwiększony Efekt
\end{itemize}

\subsubsection{Wojownik Piątego Poziomu}

Wybierz trzy zdolności z poniższej listy (lub z niższego poziomu) i dodaj je do swoich zdolności. Dodatkowo, możesz zamienić jedną ze zdolności niższego poziomu na inną zdolność niższego poziomu.

\begin{itemize}
\item Atak z Wyskoku
\item Blok
\item Mistrzostwo Ataków
\item Mistrzostwo Obrony
\item Mistrzowska Biegłość w Pancerzach
\item Ulepszony Sukces
\item Potrójny Wystrzał
\item Zaawansowany Użytkownik Cypherów
\end{itemize}

(Pamiętaj, że na wyższych poziomach, można wybrać zdolności z niższych poziomów. Czasami jest to najlepszy sposób, by uzyskać dokładnie taką postać, jakiej pragniesz. Jest to zwłaszcza prawdziwe odnośnie zdolności, które zapewniają umiejętności, które zazwyczaj można wybrać więcej niż jeden raz.)

\subsubsection{Wojownik Szóstego Poziomu}

Wybierz dwie ze zdolności z poniższej listy (lub z niższego poziomu) i dodaj je do swoich zdolności. Dodatkowo, możesz zamienić jedną ze zdolności niższego poziomu na inną z niższego poziomu.

\begin{itemize}
\item Broń i Cios
\item Chwila Wspaniałości
\item Morderca
\item Ostateczny Cios
\item Wielokrotny Atak
\item Znowu i Znowu
\end{itemize}

\subsubsection{Przykładowy Wojownik}

Ray chce stworzyć Wojownika do współczesnej kampanii. Decyduje się on na byłego członka armii, który jest silny i szybki. 3 z wolnych punktów idą do Puli Mocy, a pozostałe 3 do Puli Szybkości. Jego Statystyki to teraz Moc 13, Szybkość 13 i Intelekt 8. Jako, że postać jest na 1-szym  poziomie, jej Wysiłek to 1, jej Skupienie w Mocy to 1, a Skupienie w Szybkości i Intelekcie to 0. Jego postać nie jest szczególnie mądra lub charyzmatyczna.

Chce on korzystać z dużego noża bojowego (średnia broń, która zadaje 4 punkty obrażeń) i .357 Magnum (ciężki pistolet, który zadaje 6 punktów obrażeń, ale wymaga dwóch rąk do korzystania). Ray decyduje się na nie noszenie żadnej zbroi, gdyż nie pasuje to do settingu, tak więc, jako swoją pierwszą zdolność wybiera \mytext{Wyszkolony Bez Zbroi}, co ułatwia jego Obronę Szybkości. Jako drugą zdolność wybiera \mytext{Zdolności Bojowe}, by zadawać większe obrażenia swoim wielkim nożem. 

Ray chce być zarówno szybki, jak i wytrzymały, wybiera więc \mytext{Ulepszone Skupienie}. Daje mu to Skupienie w Szybkości na 1. Jako ostatnią zdolność, wybiera \mytext{Umiejętności Fizyczne} i wybiera pływanie i bieganie. 

Wojownik może mieć przy sobie maksymalnie 2 cyphery. GM decyduje, że pierwszy cypher Raya to pigułka, która regeneruje 6 punków Mocy po połknęciu, a jego drugi cypher to mały, łatwy do ukrycia granat, który eksploduje jak ognista bomba, gdy go się rzuci, zadając 3 punkty obrażeń wszystkim w bliskim zasięgu. 

Ray dalej musi wybrać deskryptor i specjalizację. Przeglądając deskryptory, Ray wybiera \mytext{Silnego}, co zwiększa jego Pulę Mocy do 17. Jest także wytrenowany w skakaniu i niszczeniu przedmiotów. (Jeśli Ray wybrałby skakanie jako jedną ze swoich umiejętności fizycznych, teraz dzięki deskryptorowi byłby wyspecjalizowany w skakaniu, zamiast być wytrenowanym). Bycie Islnym daje też Ray’owi dodatkową średnią bądź ciężką broń. Wybiera kij baseballowy, który przechowuje w bagażniku swojego auta. 

Jako swoją specjalizację, Ray wybiera \mytext{Mistrzowsko Posługuje się Bronią}. Daje mu to kolejną broń wysokiej jakości. Wybiera dodatkowy nóż bojowy i pyta MG, czy może z niego korzystać w lewej ręce – nie do wykonywania ataków, lecz jako tarczę. To ułatwi jego rzuty na Obronę Szybkości, jeśli ma obydwie bronie w rękach (“tarcza” liczy się jako atut). MG się zgadza. Podczas gry, ciężko będzie trafić Wojownika Ray’a – kjest wytrenowany w rzutach na Obronę Szybkości, a jego dodatkowy nóż obniża rzuty o kolejny stopień. 

Dzięki jego specjalizacji, zadaje także dodatkowy 1 punkt obrażeń w walce swoją wybranąbronią. Teraz zadaje 6 punktów obrażeń swoim ostrzem. Postać Raya to śmiercionośny wojownik, zapewne rozpoczynający grę z reputacją jako walczący nożami. 

Jako swój motyw fabularny, Ray wybiera \mytext{Pokonać Wroga}. Ten wróg, Ray decyduje, to nikt inny jak jego stary przyjaciel z armii, który wszedł na ścieżkę zła.