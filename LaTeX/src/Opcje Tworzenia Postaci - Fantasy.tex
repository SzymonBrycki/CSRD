\subsubsection{Opcje Tworzenia Postaci - Fantasy}\index{Opcje Tworzenia Postaci!Fantasy}

W pewnych przypadkach, poniższe pomysły wymagają pewnych zmian zgodnie z Posmakiem, co opisano w opcjach postaci; powinieneś pracować ze swoim MG w celu aplikacji owych zmian, zgodnie z duchem kampanii. Większość specjalizacji w tej sekcji występuje w Cypher System – specjalizacje z gwiazdką (*) można znaleźć dalej w tym dokumencie. Niektóre z tych opcji sugerują zamianę zdolności z typu na zdolność z Posmaku takiego jak walka, magia lub skradanie się.

Alchemik: W rozumieniu tego, że alchemik to ktoś, kto robi magiczne przedmioty i tym podobne, Adept i Odkrywca to odpowiednie typy dla alchemika-naukowca. Aby stworzyć ogólnego alchemika, który robi mikstury z magicznymi właściwościami, wybierz specjalizację Włada Zaklęciami (zamiast zaklęć, masz eliksiry). Aby stworzyć alchemika, który zamienia się w potężną i niebezpieczną istotę, wybierz Wyje do Księżyca. Dla alchemika, który kocha rzucać bombami, wybierz Nosi Halo Ognia. Aby stworzyć uzdrowiciela, wybierz Uzdrawia.

Barbarzyńca: Barbarzyńca to najpewniej Wojownik lub (jeśli wolisz się skupić nie tylko na walce) Odkrywca. Dobre specjalizacje ,które można wybrać, to: Żyje w Dziczy, Mistrzowsko Posługuje się Bronią, Nie Potrzebuje Broni, Nigdy się Nie Poddaje, Jest Bardzo Silny i Wpada w Furię. 

Bard: Bardowie w fikcji fantasy i grach są trubadurami, minstrelami i opowiadaczami historii, być może z magicznymi zdolnościami. Bardowie to zazwyczaj Odkrywcy lub Mówcy. Odpowiednie specjalizacje to: Zabawia, Pomaga Swoim Przyjaciołom, Infiltruje i Włada Zaklęciami.

Kleryk lub Kapłan: Kapłani z dobrym wykształceniem to zazwyczaj Adepci lub Mówcy, ale wojowniczy są zazwyczaj Wojownikami (możliwe, że z Posmakiem magia). Aby stworzyć typowego kleryka z szerokim wachlarzem zdolności, wybierz specjalizację Otrzymuje Boskie Błogosławieństwo.

\begin{itemize}
\item Kleryk (burza): Ujeżdża Błyskawicę, Grzmi
\item Kleryk (oszustwo): Przyjmuje Zwierzęcy Kształt (patrz także opcje dla łotrzyków) 
\item Kleryk (śmierć): Zadaje się z Martwymi, Mówi z Duchami
\item Kleryk (światło): Jaśnieje Światłem, Otrzymuje Boskie Błogosławieństwo
\item Kleryk (wiedza): Szybko się Uczy, Jest Jasnowidzem, Wolałby Czytać
\item Kleryk (wojna): Mistrzowsko Posługuje się Bronią (patrz także opcje dla wojowników)
\item Kleryk (życie): Chroni Słabszych, Wspiera Społeczność, Uzdrawia
\end{itemize}

Zabójca/Szpieg: Odkrywca i Wojownik są dobrymi typami dla takiej postaci. Stosowne specjalizacje to Mistrzowsko Posługuje się Bronią, Porusza się jak Kot, Morduje i Pracuje w Ciemnych Uliczkach.

Druid: Jako bardzo specyficzny rodzaj kapłana natury, druid to zazwyczaj Adept lub Odkryca (obydwie opcje być może z Posmakiem magii). Typowy druid to ma najpewniej specjalność Otrzymuje Boskie Błogosławieństwo lub Żyje w Dziczy, a;e po bardziej specyficzne opcje, patrz niżej:

\begin{itemize}
\item Druid (transformacja): Jest Stworzony z Kamienia,  Przyjmuje Zwierzęcy Kształt*, Spaceruje w Dzikich Lasach*
\item Druid (więź z naturą): Mówi Głosem Ziemi
\item Druid (zwierzęcy towarzysz): Kontroluje Bestie, Włada Rojem
\item Druid (żywiołak): Jest Stworzony z Kamienia, Nosi Halo Ognia, Porusza się jak Wiatr, Ujeżdza Błyskawicę,  Rides the Lightning, Przywdziewa Połyskliwy Lód
\end{itemize}

Wojownik: Jak sama nazwa wskazuje, wojownik prawie zawsze będzie Wojownikiem, ale niektórzy to Badacze. Typowy wojownik najpewniej posiada bezpośrednią specjalność, taką jak Mistrzowski Posługuje się Bronią lub Dzierży Magiczną Broń*. Po dodatkowe opcje w zależności od specjalizacji, patrz poniżej:

\begin{itemize}
\item Wojownik (strażnik): Nosi Egzotyczną Tarczę, Chroni Wrót, Masters Defense, Nigdy się Nie Poddaje, Jest Jednoosobowym Bastionem.
\item Wojownik (walka na dystans): Ma Licencję na Broń, Rzuca ze Śmiertelną Dokładnością
\item Wojownik (wręcz): Walczy Nieczysto, Walcząc, Porywa Tłum, Szuka Kłopotów, Nie Potrzebuje Broni, Dzierży Dwie Bronie Naraz
\end{itemize}

Rewolwerowiec: Rewolwerowiec to najpewniej Wojownik lub Eksplorer, ale niektórzy są Mówcami z Posmakiem walki. Stosowne specjalności to Ma Licencję na Broń, Mistrzowsko Posługuje się Bronią, Pływał z Piratami i Dzierży Magiczną Broń*.

Inkwizytor: Inkwizytorzy to zazwyczaj Odkrywcy, Mówcy lub Wojownicy, w zależności od tego, czy gracz chce mieć wiele umiejętności, być dobrym w interakcji społecznej, lub w walce. Stosowne specjalności to Infiltruje, Zaprowadza Sprawiedliwość lub Działa pod Przykrywką.

Kupiec: Odkrywca ze specjalizacją skupioną na interakcjach społecznych, taką jak Zabawia lub Przewodzi, mógłby być dobrym kupcem, ale bardziej oczywistym wyborem byłby Mówca.

Mnich lub Mistrz Sztuk Walki: Jako mistrzowie walki bez broni, są to zazwyczaj Wojownicy lub Odkrywcy (możliwe, że z Posmakiem walki). Odpowiednie specjalizacje to Walcząc, Porywa Tłum, Nie Potrzebuje Broni i Rzuca ze Śmiertelną Dokładnością. 

Paladyn/Święty Rycerz: Jako święci wojownicy, którzy mają do dyspozycji magię i moce walki, paladyni to zazwyczaj Wojownicy lub Odkrywcy (w obydwu przypadkach zmodyfikowani Posmakiem magii). Dobre specjalności dla takiej postaci to Chroni Wrót, Chroni Słabszych, Zaprowadza Sprawiedliwość, Zabije Potwory i Dzierży Magiczną Broń. 

Łowca: Łowcy mają mieszankę moc walki i magicznych, i z tego względu są zazwyczaj Odkrywcami (możliwe, że z Posmakiem walki) lub Wojownikiem (możliwe, że z Posmakiem umiejętność i wiedza). Odpowiednie specjalności dla łowcy to: Kontroluje Bestie, Poluje, Żyje w Dziczy, Zabija Potwory, Rzuca ze Śmiertelną Dokładnością i Dzierży Dwie Bronie Naraz.

Łotrzyk lub Złodziej: Większość łotrzyków to Odkrywcy, ale postać skupiona na interakcjach społecznych mogłaby być Mówcą (możliwe, że z Posmakiem skradanie się). Specjalności dobre dla łotrzyka to Bada Ciemne Miejsca, Walczy Nieczysto, Poluje, Infiltruje, Jest Poszukiwany Przez Prawo, Porusza się jak Kot, Pływał z Piratami i Pracuje w Ciemnych Uliczkach.

Zaklinacz: Zaklinacza, to dla naszych potrzeb, magowie, którzy posiadają wrodzoną moc magiczną (w przeciwieństwie do czarodziejów, którzy muszą się tego nauczyć). Większość Zaklinaczy to Adepci, ale niektórzy są Odkrywcami lub Mówcami. Specjalność Włada Zaklęciami daje typowemu zaklinaczowi różne zdolności, a większość specjalności zapewnia zaklęcia tematyczne. Po zaklinaczy z poszczególnych linii krwi, patrz poniżej:

\begin{itemize}
\item Zaklinacz (anioł): Jaśnieje Światłem, Otrzymuje Boskie Błogosławieństwo, Posiada Magicznego Sprzymierzeńca
\item Zaklinacz (przeznaczenie): Ma Szlachetną Krew, Został Przepowiedziany
\item Zaklinacz (smok): Nosi Halo Ognia, Ujeżdża Błyskawicę, Przywdziewa Połyskliwy Lód
\item Zaklinacz (żywiołak): Jest Stworzony z Kamienia, Nosi Halo Ognia, Włada Magnetyzmem, Porusza się jak Wiatr, Ujeżdża Błyskawicę, Przywdziewa Połyskliwy Lód
\item Zaklinacz (fae): Przyjmuje Zwierzęcy Kształt
\item Zaklinacz (demon): Nosi Halo Ognia, Posiada Magicznego Sprzymierzeńca
\item Zaklinacz (nieumarły): Zadaje się z Martwymi, Mówi do Duchów
\end{itemize}

Trikster lub Oszust: Te bystrzaki to zazwyczaj Mówcy, ale niekiedy są Adeptami, jeśli są bardzo magiczni (lub Odkrywcami, jeśli nie są magiczni w ogóle). Wybór specjalności to między innymi Walczy Nieczysto, Pracuje w Ciemnych Uliczkach lub Zabawia.

Czarodziej wojenny: Te nietypowe postaci mieszają korzystanie z broni z magią – wybierz Wojownika z Posmakiem magii lub Odkrywcę z Posmakami magii lub walki. Specjalności, które mogę Cię zainteresować, to Walczą,c Porywa Tłum, Mistrzowsko Włada Broniami, lub Dzierży Magiczną Broń.  

Czarownik lub Wiedźma: Dla celów tej listy, czarownik i wiedźma to magowie, którzy uzyskali moc magiczną z paktu, który zawarli z bytami spoza rzeczywistości. Większość czarowników to Adepci, ale Odkrywcy lub Mówcy (możliwe, że z Posmakiem magia) mogą być ciekawymi opcjami. Odpowiednie specjalności to Tańczy z Czarną Materią, Posiada Magicznego Sprzymierzeńca, Włada Rojem, Izoluje Umysł od Ciała i Został Przepowiedziany. W zależności od patrona i paktu, większość specjalności zaklinacza i czarodzieja będzie ok.

Dziki mag: Ci, którzy korzystają z chaotycznej magii, to zazwyczaj Adepci, ale może t obyć także Odkrywca lub Mówca z Posmakiem magii. Najlepszą specjalizacją byłoby Włada Dziką Magią.

Czarodziej: Dla celów tej listy, czarodzieje uczą się magicznej wiedzy przez wiele lat, aby zdobyć zdolność rzucania zaklęć (w przeciwieństwie do zaklinaczy, czarnoksiężników itp.). Czarodzieje to zazwyczaj Adepci, ale czarodziej zorientowany na ludzi może być Mówcą (być może z Posmakiem magii). Aby stworzyć ogólnego czarodzieja, wybierz specjalizację Włada Zaklęciami. Po bardziej wyspecjalizowanych czarodziejów, patrz niżej

\begin{itemize}
\item Czarodziej (znawca odrzuceń): Absorbuje Energię, Stawia Umysł Ponad Materią, Przywdziewa Połyskliwy Lód
\item Czarodziej (znawca przywołań): Kontroluje Bestie, Posiada Magicznego Sprzymierzeńca
\item Czarodziej (znawca poznań): Szybko się Uczy, Jest Jasnowidzem, Izoluje Umysł od Ciała, Rozwiązuje Zagadki
\item Czarodziej (znawca zauroczeń): Włada Mocami Mentalnymi, Przewodzi
\item Czarodziej (znawca wywołań): Nosi Halo Ognia, Jaśnieje Światłem, Ujeżdza Błyskawicę, Grzmi, Przewdziewa Połyskliwy Lód
\item Czarodziej (iluzjonista): Przebudza Sny, Tworzy Iluzje
\item Czarodziej (nekromanta): Zadaje sięz Umarłymi, Mówi do Duchów
\item Czarodziej (znawca transmutacji): Kontroluje Grawitację, Stawia Umysł Ponad Materią, Przyjmuje Zwierzęcy Kształt
\end{itemize}

\paragraph{Zaklęcia Przygotowane i Spontaniczne}\index{Zaklęcia Przygotowane i Spontaniczne}

Magiczne postaci otrzymują swoje zdolności (które mogą być zaklęciami, rytuałami lub czymś innym) ze swojego typu i specjalności, i mogą korzystać z owych zdolności jak uzn ają za stosowne tak długo, jak wydadzą punkty z Puli. To technicznie czyni ich bardziej jak spontaniczny czarujący. Jeśli wolisz zagrać czymś bardziej jak czarodziej przygotowujący zaklęcia, z większą ilością czarów, z których wybierasz małą ilość każdego dnia, rozważ specjalność skupioną na zaklęciach, taką jak Otrzymuje Boskie Błogosławieństwo, Włada Zaklęciami lub Mówi Głosem Ziemi i rozważ dalszą customizację opcjonalną zasadą rzucania zaklęć. 