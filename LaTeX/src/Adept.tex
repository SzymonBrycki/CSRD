\subsection{Adept}\index{Typ!Adept}

Fantasy/baśń: mag, czarodziej, czarnoksiężnik, kleryk, druid, jasnowidz, diabolista, dotknięty przez Fae.

Współczesność/Horror/Romans: psionik, okultysta, wiedźma, praktykujący magię, medium, szalony naukowiec.

Science fiction: psionik, telepata, jasnowidz, skanujący, ESP-er, abominacja.

Superbohaterowie/Post-apokalipsa: mag, czarownik, dzierżący moc, psionik, telepata.

Władasz mocami i zdolnościami poza ludzkim doświadczeniem, zrozumienie i czasami wiarą. Może to być magia, psionika, zdolności mutanta, lub po prostu skomplikowane urządzenia, w zależności od settingu. (“Magia” to termin, który stosujemy tutaj bardzo luźno. To termin na wszystkie wspaniałe, możliwie nadnaturalne rzeczy, które może zrobić Twoja postać, a inne nie mogą. Może to być skutek posługiwania się odpowiednim sprzętem, kontaktu z duchami, mutacji, psioniki, nanotechnologii lub innych źródeł.)

Rola w grze:Adepci to zazwyczaj inteligentni, myślący ludzie. Bardzo często myślą ostrożnie, zanim podejmą akcję i polegają na swoich nadnaturalnych zdolnościach.

Rola w drużynie: Adepci nie są potężni w bezpośredniej walce, choć często posiadają zdolności, które są wspaniałym uzupełnieniem zdolności bojowych ich towarzyszy, zarówno defensywnie, jak i ofensywnie. Czasami posiadają zdolności, które pomagają im przezwyciężać trudności i wyzwania. Dla przykładu, jeśli grupa musi się przedostać przez zamknięte drzwi, Adept może być w stanie je zniszczyć lub przeteleportować wszystkich na ich drugą stronę.

Rola społeczna: W settingach w których moce nadnaturalne są rzadkie, tajemnicze lub wywołują strach, Adepci są zazwyczaj także rzadcy i wywołujący strach. Pozostają wtedy w ukryciu. Kiedy jest inaczej, Adepci są częstsi i bardziej bezpośredni. Mogą nawet zostać liderami swoich społeczności.

Zaawansowani Adepci: Nawet na niższych poziomach, moce Adeptów zapierają dech w piersiach. Na wyższych poziomach, Adepci mogą dokonać prawdziwie wielkich czynów, które mogą przekształcić materię i środowisko wokół nich.
(Adepci prawie zawsze są paranormalni lub nadludzcy w jakimś sensie – czarodzieje, psionicy itp. Jeśli gra, w którą gracie, nie posiada takich postaci, Adept mógłby być szarlatanem, który udaje magiczne zdolności przy pomocy trików i ukrytych urządzeń, lub gadżeciarzem z “przydatnym paskiem” pełnym dziwnych narzędzi. Lub w Twoim świecie może nie być Adeptów. To także jest ok.)

\subsubsection{Adept - Wtrącenia Gracza}

Kiedy grasz Adeptem, możesz wydać 1 PD na jedne z poniższych wtrąceń gracza, jeśli sytuacja jest stosowna i MG się zgodzi.
 
Przydatna Awaria: Urządzenie, z którego korzysta się przeciwko Tobie, ulega awarii. Może ono zranić użytkownika lub jednego z jego sprzymierzeńców w ciągu jednej tury, lub aktywować dramatyczny i rozpraszający efekt uboczny, trwający parę tur.

Nagłe Olśnienie: Doświadczasz nagłego olśnienia, które zapewnia jasną odpowiedz lub sugeruje następne kroki w temacie ważnego pytanie, problemu lub przeszkody na Twojej drodze. 

Cudowna Aktywacja: Nieaktywne, zrujnowane lub najwyraźniej-zniszczone urządzenia chwilowo się aktywuje i wykonuje przydatną akcję w kontekście obecnej sytuacji. Może to kupić Ci trochę czasu na znalezienie lepszego rozwiązania, przezwyciężyć komplikację która wpływa na Twoje moce, lub po prostu umożliwić skorzystanie z zużytego cyphera lub artefaktu jeszcze raz. 

\begin{table*}[t]
 \centering
 \begin{tabularx}{\textwidth}{ | X | X |}
  \hline
  \textbf{ Statystyka} & \textbf{Początkowa Wartość Puli}  \\ \hline
    Moc & 7 \\ \hline
    Szybkość & 9 \\ \hline
    Intelekt & 12 \\ \hline
 \end{tabularx}
  \caption {Pule Statystyk Adepta}
  \label {Pule Statystyk Adepta}
 \end{table*}
 
 Otrzymujesz 6 dodatkowych punktów do podziału pomiędzy Pule statystyk, zgodnie z własną wolą.
 
\subsubsection{Historia Adepta}

Twój typ pomaga Ci określić Twoje miejsce w settingu. Rzuć k20 lub wybierz z poniższej listy, by określić konkretny fakt odnośnie Twojej historii, która łączy Cię z resztą świata. Możesz także stworzyć swój własny fakt. 

 \begin{table*}[t]
 \centering
 \begin{tabularx}{\textwidth}{| p{0.10\textwidth} | X |}
  \hline
  \textbf{d20} & \textbf{Historia Adepta}  \\ \hline
    1 & Służyłeś jako uczeń u Adepta, którego respektowało i bało się wielu ludzi. Teraz nosisz jego brzemię. \\ \hline
    2 & Studiowałeś w szkole słynącej z jej mrocznych nauczycieli i absolwentów. \\ \hline
    3 & Nauczyłeś się swoich zdolności w świątyni mało znanego boga. Jego kapłani i wierni, choć niezbyt liczni, respektują i adorują Twoje talenty i potencjał. \\ \hline
    4 & Kiedy podróżowałeś samotnie, ocaliłeś życie potężnej osoby. Ma ona względem Ciebie dług wieczności. \\ \hline
    5 & Twoja matka była potężnym Adeptem za życia, pomagała też ludziom w okolicy. Patrzą oni na Ciebie ciepło, ale także spodziewają się wiele po Tobie. \\ \hline
    6 & Wisisz pieniądze wielu ludziom i nie masz pieniędzy, by spłacić swój dług. \\ \hline
    7 & Zaliczyłeś gigantyczną klęskę w swoich początkowych studiach z nauczycielem i teraz uczysz się na własną rękę. \\ \hline
    8 & Nauczyłeś się swoich zdolności szybciej, niż Twoi nauczyciele widzieli u któregokolwiek ze swoich uczniów. Potężni tego świata zwrócili na Ciebie swoją uwagę i obserwują Cię intensywnie.  \\ \hline
    9 & Zabiłeś dobrze znanego kryminalistę w samoobronie, zyskując respekt wielu i nieprzyjaźń paru niebezpiecznych ludzi. \\ \hline
    10 & Uczyłeś się na Wojownika, ale Twoje uzdolnienia w kierunku Adepta ostatecznie skierowały Cię na odmienną ścieżkę. Twoi dawni kompani nie rozumieją Cię, ale mimo to Cię szanują. \\ \hline
    11 & Kiedy studiowałeś na Adepta, pracowałeś jako asystant w banku, zaprzyjaźniajac się z właścicielem i klientami. \\ \hline
    12 & Twoja rodzina posiada wielką winnicę niedaleko, znaną ze swojego dobrego wina i uczciwości biznesowej. \\ \hline
    13 & Trenowałeś przez pewien czas z grupą wpływowych Adeptów, którzy dalej darzą Cię przyjaźnią. \\ \hline
    14 & Pracowałeś w ogrodach pałacowych wpływowego szlachcica lub bogatej osoby. Nie pamięta ona Cię, ale zaprzyjaźniłeś się z jej młodą córką. \\ \hline
    
 \end{tabularx}
  \caption {Historia Adepta}
  \label {Historia Adepta}
 \end{table*}