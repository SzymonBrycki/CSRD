\cleardoublepage

\subsection{Adept}\index{Typ!Adept}

Fantasy/baśń: mag, czarodziej, czarnoksiężnik, kleryk, druid, jasnowidz, diabolista, dotknięty przez Fae.

Współczesność/Horror/Romans: psionik, okultysta, wiedźma, praktykujący magię, medium, szalony naukowiec.

Science fiction: psionik, telepata, jasnowidz, skanujący, ESP-er, abominacja.

Superbohaterowie/Post-apokalipsa: mag, czarownik, dzierżący moc, psionik, telepata.

Władasz mocami i zdolnościami poza ludzkim doświadczeniem, zrozumienie i czasami wiarą. Może to być magia, psionika, zdolności mutanta, lub po prostu skomplikowane urządzenia, w zależności od settingu. (“Magia” to termin, który stosujemy tutaj bardzo luźno. To termin na wszystkie wspaniałe, możliwie nadnaturalne rzeczy, które może zrobić Twoja postać, a inne nie mogą. Może to być skutek posługiwania się odpowiednim sprzętem, kontaktu z duchami, mutacji, psioniki, nanotechnologii lub innych źródeł.)

Rola w grze:Adepci to zazwyczaj inteligentni, myślący ludzie. Bardzo często myślą ostrożnie, zanim podejmą akcję i polegają na swoich nadnaturalnych zdolnościach.

Rola w drużynie: Adepci nie są potężni w bezpośredniej walce, choć często posiadają zdolności, które są wspaniałym uzupełnieniem zdolności bojowych ich towarzyszy, zarówno defensywnie, jak i ofensywnie. Czasami posiadają zdolności, które pomagają im przezwyciężać trudności i wyzwania. Dla przykładu, jeśli grupa musi się przedostać przez zamknięte drzwi, Adept może być w stanie je zniszczyć lub przeteleportować wszystkich na ich drugą stronę.

Rola społeczna: W settingach w których moce nadnaturalne są rzadkie, tajemnicze lub wywołują strach, Adepci są zazwyczaj także rzadcy i wywołujący strach. Pozostają wtedy w ukryciu. Kiedy jest inaczej, Adepci są częstsi i bardziej bezpośredni. Mogą nawet zostać liderami swoich społeczności.

Zaawansowani Adepci: Nawet na niższych poziomach, moce Adeptów zapierają dech w piersiach. Na wyższych poziomach, Adepci mogą dokonać prawdziwie wielkich czynów, które mogą przekształcić materię i środowisko wokół nich.
(Adepci prawie zawsze są paranormalni lub nadludzcy w jakimś sensie – czarodzieje, psionicy itp. Jeśli gra, w którą gracie, nie posiada takich postaci, Adept mógłby być szarlatanem, który udaje magiczne zdolności przy pomocy trików i ukrytych urządzeń, lub gadżeciarzem z “przydatnym paskiem” pełnym dziwnych narzędzi. Lub w Twoim świecie może nie być Adeptów. To także jest ok.)

\subsubsection{Adept - Wtrącenia Gracza}

Kiedy grasz Adeptem, możesz wydać 1 PD na jedne z poniższych wtrąceń gracza, jeśli sytuacja jest stosowna i MG się zgodzi.
 
Przydatna Awaria: Urządzenie, z którego korzysta się przeciwko Tobie, ulega awarii. Może ono zranić użytkownika lub jednego z jego sprzymierzeńców w ciągu jednej tury, lub aktywować dramatyczny i rozpraszający efekt uboczny, trwający parę tur.

Nagłe Olśnienie: Doświadczasz nagłego olśnienia, które zapewnia jasną odpowiedz lub sugeruje następne kroki w temacie ważnego pytanie, problemu lub przeszkody na Twojej drodze. 

Cudowna Aktywacja: Nieaktywne, zrujnowane lub najwyraźniej-zniszczone urządzenia chwilowo się aktywuje i wykonuje przydatną akcję w kontekście obecnej sytuacji. Może to kupić Ci trochę czasu na znalezienie lepszego rozwiązania, przezwyciężyć komplikację która wpływa na Twoje moce, lub po prostu umożliwić skorzystanie z zużytego cyphera lub artefaktu jeszcze raz. 

\begin{table*}[t]
 \centering
 \begin{tabularx}{\textwidth}{ | X | X |}
  \hline
  \textbf{ Statystyka} & \textbf{Początkowa Wartość Puli}  \\ \hline
    Moc & 7 \\ \hline
    Szybkość & 9 \\ \hline
    Intelekt & 12 \\ \hline
 \end{tabularx}
  \caption {Pule Statystyk Adepta}
  \label {Pule Statystyk Adepta}
 \end{table*}
 
 Otrzymujesz 6 dodatkowych punktów do podziału pomiędzy Pule statystyk, zgodnie z własną wolą.
 
\subsubsection{Historia Adepta}

Twój typ pomaga Ci określić Twoje miejsce w settingu. Rzuć k20 lub wybierz z poniższej listy, by określić konkretny fakt odnośnie Twojej historii, która łączy Cię z resztą świata. Możesz także stworzyć swój własny fakt. 

 \begin{table*}[t]
 \centering
 \begin{tabularx}{\textwidth}{| p{0.10\textwidth} | X |}
  \hline
  \textbf{k20} & \textbf{Historia Adepta}  \\ \hline
    1 & Służyłeś jako uczeń u Adepta, którego respektowało i bało się wielu ludzi. Teraz nosisz jego brzemię. \\ \hline
    2 & Studiowałeś w szkole słynącej z jej mrocznych nauczycieli i absolwentów. \\ \hline
    3 & Nauczyłeś się swoich zdolności w świątyni mało znanego boga. Jego kapłani i wierni, choć niezbyt liczni, respektują i adorują Twoje talenty i potencjał. \\ \hline
    4 & Kiedy podróżowałeś samotnie, ocaliłeś życie potężnej osoby. Ma ona względem Ciebie dług wieczności. \\ \hline
    5 & Twoja matka była potężnym Adeptem za życia, pomagała też ludziom w okolicy. Patrzą oni na Ciebie ciepło, ale także spodziewają się wiele po Tobie. \\ \hline
    6 & Wisisz pieniądze wielu ludziom i nie masz pieniędzy, by spłacić swój dług. \\ \hline
    7 & Zaliczyłeś gigantyczną klęskę w swoich początkowych studiach z nauczycielem i teraz uczysz się na własną rękę. \\ \hline
    8 & Nauczyłeś się swoich zdolności szybciej, niż Twoi nauczyciele widzieli u któregokolwiek ze swoich uczniów. Potężni tego świata zwrócili na Ciebie swoją uwagę i obserwują Cię intensywnie.  \\ \hline
    9 & Zabiłeś dobrze znanego kryminalistę w samoobronie, zyskując respekt wielu i nieprzyjaźń paru niebezpiecznych ludzi. \\ \hline
    10 & Uczyłeś się na Wojownika, ale Twoje uzdolnienia w kierunku Adepta ostatecznie skierowały Cię na odmienną ścieżkę. Twoi dawni kompani nie rozumieją Cię, ale mimo to Cię szanują. \\ \hline
    11 & Kiedy studiowałeś na Adepta, pracowałeś jako asystant w banku, zaprzyjaźniajac się z właścicielem i klientami. \\ \hline
    12 & Twoja rodzina posiada wielką winnicę niedaleko, znaną ze swojego dobrego wina i uczciwości biznesowej. \\ \hline
    13 & Trenowałeś przez pewien czas z grupą wpływowych Adeptów, którzy dalej darzą Cię przyjaźnią. \\ \hline
    14 & Pracowałeś w ogrodach pałacowych wpływowego szlachcica lub bogatej osoby. Nie pamięta ona Cię, ale zaprzyjaźniłeś się z jej młodą córką. \\ \hline
    15 & Eksperyment, który przeprowadziłeś w przeszłości, kompletnie nie wypalił. Ludzie z tamtej okolicy zapamiętali Cię jako niebezpiecznego i bezmyślnego typka. \\ \hline
    16 & Pochodzisz z dalekiego miejsca, gdzie byłeś dobrze znany i traktowany, ale ludzie tutaj traktują Cię z dużą podejrzliwością. \\ \hline
    17 & Ludzie, których spotykasz, wydają się trzymać na dystans ze względu na dziwne piętna na Twojej twarzy. \\ \hline
    18 & Twój najlepszy przyjaciel to także Adept. Ty i Twój przyjaciel dzielicie się odkryciami i sekretami. \\ \hline
    19 & Znasz lokalnego kupca bardzo dobrze. Ponieważ zapewniłeś mu dużo przychodu, oferuje Ci on zniżki i specjalne traktowanie.  \\ \hline
    20 & Należysz to sekretnego klubu, który spotyka się co miesiąc, by wypić i porozmawiać. \\ \hline
 \end{tabularx}
  \caption {Historia Adepta}
  \label {Historia Adepta}
 \end{table*}
 
 \afterpage{\clearpage}
 
 \subsubsection{Adept Pierwszego Poziomu}
 
 Pierwszo-poziomowi Adepci posiadają następujące zdolności:
 
Wysiłek: Twój Wysiłek to 1.

Geniusz: Masz Skupienie w Intelekcie 1 oraz Skupienie w Mocy i Szybkości 0.

Eksperckie Korzystanie z Cypherów: Możesz nosić 3 cyphery w danym czasie.

Początkowy Ekwipunek: Stosowne ubranie, plus 2 drogie przedmioty, dwa przedmioty średniej ceny i do 4 niedrogich przedmiotów Twojego wyboru.

Bronie: Możesz korzystać z lekkich broni bez żadnej kary. Posiadasz nieumiejętność w średnich i ciężkich broniach – Twoje ataki z średnimi i ciężkimi broniami są utrudnione.

Specjalne zdolności: Wybierz 4 zdolności z poniższej listy. Nie możesz wybrać danej zdolności więcej niż 1 raz, chyba, że jej opis stanowi inaczej. Pełny opis każdej z dostępnych zdolności znajduje się w rozdziale \mytext{Zdolności}, który zawiera także zdolności Posmaków i specjalizacje w jednym, rozbudowanym katalogu. (Zdolności Adepta wymagają przynajmniej jednej wolnej ręki, chyba, że MG mówi inaczej.)

\begin{itemize}
\item Zamglenie
\item Usunięcie Wspomnień
\item Daleki Krok
\item Sztuczki Magiczne
\item Trening Magiczny
\item Pocisk
\item Pchnięcie
\item Pole Renozansowe
\item Skan
\item Strzaskanie
\item Magia Obronna
\end{itemize}

\subsubsection{Adept Drugiego Poziomu}

Wybierz jedną ze zdolności z poniższej listy (lub z niższego poziomu) i dodaj do swoich zdolności. Dodatkowo, możesz zamienić jedną ze zdolności z niższego poziomu na inną zdolność z niższego poziomu.

\begin{itemize}
\item Adaptacja
\item Czytanie Myśli
\item Odzyskanie Wspomnień
\item Ujawnienie
\item Unoszenie Się
\item Tnące Światło
\item Zastój
\end{itemize}

\subsubsection{Adept Trzeciego Poziomu}

Wybierz dwie zdolności z poniższej listy (lub z niższego poziomu) i dodaj do swoich zdolności. Dodatkowo, możesz wybrać jedną ze zdolności niższego poziomu i zamienić na inną z niższego poziomu. 

\begin{itemize}
\item Bariera Pola Siłowego
\item Ochrona Przed Energią
\item Ogień i lód
\item Sokole Oko
\item Sensor
\item Środki Zaradcze
\item Zaawansowany Użytkownik Cypherów
\end{itemize}

\subsubsection{Adept Czwartego Poziomu}

Wybierz jedną ze zdolności z poniższej listy (lub z niższego poziomu) i dodaj do swoich zdolności. Dodatkowo, możesz zamienić jedną ze zdolności niższego poziomu na inną niższego poziomu.

\begin{itemize}
\item Chmura Drobiazgów
\item Dotyk Śmierci
\item Kontrola Umysłu
\item Niewidzialność
\item Projekcja
\item Przebudowa
\item Regeneracja
\item Szybka Reakcja
\item Tunel Czasoprzestrzenny
\item Wygnanie
\end{itemize}

\subsubsection{Adept Piątego Poziomu}

Wybierz dwie ze zdolności z poniższej listy (lub z niższego poziomu) i dodaj do swoich zdolności. Dodatkowo, możesz zamienić jedną ze zdolności niższego poziomu na inną zdolność niższego poziomu.

\begin{itemize}
\item Absorpcja Energii
\item Mistrzowskie Korzystanie z Cypherów
\item Prawdziwe Zmysły
\item Przywołanie
\item Pył w Pył
\item Stworzenie
\item Teleportacja
\item Wiedza o Nieznanym
\item Wybuch
\end{itemize}

\subsubsection{Adept Szóstego Poziomu}

Wybierz jedną ze zdolności z poniższej listy (lub z niższego poziomu) i dodaj do swoich zdolności. Dodatkowo, możesz zamienić jedną ze zdolności niższego poziomu na inną zdolność niższego poziomu.

\begin{itemize}
\item Absorpcja Cyphera
\item Kontrola Pogody
\item Podróż Między Światami
\item Poruszenie Góry
\item Trzęsienie Ziemi
\end{itemize}

\subsubsection{Przykładowy Adept}

Jen chce stworzyć Adepta – czarownika w kampanii fantasy. Decyduje się być dobra we wszystkim po trochu, więc przydziela 2 do każdej z Pul Statystyk, co daje jej Moc 9, Szybkość 11 i Intelekt 14. Jej Adept jest szybki i bystry. Posiada ona Skupienie w Intelekcie 1, w Mocy i Szybkości zaś – 0. Jako postać pierwszego poziomu, jej Wysiłek to 1. Jako początkowe zdolności, wybiera ona \mytext{Pocisk} i \mytext{Magię Obronną}, co daje jej dobre moce defensywne i ofensywne. Wybiera także \mytext{Trening Magiczny} i dopełnia postać za pomocą \mytext{Skanu}, co ma jej zapewnić dodatkowe źródła informacji. Oznacza to, że Pocisk, Magi Obronna i Skan to zaklęcia, których się nauczyła w ciągu wielu lat nauki.
Jen może mieć trzy cyphery. MG daje jej miksturę, która działa jak teleporter krótkiego zasięgu, mały medalion, który regeneruje  punktów Puli Intelektu i flakonik wypełniony płynem, który eksploduje niczym ognista bomba. Czarownik Jen jest wyszkolony w lekkich broniach, więc wybiera ona sztylet.

Na swój deskryptor Jen wybiera \mytext{Pełen Gracji}, co dodaje 2 punkty do jej Puli Szybkości, zwiększając ją do 2. Ten deskryptor oznacza także, że Jen jest wytrenowana w balansowaniu i ostrożnych ruchach, fizycznych występach i rzutach na Obronę Szybkości. Może jest ona tancerzem. Po prawdzie, zaczyna ona tworzyć historię, w której jej postać rzuca swoje zaklęcia przy pomocy lekkich, zwinnych ruchów.
Na swoją specjalizację, wybiera ona \mytext{Przewodzi}. To daje jej wytrenowanie w społecznych interakcjach, co znowu daje uniwersalną postać – jest dobra we wszystkich sytuacjach. Co więcej, uzyskuje ona zdolność Dobra Porada, co pozwala jej na bycie w centrum jej drużyny.
Jje zaklęcia i zdolności specjalizacyjne kosztują punkty Intelektu, by je aktywowac, jest więc szczęśliwa, że ma wiele punktów w Intelekcie. Dodatkowo, jej Skupienie w Intelekcie obniży ten koszt. Jeśli użyje ona swojego Pocisku bez stosowania Wysiłku, kosztować ją będzie to 0 punktów Intelektu i zada 4 punkty obrażeń. Jej Skupienie w Intelekcie pozwoli jej zaoszczędzić punkty, które będzie mogła wydać w innym celu, być może aby zwiększyć dokładność jej Pocisku.

Na swój motyw fabularny Jen wybiera \mytext{Wsparcie Przyjaciela}. Decyduje ona, że kiedy jej czarowniczka była młoda, miała magicznego mentora. Później ów mentor został wzięty w niewolę przez demona, więc jej postać zawsze szuka wskazówek odnośnie tego demona, aby wyzwolić jej przyjaciela z niewoli.

(MG ma możliwość pre-selekcji specjalnych zdolności danego typu na danym poziomie, by kształtować setting. Przykładowo, w powyższym przypadku MG może określić, że wszyscy czarownicy (Adepci) zaczynają grę z Treningiem Magicznym jako jedną z ich zdolności pierwszego poziomu. Nie czyni to postaci mniej potężną lub specjalną, ale mówi coś o jej roli w świecie i tym, czego można się spodziewać w trakcie gry.)