% hyperref !!! https://tex.stackexchange.com/questions/180571/making-clickable-links-to-sections-with-hyperref

\chapter{Zdolności w kolejności alfabetycznej}

\section{A}

\textbf{Uśmiech i Słowo}\index{Zdolności!Alfabetycznie!Uśmiech i Słowo}\label{sec:Uśmiech i Słowo} - kiedy korzystasz z Wysiłku do dowolnej akcji interakcji społecznej - nawet takiej która polega na uspokajaniu zwierząt lub komunikowania się z kimś, czyim językiem nie mówisz - uzyskujesz darmowy poziom Wysiłku na tym zadaniu. Akcja.

\textbf{Przydatna Pomoc}\index{Zdolności!Alfabetycznie!Przydatna Pomoc}\label{sec:Przydatna Pomoc} - kiedy pomagasz komuś z zadaniem i stosuje on poziom wysiłku, zyskuje on darmowy poziom Wysiłku na tym zadaniu. Umożliwienie. 

\textbf{Absorpcja Energii}\index{Zdolności!Alfabetycznie!Absorpcja Energii}\label{sec:Absorpcja Energii} (7 punktów Intelektu) - dotykasz obiektu i absorbujesz jego energię. Jeśli dotykasz zamanifestowanego Cyphera, czynisz go bezużytecznym. Jeśli dotykasz artefaktu, rzuć na jego wyczerpanie. Jeśli dotykasz innego rodzaju zasilanego urządzenia lub maszyny, GM określa, czy jego moc jest w pełni wyssana. W każdym razie, absorbujesz energię z obiektu i odzyskujesz 1k10 punktów Intelektu. Jeśli to dałoby Ci więcej punktów Intelektu niż maksimum Twojej Puli, dodatkowe punkty są utracone, i musisz wykonać rzut na Obronę Mocy. Trudność tego rzutu to numer punktów powyżej Twojego maksimum, które zaabsorbowałeś. Jeśli oblejesz ten rzut, otrzymujesz 5 punktów obrażeń i nie możesz podejmować działań przez jedną rundę. Możesz wykorzystać tę zdolność jako akcję obronną kiedy jesteś celem ataku zdolnością. Taka akcja niweluje atak zdolnością, a ty absorbujesz energię, jakby pochodziła z urządzenia. Akcja.

\textbf{Absorpcja Energii Kinetycznej}\index{Zdolności!Alfabetycznie!Absorpcja Energii Kinetycznej}\label{sec:Absorpcja Energii Kinetycznej} - absorbujesz porcję energii ataku fizycznego lub uderzenia. Negujesz 1 punkt obrażeń, które normalnie byś poniósł i przechowujesz tę energię. Po tym, jak zaabsorbujesz 1 punkt energii, kontynuujesz obniżać obrażenia o 1 punkt z nadchodzących ataków, ale pozostała energia wycieka z Ciebie w formie błysku nieszkodliwego światła (nie możesz przechowywać na raz więcej niż 1 punktu energii w tym samym czasie). Umożliwienie. 

\textbf{Absorpcja Czystej Energii}\index{Zdolności!Alfabetycznie!Absorpcja Czystej Energii}\label{sec:Absorpcja Czystej Energii} - kiedy korzystasz z Absorpcji Energii Kinetycznej, możesz także absorbować i przechowywać energię ataków bazujących na czystej energii (światło, promieniowanie, energie międzywymiarowe, psioniczne itp.) lub z przekaźników owej energii, gdy masz z nimi bezpośredni kontakt. Ta zdolność nie zmienia tego, ile punktów energii możesz przechowywać. Jeśli masz również Ulepszoną Absorpcję Energii Kinetycznej, możesz również absorbować do 2 punktów obrażeń ze źródeł czystej energii. Umożliwienie. 

\textbf{Akceleracja}\index{Zdolności!Alfabetycznie!Akceleracja}\label{sec:Akceleracja} (4+ punkty Intelektu) - Twoje słowa umacniają ducha postaci w bliskim zasięgu, która jest w stanie zrozumieć Cię, przyspieszając ją, tak, że zyskuje ona atut na testach inicjatywy i rzutach na Obronę Szybkości przez 10 minut. Dodatkowo, poza zwykłymi opcjami korzystania z Wysiłku, możesz z niego skorzystać, by objąć celem tej zdolności więcej postaci - każdy poziom Wysiłku obejmuje dodatkowy cel. Musisz przemówić do dodatkowych celów, by je przyspieszyć, jeden cel na rundę. Jednak akcja na jeden cel by rozpocząć. 

\textbf{Akrobatyczny Atak}\index{Zdolności!Alfabetycznie!Akrobatyczny Atak}\label{sec:Akrobatyczny Atak} (1+ punktów Szybkości) - wyskakujesz w ataku, przesuwając się przez powietrze. Jeśli wyrzucasz naturalne 17 lub 18, możesz wybrać mniejszy efekt zamiast dodatkowych obrażeń. Jeśli zastosujesz Wysiłek do tego ataku, uzyskujesz darmowy poziom Wysiłku na zadaniu. Nie możesz skorzystać z tej zdolności, jeśli Twój Wysiłek Szybkości jest zredukowany wskutek noszenia zbroi. Umożliwienie. 

\textbf{Procesor Akcji}\index{Zdolności!Alfabetycznie!Procesor Akcji}\label{sec:Procesor Akcji} (4 punkty Intelektu) - korzystając z przechowywanych informacji i zdolności analizowania nadchodzących danych z wielką szybkością, jesteś wyszkolony w jednym fizycznym zadaniu Twojego wyboru na 10 minut. Dla przykładu, możesz wybrać bieg, wspinaczkę, pływanie, Obronę Szybkości lub atak specyficzną bronią. Akcja by rozpocząć.

\textbf{Adaptacja}\index{Zdolności!Alfabetycznie!Adaptacja}\label{sec:Adaptacja} - dzięki ukrytej mutacji, urządzeniu wbudowanemu w Twój kręgosłup, rytuałowi krwii smoka, lub jakiemuś innemu darowi, jesteś teraz w komfortowej temperaturze; nie musisz sie nigdy martwić o niebezpieczne promieniowanie, choroby lub gazy; i możesz zawsze oddychać w dowolnym środowisku (nawet w próżni kosmosu). Umożliwienie.

\textbf{Zaawansowany Użytkownik Cypherów}\index{Zdolności!Alfabetycznie!Zaawansowany Użytkownik Cypherów}\label{sec:Zaawansowany Użytkownik Cypherów} - możesz mieć przy sobie 4 Cyphery w danym czasie. Umożliwienie. 

\textbf{Zaawansowany Rozkaz}\index{Zdolności!Alfabetycznie!Zaawansowany Rozkaz}\label{sec:Zaawansowany Rozkaz} (7 punktów Intelektu) - cel w średnim zasięgu słucha każdej komendy, którą mu wydasz, tak długo, jak słyszy Cię i rozumie. Co więcej, tak długo, jak nie robisz nic innego niż wydawanie komend (nie wolno Ci wziąć żadnej innej akcji) możesz dać temu samemu celowi nową komendę. Ten efekt kończy się, gdy kończysz wydawać komendy lub gdy cel opuszcza średni zasięg względem Ciebie. Akcja by rozpocząć. 

\textbf{Atak z Rozbrojeniem}\index{Zdolności!Alfabetycznie!Atak z Rozbrojeniem}\label{sec:Atak z Rozbrojeniem} (3 punkty Szybkości) - za pomocą serii szybkich ruchów, wykonujesz atak przeciwko uzbrojonemu przeciwnikowi, zadając mu obrażenia i rozbrajając go, tak, że jego broń jest teraz w Twoich rękach lub 3 metry od niego na ziemi - Ty wybierasz. Ten atak rozbrajający jest utrudniony. Akcja.

\textbf{Zalety Bycia Dużym}\index{Zdolności!Alfabetycznie!Zalety Bycia Dużym}\label{sec:Zalety Bycia Dużym} - kiedy korzystasz ze Wzrostu, jesteś tak duży, że możesz łatwiej przenosić duże obiekty, wspinać sie na budynki korzystając z uchwytów niedostępnych dla zwykłych ludzi i skakać znacznie dalej. Kiedy korzystasz ze Wzrostu, wszystkie zadania wspinaczki, podnoszenia ciężarów i skakania są dla Ciebie ułatwione. Umożliwienie.

\textbf{Zalety Bycia Małym}\index{Zdolności!Alfabetycznie!Zalety Bycia Małym}\label{sec:Zalety Bycia Małym} - nauczyłeś się, jak wykorzystać swój rozmiar, siłę i dokładność. Twoje obrażenia już się nie dzielą na pół gdy korzystasz ze Zmniejszenia się, a zadania wspinaczki i skakania są ułatwione. Umożliwienie.

\textbf{Porada od Przyjaciela}\index{Zdolności!Alfabetycznie!Porada od Przyjaciela}\label{sec:Porada od Przyjaciela} (1 punkt Intelektu) - znasz słabe i mocne strony swojego przyjaciela, i wiesz jak go zmotywować, by osiągnął sukces. Kiedy dajesz przyjacielowi sugestię powiązaną z jego następną akcję, postać ta jest wyszkolona w tej akcji na jedną rundę. Akcja. 

\textbf{Znowu i Znowu}\index{Zdolności!Alfabetycznie!Znowu i Znowu}\label{sec:Znowu i Znowu} (8 punktów Szybkości) - możesz wziąć kolejną akcję w rundzie, w której już podjąłeś akcję. Umożliwienie.

\textbf{Nieśmiertelny}\index{Zdolności!Alfabetycznie!Nieśmiertelny}\label{sec:Nieśmiertelny} - Twoje ciało i umysł się nie starzeją. Jeśli nie zostaniesz zabity przez akt przemocy (lub jakąś zewnętrzną siłę jak trucizna lub infekcja), nigdy nie umrzesz. Umożliwienie.  

\textbf{Agent-Prowokator}\index{Zdolności!Alfabetycznie!Agent-Prowokator}\label{sec:Agent-Prowokator} - wybierz jedna z poniższych, by być wytrenowanym w: atakowanie bronią swojego wyboru, ładunki wybuchowe, lub skradanie się i otwieranie zamków (jeśli wybierzesz ostatnią opcję, posiadasz trening w dwóch umiejętnościach). Umożliwienie.

\textbf{Agresja}\index{Zdolności!Alfabetycznie!Agresja}\label{sec:Agresja} (2 punkty Mocy) - skupiasz się na atakowaniu w tak wielki sposób, że zostawiasz siebie wysuniętego na ataki wrogów. Kiedy ta zdolność jest aktywna, zyskujesz atut na atakach wręcz i Twoje rzuty na Obronę Szybkości przeciwko atakom wręcz i dystansowym są utrudnione. Ten efekt trwa tak długo, jak sobie życzysz ale kończy się, jeśli walka nie ma miejsca w zasięgu Twoich zmysłów. Umożliwienie.

\textbf{Szybki Umysł}\index{Zdolności!Alfabetycznie!Szybki Umysł}\label{sec:Szybki Umysł} - kiedy próbujesz wykonać zadanie Szybkości, możesz zamiast tego rzucić (i wydać punkty z puli) jakby to była akcja Intelektu. Jeśli stosujesz Wysiłek do tego zadania, możesz wydać punkty z Puli Intelektu zamiast Puli Szybkości (wtedy stosujesz też Skupienie w Intelekcie zamiast w Szybkości). Umożliwienie. 

\textbf{Wysokie Skupienie}\index{Zdolności!Alfabetycznie!Wysokie Skupienie}\label{sec:Wysokie Skupienie} (7 punktów Intelektu) - wkładasz w swoje zadanie wszystko. Dodajesz trzy darmowe poziomy Wysiłku to następnego zadania, które podejmujesz. Nie możesz wykorzystać tej zdolności znowu, dopóki nie zakończysz 10-godzinnego odpoczynku. Akcja.

\textbf{Uzdrowienie}\index{Zdolności!Alfabetycznie!Uzdrowienie}\label{sec:Uzdrowienie} (3 punkty Intelektu) - możesz spróbować uzdrowić jedno schorzenie (np: chorobę lub truciznę) dotyczące jednej istoty. Akcja.

\textbf{Szczur Miejski}\index{Zdolności!Alfabetycznie!Szczur Miejski}\label{sec:Szczur Miejski} (6 punktów Intelektu) - kiedy jesteś w mieście, odnajdujesz lub tworzysz znaczące skróty, sekretne wejścia lub ostateczne trasy ucieczki tam, gdzie wcześniej ich nie było. Aby to zrobićm musisz uzyskać sukces na kacji Intelektu, której trudność określa MG bazując na danej sytuacji. Powinieneś ustalić detale wraz ze swoim MG. Akcja.

\textbf{Zawsze Majsterkując}\index{Zdolności!Alfabetycznie!Zawsze Majsterkując}\label{sec:Zawsze Majsterkując} - jeśli masz narzędzia i materiały i nosisz mniej cypherów niż Twój limit, możesz stworzyć zamanifestowany cypher, jeśli poświęcisz na to godzinę. Nowy cypher jest wybierany przypadkowo i zawsze o 2 poziomy mniej niż normalnie (minimum to 1-szy poziom). Jest on także chwilowy i wrażliwy na uszkodzenia. Nazywa się go chwilowym cypherem. Jeśli dasz go komuś, by z niego korzystał, rozpada się on natychmiast w bezużyteczne śmieci. Akcja by rozpocząć; 1 godzina by ukończyć.

\textbf{Cudowne Kopiowanie}\index{Zdolności!Alfabetycznie!Cudowne Kopiowanie}\label{sec:Cudowne Kopiowanie} - możesz skorzystać ze zdolności Skopiuj Moc, aby skopiować potężniejsze zdolności. W dodatku do normalnych opcji korzystania z Wysiłku przy użyciu Skopiuj Moc, jeśli zaaplikujesz 2 poziomy Wysiłku, MG wybiera moc wysokiego poziomu, która najbardziej przypomina moc, którą pragniesz skopiować (zamiast zdolności niskiego poziomu). Umożliwienie.

\textbf{Dodatkowy Wysiłek}\index{Zdolności!Alfabetycznie!Dodatkowy Wysiłek}\label{sec:Dodatkowy Wysiłek} - kiedy stosujesz przynajmniej jeden poziom Wysiłku do akcji niebojowej, otrzymujesz darmowy, dodatkowy poziom Wysiłku na tym zadaniu. Kiedy wybierasz tę zdolność, musisz zdecydować, czy dotyczy ona Wysiłku Mocy, czy też Wysiłku Szybkości. Umożliwienie.

\textbf{Wielki Skok}\index{Zdolności!Alfabetycznie!Wielki Skok}\label{sec:Wielki Skok} (2 punkty Mocy) - skaczesz w powietrze i lądujesz bezpiecznie w pewnej odległości. Możesz skoczyć wzwyż, w dół lub w poziomie gdziekolwiek w dalekim zasięgu  jeśli masz czystą trasę do tego miejsce, bez żadnych przeszkód. Jeśli masz 3 lub więcej punktów mocy zainwestowanych w siłę, Twój zasięg się ulepsza do bardzo dalekiego. Jeśli masz 5 lub więcej punktów mocy zainwestowanych w siłę, Twój zasięg skoku zostaje ulepszony do 300 metrów. Akcja.

\textbf{Czatownik}\index{Zdolności!Alfabetycznie!Czatownik}\label{sec:Czatownik} - kiedy atakujesz istotę, która jeszcze nie wzięła swojej pierwszej rundy w walce, Twój atak jest ułatwiony. Umożliwienie.

\textbf{Wzmocnienie Dźwięku}\index{Zdolności!Alfabetycznie!Wzmocnienie Dźwięku}\label{sec:Wzmocnienie Dźwięku} (2 punkty Mocy) - na jedną minutę, możesz wzmocnić dalekie lub ciche dźwięki, tak, byś mógł je słyszeć wyraźnie, nawet jeśli jest to rozmowa lub dźwięk małego zwierzęcia poruszającego się w podziemnej norze w bardzo dalekim zasięgu. Możesz spróbować usłyszeć dźwięk, nawet jeśli istnieją bariery blokujące dźwięk lub jest on bardzo cichy, choć to wymaga paru dodatkowych rund koncentracji. Aby odróżnić dźwięk, którego poszukujesz, od głośnego środowiska, także powinieneś poświęcić parę rund na skupienie, gdy przeszukujesz słuchem swoją okolicę. Mając odpowiednio dużo czasu, możesz wyśledzić każdą konwersację, oddychającą istotę i każde urządzenie wydające dźwięk w zasięgu. Akcja by rozpocząć, do paru rund by ją zakończyć, w zależności od trudności zadania.

\textbf{Anegdota}\index{Zdolności!Alfabetycznie!Anegdota}\label{sec:Anegdota} (2 punkty Intelektu) - możesz polepszyć morale grupy istot i pomóc im w nawiązaniu więzi, poprzez zabawianie ich podnoszącą na duchu anegdotą. Przez następną godzinę, ci którzy słuchali Twojej historii są wyszkoleni w jednym zadaniu Twojego wyboru, które jest powiązane z anegdotą, tak długo, jak nie jest to atak lub obrona. Akcja by rozpocząć, jedna minuta by zakończyć.

\textbf{Zwierzęce Szpiegowanie}\index{Zdolności!Alfabetycznie!Zwierzęce Szpiegowanie}\label{sec:Zwierzęce Szpiegowanie} (4+ punkty Intelektu) - jeśli znasz ogólną lokalizację zwierzęcia, które jest przyjazne względem Ciebie i w zasięgu 1.5 km od Ciebie, możesz postrzegać świat jego zmysłami do 10 minut. Jeśli nie jesteś w formie zwierzęcej lub w formie podobnej do tego zwierzęcia, musisz zastosować poziom Wysiłku do korzystania z tej umiejętności. Akcja by rozpocząć. 

\textbf{Zwierzęcy Kształt}\index{Zdolności!Alfabetycznie!Zwierzęcy Kształt}\label{sec:Zwierzęcy Kształt} (3+ punkty Intelektu) - zmieniasz się w zwierzę tam małe jak szczur lub tak duże jak ty (np: duży pies lub mały niedźwiedź) na 10 minut. Za każdym razem, gdy zmieniasz kształt, możesz wybrać inne zwierzę. Twój ekwipunek staje się częścią owej transformacji, co czyni go nieużytecznym, o ile nie ma pasywnego efektu, takiego jak zbroja. W tej formie Twoje Statystyki pozostają takie same jak w Twojej normalnej formie, ale możesz się ruszać i atakować zgodnie z Twoim zwierzęcym kształtem (ataki większości zwierząt tego rozmiaru to bronie średnie, z których możesz korzystać bez żadnej kary). Zadania wymagające rąk - takie jak naciskanie klamek lub przycisków są utrudnione kiedy jesteś w formie zwierzęcej. Nie możesz mówić, ale dalej możesz korzystać ze zdolności, które nie polegają na ludzkiej mowie. Uzyskujesz dwie pomniejsze zdolności powiązane z istotą, w którą sie zmieniłeś (patrz tabela Mniejsze Zdolności Zwierzęcego Kształtu). Dla przykładu, jeśli zamieniasz się w nietoperza, jesteś wyszkolony w percepcji i możesz latać na daleki zasięg w każdej rundzie. Jeśli zamienisz się w ośmiornicę, jesteś wyszkolony w skradaniu się i oddychasz pod wodą. Jeśli zastosujesz poziom Wysiłku do stosowania tej zdolności, możesz albo przybrać kształt mówiącego zwierzęcia, albo hybrydowy. 