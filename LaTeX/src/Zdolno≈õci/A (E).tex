% hyperref !!! https://tex.stackexchange.com/questions/180571/making-clickable-links-to-sections-with-hyperref

\chapter{Zdolności w kolejności alfabetycznej}

\section{A}

\textbf{Uśmiech i Słowo}\index{Zdolności!Alfabetycznie!Uśmiech i Słowo}\label{sec:Uśmiech i Słowo} - kiedy korzystasz z Wysiłku do dowolnej akcji interakcji społecznej - nawet takiej która polega na uspokajaniu zwierząt lub komunikowania się z kimś, czyim językiem nie mówisz - uzyskujesz darmowy poziom Wysiłku na tym zadaniu. Akcja.

\textbf{Przydatna Pomoc}\index{Zdolności!Alfabetycznie!Przydatna Pomoc}\label{sec:Przydatna Pomoc} - kiedy pomagasz komuś z zadaniem i stosuje on poziom wysiłku, zyskuje on darmowy poziom Wysiłku na tym zadaniu. Umożliwienie. 

\textbf{Absorpcja Energii}\index{Zdolności!Alfabetycznie!Absorpcja Energii}\label{sec:Absorpcja Energii} (7 punktów Intelektu) - dotykasz obiektu i absorbujesz jego energię. Jeśli dotykasz zamanifestowanego Cyphera, czynisz go bezużytecznym. Jeśli dotykasz artefaktu, rzuć na jego wyczerpanie. Jeśli dotykasz innego rodzaju zasilanego urządzenia lub maszyny, GM określa, czy jego moc jest w pełni wyssana. W każdym razie, absorbujesz energię z obiektu i odzyskujesz 1k10 punktów Intelektu. Jeśli to dałoby Ci więcej punktów Intelektu niż maksimum Twojej Puli, dodatkowe punkty są utracone, i musisz wykonać rzut na Obronę Mocy. Trudność tego rzutu to numer punktów powyżej Twojego maksimum, które zaabsorbowałeś. Jeśli oblejesz ten rzut, otrzymujesz 5 punktów obrażeń i nie możesz podejmować działań przez jedną rundę. Możesz wykorzystać tę zdolność jako akcję obronną kiedy jesteś celem ataku zdolnością. Taka akcja niweluje atak zdolnością, a ty absorbujesz energię, jakby pochodziła z urządzenia. Akcja.

\textbf{Absorpcja Energii Kinetycznej}\index{Zdolności!Alfabetycznie!Absorpcja Energii Kinetycznej}\label{sec:Absorpcja Energii Kinetycznej} - absorbujesz porcję energii ataku fizycznego lub uderzenia. Negujesz 1 punkt obrażeń, które normalnie byś poniósł i przechowujesz tę energię. Po tym, jak zaabsorbujesz 1 punkt energii, kontynuujesz obniżać obrażenia o 1 punkt z nadchodzących ataków, ale pozostała energia wycieka z Ciebie w formie błysku nieszkodliwego światła (nie możesz przechowywać na raz więcej niż 1 punktu energii w tym samym czasie). Umożliwienie. 

\textbf{Absorpcja Czystej Energii}\index{Zdolności!Alfabetycznie!Absorpcja Czystej Energii}\label{sec:Absorpcja Czystej Energii} - kiedy korzystasz z Absorpcji Energii Kinetycznej, możesz także absorbować i przechowywać energię ataków bazujących na czystej energii (światło, promieniowanie, energie międzywymiarowe, psioniczne itp.) lub z przekaźników owej energii, gdy masz z nimi bezpośredni kontakt. Ta zdolność nie zmienia tego, ile punktów energii możesz przechowywać. Jeśli masz również Ulepszoną Absorpcję Energii Kinetycznej, możesz również absorbować do 2 punktów obrażeń ze źródeł czystej energii. Umożliwienie. 

\textbf{Akceleracja}\index{Zdolności!Alfabetycznie!Akceleracja}\label{sec:Akceleracja} (4+ punkty Intelektu) - Twoje słowa umacniają ducha postaci w bliskim zasięgu, która jest w stanie zrozumieć Cię, przyspieszając ją, tak, że zyskuje ona atut na testach inicjatywy i rzutach na Obronę Szybkości przez 10 minut. Dodatkowo, poza zwykłymi opcjami korzystania z Wysiłku, możesz z niego skorzystać, by objąć celem tej zdolności więcej postaci - każdy poziom Wysiłku obejmuje dodatkowy cel. Musisz przemówić do dodatkowych celów, by je przyspieszyć, jeden cel na rundę. Jednak akcja na jeden cel by rozpocząć. 

\textbf{Akrobatyczny Atak}\index{Zdolności!Alfabetycznie!Akrobatyczny Atak}\label{sec:Akrobatyczny Atak} (1+ punktów Szybkości) - wyskakujesz w ataku, przesuwając się przez powietrze. Jeśli wyrzucasz naturalne 17 lub 18, możesz wybrać mniejszy efekt zamiast dodatkowych obrażeń. Jeśli zastosujesz Wysiłek do tego ataku, uzyskujesz darmowy poziom Wysiłku na zadaniu. Nie możesz skorzystać z tej zdolności, jeśli Twój Wysiłek Szybkości jest zredukowany wskutek noszenia zbroi. Umożliwienie. 

\textbf{Procesor Akcji}\index{Zdolności!Alfabetycznie!Procesor Akcji}\label{sec:Procesor Akcji} (4 punkty Intelektu) - korzystając z przechowywanych informacji i zdolności analizowania nadchodzących danych z wielką szybkością, jesteś wyszkolony w jednym fizycznym zadaniu Twojego wyboru na 10 minut. Dla przykładu, możesz wybrać bieg, wspinaczkę, pływanie, Obronę Szybkości lub atak specyficzną bronią. Akcja by rozpocząć.

\textbf{Adaptacja}\index{Zdolności!Alfabetycznie!Adaptacja}\label{sec:Adaptacja} - dzięki ukrytej mutacji, urządzeniu wbudowanemu w Twój kręgosłup, rytuałowi krwii smoka, lub jakiemuś innemu darowi, jesteś teraz w komfortowej temperaturze; nie musisz sie nigdy martwić o niebezpieczne promieniowanie, choroby lub gazy; i możesz zawsze oddychać w dowolnym środowisku (nawet w próżni kosmosu). Umożliwienie.

\textbf{Zaawansowany Użytkownik Cypherów}\index{Zdolności!Alfabetycznie!Zaawansowany Użytkownik Cypherów}\label{sec:Zaawansowany Użytkownik Cypherów} - możesz mieć przy sobie 4 Cyphery w danym czasie. Umożliwienie. 

\textbf{Zaawansowany Rozkaz}\index{Zdolności!Alfabetycznie!Zaawansowany Rozkaz}\label{sec:Zaawansowany Rozkaz} (7 punktów Intelektu) - cel w średnim zasięgu słucha każdej komendy, którą mu wydasz, tak długo, jak słyszy Cię i rozumie. Co więcej, tak długo, jak nie robisz nic innego niż wydawanie komend (nie wolno Ci wziąć żadnej innej akcji) możesz dać temu samemu celowi nową komendę. Ten efekt kończy się, gdy kończysz wydawać komendy lub gdy cel opuszcza średni zasięg względem Ciebie. Akcja by rozpocząć. 

\textbf{Atak z Rozbrojeniem}\index{Zdolności!Alfabetycznie!Atak z Rozbrojeniem}\label{sec:Atak z Rozbrojeniem} (3 punkty Szybkości) - za pomocą serii szybkich ruchów, wykonujesz atak przeciwko uzbrojonemu przeciwnikowi, zadając mu obrażenia i rozbrajając go, tak, że jego broń jest teraz w Twoich rękach lub 3 metry od niego na ziemi - Ty wybierasz. Ten atak rozbrajający jest utrudniony. Akcja.

\textbf{Zalety Bycia Dużym}\index{Zdolności!Alfabetycznie!Zalety Bycia Dużym}\label{sec:Zalety Bycia Dużym} - kiedy korzystasz ze Wzrostu, jesteś tak duży, że możesz łatwiej przenosić duże obiekty, wspinać sie na budynki korzystając z uchwytów niedostępnych dla zwykłych ludzi i skakać znacznie dalej. Kiedy korzystasz ze Wzrostu, wszystkie zadania wspinaczki, podnoszenia ciężarów i skakania są dla Ciebie ułatwione. Umożliwienie.

\textbf{Zalety Bycia Małym}\index{Zdolności!Alfabetycznie!Zalety Bycia Małym}\label{sec:Zalety Bycia Małym} - nauczyłeś się, jak wykorzystać swój rozmiar, siłę i dokładność. Twoje obrażenia już się nie dzielą na pół gdy korzystasz ze Zmniejszenia się, a zadania wspinaczki i skakania są ułatwione. Umożliwienie.

\textbf{Porada od Przyjaciela}\index{Zdolności!Alfabetycznie!Porada od Przyjaciela}\label{sec:Porada od Przyjaciela} (1 punkt Intelektu) - znasz słabe i mocne strony swojego przyjaciela, i wiesz jak go zmotywować, by osiągnął sukces. Kiedy dajesz przyjacielowi sugestię powiązaną z jego następną akcję, postać ta jest wyszkolona w tej akcji na jedną rundę. Akcja. 

\textbf{Znowu i Znowu}\index{Zdolności!Alfabetycznie!Znowu i Znowu}\label{sec:Znowu i Znowu} (8 punktów Szybkości) - możesz wziąć kolejną akcję w rundzie, w której już podjąłeś akcję. Umożliwienie.

\textbf{Nieśmiertelny}\index{Zdolności!Alfabetycznie!Nieśmiertelny}\label{sec:Nieśmiertelny} - Twoje ciało i umysł się nie starzeją. Jeśli nie zostaniesz zabity przez akt przemocy (lub jakąś zewnętrzną siłę jak trucizna lub infekcja), nigdy nie umrzesz. Umożliwienie.  

\textbf{Agent-Prowokator}\index{Zdolności!Alfabetycznie!Agent-Prowokator}\label{sec:Agent-Prowokator} - wybierz jedna z poniższych, by być wytrenowanym w: atakowanie bronią swojego wyboru, ładunki wybuchowe, lub skradanie się i otwieranie zamków (jeśli wybierzesz ostatnią opcję, posiadasz trening w dwóch umiejętnościach). Umożliwienie.

\textbf{Agresja}\index{Zdolności!Alfabetycznie!Agresja}\label{sec:Agresja} (2 punkty Mocy) - skupiasz się na atakowaniu w tak wielki sposób, że zostawiasz siebie wysuniętego na ataki wrogów. Kiedy ta zdolność jest aktywna, zyskujesz atut na atakach wręcz i Twoje rzuty na Obronę Szybkości przeciwko atakom wręcz i dystansowym są utrudnione. Ten efekt trwa tak długo, jak sobie życzysz ale kończy się, jeśli walka nie ma miejsca w zasięgu Twoich zmysłów. Umożliwienie.

\textbf{Szybki Umysł}\index{Zdolności!Alfabetycznie!Szybki Umysł}\label{sec:Szybki Umysł} - kiedy próbujesz wykonać zadanie Szybkości, możesz zamiast tego rzucić (i wydać punkty z puli) jakby to była akcja Intelektu. Jeśli stosujesz Wysiłek do tego zadania, możesz wydać punkty z Puli Intelektu zamiast Puli Szybkości (wtedy stosujesz też Skupienie w Intelekcie zamiast w Szybkości). Umożliwienie. 

\textbf{Wysokie Skupienie}\index{Zdolności!Alfabetycznie!Wysokie Skupienie}\label{sec:Wysokie Skupienie} (7 punktów Intelektu) - wkładasz w swoje zadanie wszystko. Dodajesz trzy darmowe poziomy Wysiłku to następnego zadania, które podejmujesz. Nie możesz wykorzystać tej zdolności znowu, dopóki nie zakończysz 10-godzinnego odpoczynku. Akcja.

\textbf{Uzdrowienie}\index{Zdolności!Alfabetycznie!Uzdrowienie}\label{sec:Uzdrowienie} (3 punkty Intelektu) - Możesz spróbować uzdrowić jedno schorzenie (np: chorobę lub truciznę) dotyczące jednej istoty. Akcja.

