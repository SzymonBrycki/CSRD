\section{Deskryptor}\index{Deskryptory}

Twój deskryptor definiuje Twoją postać – koloruje wszystko, co robisz. Różnice między Urokliwym Odkrywcą a Złośliwym Odkrywcą są dosyć spore. Deskryptory zmieniają praktycznie wszystkie akcje owych postaci. Twój deskryptor stawia postać w sytuacji (pierwszej przygodzie, która zaczyna kampanię) i pomaga dać jej motywację. Jest to przymiotnik w zdaniu “Jestem przymiotnik rzeczownik który czasownikuje". 

Deskryptory oferują jednorazową paczkę dodatkowych umiejętności, zdolności i modyfikatorów do Twoich Statystyk. Nie wszystkie oferowane przez deskryptor modyfikacje są pozytywne. Dla przykładu, niektóre deskryptory posiadają nieumiejętności – zadania, w których postać sobie nie radzi. Możesz myśleć o nieumiejętnościach jak o negatywnych umiejętnościach – zamiast ułatwiać zadanie o krok, czynią one je o krok trudniejszym. Jeśli zdobywasz trening w sferze, w której masz nieumiejętność, nawzajem się one znoszą. Pamiętaj, że postaci są zdefiniowane równie mocno przez to, z czym sobie nie radzą, jak i przez to, w czym są dobre. 

Deskryptory oferują także krótkie sugestie odnośnie tego, jak postać zapoznała się z resztą grupy na swojej pierwszej przygodzie. Możesz z nich skorzystać, jeśli tylko sobie tego życzysz (lub nie, Twoja wola).

Ta sekcja zawiera 50 deskryptorów. Wybierz jeden z nich dla swojej postaci. Możesz wybrać dowolny deskryptor, niezależnie od swojego typu. Na końcu rozdziału jest parę opcji customizacyjnych, wliczając w to tworzenie rasy jako deskryptora.
(Deskryptor ma największe znaczenie dla początkującej postaci. Benefity (i utrudnienia) związane z deskryptorem będą ostatecznie przyciemnione przez rosnące znaczenie typu i specjalności. Jednakże, deskryptor dalej będzie odgrywał pewne znaczenie w ciągu życia postaci). 

\subsection{Lista Deskryptorów}\index{Deskryptory!Lista}

\textbf{Bystrooki}\index{Deskryptory!Bystrooki}

Jesteś percepcyjny i dobrze świadom swojego otoczenia. Dostrzegasz małe detale i zapamiętujesz je. Jesteś trudny do zaskoczenia.

Otrzymujesz poniższe cechy:
\begin{itemize}
\item Umiejętność: Jesteś wyszkolony w inicjatywie.
\item Umiejętność: Jesteś wyszkolony w percepcji.
\item Znaleźć Wadę: Jeśli Twój przeciwnik ma jakaś oczywistą wadę (otrzymuje więcej obrażeń od ognia, nie widzi na lewe oko itp.) MG powie Ci o tym.
\end{itemize}
    
Początki Przygód: Z poniższej listy opcji, wybierz jak Twoja postać wzięła udział w pierwszej przygodzie.

1. Słyszałeś, co się święciło, dostrzegłeś wadę w planach BG i dołączyłeś do nich, by im pomóc.
2. Zauważyłeś, że BG mieli wroga (lub przynajmniej ktoś ich śledził) i nie byli tego świadomi.
3. Dostrzegłeś, że inni BG robili coś ciekawego i się do nich przyłączyłeś.
4. Od pewnego czasu działy się dziwne rzeczy, i to wszystko wydaje się być powiązane.

\textbf{Bystry}\index{Deskryptory!Bystry}

Myślisz szybko i bystro. Rozumiesz ludzi i możesz ich oszukiwać, ale sam rzadko kiedy dajesz się omamić. Ponieważ łatwo widzisz rzeczy takimi, jakie są, wchodzisz szybko, oceniasz zagrożenia i sprzymierzeńców, a potem robisz swoją robotę. Może jesteś fizycznie atrakcyjny, lub może korzystasz ze swojego mózgu, by przezwyciężyć fizyczne i psychiczne niedoskonałości. 

Otrzymujesz poniższe cechy:
\begin{itemize}
\item Bystry: +2 do Puli Intelektu.
\item Umiejętność: Jesteś wyszkolony we wszystkich interakcjach typu kłamstwa lub sztuczki.
\item Umiejętność: Jesteś wyszkolony w rzutach na przeciwstawianie się efektom mentalnym.
\item Umiejętność: Jesteś wyszkolony we wszystkich zadaniach identyfikowania lub oceniania niebezpieczeństwa, kłamstw, jakości, ważności, funkcji lub mocy.
\item Nieumiejętność: Nigdy nie byłeś dobry w nauce lub przywoływaniu z pamięci trywialnej wiedzy. Każde zadanie związane z wiedzą lub zrozumieniem jest utrudnione.
\item Dodatkowy ekwipunek: Łatwo dostrzegasz plany innych i czasami przekonujesz ich, by Tobie wierzyli – nawet kiedy nie powinni. Dzięki swojemu bystremu zachowaniu, masz dodatkowy drogi przedmiot.
\end{itemize}

Początki Przygód: Z listy poniższych opcji, wybierz jak Twoja postać wzięła udział w pierwszej przygodzie.

1. Przekonałeś jednego z reszty BG, by powiedział Ci co robią.
2. Z daleka dostrzegłeś, że coś ciekawego ma miejsce.
3. Wplątałeś się w tę sytuację, bo myślałeś, że możesz w ten sposób zarobić trochę pieniędzy.
4. Spodziewałeś się, że bez Ciebie BG nie odniosą sukcesu.

\textbf{Chaotyczny}\index{Deskryptory!Chaotyczny}

Niebezpieczeństwo nie ma dla Ciebie dużego znaczenia, głównie dlatego, że nie myślisz o konsekwencjach. Po prawdzie, rozkoszujesz się zaskakiwaniem, żeby tylko zobaczyć, co się stanie. Im bardziej niespodziewany rezultat, tym jesteś szczęśliwszy. Czasami jesteś szczególnie maniakalny i dla swoich towarzyszy powstrzymujesz się przed podejmowaniem akcji, które mogą prowadzić do zagłady. 

Otrzymujesz następujące cechy:
\begin{itemize}
 \item Wzburzony: +4 do Puli Szybkości.
 \item Umiejętność: Jesteś wyszkolony w Obronie Intelektu.
 \item Chaotyczny: Raz po każdym 10-godzinnym rzucie na odzyskanie zdrowia, jeśli nie lubisz swojego wyniku, możesz przerzucić kostkę. Jeśli to zrobisz, niezależnie od wyniku, MG stosuje względem Ciebie wtrącenie MG.
\item Nieumiejętność: Twoje ciało jest nieco zmęczona całą tąchaotycznością. Obrona Mocy jest dla Ciebie utrudniona.
\end{itemize}

Początki Przygód: Z listy poniższych opcji, wybierz jak Twoja postać wzięła udział w pierwszej przygodzie.

1. Inny BG zrekrutował Cię, kiedy dobrze się zachowywałeś, nie zdawszy sobie sprawy z tego, jak bardzo jesteś chaotyczny.
2. Masz powody by wierzyć, że inni BG pomogą Ci kontrolować swoje chaotyczne zachowanie.
3. Inny BG wyzwolił Cię z niewoli, i aby mu podziękować, zaoferowałeś swoją pomoc.
4. Nie masz bladego pojęcia, czemu dołączyłeś do BG. Po prostu to robisz, a odpowiedzi poznasz z czasem.

\textbf{Ciekawy}\index{Deskryptory!Ciekawy}

Świat jest wielki i tajemniczy, z cudami i sekretami, które starczą na kilka żywotów. Czujesz wołanie w swoim sercu, zew, by badać pozostałości dawnych cywilizacji, by odkryć nowe ludy, nowe miejsca, i jakiekolwiek dziwne cuda znajdziesz po drodze. Jednakże, choć czujesz potężną chęć, by podróżować po świecie, wiesz, że jest pełen niebezpieczeństw, i musisz się zabezpieczyć na każdą możliwość. Badania, przygotowania i gotowość pomogą Ci żyć dostatecznie długo, by zobaczyć wszystko co chcesz zobaczyć i zrobić wszystko, co chcesz zrobić.

Najpewniej masz tuzin książek i map przy sobie w dowolnym czasie. Kiedy nie podróżujesz i nie chłoniesz widoków, spędzasz czas z nosem w książce, uczać się o miejscu, do którego podrożujesz, żebyś wiedział, czego się spodziewać w tym miejscu.

Otrzymujesz poniższe cechy:
\begin{itemize}
\item  Bystry: +4 do Puli Intelektu.
\item Umiejętność: Jesteś chętny do nauki. Jesteś wyszkolony w każdym zadaniu, które jest nauką czegoś nowego, niezależnie, czy to lokalna informacja, czy przeszukiwanie starych ksiąg wiedzy.
\item  Umiejętność: Uczyłeś się o świecie. Jesteś wyszkolony w każdym zadaniu powiązanych z geografią lub historią.
\item Nieumiejętność: Skupiasz się na detalach, co sprawia, że jesteś nieco zagubiony. Wszelkie zadania w celu usłyszenia lub dostrzeżenia czegoś są dla Ciebie utrudnione.
\item Nieumiejętność: Kiedy widzisz coś interesującego, wahasz się, aavy dostrzec wszystkie detale. Rzuty na inicjatywę (określające kto rusza pierwszy w walce) są dla Ciebie utrudnione.
\item Dodatkowy Ekwipunek: Masz trzy ksiażki o dowolnych tematach, które wybierzesz.
\end{itemize}

Początki Przygód: Z listy poniższych opcji, wybierz jak Twoja postać wzięła udział w pierwszej przygodzie.

1. Jeden z BG podszedł do Ciebie, aby zdobyć informację powiązaną z misją, słysząc, że jesteś ekspertem.
2. Zawsze chciałeś zobaczyć miejsce, do którego podróżuje reszta BG.
3. Zainteresowało Cię to, co chcieli zrobić inni BG i zadecydowałeś, że będziesz im towarzyszyć.
4. Jeden z BG Cię fascynuje, może ze względu na specjalną lub dziwną zdolność, którą on dysponuje.


\textbf{Dziki}\index{Deskryptory!Dziki}

Kochasz dziką przyrodę i jesteś przyzwyczajony do życia w trudnych warunkach, wśród dziczy. Najpewniej jesteś uzdolnionych łowcą lub naturalistą. Lata życia w dziczy zostawiły swoje znaki na Twoim ciele – masz znoszone ubrania, burzę dzikich włosów lub blizny. Twoje ubrania są najpewniej znacznie mniej modne niż ubrania ludzi mieszkających w miastach. 

Otrzymujesz poniższe cechy:
\begin{itemize}
 \item Umiejętność: Jesteś wyszkolony w wszystkich zadaniach typu wspinaczka, skakanie, bieganie i pływanie.
\item Umiejętność: Jesteś wyszkolony we wszystkich zadaniach typu trenowanie, ujeżdzanie i uspokajanie naturalnych zwierząt.
\item Umiejętność: Jesteś wyszkolony we wszystkich zadaniach polegających na identyfikacji lub używaniu naturalnych roślin.
\item Nieumiejętność; Nie jesteś bardzo społeczny i wolisz towarzystwo zwierząt niż ludzi. Wszelkie zadania polegające na oczarowaniu innych ludzi, perswazji, etykiecie lub oszustwie są dla Ciebie utrudnione.
\item Dodatkowy Ekwipunek: Posiadasz pakiet odkrywcy, z liną, dwoma dniami racji żywnościowych, matą do spania i innymi narzędziami potrzebnymi do przetrwania na zewnątrz.
\end{itemize}    
    
Początki Przygód: Z listy poniższych opcji, wybierz jak Twoja postać wzięła udział w pierwszej przygodzie.

1. Pomimo Twojego lepszemu osądowi, dołączyłsz do BG, ponieważ widziałeś, że są w niebezpieczeństwie.
2. Jeden z BG przekonał Cię, że dołączenie do grupy jest w Twoim najlepszym interesie.
3. Boisz się, co może się wydarzyć, jeśli BG odniosą klęskę.
4. Jest mowa o nagrodzie, a Ty potrzebujesz pieniędzy.

\textbf{Dziwny}\index{Deskryptory!Dziwny
}
Nie jesteś taki jak inni, i to w porządku (według Ciebie). Ludzie nie są w stanie Cię zrozumieć – niektórzy nawet się Ciebie boją – ale kogo to obchodzi? Rozumiesz świat lepiej niż oni, ponieważ jesteś dziwny, zupełnie jak świat, w którym żyjesz. Koncept “dziwności” jest Tobie dobrze znany. Dziwne urządzenia, antyczne miejsca, dziwne istoty, burze, które mogą Cię zmutować, żyjące pola energii, konspiracje, obcy i rzeczy, których większość ludzi nie mogłaby nazwać istnieją w tym świecie, a Ty masz się z tym świetnie. Masz specjalną więź z tym wszystkim, a im więcej odkrywasz dziwności w świecie, tym lepiej rozumiesz samego siebie. Dziwne postaci mogą być mutantami lub ludźmi z wrodzonymi dziwnymi cechami, ale czasami zaczęli jako “normalni” i dziwaczeli dopiero w trakcie życia.

Otrzymujesz poniższe cechy:
\begin{itemize}
\item Wewnętrzne Światło: +2 do Puli Intelektu.
\item Fizyczne Znamię Dziwności: Masz unikalną cechę fizyczną, która jest, no cóż, dziwna. W zależności od settingu, mogą to być różne rzeczy. Może masz fioletowe włosy lub metalowe kolce w swojej głowie. Może Twoje ręce nie łączą się z Twoimi ramionami, choć poruszają się tak, jakby były połączone. Może masz trzecie oko na czole, a może bezużyteczne wici wyrastają z Twoich pleców. Cokolwiek to jest, Twoje znamię może być mutacją, cechą nadprzyrodzoną (błogosławieństwem lub klątwą), nie mieć żadnego wyjaśnienia lub być po prostu naprawdę dzikim tatuażem, który przyciąga mnóstwo uwagi.
\item Zmysł Dziwności: Czasami – za uznaniem MG – dziwne rzeczy związane z zjawiskami nadprzyrodzonymi lub ich wpływem na świat coś w Tobie budzą. Możesz je wyczuć z daleka, a jeśli znajdziesz się w dalekim zasięgu od takiej rzeczy, możesz wyczuć czy jest niebezpieczna czy nie.
\item Umiejętność: Jesteś wyszkolony w nadprzyrodzonej wiedzy.
 \item Nieumiejętność: Ludzie uważają Cięza dziwnego. Wszystkie zadania związane z przyjemną interakcją społeczną są dla Ciebie utrudnione.
\end{itemize}
    
Początki Przygód: Z listy poniższych opcji, wybierz jak Twoja postać wzięła udział w pierwszej przygodzie.

1. Wyglądało to na dziwne, czemu więc nie?
2. Niezależnie od tego, czy inni BG zdają sobie z tego sprawę, czy też nie, ich misja jest powiązana z czymś dziwnym, o czym wiesz, więc wziąłeś w niej udział.
3. Jako ekspert od dziwności, zostałeś zrekrutowany przez innych BG.
4. Poczułeś dążenie, by dołączyć do innych BG, ale nie wiesz czemu.

\textbf{Empatyczny}\index{Deskryptory!Empatyczny}

Inni ludzie to dla Ciebie otwarte księgi. Możesz mieć talent do odczytania ludzkich uczuć, tych subtelnych ruchów, które zawierają w sobie nastrój i emocje. Lub możesz otrzymywać informacje w bardziej bezpośredni sposób, czując emocje danej osoby jakby były materialne, wrażenia, które odbiera Twój umysł. Twój dar empatii pomaga Ci nawigować w sytuacjach społecznych i kontrolować je, by uniknąć nieporozumień i uniemożliwić erupcje bezsensownych konfliktów. 
Ciągłe bombardowanie emocjami ludzi wokół Ciebie jest jednak męczące. Możesz się dać porwać dominującemu nastrojowi, mieć wahania nastrojów od radości do smutku bez żadnego ostrzeżenia. Lub możesz się zamknąć w sobie i pozostać zagadką dla innych, ze względu na chęć chronienia siebie i podświadomy lęk, że ktoś może się dowiedzieć, jak się naprawdę czujesz.
Otrzymujesz poniższe cechy:
    • Otwarty Umysł: +4 do Puli Intelektu
    • Umiejętność: Jesteś wyszkolony w zadaniach związanych z odczuwaniem emocji innych i przeczuć odnośnie ludzi wokół Ciebie.
    • Umiejętność: Jesteś wyszkolony we wszystkich akcjach związanych ze społecznymi interakcjami, przyjemnymi bądź nie.
    • Nieumiejętność: Bycie tak bardzo otwartym na myśli i nastroje innych czyni Cię narażonym na ataki mentalne. Twoja Obrona Intelektu jest utrudniona.
Początki Przygód: Z listy poniższych opcji, wybierz jak Twoja postać wzięła udział w pierwszej przygodzie.
1. Wyczułeś oddanie do zadania innych BG i postanowiłeś im pomóc.
2. Utworzyłeś bliską więź z innym BG i nie chcesz się z nim rozstawać.
3. Wyczułeś coś dziwnego w jednym z BG i zdecydowałeś się dołączyć do ich grupy by sprawdzić, czy wyczujesz to raz jeszcze i odkryjesz prawdę.
4. Dołączyłeś do BG by uciec od nieprzyjemnej relacji lub negatywnego środowiska. 
Honorowy
Jesteś godny zaufania, uczciwy i szczery. Próbujesz zrobić to, co powinno być zrobione, pomagać innym i traktować ich dobrze. Kłamanie i oszukiwanie nie są sposobem, by triumfować w życiu – to wybory słabych, leniwych i godnych potępienia. Najpewniej spędzasz dużo czasu myśląc o Twoim osobistym honorze, jak najlepiej go utrzymać i jak bronić go jeśli jest zagrożony. W walce jesteś bezpośredni i oferujesz litość każdemu z wrogów.
Najpewniej poczucie honoru zaszczepił w Tobie rodzic bądź mentor. Czasami rozróżnienie między tym co jest, a co nie jest honorowe jest zależne od szkoły myślowej, ale ogólnie rzecz ujmując, honorowi ludzie mogą się zgodzić co do większości aspektów znaczenia honoru.
Otrzymujesz następujące cechy:
    • Dzielny: +2 do Puli Mocy.
    • Umiejętność: Jesteś wyszkolony w przyjemnych interakcjach społecznych.
    • Umiejętność: Jesteś wyszkolony w odczytywaniu prawdziwych intencji innych osób i dostrzeganiu kłamstw.
Początki Przygód: Z listy poniższych opcji, wybierz jak Twoja postać wzięła udział w pierwszej przygodzie.
1. Cele BG wydają się być honorowe.
2. Widzisz, ze to, co chcą zrobić BG, jest niebezpieczne, i chciałbyś im pomóc.
3. Jeden z BG zaprosił Cię, słysząc, że jesteś godny zaufania.
4. Zapytałeś grzecznie, czy mógłbyś dołączyć do BG.
Idiotyczny
Nie każdy może być bystry jak rzeka.  Oh nie, nie myślisz o sobie jak o głupim, co to to nie. Po prostu inni zdają się mieć więcej… mądrości. Wejrzenia w rzeczy. Preferujesz ruszać prosto do celu poprzez życie i pozwalasz innym martwić się rzeczami. Zamartwianie się nigdy Ci nie pomogło, więc po co to robić? Bierzesz rzeczy na klatę i nie martwisz się dniem jutrzejszym.
Ludzie swą Cię “idiotą” bądź “głupcem”, ale nie wpływa to jakoś bardzo mocno na Ciebie. 
(Może być bardzo fajnym doświadczeniem odgrywaniem idiotycznej postaci. W pewien sposób zrzuca to presję, by zawsze robić dobre, mądre rzeczy. Z drugiej strony, jeśli grasz taką postacią jako jawnym głupcem w każdej sytuacji, może to być denerwujące dla innych przy stole. Jak ze wszystkim, należy znaleźć złoty środek i rozmawiać z innymi graczami o naszych potrzebach i uwagach krytycznych.)
Zyskujesz poniższe cechy:
    • Niemądry: -4 do puli Intelektu
    • Wolny: Polegasz na szczęściu bardziej niż na czymkolwiek innym. Za każdym razem, gdy rzucasz kością na jakieś zadanie, rzuć dwa razy i weź wyższy wynik.
    • Słabość Intelektu: Za każdym razem, gdy wydajesz punkty z Puli Intelektu, kosztuje Cię to o 1 punkt więcej niż normalnie.
    • Nieumiejętność: Twoja Obrona Intelektu jest utrudniona.
    • Nieumiejętność: Każde zadanie, które wymaga dostrzeżenia kłamstwa, iluzji lub pułapki jest utrudnione.
Początki Przygód: Z listy poniższych opcji, wybierz jak Twoja postać wzięła udział w pierwszej przygodzie.
1. Kto wie? Wyglądało to jak dobry pomysł.
2. Ktoś poprosił Cię o dołączenie do BG. Ta osoba chciała, byś nie zadawał zbyt wiele pytań, więc tak postąpiłeś.
3. Twój ojciec (lub rodzic/mentor) chciał, żebyś miał zajęcie i może “zdobył trochę rozsądku”.
4. Inni BG potrzebowali siłacza, który nie zastanawiałby się zbyt długo nas swoimi zadaniami.
Impulsywny
Masz problem z ograniczaniem swojego entuzjazmu. Po co czekać, kiedy można to po prostu zrobić (cokolwiek to jest) i mieć problem z głowy? Radzisz sobie z problemami w momencie, gdy powstają, zamiast planować w przyszłość. Gaszenia małych ognisk sprawia, że potem nie będzie wielkich pożarów. Jesteś pierwszym, który ryzykuje i udziela pomocy, który wstępuje w ciemne korytarze i który znajduje niebezpieczeństwo.
Twoja impulsywność z pewnością sprawia, że niekiedy masz kłopoty. Kiedy inni mogą poświęcić czas na studiowanie przedmiotów, które pozyskali, TY korzystasz z nich bez chwili wahania. Przecież najlepszym sposobem nauki jest akt sprawczy. Kiedy ostrożny odkrywca może rozejrzeć się i poszukać niebezpieczeństw, Ty musisz powstrzymać się siłą przed ruszeniem naprzód. Po co czekać, skoro wokół jest tyle ekscytujących rzeczy?
(Impulsywne postaci ściągają na siebie kłopoty. To ich cecha stała i to jest ok. Ale jeśli ciągle ściągasz innych BG w środek kłopotów (lub, co gorsza, sprawiasz, że są zranieni lub martwi), to będzie to bardzo wkurzające, oględnie mówiąc. Dobrą regułą jest stwierdzenie, że impulsywność nie zawsze oznacza robienie złych rzeczy. Czasami, jest to żądza, by zrobić coś właściwego.)
Otrzymujesz następujące cechy:
    • W Gorącej Wodzie Kąpany: +2 do puli Szybkości.
    • Umiejętność: Jesteś wyszkolony w rzutach na inicjatywę (by określić, kto działa pierwszy w walce)
    • Umiejętność: Jesteś wyszkolony w Obronie Szybkości.
    • Nieumiejętność: Możesz spróbować wszystkiego jeden raz, ale następne próby Cię nużą. Każde zadanie, które wymaga cierpliwości, siły woli lub dyscipliny jest dla Ciebie utrudnione. 
Początki Przygód: Z listy poniższych opcji, wybierz jak Twoja postać wzięła udział w pierwszej przygodzie.
1. Usłyszałeś co planowali inni BG i nagle zadecydowałeś, że do nich dołączysz.
2. Zebrałeś wszystkich razem po usłyszeniu plotki o czymś interesującym, co chcesz zobaczyć lub zrobić.
3. Wydałeś wszystkie swoje pieniądze i teraz potrzebujesz źródła dochodu.
4. Jesteś w kłopotach po podążaniu za głosem swojego serca. Dołączyłeś do BG, bo oferują sposób na wyjście ze swoich problemów.
Inteligentny
Jesteś dosyć bystry. Twoja pamięć jest ostra jak brzytwa, i z łatwością pojmujesz koncepty, których inni nie rozumieją. Nie oznacza to koniecznie, że masz za sobą lata formalnej edukacji, ale sporo się nauczyłeś w życiu, głównie przy okazji.
Zyskujesz poniższe cechy:
    • Bystry: +2 do Puli Intelektu.
    • Umiejętność:Jesteś wyszkolony w jednej domenie wiedzy swojego wyboru.
    • Umiejętność: Jesteś wyszkolony we wszystkich zadaniach polegającym a zapamiętywaniu tego, czego bezpośrednio doświadczyłeś. Dla przykładu, zamiast przypominać sobie o szczegółach geograficznych, o których czytałeś w książce, możesz pamiętać ścieżkę przez tunele, którymi wcześniej podążałeś.
Początki Przygód: Z listy poniższych opcji, wybierz jak Twoja postać wzięła udział w pierwszej przygodzie.
1. Inny z BG zapytał Cię o Twoją opinię o tej misji, wiedząc, że jeśli powiesz, że to dobry pomysł, to zapewne tak będzie.
2. Dostrzegłeś wartość w tym, co robią inni BG.
3. Wierzysz, że to zadanie może prowadzić do ważnych i ciekawych odkryć.
4. Kolega poprosił cię o wzięcie udziału w tej misji jako przysługę, którą mu byłeś winny.
Intuitywny
Często masz przeczucie co ktoś inny powie, jak zareaguje bądź jak wydarzenia się potoczą. Może masz jakiś zmysł mutanta, może możesz na parę chwil spojrzeć w bliską przyszłość, a może po prostu jesteś w stanie odczytać mimikę i mowę ciała ludzi. Niezależnie od powodu, wielu z tych, którzy patrzą w Twe oczy, natychmiast odwraca wzrok, jakby byli przerażeni tym, co w nich odczytasz.
Zyskujesz poniższe cechy:
    • Intuicja: +2 do Puli Intelektu.
    • Umiejętność: Jesteś wyszkolony w percepcji.
    • Wiesz, co Czynić: Możesz zareagować natychmiast, nawet jeśli to jeszcze nie Twoja tura. Potem, w Twojej następnej normalnej turze, każda akcja, którą wykonujesz, jest utrudniona. Możesz to zrobić jeden raz, ale ta opcja odnawia się po każdym rzucie na odzyskanie zdrowia.
Początki Przygód: Z listy poniższych opcji, wybierz jak Twoja postać wzięła udział w pierwszej przygodzie.
1. Po prostu wiedziałeś, że musisz z nimi wyruszyć.
2. Przekonałeś jednego z BG, że Twoja intuicja jest bezcennna.
3. Wyczułeś, że stanie się coś złego, jeśli z nimi nie wyruszysz.
4. Jesteś pewien, że powód, dla którego z nimi wyruszyłeś wkrótce stanie się jasny.
Kreatywny
Możesz mieć notatnik, w którym notujesz pomysły, które później rozwiniesz. Może wysyłasz samemu sobie maile w momencie, gdy nachodzi Cię inspiracja, by potem je zanotować w elektronicznym dokumencie. A może po prostu siadasz, patrzysz w ekran i niemożliwą siłą woli, tworzysz coś z niczego.
Niezależnie od tego, jak Twój dar działa, jestes kreatywny – kodujesz, piszesz, komponujesz, rzeźbisz, projektujesz, reżyserujesz lub w inny sposób tworzysz narracje, które zachwycająinnych ludzi.
Otrzymujesz następujące cechy:
    • Kreatywny: +2 do Puli Intelektu.
    • Oryginalny: Zawsze robisz coś nowego. Jesteś wyszkolony w każdym zadaniu, związanym z tworzeniem narracji (takiej jak historia, przedstawienie teatralne lub scenariusz). Wlicza się w to oszustwo, jeśli oszustwo jest częścią narracji, którą tworzysz.
    • Umiejętność: Jesteś naturalnym twórcą. Jesteś wyszkolony w jednej konkretnej umiejętności kreatywnej Twojego wyboru – pisaniu, tworzeniu oprogramowania, komponowaniu muzyki, malowaniu, rysowaniu itp.
    • Umiejętność: Kochasz rozwiązywać zagadki itp. Jesteś wyszkolony w rozwiązywaniu zagadek.
    • Umiejętność: Bycie kreatywnym wymaga ciągłego zdobywania wiedzy. Jesteś wyszkolony w każdym zadaniu związanym ze zdobywaniem nowych informacji, tak jak wtedy, gdy przekopujesz się przez bibliotekę, dane bankowe, archiwum newsów lub malą kolekcję źródeł wiedzy.
    • Nieumiejętność: Jesteś kreatywny, ale nie urokliwy. Wszystkie zadania związane z przyjemną interakcją społeczną są dla Ciebie utrudnione.
Początki Przygód: Z listy poniższych opcji, wybierz jak Twoja postać wzięła udział w pierwszej przygodzie.
1. Robiłeś badania związane z projektem i przekonałeś BG, by Cię z sobą zabrali.
2. Szukałeś nowych rynków zbytu na swój kreatywny output.
3. Wpadłeś w niewłaściwe towarzystwo, ale zaczęli Cię lubić.
4. Kreatywne życie zazwyczaj oznacza problemy finansowe. Dołączyłeś do BG, bo miałeś nadzieję na zarobienie pieniędzy.
Mechaniczny
Masz specjalny talent do maszyn wszelkiego rodzaju i jesteś dobry jeśli chodzi o zrozumienie i, jeśli zajdzie potrzeba, naprawę owych maszyn. Może jesteś nieco wynalazcą, tworzącym nowe maszyny od czasu do czasu. Jesteś nazywany rożnymi nickami, w tym “złotą rączką”. Mechanicy zazwyczaj noszą praktyczne stroje i noszą ze sobą dużo narzędzi.
Zyskujesz poniższe cechy:
    • Bystry: +2 do Puli Intelektu.
    • Umiejętność: Jesteś wyszkolony we wszystkich akcjach związanych z identyfikacją i zrozumieniem maszyn.
    • Umiejętność: Jesteś wyszkolony we wszystkich zadaniach związanych z używaniem, naprawą lub tworzeniem maszyn.
    • Dodatkowy Ekwipunek: Zaczynasz grę z rożnymi narzędziami do naprawy maszyn.
Początki Przygód: Z listy poniższych opcji, wybierz jak Twoja postać wzięła udział w pierwszej przygodzie.
1. Kiedy naprawiałeś pobliską maszynę, podsłuchałeś rozmawiających BG.
2. Potrzebujesz pieniędzy na części i narzędzia.
3. Było oczywiste, że misja nie uda się bez Twoich umiejętności i wiedzy.
4. Inny BG poprosił Cię ,byś do nich dołączył.
Milkliwy 
Nigdy nie byłeś zbyt rozmowny. Kiedy jesteś zmuszony, by wejść w interakcję społeczną, nie masz pomysłu, jakich słów użyć – zawodzą one Ciebie lub wychodzą nie takie, jak trzeba. Zazwyczaj mówisz dokładnie złą rzecz i przez przypadek kogoś znieważasz. Przez większość czasu, po prostu jesteś cichy. Czyni to z Ciebie słuchacza – uważnego obserwatora. To oznacza także, że jesteś lepzy w robieniu rzeczy, niż w mówieniu. Jesteś szybki, by zacząć działać.
Zyskujesz poniższe cechy:
    • Akcja, nie Słowa: +2 do puli Mocy i +2 do Puli Szybkości.
    • Umiejętność: Jesteś wyszkolony w percepcji.
    • Umiejętność: Jesteś wyszkolony w inicjatywie (chyba, że starcie jest społeczne).
    • Nieumiejętność: Wszystkie zadania związane ze społeczną interakcją są dla Ciebie utrudnione.
    • Nieumiejętność: Wsyzstkie zadania powiązane z komunikacją werbalną lub przekazywaniem informacji są dla Ciebie utrudnione.
Początki Przygód: Z listy poniższych opcji, wybierz jak Twoja postać wzięła udział w pierwszej przygodzie.
1. Po prostu się dołączyłeś i nikt nie powiedział Ci, żebyś sobie poszedł.
2. Widziałeś coś ważnego, czego inni BG nie dostrzegli i (z pewnym wysiłkiem) powiedziałeś im o tym.
3. Zainterweniowałeś, by ocalić jednego z innych BG kiedy byli w niebezpieczeństwie.
4. Jeden z innych BG zrekrutował Cię ze względu na Twoje talenty.
Mistyczny
Myślisz o sobie jako o mistycznym, dostrojonym do tego co paranormalne i tajemnicze. Twoje prawdziwe talenty leżą w nadnaturalnym.  Najpewniej masz doświadczenie w wiedzy tajemnej, i możesz wyczuć i dzierżyć nadnaturalne – możliwe, że jako “magię”, “psionikę” lub coś odmiennego, zależy to od Ciebie i tych wokół Ciebie. Mistyczne postaci często noszą biżuterię, taką jak pierścienie i amulety, lub mają tatuaże i inne oznaczenia swoich zainteresowań.
Zyskujesz poniższe cechy:
    • Bystry: +2 do Puli Intelektu.
    • Umiejętność: Jesteś wyszkolony w akcjach powiązanych z identyfikowaniem i rozumieniem nadnaturalnego.
    • Wyczucie Magii: Możesz wyczuć, czy zjawiska nadprzyrodzone są aktywne w sytuacjach, gdzie ich obecność nie jest oczywista. Musisz badać obiekt lub lokację czujnie przez minutę, by odczuć, czy jest tu magia. 
    • Zaklęcie: Możesz dokonywać Sztuczki Magiczne jako zaklęcie gdy masz wolną dłoń i możesz zapłacić koszt w Intelekcie. 
    • Nieumiejętność: Posiadasz pewną aurę, która nieco zbija z tropu innych. Każde zadanie wymagające uroku, perswazji lub oszustwa jest dla Ciebie utrudnione. 
Początki Przygód: Z listy poniższych opcji, wybierz jak Twoja postać wzięła udział w pierwszej przygodzie.
1. Miałeś proroczy sen.
2. Potrzebujesz pieniędzy, by sfinansować swoje studia.
3. Wierzyłeś, że ta misja byłaby świetnym sposobem, by dowiedzieć się więcej o zjawiskach nadprzyrodzonych.
4. Różne znaki i omeny przywiodły tu Ciebie.
Miły
To zawsze było dla Ciebie proste, by dostrzec rzeczy z perspektywy innych ludzi. Ta zdolność uczyniła z Ciebie sympatyczną osobę, która rozumie czego inni pragną lub chcą. Z Twojej perspektywy, po prostu stosujesz stare przysłowie “łatwiej jest złapać muchy na miód niż na ocet” ale inni patrzą na Twoje zachowanie jak na bycie miłym. Oczywiście, bycie miłym wymaga czasu, a Twój jest ograniczony. Nauczyłeś się, żepewna czesć ludzi nie zasługuje na Twoje miłosierdzie – prawdziwi sadyści, narcyzy i tym podbni tylko marnują Twoją energię. Tak więc załatwiasz ich szybko, oszczędzając swoje bycie miłym na tych, którzy tego zasługują i na tym skorzystają. 
Zyskujesz poniższe cechy:
    • Emocjonalnie Intuitywny: +2 do Puli Intelektu.
    • Umiejętność: Wiesz, co to znaczy przejść milę w czyichś butach. Jesteś wyszkolony we wszystkich zadaniach związanych z przyjemną interakcją społeczną i wyczuwaniem emocji innych ludzi.
    • Karma: Czasami, obcy po orstu Ci pomagają. By zyskać pomoc obcego człowieka, musisz zużyć jeden rzut na odzyskiwanie zdrowia – taki, który zajmuje akcję, 10 minut lub 10 godzin (bez regeneracji zdrowia) i MG określa naturę pomocy, którą otrzymujesz. Zazwyczaj, akt dobroci nie jest dostateczny, by kompletnie przezwyciężyć trudną sytuację, ale może w niej pomóc i prowadzić do nowych okazji. Dla przykładu, jeśli jesteś pojmany, strażnik może nieco poluzować Twoje więzy, przynieść Ci wodę lub przekazać wiadomość.
    • Nieumiejętność: Bycie miłym nosi z sobą pewne ryzyko. Wszystkie zadania związane z odczytywaniem kłamstw są utrudnione. 
Początki Przygód: Z listy poniższych opcji, wybierz jak Twoja postać wzięła udział w pierwszej przygodzie.
1. BG potrzebował Twojej pomocy i zgodziłeś się mu towarzyszyć.
2. Dałeś złej osobie dostęp do swoich pieniędzy i teraz musisz je odzyskać.
3. Jesteś gotów zanieść swe miłosierdzie wzdłuż i w dal i pomóc większej ilości ludzi, w którym to celu dołączyłeś do BG.
4. Twoja praca, która wyglądała początkowo na świetną, nie jest taka. Dołączyłeś do BG, by od niej uciec.
Naiwny
Wychowywałeś się pod kloszem. Twoje dzieciństwo było bezpieczne i nie miałeś szansy, by nauczyć się czegoś o świecie – a nawet mniejszą szansę, by go doświadczyć. Mogłeś być szkolony w w jakimś kierunku, trzymać nos w książkach, lub po prostu byłeś w odosobnionym miejscu, tak więc nie zrobiłeś zbyt wiele, nie spotkałeś wielu ludzi i nie widziałeś zbyt wielu interesujących rzeczy. Pewnie to się szybko zmieni, ale jak wchodzisz w większy, obcy świat, robisz to bez pewnego zrozumienia, które posiadają inni ludzie wokół. 
Otrzymujesz poniższe cechy:
    • Świeży: Dodajesz +1 do swoich rzutów na odzyskanie zdrowia.
    • Szlachetny Umysł: Jesteś wyszkolony w Obronie Intelektu i we wszystkich zadaniach powiązanych z odpieraniem się kuszeniu.
    • Umiejętność: Wszystko dla Ciebie jest nowe i ciekawe. Jesteś wyszkolony w percepcji.
    • Nieumiejętność: Każde zadanie polegające na rozpoznaniu fałszu lub odkryciu czyisć sekretnych motywów jest dla Ciebie utrudnione.
Początki Przygód: Z listy poniższych opcji, wybierz jak Twoja postać wzięła udział w pierwszej przygodzie.
1. Ktoś powiedział Ci, że powinieneś się w to wmieszać.
2. Potrzebujesz pieniędzy, i to wyglądało na dobry sposób ich zarobku.
3. Wierzysz, że możesz się dużo nauczyć od jednego z BG.
4. To brzmi jak dobra zabawa.
Niehonorowy
Złodzieje nie mają honoru – i zdrajcy, wbijający noże w plecy, kłamcy i oszuści. Jesteś nimi wszystkimi i albo nie prześladuje Cię to po nocach, albo okłamujesz samego siebie i innych. Niezależnie, jesteś zdolny zrobić wszystko, by było tak, jak tego pragniesz. Honor, etyka i zasady to ledwie słowa. Według Ciebie, nie mają one miejsca w prawdziwym świecie.
Otrzymujesz poniższe cechy:
    • Ukradkowy: +4 do Puli Szybkości.
    • Po Obiedzie: Kiedy MG daje innemu graczowi punkt doświadczenia jako nagrodą za Wtrącenie MG, ten gracz nie może dać Tobie drugiego punktu.
    • Umiejętność: Jesteś wyszkolony w oszustwie.
    • Umiejętność: Jesteś wyszkolony w skradaniu się.
    • Umiejętność: Jesteś wyszkolony w zastraszaniu.
    • Nieumiejętność: Ludzie nie lubią Cię i Ci nie ufają. Przyjemna interakcje społeczne są dla Ciebie utrudnione. 
Początki Przygód: Z listy poniższych opcji, wybierz jak Twoja postać wzięła udział w pierwszej przygodzie.
1. Jesteś zainteresowany tym, co robią BG, więc im nakłamałeś, aby Cię przyjęli do grupy.
2. Przypadkiem podsłuchałeś BG omawiających swoje plany i zdałeś sobie sprawę, że chcesz do nich dołączyć.
3. Jeden z innych BG zaprosił Cię, nie mając pojęcia jaki naprawdę jesteś.
4. Wprosiłeś się przy pomocy zastraszania i pogróżek.
Niezręczny
Bez gracji ruchów i niezręczny, wszysci Ci mówili, że “kiedyś to minie”, ale nigdy tak się nie stało. Często upuszczasz rzeczy, potykasz sięo własne stopy, lub przewracasz rzeczy (i ludzi). Niektórych ludzi to irytuje, lecz dla większości jest to zabawne, a może nawet urocze.
(Pewnie gracze mogą nie chcieć być definiowani przez “negatywny” deskryptor jak Niezręczny, ale po prawdzie, nawet ten rodzaj deskryptora ma tyle zalet, że czyni postać zdolną i utalentowaną. Negatywne deskryptory czynią postać bardziej interesującą i wieloaspektową i często świetnie się nimi gra.)
Zyskujesz poniższe cechy:
    • Maślane palce: -2 do Puli Szybkości.
    • Umięśniony: +2 do Puli Mocy.
    • Uroczy: Masz pewien wrodzony urok. Jesteś wyszkolony we wszystkich przyjemnych interakcjach społecznych kiedy wykazujesz lekkie podejście na swojej niezręczności.
    • Głupie Szczęście: MG może CI ofiarować Wtrącenie MG, bazujące na Twojej niezręczności, bez dawania Ci PD-ka (jakbyś wylosował 1 na k20). Jednakże, jeśli to się wydarzy, istnieje szansa 50%, że Twoja niezręczność zadziała na Twoją korzyść. Zamiast zranić Ciebie (mocno), pomaga to Tobie lub rani Twoich wrogów. Poślizgujesz się, ale w odpowiednim momencie, by uniknąć ataku. Upadasz, ale przewracasz też swoich wrogów, podcinając im nogi. Odwracasz się zbyt szybko, ale w końcu wytrącasz wrogowi broń z rąk. Powinieneś pracować razem z MG by określić szczegóły. Jeśli MG sobie tego życzy, może on zastosować Wtrącenia MG związane z Twojąniezdarnością korzystając z normalnych zasad (przyznając PD-ki).
    • Umiejętność: Masz pewne cechy byka. Jesteś wyszkolony w zadaniach powiązanych z łamaniem i niszczeniem rzeczy.
    • Nieumiejętność: Wszelkie zadania związane z równowagą, gracją i koordynację ręka-oko są dla Ciebie utrudnione. 
Początki Przygód: Z listy poniższych opcji, wybierz jak Twoja postać wzięła udział w pierwszej przygodzie.
1. Byłeś we właściwym miejscu o właściwej porze.
2. Miałeś informację, której inni BG potrzebowali, by ruszyć do przodu.
3. Rodzeństwo zarekomendowało Cię innym BG.
4. Natknąłeś się na BG, gdy Ci dyskutowali o swojej misji, i polubili Cię.
Odporny
Możesz wytrzymać wiele, zarówno fizycznie, jak i przychicznie, i dalej stać na nogach. Trzeba wiele, by Cię powstrzymać. Ani fizyczne, ani mentalne trudy i obrażenie nie mają na Ciebie długotrwałego wpływu. Jesteś wytrzymały. Niezłomny. Niepowstrzymany. 
Otrzymujesz poniższe cechy:
    • Odporny: +2 do Puli Mocy, +2 do Puli Intelektu.
    • Odzyskiwanie zdrowia: Możesz wykonać dodatkowy rzut na odzyskanie zdrowia każdego dnia. Ten rzut to tylko jedna akcja. Tak więc możesz wykonać dwa rzuty, które trwają 1 akcję, 1 1-minutowy, i czwarty, który zajmuje 1 godzinę, w końcu zaś piąty, który zajmuje 10 godzin odpoczynku.
    • Umiejętność: Jesteś wyszkolony w Obronie Mocy.
    • Umiejętność: Jesteś wyszkolony w Obronie Intelektu.
    • Nieumiejętność: Jesteś wytrzymały, ale niekoniecznie silny. Wszelkie akcje związane z przemieszczeniem, naginaniem lub niszczeniem rzeczy są dla Ciebie utrudnione.
    • Nieumiejętność: Masz mnóstwo siły woli, ale niekoniecznie jesteś mądry. Wszelkie akcje powiązane z wiedzą i rozwiązywaniem problemów są dla Ciebie utrudnione.
Początki Przygód: Z listy poniższych opcji, wybierz jak Twoja postać wzięła udział w pierwszej przygodzie.
1. Dostrzegłeś, że BG z pewnością potrzebują kogoś takiego jak Ty.
2. Ktoś poprosił Cię, byś miał na oku jednego z BG w szczególności, i przychyliłeś się do tej prośby.
3. Jesteś znużony i w desperackiej potrzebie jakiegoś wyzwania.
4. Przegrałeś zakład – nie fair, wierzysz – i musiałeś zająć czyjeś miejsce w tej misji.
Odporny Psychicznie
Jesteś osobą o silnej woli i bardzo niezależną. Nikt nie może Ci niczego wmówić lub zmienić Twojego zdania kiedy nie chcesz sam go zmienić. Ta cechan iekoniecznie czyni Cięmądrym, ale na pewno jesteś bastionem siły woli. Możliwe, że sięubierasz i działasz z unikalnym stylem i mocą, nie dbając o to, co myślą inni. 
Otrzymujesz poniższe cechy:
    • Wytrzymały psychicznie: +4 do Puli Intelektu.
    • Umiejętność: Jesteś wyszkolony w opieraniu się efektom mentalnym.
    • Umiejętność: Jesteś wyszkolony w zadaniach wymagających wielkiego skupienia i koncentracji.
    • Nieumiejętność: Silna wola nie oznacza bycia bystrym. Wszelkie zadania powiązane z zagadkami lub problemami, zapamiętywaniem rzeczy lub korzystania z wiedzy są dla Ciebie utrudnione. 
Początki Przygód: Z listy poniższych opcji, wybierz jak Twoja postać wzięła udział w pierwszej przygodzie.
1. Pomimo Twojego osądu, dołączyłeś do BG ponieważ byli w niebezpieczeńtwie.
2. Jeden z BG przekonał Cię, że dołączenie do grupy byłoby w Twoim najlepszym interesie.
3. Boisz się, co się stanie, jeśli reszta BG zawiedzie.
4. W tle jest nagroda, a Ty potrzebujesz pieniędzy.
Okrutny
Nieszczęścia i cierpienie Cię nie obchodzą. Kiedy ktoś inny przechodzi w swoim życiu trudy, ciężko Ci się przejąć, co więcej, jego cierpienie i trudności mogą Cię ucieszyć, jeśli w przeszłości zaszedł Ci on za skórę. Twoje okrucieństwo może wynikać z rozżalenie światem i własnymi problemami. Możesz być twardym pragmatystą, robiąc to, co jak czujesz musisz zrobić ,nawet jeśli inni przez to cierpią. Lub możesz być sadystą, rozkoszującym się cierpieniem, które zadajesz. 
Bycie okrutnym niekoniecznie czyni z Ciebie złoczyńcę. Twoje okrucieństwo może być zarezerwowane tych, którzy staną na Twojej drodze lub innych ludzi, gdzie będzie to miało swoje użycie. Mogłeś stać się okrutny po bardzo smutnym wydarzeniu życiowym. Przemoc i tortury, na ten przykład, mogą odebrać litość dla innych istot żywych.
Podobnie, nie musisz być okrutny w każdej sytuacji. Po prawdzie, możesz być lubiany, przyjazny i nawet pomocny. Ale kiedy jesteś wściekły lub sfrustrowany, Twoja podwójna natura daje o sobie znać i Ci, którzy cię wkurzyli, z pewnością tego pożałują.
Otrzymujesz poniższe cechy:
    • Bystry: +2 do Puli Intelektu.
    • Okrucieństwo: Kiedy korzystasz ze swojej siły, możesz wybrać okaleczanie swojego wroga lub zadawanie mu bolesnych obrażeń, by przedłużyć jego cierpienie. Kiedykolwiek zadajesz obrażenia, możesz wybrać zadanie 2 punktów obrażeń mniej, by ułatwić Twój następny atak przeciwko temu wrogowi.
    • Umiejętność: Jesteś wyszkolony w zadaniach powiązanych z oszukiwaniem, zastraszaniem i perswazją, gdy wchodzisz w interakcję z postaciami cierpiącymi fizycznie lub emocjonalnie. 
    • Nieumiejętność: Masz problemy w wyciąganiu ręki do innych ludzi, rozumieniu ich pobudek lub dzieleniu się swoimi uczuciami. Wszystkie zadania powiązane z zrozumieniem motywów, uczuć i stanu umysłu są dla Ciebie utrudnione.
    • Dodatkowy Ekwipunek: Masz cenną pamiątkę o ostatniej osobie, którą zniszczyłeś. Ta pamiątka jest średnio wielkiej ceny i możesz jąsprzedać lub wymienić na przedmiot o takiej samej lub niższej wartości.
Początki Przygód: Z listy poniższych opcji, wybierz jak Twoja postać wzięła udział w pierwszej przygodzie.
1. Uważasz, że możesz wiele zyskać w dłuższym czasie, jeśli pomożesz BG i możesz wykorzystać ową przewagę przeciwko Twoim wrogom.
2. Poprzez dołączanie do BG, widzisz szansę na zwiększenie swojej własnej potęgi i statusu kosztem innych.
3. Pragniesz uczynić życie innego BG ciężkim, poprzez dołączenie do grupy.
4. Dołączenie do BG daje Ci szansę na ucieczkę od sprawiedliwości za przestępstwo, które popełniłeś. 
Pewny Siebie
Jesteś pewien własnych zdolności, energetyczny, i najpewniej nieco prześmiewczy odnośnie pomysłów, z którymi się nie zgadzasz.  Niektóry nazywają Cię odważnym, ale Ci, którym pokazałeś ich miejsce, zwą Cię aroganckim. Ojtam. To nie w Twojej naturz,e przejmować sięco inni myślą o Tobie – chyba, że to rodzina lub przyjaciele. Nawet ktoś tak pewny siebie jak tywie, że przyjaciele czasami muszą być na pierwszym miejscu.
Otrzymujesz poniższe cechy:
    • Energetyczny: +2 do Puli Szybkości.
    • Umiejętności: Jesteś wyszkolony w inicjatywie. 
    • Odważny:Jesteś wyszkolony we wszystkich akcjach które polegają na przezwyciężaniu strachu i zastraszania. 
Początki Przygód: Z listy poniższych opcji, wybierz jak Twoja postać wzięła udział w pierwszej przygodzie.
1. Dostrzegłeś, ze dzieje się coś dziwnego, i bez większego namysłu, wkroczyłeś do akcji.
2. Pokazałeś się tam i wtedy, ponieważ ktoś stwierdził, że tego nie zrobisz – hej, musiałeś pokazać, że nie miał racji!
3. Ktoś Cię wyzwał, ale zamiast zacząć walkę, zaczęło to Twoją obecną przygodę.
4. Powiedziałeś przyjacielowi, że nic niem oże Cię przerazić, i że nic, co zobaczysz, nie zmieni Twojego zdania. To zaprowadziło Cię do wydarzeń obecnych. 
Pełen Wdzięku
Masz perfekcyjny zmysł równowagi, poruszasz się i mówisz z gracją i pięknem. Jesteś szybki i zwinny. Twoja cieło nadaje się perfekcyjnie do tańca, i używasz tej przewagi podczas walki do unikania ciosów. Możesz nosisz ubrania, które umożliwiają Twoje zwinne ruchy i stanowią o Twoim wyczuciu stylu.
Otrzymujesz poniższe cechy:
    • Zwinny: +2 do Puli Szybkości.
    • Umiejętność: Jesteś wyszkolony we wszystkich zadaniach powiązanych z równowagą i ostrożnymi ruchami.
    • Umiejętność: Jesteś wyszkolony we wszystkich akcjach powiązanych z fizycznymi sztukami performatywnymi.
    • Umiejętność: Jesteś wyszkolony w Obronie Szybkości.
Początki Przygód: Z listy poniższych opcji, wybierz jak Twoja postać wzięła udział w pierwszej przygodzie.
1. Pomimo Twojego lepszego osądu, dołączyłeś do innych BG, bo dostrzegłeś, że są w niebezpieczeństwie.  
2. Inny z BG przekonał Cię, że dołączenie do grupy będzie w Twoim najlepszym interesie.
3. Boisz się, co się stanie, jeśli reszta BG poniesie klęskę.
4. Można wygrać nagrodę, a Ty potrzebujesz pieniędzy.
Piękny
Jesteś atrakcyjny dla innych, ale co być może ważniejsze, jesteś lubiany i charyzmatyczny. Masz to “specjalne coś” co przyciąga do Ciebie innych. Wiesz zazwyczaj, co powiedzieć, by kogoś rozbawić, uspokoić lub zmusić do akcji. Ludzie lubią Cię, pragną Ci pomóc i chcą być Twoimi przyjaciółmi. 
Otrzymujesz poniższe cechy:
    • Charyzmatyczny: +2 do Puli Intelektu.
    • Umiejętność: Jesteś wyszkolony w przyjemnych interakcjach społecznych.
    • Odporny na Wdzięki: Jesteś siadom, jak inni mogą manipulować i czarować ludzi i dostrzegasz, gdy ta taktyka jest używana na Tobie. Z powodu tej świadomości, jesteś wyszkolony w odpieraniu perswazji i uwodzenia.
Początki Przygód: Z listy poniższych opcji, wybierz jak Twoja postać wzięła udział w pierwszej przygodzie.
1. Spotkałeś kompletnego obcego (jednego z BG) i tak go oczarowałeś, że zaprosił Cię do grupy.
2. BG szukali kogoś innego, ale przekonałeś ich, że jesteś perfekcyjnym kandydatem.
3. Czysty przypadek – ponieważ ruszasz z prądem i wszystko zazwyczaj działa, jak powinno.
4. Twoja charyzma pomogła jednego BG wyjść z trudnej sytuacji dawno temu i zawsze teraz prosi Cię on o wzięcie udziału w nowych przygodach.
Pomocny
Pomaganie innym to Twoje powołanie. To dlatego tutaj jesteś. Inni doceniają twoją przyjazną i pomocną naturę, a Ty rozkoszujesz się ich szczęściem. Kochasz pomagać ludziom, czy to poprzez wyjaśnianie im, jak mogą najlepiej poradzić sobie z problemem, czy też demonstrując im to.
Zyskujesz poniższe cechy:
    • Życzliwy: Wszyscy, którzy spędzili z Tobą cały poprzedni dzień, zyskują +1 do swoich rzutów na odzyskanie zdrowia.
    • Altruistyczny: Jeśli stoisz obok istoty, która otrzymuje obrażenia, możesz przejąć 1 punkt tych obrażeń na samego siebie (redukując obrażenia istoty o 1). Jeśli masz Pancerz, nie korzystasz z niego przy korzystaniu z tej zdolności.
    • Umiejętność: Jesteś wyszkolony we wszystkich zadaniach związanych z przyjemną interakcją społeczną, pomaganiu się innym zrelaksować i pozyskiwaniu zaufania.
    • Pomocny: Zawsze, gdy pomagasz innej postaci, ta postać zyskuje bonus, jakbyś był wyszkolony, nawet, jeśli ni jesteś wyszkolony lub wyspecjalizowany w danym zadaniu.
    • Nieumiejętność: Gdy jesteś sam, wszystkie zadania Intelektu i Szybkości są dla Ciebie utrudnione.
Początki Przygód: Z listy poniższych opcji, wybierz jak Twoja postać wzięła udział w pierwszej przygodzie.
1. Pomimo tego, że nie znałeś większości BG przedtem, wprosiłeś się w ich drużynę.
2. Widziałeś, że BG mają kłopot i ruszyłeś im na pomoc.
3. Jesteś pewien, że BG zaliczą klęskę bez Ciebie.
4. Twój wybór to było  jednej strony.
Ryzykujący
Jest częścią Twojej natury kwestionowanie co inni sądzą w temacie tego, co nie powinno być robione. Nie jesteś szalony, oczywiście – nie chciałbyś przeskoczyć przez przepaść o szerokości mili ylko dlatego, że ktoś Cię wyzwał. Istnieją rzeczy niemożliwe i prawie niemożliwe. Lubisz próbować swoich sił w tych drugich, ponieważ daje Ci to satysfakcję i przyjemność, gdy odnosisz sukces. Im większy Twój sukces, tym bardziej szukasz kolejnego ryzykownego wyzwania.
Otrzymujesz poniższe cechy:
    • Zwinny: +4 do Puli Szybkości.
    • Umiejętność: Jesteś dobry w ocenaniu ryzyka, i wyszkolony w zadaniach, które mają pewien element losowości, jak granie w gry hazardowe lub wybieranie pomiędzy dwoma lub trzema podobnymi opcjami.
    • Szczęście: Możesz wybrać automatyczny sukces na jednym teście bez rzucania kością, tak długo, jak zadanie ma trudność nie większą niż 6. Kiedy to jednak robisz, MG dokonuje Wtrącenia, jakbyś wylosował 1. Wtrącenie nie neguje sukcesu, ale może na niego wpłynąć. Możesz to zrobić jeden raz, lecz zdolność odnawia się po każdym 10-godzinnym rzucie na odzyskanie zdrowia.
    • Nieumiejętność: Jesteś zwinny, ale nie ukradkowy. Zadania powiązane z zakradaniem się i byciem cicho są dla Ciebie utrudnione.
Początki Przygód: Z listy poniższych opcji, wybierz jak Twoja postać wzięła udział w pierwszej przygodzie.
1. Wyglądało na to, że była szansa 50/50 na to, że BG odniosą sukces, co dla Ciebie było odpowiednio dużą.
2. Myślisz, że zadanie będzie miało wyjątkowo i satysfakcjonujące wyzwania.
3. Jedno z Twoich ryzyk się nie udało, i potrzebujesz pieniędzy, by je spłacić.
4. Chwaliłeś się, że nigdy nie widziałeś ryzyka, którego byś nie lubił, co doprowadziło Cię do sytuacji bieżącej. 
Sceptyczny
Jesteś bardzo sceptyczny odnośnie twierdzeń, które inni biorą za ozywisa prawdę. Nie jesteś koniecznie “wątpiącym Tomaszem” (sceptykiem, który wątpi we wszystko bez stuprocentowych dowodów), ale często odnosiłeś korzyści, kwestionując twierdzenia, opinie i wiedzę innych ludzi.
Otrzymujesz poniższe cechy:
    • Myśliciel: +2 do Puli Intelektu.
    • Umiejętność: Jesteś wyszkolony w identyfikowaniu.
    • Umiejętność: Jesteś wyszkolony w każdej akcji, która polega na przejrzeniu trików, iluzji bądź sztuczek retorycznych które unikają prawdy lub jawnego kłamstwa. Dla przykładu, jesteś lepszy w stwierdzeniu, który kubek zawiera ukrytą kulkę, wyczuwaniu iluzji, lub zorientowaniu się, że ktoś Cię okłamuje (ale tyko, jeśli się skoncentrujesz i użyjesz tej zdolności).
Początki Przygód: Z listy poniższych opcji, wybierz jak Twoja postać wzięła udział w pierwszej przygodzie.
1. Podsłuchałeś innych BG rozmawiających o czymś, odnośnie czego byłeś sceptyczny, więc podszedłeś do nich i poprosiłeś o dowód.
2. Śledziłeś jednego BG, którego o coś podejrzewałeś, co wpłątało Cię w wydarzenia.
3. Twoja teoria o nieistnieniu nadnaturalnego może być udowodniona jako fałszywa tylko przez Twoje własne zmysły, więc wyruszyłeś w podróż z BG.
4. Potrzebujesz pieniędzy, by sfinansować własne badania.
Silny
Jesteś wyjątkowo silny i potężny fizycznie i korzystasz właściwie z tych cech, czy to przy pomocy przemocy, czy podnoszenia ciężarów. Najpewniej jesteś silnie zbudowany i masz okazałą muskulaturę.
Otrzymujesz poniższe cechy:
    • Bardzo Silny: +4 do Puli Mocy.
    • Umiejętność: Jesteś wyszkolony we wszystkich zadaniach niszczenia nieruchomych obiektów.
    • Umiejętność: Jesteś wyszkolony w skakaniu.
    • Dodatkowy Ekwipunek: Masz dodatkową średnią lub ciężką broń.
Początki Przygód: Z listy poniższych opcji, wybierz jak Twoja postać wzięła udział w pierwszej przygodzie.
1. Pomimo Twojego lepszego osądu, dołączyłeś do reszty BG ponieważ byli w niebezpieczeństwie.
2. Inny z BG przekonał Cię, że dołączenie byłoby w Twoim najlepszym interesie.
3. Boisz się, co się wydarzy, jeśli BG poniosą klęskę.
4. Jest oferowana nagroda, a Ty potrzebujesz pieniędzy.
Skryty
Ukrywasz swoją prawdziwą naturę za maską i nie chcesz, by ktokolwiek poznał, kim naprawdę jesteś. Chronienie siebie, fizycznie i emocjonalnie, jest tym, na czym Ci zależy, i trzymasz wszystkich na bezpieczny dystans. Możesz być podejrzliwy względem każdego, kogo spotykasz, spodziewać się najgorszego od ludzi, tak więc nie jesteś zaskoczony, gdy okazuje się, że masz rację. Lub po prostu możesz być nieco zdystansowany, ostrożny odnośnie pozwolenia ludziom na przejrzenie kim naprawdę jesteś.
Nikt nie może być tak zdystansowany jak Ty i mieć wielu przyjaciół. Najpewniej posiadasz szorstką osobowość i masz pesymistyczne spojrzenie na życie. Możliwe, że rozdrapujesz rany i odkryłeś, że jedyny sposób, w jaki możesz iść do przodu, to zamknąć się w sobie.
Otrzymujesz poniższe cechy:
    • Podejrzliwy: +2 do Puli Intelektu.
    • Umiejętność: Jesteś wyszkolony w Obronie Intelektu.
    • Umiejętność: Jesteś wyszkolony w zadaniach polegających na ustalaniu prawdy, dostrzeganiu przebrań i rozpoznawaniu kłamstw i innych oszustw.
    • Nieumiejętność: Twoja podejrzliwa natura czyni Cię nielubianym. Wszystkie zadania związana z oszustwem lub perswazją są dla Ciebie utrudnione.
Początki Przygód: Z listy poniższych opcji, wybierz jak Twoja postać wzięła udział w pierwszej przygodzie.
1. Jeden z BG zdołał się przebić przez Twoją skrytość i zostać Twoim przyjacielem.
2. Chcesz odkryć, co robią BG, więc im towarzyszysz, chcąc ich złapać w akcie.
3. Masz paru wrogów i dołączyłeś do BG w celach obronnych.
4. BG to jedyni ludzie, którzy mogą z Tobą wytrzymać.
Skazany na Zagładę
Dobrze wiesz jaki los został Ci przepowiedziany – nieuchronny, okropny koniec. To przeznaczenie może być tylko Twoje, lu może też dotyczyć Twoich najbliższych.
Otrzymujesz poniższe cechy:
    • Nerwowy: +2 do Puli Szybkości.
    • Umiejętność: Zawsze rozglądasz sieza niebezpieczeństwem, jesteświęc wyszkolony w percepcji.
    • Umiejętność: Koncentrujesz się na obronie, jesteś więc wyszkolony w Obronie Szybkości.
    • Umiejętność: Jesteś cyniczny i zawsze spodziewasz się najgorszego. Tak więc, jesteś odporny na szoki mentalne. Jesteś wyszkolony w Obronie Intelektu, która polega na unikaniu szaleństwa i osiąganiu spokoju umysłu.
    • Zagłada: Co drugi raz, gdy MG używa Wtrącenia MG na Twojej postaci, nie możesz odmówić i nie uzyskujesz z tego tytułu PD (ale dalej dajesz PD innemu graczowi). Dzieje się tak, gdyż jesteś skazany na zagładę. Kosmos to zimne, nieczułe miejsce, a Twoje wysiłki są daremne.
Początki Przygód: Z listy poniższych opcji, wybierz jak Twoja postać wzięła udział w pierwszej przygodzie.
1. Chciałeś uniknąć tego losu, ale wydarzenia zdawały się spiskować przeciwko Tobie.
2. Czemu nie? To bez znaczenia. I tak jesteś skazany na zagładę, niezależnie od wszystkiego.
3. Jeden z BG ocalił Twoje życie, a teraz spłacasz dług, pomagając mu w różnych zadaniach.
4. Podejrzewasz, że jedyna nadzieja jaką masz, by uniknąć swojemu losowi, znajduje się na obranym przez Ciebie szlaku.
Spokojny
Poświęciłeś całe swój czas na życie w spokoju – książki, filmy, hobby itp. - zamiast poświęcić je na jakieś aktywności. Dobrze znasz sięna wielu problemach naukowych lub intelektualnych, ale nie na czymś fizycznej natury. Nie jesteś słaby czy coś, raczej (choć to dobry deskryptor dla postaci, które są starszymi osobami) ale nie masz doświadczenia w fizycznych aktywnościach.
(Spokojny to świetny deskryptor dla postaci, które nigdy nie chciały ruszyć na przygody, lecz życie ich do tego zmusiło – wątek powszechny we współczesnych grach, zwłaszcza w horrorach.)
Otrzymujesz poniższe cechy:
    • Z nosem w książkach: +2 do puli Intelektu.
    • Umiejętności: Jesteś wyszkolony w 4 niefizycznych umiejętnościach Twojego wyboru.
    • Trivia:Kiedy sobie tego życzysz, możesz podać z pamięci fakty przydatne w danej sytuacji. Zawsze są to fakty, an ie przesądy lub wierzenia, i musi to być coś, o czym mógłbyś przeczytać lub widzieć w przeszłości. Możesz zrobić tak jedne raz, ale zdolność odświeża się po wykonaniu rzutu na odzyskanie zdrowia.
    • Nieumiejętność: Nie jesteś wojownikiem. Wszystkie Twoje fizyczne akcje są utrudnione. 
    • Nieumiejętność: Nie lubisz wychodzić na zewnątrz. Wszystkie zadania związane ze wspinaczką, bieganiem, skakaniem i pływaniem są dla Ciebie utrudnione.
Początki Przygód: Z listy poniższych opcji, wybierz jak Twoja postać wzięła udział w pierwszej przygodzie.
1. Czytałeś o obecnej sytuacji i zadecydowałeś, że chcesz są sprawdzić osobiście.
2. Byłeś w złym (dobrym?) miejscu o złej (dobrej?) porze.
3. Kiedy unikałeś kompletnie odmiennej sytuacji, napotkałeś na obecny problem.
4. Jeden z innych BG wmieszał Cię w tą sytuację.
Szalony
Zagłębiłeś się zbyt głęboko w rzeczy, o których ludzkość nie powinna wiedzieć. Jesteś dobrze wykształcony w ezoterycznych tematach dostępnych niewielu, ale ta wiedza miała przerażającą cenę. Zapewne jesteś w złym stanie fizycznym i czasami masz tiki nerwowe. Czasami mówisz sam do siebie bez zdania sobie z tego sprawy.
Otrzymujesz poniższe cechy:
    • Ten, Który Wie: +4 do Puli Intelektu.
    • Przebłysk Olśnienia: Kiedykolwiek posiadanie takiej wiedzy jest stosowne, GM może dać ci ją, choć nie istnieje jasne wyjaśnienie skąd masz ową wiedzę. Leży to w kwestii MG, ale powinno się to zdarzać przynajmniej raz na sesję.
    • Dziwne Zachowanie: Masz tendencję do dziwnych, irracjonalnych zachowań. Kiedy jesteś w przededniu wielkiego odkrycia lub poddany wielkiemu stresowi (no: podczas zagrożenia fizycznego) MG może wprowadzić Wtrącenie MG bez nagradzania PD, które pokieruje Twoja następną akcją. Możesz dalej wydać 1 PD, by odmówić. Wpływ MG to manifestacja Twojego szaleństwa i zawsze jest czymś, czego normalnie byś nie zrobił,, ale nie jest ono bezpośrednio, oczywiście szkodliwe, chyba, że zajdą specjalne okoliczności (Dla przykładu, jeśli wróg nagle wyskakuje z ciemności, możesz spędzić pierwszą turę mrucząc coś pod nosem albo krzycząc imię swojej pierwszej prawdziwej miłości). 
    • Umiejętność: Jesteś wyszkolony w jednej dziedzinie wiedzy (najpewniej czymś dziwnym lub ezoterycznym).
    • Nieumiejętność: Twó umysł jest bardzo wrażliwy. Zadania związane z odpieraniem ataków mentalnych są dla Ciebie utrudnione.
Początki Przygód: Z listy poniższych opcji, wybierz jak Twoja postać wzięła udział w pierwszej przygodzie.
1. Głosy w Twojej głowie kazały ci to uczynić.
2. Rozpocząłeś całą przygodę i nakłoniłeś innych, by do Ciebie dołączyli.
3. Jeden z innych BG zyskał księgę wiedzy dla Ciebie, a teraz spłacasz dług.
4. Czułeś się kuszony przez niejasną intuicje.
Szczęśliwy
Polegasz na szansach i szczęściu w wielu sytuacjach. Kiedy ludzie mówią, że ktoś był urodzony pod szczęśliwą gwiazdą, mają na myśli Ciebie. Kiedy próbujesz czegoś nowego, niezależnie od tego, jak bardzo nie wiesz, co robisz, częściej niż nie odnosisz sukces. Nawet gdy nadciąga katastrofa, rzadko jest ona tak zła, jak mogłaby być. Bardzo często, drobnostki idą po Twojej myśli, wygrywasz konkursy i znajdujesz się we właściwym miejscu o właściwej porze.
Otrzymujesz poniższe cechy:
    • Pula Szczęścia: Masz dodatkową pule zwaną Szcześciem, która zaczyna się od 3 punktów i ma maksymalna wartość 3 punktów. Kiedy wydajesz punkty z dowolnej innej Puli, możesz wziąć jeden, trochę lub wszystkie punkty ze swojej Puli Szczęścia. Kiedy wykonujesz rzut na odzyskanie zdrowia, Twoja Pula Szczęście regeneruje się o tyle samo punktów. Kiedy Twoja Pula Szczęścia zawiera 0 punktów, nie wlicza się to do obniżenia licznika obrażeń. 
    • Przewaga: Gdy wydajesz 1 PD, by przerzucić k20 na każdym rzucie, który dotyczy tylko Ciebie, dodajesz +3 do wartości ponownie rzuconej kości. 
Początki Przygód: Z listy poniższych opcji, wybierz jak Twoja postać wzięła udział w pierwszej przygodzie.
1. Wiedząc, że szczęściarze dostrzegają i wykorzystują sposobność, wziąłeś udział w pierwszej przygodzie z wyboru. 
2. Dosłownie wpadłeś na kogoś na tej przygodzie poprzez czyste szczęście.
3. Odkryłeś teczkę leżącą na drodze. Była znoszona, ale w  środku znalazłeś dużo dziwnych dokumentów, które tu Cię przywiodły.
4. Twoja szczęście Cię ocaliło, gdy uniknąłeś pędzącego pojazdu poprzez wpadnięcie w dziurę w ziemi. Poniżej poziomu gruntu, odnalazłeś coś, czego nie mogłeś zignorować. 
Szlachetny
Robienie tego, co słuszne, to sposób życia. Żyjesz zgodnie z kodeksem, a ten kodeks to ,coś, co dyktuje Twoje codzienne życie. Kiedy zbaczasz z drogi, bijesz się w pierś za swoją słabość i zaczynasz od nowa. W Twój kodeks najpewniej wlicza się umiarkowanie, szacunek dla innych, czystość i inne charakterystyki, które większość ludzi uzna za cnoty. Odrzucasz także ich przeciwieństwa: lenistwo, rządzę pieniądza, obżarstwo itp.
Otrzymujesz poniższe cechy:
    • Silny: Otrzymujesz +2 do Puli Mocy.
    • Umiejętność: Jesteś wyszkolony w ocenaniu prawdziwych motywów innych ludzi i dostrzeganiu kłamstw.
    • Umiejętność: Twoje podążanie ścisłym kodeksem moralnym uodporniło Twój umysł na strach, wątpliwości i zewnętrzne wpływy. Jesteś wyszkolony w Obronie Intelektu.
Początki Przygód: Z listy poniższych opcji, wybierz jak Twoja postać wzięła udział w pierwszej przygodzie.
1. BG robią coś szlachetnego, i wspierasz ich całym sercem.
2. BG są na drodze do zatracenia i uważasz, że musisz ich sprowadzić na właściwą drogę.
3. Inny z BG zaprosił Ciędo przygody, słysząc o Twej szlachetnej naturze.
4. Postawiłeś cnotę przed zdrowym rozsądkiem i broniłeś czyjegoś honoru w obliczu organizacji lub mocy znacznie większej od Ciebie samego. Dołączyłeś do BG, ponieważ zaoferowali Ci pomoc i przyjaźń wtedy, gdy z obawy przed konsekwencjami, nikt inny tego nie zrobił.
Szpetny
Jesteś ohydny zgodnie z prawie każdym ludzkim standardem. Może miałęś tragiczny wypadek, szkodliwą mutację, lub złe szczęście do genów, ale jesteś niemożliwie wprost szpetny.
Nadrabiasz wygląd innymi cechami. Ponieważ musisz ukrywać swój wygląd, jesteś dobry w skradaniu się i przebieraniu. Ale co być może najważniejsze, bycia ofiarą ostracyzmu, gdy inni się socjalizowali, dało Ci czas na dorośnięcie  - wyrosłeś na silnego lub szybkiego – a może masz bystry umysł? 
Otrzymujesz poniższe cechy:
    • Uniwersalny: Masz dodatkowe 4 punkty do rozdzielenia między swoje Statystyki.
    • Umiejętność: Jesteś wyszkolony w zastraszaniu i innych akcjach bazujących na lęku, jeśli pokażesz swoją prawdziwą twarz.
    • Umiejętność: Jesteś wyszkolony w skradaniu się i przebieraniu się.
    • Nieumiejętność: Wszystkie zadania związane z przyjemną społeczną interakcją śa dla Ciebie utrudnione.
Początki Przygód: Z listy poniższych opcji, wybierz jak Twoja postać wzięła udział w pierwszej przygodzie.
1. Jeden z BG podszedł do Ciebie kiedy byłeś w przebraniu i Cię zrekrutował, wierząc, że jesteś kimś innym.
2. Kiedy się przekradałeś, usłyszałeś jak BG planowali swoją przygodę i zdałeś sobie sprawę, że chcesz do niej dołączyć.
3. Jeden z BG zaprosił Cię, ale zastanawiasz się ,czy nie zrobił tego z litości.
4. Pogróżkami zdobyłeś sobie miejsce w drużynie.
Szybki
Jesteś naprawdę szybki. Z tego powodu, możesz wykonać niektóre zadania szybciej od innych. Nie tylko masz szybkie stopy, ale i dłonie, myślisz także i reagujesz szybko. Nawet mówisz z pośpiechem. 
Otrzymujesz poniższe cechy:
    • Energetyczny: +2 do Puli Szybkości.
    • Umiejętność: Jesteś wyszkolony w bieganiu.
    • Szybki: Możesz się poruszyć na średni dystans i dalej dokonać akcji w tej samej rundzie, lub możesz poruszyć się na daleki dystans bez konieczności wykonania rzutu.
    • Nieumiejętność: Jesteś sprinterem, ale nie biegaczem długodystansowym. Nie masz wielkiej kondycji. Obrona Mocy jest dla Ciebie utrudniona.
Początki Przygód: Z listy poniższych opcji, wybierz jak Twoja postać wzięła udział w pierwszej przygodzie.
1. Ruszyłeś by ocalić jednego z BG, który był w wielkiej potrzebie.
2. Jeden z innych BG zrekrutował Cię ze względu na Twoje wyjątkowe talenty.
3. Jesteś impulsywny, i w tamtym momencie wyglądało to na dobry pomysł.
4. Ta misja wiąże się z Twoim osobistym celem.
Tajemniczy
Mroczna figura czająca się w kącie? To Ty. Nikt naprawdę nie wie skąd przybyłeś lub jakie są Twoje motywy – chowasz to w sekrecie. Twoja maniera konfunduje innych, ale nie czyni to z Ciebie złego przyjaciela lub sprzymierzeńca. Jesteś po prostu dobry w zachowywaniu pewnych wiadomości dla siebie, poruszaniu się tak, by nikt Cię nie dostrzegł i ukrywaniu swojej obecności i tożsamości.
Otrzymujesz poniższe cechy:
    • Umiejętność: Jesteś wyszkolony w skradaniu się.
    • Umiejętność: Jesteś wyszkolony w opieraniu się przesłuchaniom i sztuczkom, by wydobyć z Ciebie prawdę.
    • Konfuzja: Masz talenty i zdolności Bóg wie skąd. Możesz spróbować wykonać jedno zadanie, do którego nie masz przeszkolenia, jakbyś był wyszkolony, zadanie,w którym jesteś wyszkolony, jakbyś był wyspecjalizowany, lub zyskać darmowy poziom Wysiłku dla zadania, w którym jesteś wyspecjalizowany. Ta zdolność restartuje się za każdym razem, gdy wykonasz rzut na odzyskanie zdrowia, ale jej użycia nigdy się nie kumulują.
    • Nieumiejętność: Ludzie nigdy nie wiedzą, co Cięz nimi łączy. Wszelkie zadania, mające na celu sprawienie, by ludzie Ci uwierzyli lub Ci zaufali, są dla Ciebie utrudnione.
Początki Przygód: Z listy poniższych opcji, wybierz jak Twoja postać wzięła udział w pierwszej przygodzie.
1. Po prostu się stawiłeś pewnego dnia.
2. Przekonałeś jednego z BG, że masz cenne umiejętności.
3. Równie tajemnicza osoba powiedziała Ci gdzie być i kiedy (ale nie dlaczego) aby dołączyć do grupy.
4. Coś – przeczucie lub być może sen – powiedziało Ci, aby być w danym miejscu i dołączyć do BG.
Tchórzliwy
Odwaga zawodzi Cię za każdym razem. Nie masz siły woli i mocnego postanowienia, by spojrzeć twarzą w twarz niebezpieczeństwu. Strach pożera Twe serce. Słuchanie swoich lęków pomogło Ci w ucieczce od niebezpieczeństwa i unikaniu zbędnego ryzyka. Inni mogli cierpieć w Twoim miejscu, i możesz być pierwszym, by się do tego przyznać, ale w sekrecie czujesz wielką ulgę, uniknąwszy niewyobrażalnego i strasznego losu.
(Deskryptory takie jak Tchórzliwy, Okrutny lub Niehonorowy mogą nie być stosowne dla każdej grupy. Są to cechy złoczyńców i niektórzy ludzie nie chcą, by ich BG nie byli niczym mniej niż w 100% herosami. Ale inni nie mają nic przeciwko odrobinie molarnej szarości. Inni z kolei patrzą na Okrutnego i Tchórzliwego jak na cechy, które trzeba przezwyciężyć w trakcie gry (najpewniej zyskując po drodze odmienne deskryptory).)
Otrzymujesz poniższe cechy:
    • Ukradkowy: +2 do Puli Szybkości.
    • Umiejętność: Jesteś wyszkolony w skradaniu się.
    • Umiejętność: Jesteś wyszkolony w bieganiu.
    • Umiejętność: Jesteś wyszkolony w każdej akcji polegającej na uciekaniu od niebezpieczeństwa, groźnej sytuacji lub unikaniu kłopotów. 
    • Nieumiejętność: Nie wchodzisz z własnej woli w niebezpieczne sytuacje. Wszelkie rzuty na inicjatywę (by określić kto pierwszy działa w walc) są dla Ciebie utrudnione.
    • Nieumiejętność: Trzęsiesz się jak galareta gdy masz wykonać potencjalnie niebezpieczne zadanie sam. Wszelkie takie akcje (takie jak atakowanie istoty, gdy jesteś sam) są dla Ciebie utrudnione.
    • Dodatkowy Ekwipunek: Masz amulet przynoszący szczęście lub urządzenie, chroniące Cię od niebezpieczeństw. 
Początki Przygód: Z listy poniższych opcji, wybierz jak Twoja postać wzięła udział w pierwszej przygodzie.
1. Wierzysz, że jesteś nawiedzany i zatrudniłeś jednego z BG jako swojego obrońcę.
2. Chcesz uciec od wstydu i zebrałeś zdolne jednostki w nadziei na poprawienie swojej reputacji.
3. Jeden z BG zmusił Cię groźba, byś im towarzyszył.
4. Grupa odpowiedziała na Twoje krzyki o pomoc, kiedy miałeś kłopoty.
Twardy
Jesteś silny i możesz wytrzymać naprawdę wiele. Możesz być wielki i mieć kanciaste szczęki. Twarde postaci często mają widoczne blizny.
Otrzymujesz poniższe cechy:
    • Odporny: +1 do Pancerza
    • Zdrowy: Dodaj 1 do punktów które odzyskujesz gdy rzucasz na odzyskanie zdrowia.
    • Umiejętność: Jesteś wyszkolony w Obronie Mocy.
    • Dodatkowy Ekwipunek: Masz dodatkową lekką broń.
Początki Przygód: Z listy poniższych opcji, wybierz jak Twoja postać wzięła udział w pierwszej przygodzie.
1. Jesteś ochroniarzem jednego z innych BG.
2. Jeden z BG to Twoja rodzeństwo i masz na nie oko.
3. Potrzebujesz pieniędzy, gdyż Twoja rodzina jest zadłużona.
4. Ochroniłeś innego BG, gdy ten był zagrożony. Kiedy rozmawiałeś z nim potem, usłyszałeś o celu grupy. 
Uczony
Uczyłeś się dużo, albo samodzielnie, albo z nauczycielem. Wiesz wiele rzeczy i jesteś ekspertem w kilku tematach, takich jak historia, biologia, geografia, mitologia, nature lub inne dziedziny wiedzy. Uczone postaci zazwyczaj mają przy sobie parę książek i spędzają wolny czas na czytaniu.
Otrzymujesz poniższe cechy:
    • Bystry: +2 do Puli Inteligencji.
    • Umiejętność: Jesteś wyszkolony w trzech dziedzinach wiedzy własnego wyboru.
    • Nieumiejętność: Masz mało uroku osobistego. Wszelkie zadania powiązane z byciem miłym, perswazją i etykietą są dla Ciebie utrudnione.
    • Dodatkowy Ekwipunek: Masz dwie książki traktujące o tematach własnego wyboru.
Początki Przygód: Z listy poniższych opcji, wybierz jak Twoja postać wzięła udział w pierwszej przygodzie.
1. Jeden z BG poprosił Cię o pomoc ze względu na Twoją wiedzę.
2. Potrzebujesz pieniędzy, by sfinansować swoje badania.
3. Wierzyłeś, że to zadanie doprowadzi do ważnych i interesujących odkryć.
4. Kolega po fachu poprosił Cię o wzięcie udziału w misji.
Ukradkowy
Jesteś zwinny, śliski i szybki. Te talenty pomagają Ci się ukrywać, poruszać cicho i wykonywać sztuczki z dłońmi. Najpewniej jesteś mały. Jednakże, zły z Ciebie sprinter – jesteśbardziej zwinni niż szybko biegnący.
Otrzymujesz poniższe cechy:
    • Szybki: +2 do Puli Szybkości.
    • Umiejętność: Jesteś wyszkolony w skradaniu się.
    • Umiejętność: Jesteś wyszkolony w interakcjach polegających na kłamaniu lub oszustwie.
    • Umiejętność: Jesteś wyszkolony we wszystkich specjalnych zdolnościach powiązanych z iluzjami lub oszustwami.
    • Nieumiejętność:  Jesteś ukradkowy, ale nie szybki. Wszystkie zadania polegające na przemieszczaniu się są dla Ciebie utrudnione. 
Początki Przygód: Z listy poniższych opcji, wybierz jak Twoja postać wzięła udział w pierwszej przygodzie.
1. Chciałeś ukraść coś innemu BG. Złapał on Cię i zmusił, byś udał się z nim.
2. Śledziłeś jednego z BG z sobie znanych powodów, co wplątało Cię w akcję.
3. Szef (BN) w sekrecie zapłacił Ci, byś wziął udział w przygodzie.
4. Podsłuchałeś BG mówiących o temacie, który Cię zainteresował, więc podszedłeś do nich.
Urokliwy
Jesteś prawdziwym czarusiem. Niezależnie od tego, czy masz najwyraźniej nadprzyrodzone zdolności, czy po prostu jesteś dobry w słowa, możesz przekonywać innych ludzi do swoich racji. Najpewniej jesteś fizycznie atrakcyjny lub przynajmniej charyzmatyczny, a inni lubią słuchać Twojego głosu. Najpewniej dużo uwagi poświęcasz swojemu wyglądowi. Łatwo Ci nawiązywać nowe przyjaźnie. Ten deskryptor skupia się na Intelekcie jako mierze osobowości postaci – inteligencja nie jest Twoją mocną stroną. Jesteś przyjemny w obyciu, ale niekoniecznie uczony lub o silnej woli. 
Otrzymujesz poniższe cechy:
    • Lubiany: +2 do Puli Intelektu.
    • Umiejętność: Jesteś wyszkolony we wszystkich zadaniahc polegających na pozytywnej lub przyjemnej interakcji społecznej.
    • Umiejętność: Jesteś wyszkolony w używaniu specjalnych zdolności, które wpływają na umysł innych.
    • Informator: Posiadasz znajomego, który pełni ważną funkcję, np.: pomniejszego szlachcica, kapitana straży miejskiej lub głową dużego gangu złodziei. Ty i MG powinniście popracować wspólnie nad stworzeniem tej postaci.
    • Nieumiejętność: Nigdy nie byłeś dobry w szkole. Wszelkie zadania powiązanie z wiedzą, nauką i zrozumieniem są dla Ciebie utrudnione.
    • Nieumiejętność: Twoja siła woli nie jest najlepsza. Obronne akcje w celu unikania mentalnych ataków są dla Ciebie utrudnione.
    • Dodatkowy Ekwipunek: Zdołałeś pozyskać ostatnio pewne porządne zniżki. W efekcie, masz dostatecznie dużo pieniędzy, by zakupić jeden przedmiot średniej ceny. 
Początki Przygód: Z listy poniższych opcji, wybierz jak Twoja postać wzięła udział w pierwszej przygodzie.
1. Przekonałeś jednego z BG, by powiedział Ci, co zamierza zrobić.
2. Zapoczątkowałeś całą tę rzecz i nakłoniłeś innych, by do Ciebie dołączyli.
3. Jeden z innych BG zrobił Ci przysługę i teraz spłacasz zobowiązanie, pomagając mu.
4. Oferowana jest nagroda, a Ty potrzebujesz pieniędzy.
Uważny
Mało rzeczy Ci umyka. Dostrzegasz najmniejsze detale w świecie wokół Ciebie i umiejętnie dedukujesz na bazie informacji, które zdobywasz. Twoje talenty czynią z Ciebie wyśmienitego detektywa, bystrego naukowca, lub utalentowanego zwiadowcę.
Jednakże, masz trudności ze zrozumieniem społecznych interakcji. Nie zwracasz uwagi na to, kogo Twoje dedukcje obrazą, lub jak niekomfortowo inni się czują, gdy ich badasz. Masz w zwyczaju traktować innych jak ludzi mniejszego intelektu w porównaniu z Tobą, co wykorzystujesz, gdy tego potrzebujesz.
Otrzymujesz poniższe cechy:
    • Bystry: +2 do Puli Intelektu.
    • Umiejętność: Masz oko do zczegółów. Jesteś wyszkolony w każdym zadaniu, które polega na znajdywaniu lub dostrzeganiu małych detali.
    • Umiejętność: Wiesz co nieco o wszystkim. Jesteś wyszkolony w każdym zadaniu, które polega na identyfikowaniu obiektów lub przywoływaniu z pamięci detali i szczegółów oraz faktów.
    • Umiejętność: Twoja umiejętność wyciągania wniosków jest zatrważająca. Jesteś wyszkolony w zadaniach polegających na zastraszaniu innych istot.
    • Nieumiejętność: Twoja pewność siebie jest przez innych uważana za arogancję. Wszelkie zadania związana z pozytywną interakcją społeczną są dla Ciebie utrudnione.
    • Dodatkowy Ekwipunek: Masz torbę lekkich narzędzi.
Początki Przygód: Z listy poniższych opcji, wybierz jak Twoja postać wzięła udział w pierwszej przygodzie.
1. Podsłuchałeś, jak inni BG mówili o swojej misji i zgłosiłeś się na ochotnika.
2. Jeden z BG poprosił Cię ,byś im towarzyszył, wierząc, że Twoje talenty okażą się ważne podczas tej misji. 
3. Wierzysz że misja BG jest jakoś powiązana z jednym z Twoich śledztw.
4. Osoba trzecia zrekrutowała Cię, byś podążał za BG i zobaczył, co planują.
Wygnany
Podążasz długą i samotną drogą, zostawiwszy swój dom i życie za sobą. Może popełniłeś okropną zbrodnię, coś tak okropnego, że Twoi ludzie Cię wygnali, pod groźbą śmierci. Mogłeś zostać oskarżony o zbrodnię, której nie popełniłeś, i teraz płacisz cenę błędów innej osoby. Twoje wygnanie może być skutkiem gafy – może zawstydziłeś swoją rodzinę lub przyjaciela, lub samego siebie w obliczu kolegów, autorytetu lub kogoś, kogo szanujesz. Niezależnie od powodu, zostawiłeś stare życie z tyłu i teraz starasz się stworzyć dla siebie nowe.
Otrzymujesz poniższe cechy:
    • Samowystarczalny: +2 do Puli Mocy.
    • Samotnik: Nie uzyskujesz korzyści, gdy uzyskujesz pomoc od innej postaci, która jest wyszkolona lub wyspecjalizowana w danym zadaniu.
    • Umiejętność: Jesteś wyszkolony w skradaniu się.
    • Umiejętność: Jesteś wyszkolony w zbieractwie, polowaniu i znajdowaniu bezpiecznych miejsc do odpoczynku lub ukrycia się.
    • Nieumiejętność: Życie na własną rękę tak długo sprawia, że trudniej Ci ufać innym i wejść w sytuacje społeczne. Wszelkie zadania polegające na społecznej interakcji są dla Ciebie utrudnione.
    • Dodatkowy Ekwipunek: Masz pamiątkę ze swojej przeszłości – stare zdjęcie, medalik z puklem włosów, lub zapalniczkę daną Ci przez kogoś ważnego. Trzymasz ten obiekt blisko siebie i wyciągasz, by powspominać stare, dobre czasy.
Początki Przygód: Z listy poniższych opcji, wybierz jak Twoja postać wzięła udział w pierwszej przygodzie.
1. Inny BG pozyskał Twoje zaufanie pomagajac Ci, gdy tego potrzebowałeś. Towarzyszysz mu, by spłacić dług.
2. Kiedy podróżowałeś samotnie, odkryłeś coś dziwnego. Kiedy trafiłeś do osady, BG byli jedynymi, którzy Ci uwierzyli, podróżując z Tobą, by dać sobie radę z owym problemem.
3. Inny z BG przypomina Ci kogoś z Twojej przeszłości.
4. Masz dosyć swojej izolacji. Dołączenie do BG dale Ci szansę, by na nowo mieć przyjaciół.
Wytrzymały
Twoje ciało jest stworzone, by wytrzymać wiele. Niezależnie od tego, czy upijasz się w barze lub wymieniasz się ciosami z członkiem gangu, idziesz naprzód, a trudy spływają po Tobie jak po kaczce. Ani głód, ani pragnienie, pocięte ciało lub złamane kości – ni c z tego nie może Cię zatrzymać. Po prostu ignorujesz ból i przesz naprzód. 
Pomimo cieszenia się kondycją i zdrowiem, znaki zużycia malują się na Twoim ciele milionem mniejszych blizn, potrójnie złamanym nosem, poszarpanymi uszami i innymi oznakami fizycznymi, które nosisz z dumą.
Otrzymujesz poniższe cechy:
    • Potężny: +4 do Puli Mocy.
    • Szybkie Leczenie: Dzielisz na pół czas potrzebny, by wykonać rzut na odzyskanie zdrowia (co najmniej jedna tura).
    • Prawie Niepowstrzymany: Kiedy znajdujesz się na stopniu zranionego na liczniku obrażeń, funkcjonujesz jakbyś był w pełni zdrowia. Kiedy jesteś krytycznie ranny, funkcjonujesz jakbyś był zraniony. W innych słowach, konsekwencje była zranionym dotyczą Ciebie tylko gdy stajesz się krytycznie ranny i nigdy nie ponosisz konsekwencji bycia krytycznie rannym. Dalej umierasz, jeśli wszystkie Twoje punkty Pul spadną do 0.
    • Umiejętność: Jesteś wyszkolony w Obronie Mocy.
    • Nieumiejętność: Twoje silne, wielkie ciało reaguje wolno. Rzuty na inicjatywęsą dla Ciebie utrudnione.
    • Niezgrabny: Gdy stosujesz Wysiłek na rzucie Szybkości, musisz wydać 1 dodatkowy punkt ze swojej Puli Szybkości.
Początki Przygód: Z listy poniższych opcji, wybierz jak Twoja postać wzięła udział w pierwszej przygodzie.
1. BG zrekrutowali Cię, usłyszawszy o Twojej reputacji.
2. Dołączyłeś do BG, bo potrzebowałeś pieniędzy.
3. BG oferowali wyzwanie równe Twojej mocy.
4. Wierzysz, że BG przetrwają tylko wtedy, jeśli będziesz ich bronił.
Zabawny
Jesteś radosny i przyjazny. Pomagasz się innym zrelaksować za pomocą uśmiechu i dowcipu, możliwe, że z samego siebie, choć także lekko z Twoich towarzyszy.  Czasami ludzie mówią, że niczego nie bierzesz na serio. Nie jest to prawdą, oczywiście, ale nauczyłeś się, że rozmyślanie o złych rzeczach zbyt długo nie prowadzi do niczego dobrego. Zawsze masz nowy dowcip i kolekcjonujesz je tak, jak niektórzy kolekcjonują butelki wina.
Zyskujesz poniższe cechy:
    • Zabawny: +2 do Puli Intelektu.
    • Umiejętność: Jesteś biesiadny i większość ludzi lubi z Tobą przebywać. Jesteś wyszkolony we wszystkich zadaniach związanych z przyjemną interakcją społeczną.
    • Umiejętność: masz przewagę w odkrywaniu puent dowcipów, których nigdy wcześniej nie słyszałeś. Jesteś wyszkolony we wszystkich zadaniach związanych z rozwiązywaniem zagadek i łamigłówek.
Początki Przygód: Z listy poniższych opcji, wybierz jak Twoja postać wzięła udział w pierwszej przygodzie.
1. Rozwiązałeś łamigłówkę zanim zdałeś sobie sprawę, że odpowiedź rozpoczęłaby obecną przygodę.
2. Inni BG uznali, że przyniesiesz z sobą bardzo im potrzebny humor na misji.
3. Uznałeś, że sama zabawa i zero pracy nie jest najlepszym sposobem, by przeżyć życie, więc dołączyłeś do innych BG.
4. Miałeś wybór – pójść z BG, lub zmierzyć sięz czymś, co nie jest zabawne w ogóle. 
Zwinny
Poruszasz się szybko, jesteś dobrym sprinterem i pracujesz umiejętnie swoimi dłońmi. Jesteś świetny w bieganiu, ale nie zawsze wychodzi Ci to z gracją. Najpewniej jesteś chudy i muskularny.
Otrzymujesz poniższe cechy:
    • Szybki: +4 do puli Szybkości.
    • Umiejętność: Jesteś wyszkolony w rzutach na inicjatywę (by określić kto pierwszy działa w turze).
    • Umiejętność: Jesteś wyszkolony w bieganiu.
    • Nieumiejętność: Jesteś szybki, ale niekoniecznie pełen gracji. Wszelkie zadania związane z równowagą są dla Ciebie utrudnione. 
Początki Przygód: Z listy poniższych opcji, wybierz jak Twoja postać wzięła udział w pierwszej przygodzie.
1. Pomimo swojego lepszego osądu, dołączyłeś do BG ponieważ byli w niebezpieczeństwie. 
2. Jeden z BG przekonał Cię, byś dołączył do grupy, gdyż byłoby to w Twoim najlepszym interesie.
3. Boisz się co się może zdarzyć, jeśli BG odniosą klęskę.
4. W tle jest nagroda, a Ty potrzebujesz pieniędzy.
Złośliwy
Próbujesz ukryć swoje wnętrze, krzyk, by się wyzwolić i sprawić, by zapłacili, cierpieli i krwawili. Czasami udaje Ci się to dla Twoich przyjaciół – uśmiechać tak jak oni, śmiać, gdy oni się śmieją, a czasami nawet odczuwać inne emocje. Ale to zawsze czeka w środku, to poczucie szaleńczej radości zmieszane z nienawiścią, czasami wyskakujące, gdy konfrontujesz się z wrogiem. Twoi przyjaciele mogą tolerować przemoc, ale czasami martwisz się, że odkryją, że jesteś także okrutny.
Otrzymujesz poniższe cechy:
    • Umiejętność: Jesteś wyszkolony w śledzeniu istot. Jeśli istota Ciękrzywdziła, to zadanie jest ułatwione.
    • Żądny krwi: Kiedy zaczynasz walkę, widzisz tylko czerwień. Zadajesz 2 dodatkowe punkty obrażeń dowolnym atakiem. 
    • Berserk: Kiedy już zaczniesz walkę, ciężko Ci przestać. Po prawdzie, jest to zadanie Intelektu trudności 2 , nawet, gdy wróg się podda lub gdy wrogowie Ci się skończą. Jeśli nastąpi ta druga sytuacja, atakujesz najbliższego sprzymierzeńca w zasięgu.
    • Dodatkowy Ekwipunek: Posiadasz notatki z listą wszystkich, którzy zrobili Ci krzywdę.
Początki Przygód: Z listy poniższych opcji, wybierz jak Twoja postać wzięła udział w pierwszej przygodzie.
1. Inny BG widział, jak poradziłeś sobie z pijakiem w karczmie, nie wiedząc, że to Ty zacząłeś bójkę.
2. Chciałeś uciec od złej sytuacji, więc ruszyłeś z BG.
3. Chcesz się zmienić, i masz nadzieję ,że przebywanie w obecności reszty BG pomoże Ci zaznać spokoju.
4. Jeden z BG poprosił Cię, byś im towarzyszył, wierząc że Twoja złość może zostać opanowana na potrzeby misji.