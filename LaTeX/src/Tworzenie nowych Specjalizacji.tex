\section{Tworzenie nowych Specjalizacji}\index{Specjalizacje!Tworzenie nowych Specjalizacji}

Ta sekcja zawiera wszystko, czego potrzebujesz, by tworzyć nowe specjalizacje. 

Każda specjalizacja ma swój własny styl, taki jak eksploracja, manipulowanie energią, lub po prostu zadawanie wielkich obrażeń w walce. Te ogólne klasyfikacje nazywamy kategoriami specjalizacji.

Każda kategoria specjalizacji ma swój motyw, razem z selekcją porad opisujących jak wybrać zdolności dla każdego poziomu z rozdziału Zdolności, od poziomu 1 do 6.

Nowo stworzona specjalizacja powinna być nazwana w formie czasownikowej, odpowiednio odmienionej, takiej jak Kontroluje Bestie lub Jest Stworzony z Kamienia. Dla przykładu, specjalizacja skupiona na ogniu jest tworzona na bazie porad odnośnie kategorie specjalizacji manipulacja energią i może być nazwana Nosi Halo Ognia (jedna z przykładowych specjalizacji w tym rozdziale). Alternatywnie, nowo stworzona specjalizacja może otrzymać nazwę Wznieca Ognie Apokalipsy lub Rozpala Ogień Samą Myślą. 

\subsection{Kategorie Specjalizacji}\index{Specjalizacje!Kategorie Specjalizacji}

 1. Korzystanie ze Sprzymierzeńców
   
2. Podstawowa
   
3. Manipulacja Energią
   
4. Eksploracja
   
5. Wpływ
   
6. Nieregularna
   
7. Ruch
   
8. Atak w walce
   
9. Wsparcie
   
10. Przyjmowanie Obrażeń

\subsubsection{Wybieranie zdolności, bazując na ich relatywnej mocy}\index{Specjalizacje!Wybieranie zdolności}

Porady odnośnie wybierania zdolności sugerują, by wybrać zdolności z jednego z trzech zbiorów – niski poziom, średni poziom i wysoki poziom. Te poziomy odpowiadają z "poziomami" danymi każdej zdolności. Te zdolności są dalej posegregowane w kategorie zdolności bazujące na tym, co robią – zdolności, które ulepszają fizyczne ataki są w kategorii umiejętności ataków, zdolności które wspierają sprzymierzeńców są w kategorii wsparcie itp. Patrz na poziomy i kategorie w sekcji Kategorie Zdolności i Relatywnej Mocy w rozdziale Zdolności.

Zdolności niskiego poziomu najlepiej się nadają na opcje na 1 i 2 poziomie. Zdolności średniego poziomu nadają się na opcje na 3 i 4 poziomie. Zdolności wysokiego poziomy pasują do poziomów 5 i 6

Powiedziawszy to – czasami uznasz za stosowne danie zdolności niskiego poziomu na poziomach 3 lub 4, lub może zdolność średniego poziomu na poziomie 1 i 2. Rób tak rzadko, ale bądź świadom takiej możliwości. To może być jedyny sposób na otrzymanie wszystkich zdolności, których chcesz, gdy tworzysz specjalizację. "Wyższe" zdolności zazwyczaj kosztują więcej punktów z Pul. Tak więc, jeśli zdolność średniego poziomu jest dostępna na poziomie 1 lub 2, lub zdolność wysokiego poziomu jest dostępna na poziomie 3 lub 4, wyższy koszt uczyni ją bardziej zbalansowaną.

\subsubsection{Balansowanie zdolności}

Porady odnośnie każdej kategorii mają za zadanie zapewnić, że zbudowane Specjalizacja będzie zbalansowana. Czasami jest stosowne dać zdolność niskiej mocy razem ze zwykłą zdolnością na danym poziomie, w zależności od potrzeb związanych ze Specjalnością. "Zdolność niskiej mocy" nie jest nigdzie zdefiniowana i pozostaje do interpretacji MG, ale ogólnie mówiąc, nie powinna być potężniejsza niż zdolność niskiego poziomu (co znaczy, poziomiu 1 lub 2).

Dla przykładu, ktoś kto ma lodowe moce może stworzyć małe rzeźby ze śniegu w dodatku do emisji promienia zimna. Ktoś, kto korzysta z elektryczności może naładować rozładowany artefakt lub mieć atut dla korzystania z elektrycznych systemów. I tak dalej.

Często, porady odnośnie Specjalizacji notują to jako możliwość. Jednakże, masz duża dowolność w zdecydowaniu, czy Specjalizacja potrzebuje dodatkowej zdolności, nawet jeśli porady odnośnie tego poziomu nie przewidują jej. Jeśli dodajesz zdolność, lub jest tam zdolność wysokiej mocy, której normalnie nie powinno być, może to oznaczać, że wybór dany na następnym poziomie lub na poprzednim nie był całkiem dobry. Zbalansowanie Specjalizacji to poniekąd sztuka. Musisz się oprzeć chęci przeładowania mocy Specjalizacji, ale także nie może ona być zbyt słaba.

\subsubsection{Porady odnośnie zdolności nie są wyryte w kamieniu}

Każda kategoria Specjalizacji zawiera porady, jaką zdolność powinno się wybrać na jakim poziomie. Ale nie patrz an te porady jak na coś, czego nie można przeskoczyć. Nie są one wyryte w kamieniu – po po prostu startowa pozycja. Możesz chcieć zmienić zdolność na danym poziomie na taką, której nie ma w poradach. Tak długo, jak wybrana zdolność znajduje się w odpowiedniej krzywej mocy dla tego poziomu, wszystko jest ok. Porady nie zostały zaprojektowane zm yślą o ograniczaniu graczy.

Dla przykładu, jeśli budujesz Specjalność skupioną na zimnie dla gry w świecie fantasy, możesz zadecydować, że zdolność, która wzywa demona, jest lepszym wyborem na danym poziomie niż zdolność, która zadaje obrażenia obszarowe, co jest poradą dla 5 poziomu manipulacji energią. Dokonanie takiej zmiany jest szczególnie wskazane, jeśli nowa Specjalność nazywa się Sięga do Dziewiątego Kręgu Piekieł.

\subsubsection{Zamienianie zdolności}

Jeśli tworzysz Specjalizację i myślisz, że powinna zawierać zbiór zdolności na pierwszym poziomie, które by ją mechanicznie przeciążyły, masz opcję dodać jedną jako zdolność "zamienianą". Łatwo to zrobić – postać może zamienić jedną ze swoich zdolności z typu na wskazaną zdolność niskiego poziomu z Specjalizacji. Tę zdolność zyskuje się zamiast jednej zwyczajowej wynikającej z typu postaci.

\subsubsection{Koncept i kategoria}

Wybór, by stworzyć Specjalizację, która korzysta z konkretnego konceptu – np.: tworzenia iluzji – nie oznacza, że musisz stworzyć focus określonej kategorii – w tym przypadku manipulacji środowiskiem. Specjalizację można stworzyć na wiele sposobów, korzystając z określonej energii, narzędzia lub pomysłu – każdy poskutkuje Specjalizacją odmiennej formy. Wszystko zależy od Ciebie. W tym przypadku, tworzenie iluzji może być wykorzystane, by zwodzić innych, co w takim wypadku powinno być kategorią wpływu.

W ten sam sposób, jeśli Specjalizacja daje postaci moc, by korzystać z jakiejś siły lub energii, nie oznacza to automatycznie, że powinna ona należeć do kategorii manipulacji energią (ale, oczywiście, może tam należeć, jeśli atakowanie i obrona są celem korzystania z tej energii). Ale może być zbudowana Specjalność, by dawała zdolności korzystające z energii lub siły, które są skupione na wytrzymałości, co sugeruje Specjalność Przyjmowanie Obrażeń (ktoś, kto może wytrzymać wiele ataków w walce); lub, jeśli główną cechą ma być maksymalizacja zadawanych obrażeń, to sugeruje to Specjalność Atak w Walce; lub tworzenie kompana stworzonego z tej energii lub siły, co sugeruje Specjalizację Korzystanie ze Sprzymierzeńców (czyli ktoś, kto korzysta z pomocnych istot, BN-ów lub nawet zduplikowanych wersji samego siebie by zyskać przewagę).

Oto inny przykład: Specjalizacja Kontroluje Grawitację może należeć do kategorii Manipulacja Środowiskiem lub Manipulacja Energią. To zależy od tego, czy ta Specjalność bardziej koncentruje się na miażdżeniu i trzymaniu przedmiotów (Manipulacja Środowiskiem) lub na uderzaniu w rzeczy i chronieniu siebie przy pomocy grawitacji (Manipulacja Energią).
Ta sama mnogość możliwych rozwiązań jest prawdziwa w innych przypadkach. Dla przykładu, jeśli ktoś może wzywać i kształtować ziemię, może on wykorzystać swoją moc, by zamienić się w istotę z kamienia (Przyjmowanie Obrażeń), by zaatakować wrogów (Atak w Walce) lub by tworzyć ściany, barykady i tarcze, by wspierać swoich sprzymierzeńców (Wsparcie).

Jeśli szukasz zdolności i nie możesz znaleźć odpowiedniej w długaśnym katalogu zdolności, możesz chcieć zmodyfikować jedną z nich, by mieć iluzję nowości (i uzyskać to, czego chcesz). Modyfikowanie oznacza skorzystanie z mechaniki zdolności, ale zmienieniu jej szczegółów w pewien sposób. Dla przykładu, może tworzysz nową Specjalność przemieszczania ziemi ale nie możesz znaleźć odpowiedniej ilości zdolności powiązanych z ziemią, by zaspokoić Twoje potrzeby. Łatwo jest zmienić inne zdolności, tak, by korzystały z ziemi zamiast z ognia, zimna lub magnetyzmu. Dla przykładu, Skrzydła Ognia mogą zostać zmienione na Skrzydła Ziemi, Lodowa Zbroja na Zbroję Ziemi itp. Te zmiany nie zmieniają niczego poza typem obrażeń i efektami odrzucenia (dla przykładu, Skrzydła Ziemi mogą generować chmury pyłu przy swoim przelocie).

\subsubsection{Zdolności, które odnoszą się do innych zdolności}

Pewne zdolności w rozdziale im poświęconym odnoszą się do innych zdolności. Jeśli wybierasz zdolność dla swojej Specjalizacji lub typu która odnosi siędo lub modyfikuje zdolnosć niskiego poziomu, umieść ową zdolność w swoim typie lub Specjalności, jako zdolność, którą gracz może wybrać na niższym poziomie.

\subsubsection{Tworzenie kompletnie nowych zdolności}

Możesz pójść dalej niż modyfikowanie i stworzyć jedną lub więcej kompletnie nowych zdolności. Kiedy to czynisz, spróbuj znaleźć coś podobnego do niej i skorzystaj z niej jak z szablonu. W każdym wypadku, zdecydowanie jak dużo powinno kosztować korzystanie z mocy (poprzez wydawanie punktów z Pul) jest jednym z ważniejszych aspektów stworzenia odpowiedniej zdolności.

Możesz zauważyć, że zdolności wyższego poziomu są droższe. Po części dzieje się tak dlatego, że czynią więcej, ale także dlatego, że postaci wyższego poziomu mają wyższe Skupienie niż postaci mniejszego poziomu, co oznacza, że płacą mniej punktów z odpowiednich Pul. Trzeciopoziomowa postać ze Skupieniem 3 w jednej z Pul nie płaci kosztu za zdolności kosztujące 3 lub mniej punktów. To świetne odnośnie zdolności mniejszego poziomu, ale zazwyczaj będziesz chciał, by gracz pomyślał trochę o tym, jak często chce skorzystać ze swoich najpotężniejszych zdolności. To oznacza, że powinny one kosztować przynajmniej 1 punkt więcej niż prawdopodobne Skupienie postaci na danym poziomie. (Bardzo często, postać będzie miała Skupienie w odpowiedniej Puli odpowiadające jej poziomowi.)

Jako ogólna zasada, typowa zdolność powinna kosztować tyle punktów, na jakim poziomie się ona znajduje. 

\subsubsection{Wybierz wtrącenia MG}

Pomysł o tym, jakie rodzaje wydarzeń mogą zaskoczyć, zaalarmować BG lub być dla niego katastrofą, gdy tworzysz nową Specjalizację, i przypisz te wtrącenia MG do niej. Zazwyczaj wydarza się to "w biegu" na sesji. Ale pomyślenie o tym zawczasu, gdy Specjalizacja jest tworzona i masz świeże pomysły w głowie, na pewno da Ci szczególnie piekielne opcje.

\subsection{Kategorie Specjalizacji}\index{Specjalizacje!Kategorie Specjalizacji}

\subsubsection{Korzystanie ze sprzymierzeńców}\index{Specjalizacje!Kategorie Specjalizacji!Korzystanie ze Sprzymierzeńców}

Te Specjalności przede wszystkim zapewniają BN-ów (kompanów). Ci kompani zapewniają pomoc BG na różne sposoby, ale zazwyczaj w formie atutu do akcji postaci.
Istnieje wiele potencjalnych motywów w tej kategorii, od zdolności, które pozwalają postaci na przywoływanie lub tworzenie sprzymierzeńców do takich, które pozwalają im na przyciąganie sprzymierzeńców poprzez sławę, magię, lub autorytet i charyzmę. 

\textbf{Połączenia z innymi BG}: Wybierz 4 odpowiednie połączenia z powyższej listy.

\textbf{Dodatkowy Ekwipunek}: Każdy przedmiot koniczny, by bohater mógł zachować przy sobie sprzymierzeńca. Dla przykładu, ktoś ze Specjalizacją, która wymaga super-nauki do tworzenia robotów-kompanów może mieć narzędzia wymagane, by budować i naprawiać owych sprzymierzeńców. Pewne Specjalności w tej kategorii nie wymagają niczego, by zyskać lub utrzymać korzyści.

\textbf{Sugestia Mniejszego Efektu}: BN-Sprzymierzeniec otrzymuje Ułatwienie w swojej następnej turze.

\textbf{Sugestia Większego Efektu}: BN-Sprzymierzeniec zyskuje natychmiastową dodatkową akcję.

Poniższe to przykłady i nie są kompletną listą wszystkich możliwych Specjalizacji w tej kategorii.

\begin{itemize}
\item  Buduje Roboty
\item  Zadaje się z Martwymi
\item  Kontroluje Bestie
\item  Istnieje w Dwóch Miejscach na Raz
\item  Przewodzi
\item  Włada Rojem
\item Mówi do Duchów
\end{itemize}    

\textbf{Porady odnośnie wyboru zdolności}

\textbf{Poziom 1}: Wybierz zdolność niskiego poziomu, która daje kompan 2 poziomu postaci, lub daje podobną korzyść zapewnioną przez BN-a. Alternatywnie, zapewnij podstawy pod pozyskiwanie takich kompanów na wyższych poziomach przez wybór zdolności, która daje postaci wpływ na innych.

Czasami dodatkowa zdolność niskiej mocy jest wskazana, w zależności od Specjalności. Często, jest to umiejętność, która przyznaje wyszkolenie w odpowiedniej dziedzinie wiedzy lub powiązanej umiejętności. Dla przykładu, wyszkolenie w umiejętności powiązanej z rodzajem kompana, który zostanie pozyskany przez BG, byłoby wskazane. 

\textbf{Poziom 2}: Wybierz zdolność niskiego poziomu która zapewnia wpływ na zbliżone rodzaje BN-ów jak kompan pozyskany na poprzednim poziomie. Jeśli żaden kompan nie był pozyskany na poprzednim poziomie, ta zdolność powinna teraz zapewnić ów benefit. Czasami drugorzędna zdolność może być wskazana w dodatku do mocy zapewnionej powyżej, może zdolność niskiego poziomu, która dodaje 2 lub 3 punkty do Puli.

\textbf{Poziom 3}: Wybierz dwie zdolności średniego poziomu. Daj je obydwie jako opcje dla Specjalności: BG wybiera jedną lub drugą.

Jedna opcja powinna dawać zdolność średniego poziomu, która ulepsza kompana, którego już ma BG (zazwyczaj awans z poziomu 2 do 3) lub daje dodatkowego kompana.

Druga opcja powinna dawać coś korzystnego postaci – może moc ofensywną lub defensywną, lub coś, co poszerzy wpływ jaki BG ma na swoich kompanów (lub potencjalnych kompanów).

\textbf{Poziom 4}: Wybierz zdolność średniego poziomu, która daje postaci moc ofensywną lub defensywną, jeśli BG jeszcze jej nie posiada, najlepiej powiązaną tematycznie ze Specjalizacją. Dla przykładu, jeśli BG zyskuje kompanów ze względu na swoją charyzmę, ta zdolność może mu pozwolić rozkazywać wrogom na krótki czas. Jeśli postać zyskuje kompanów poprzez udowanie ich lub przywoływanie, ta zdolność może im pozwolić wpływań na byty tego samego typu, które nie są jeszcze jej kompanami.

Alternatywnie, ta zdolność może dalej ulepszyć poprzednio uzyskanego kompana z poziomu 3 na poziom 4, lub pozwolić postaci pozyskać dodatkowego kompana.

\textbf{Poziom 5}: Wybierz zdolność, która ulepsza postać poprzez zapewniania obrony, zwiększenie jednej z Pul, lub inną moc natury defensywnej.

Alternatywnie, taa zdolność może umożliwić nowy sposób wpływania na i przywoływania BN-ów, w sposób odpowiedni do tematu przewodniego Specjalizacji. Dla przykładu, ktoś kto trzyma bestie-sprzymierzeńców może zyskać zdolność wezwania hordy mniejszych bestii. Ktoś, kto buduje roboty może zyskać zdolność by zbudować kilku mniejszych robotów-pomocników. I tak dalej.

W końcu, ta zdolność może ulepszyć poprzednio zyskanego kompana do 5 poziomu.

\textbf{Poziom 6}: Wybierz dwie zdolności wysokiego poziomu. Obydwie są opcjami w tej Specjalności – BG wybiera jedną z nich.

Jedna z tych zdolności powinna ulepszyć poprzednio pozyskanego kompana do 5 poziomu, jeśli nie wydarzyło się to na poziomie 5. W takim wypadku, ta zdolność powinna być zapewniona w dodatku do dwóch innych powiązanych zdolności.

Inna opcja wysokiego poziomu może zapewnić zbiór kompanów 3 poziomu dla postaci.

Ostatnia zdolność wysokiego poziomu może zapewnić nowe sposoby wpływania na i przywoływania BN-ów w sposób odpowiedni do motywu przewodniego Specjalności. Dla przykładu, ktoś, kto zyskuje kompanów poprzez wysoką charyzmę i trening może zyskać zdolność pozyskiwania informacji, które inaczej nie byłyby możliwe do poznania.

\subsubsection{Podstawowa}\index{Specjalizacje!Kategorie Specjalizacji!Podstawowa}

Kategoria Specjalizacji zapewniająca głównie wyszkolenie w umiejętnościach, atuty do zadań i zwiększenie Pul oraz Skupień w celu ulepszenia postaci. Każda Specjalność ma także swój motyw tematyczny, jak z innymi kategoriami, co nadaje wszystkim zdolnościom sens zamiast poczucia bycia przypadkowymi.

Dodatkowo, ponieważ korzyści zapewniane przez takie Specjalizacje sązazwyczaj bezpośrednie (z nielicznymi wyjątkami), większość Podstawowych Specjalizacji byłaby w porządku w niefantastycznej kampanii gdzie magia, super-nauka i zdolności psioniczne zazwyczaj nie istnieją. Powiedziawszy to, ponieważ zdolności zapewniane przez podstawowe Specjalizacje są bezpośrednie, nie oznacza to, że nie są potężne gdy wymiesza się je ze zdolnościami zapewnianymi przez typ, deskryptor, cyphery i inne cechy postaci.

\textbf{Połączenia z innymi BG}: Wybierz 4 stosowne połączenia z powyższej listy.

\textbf{Dodatkowy Ekwipunek}: Każdy obiekt konieczny, by wypełniać motyw przewodni Specjalności. Dla przykładu, Specjalizacja Wolałby Czytać może dać postaci parę książek. Specjalizacja Pracuje, by Żyć powinna zapewniać zestaw narzędzi.

\textbf{Sugestie Mniejszego Efektu}: Następna akcja jest Ułatwiona.

\textbf{Sugestie Większego Efektu}: Wykonaj darmowy, niezajmujący akcji test na odzyskanie zdrowia który nie liczy się do dziennego limitu.

Poniższa lista to przykłady i nie jest kompletną listą wszystkich możliwych Specjalności w tej kategorii.

\begin{itemize}
\item Nie Robi Zbyt Dużo
\item Interpretuje Prawo
\item Szybko się Uczy
\item Pracuje, by Żyć
\item Wolałby Czytać
\end{itemize}

\textbf{Porady odnośnie wyboru zdolności}

\textbf{Poziom 1}: Wybierz zdolność która daje wyszkolenie lub atut do umiejętności powiązanej z motywem przewodnim Specjalizacji, lub który daje 5 lub 6 punktów do jednej Puli.

Alternatywnie, wybierz zdolność która daje tylko 2 lub 3 punkty do jednej Puli i zdolność, która zapewnia trening lub atut do jednego zadania.

\textbf{Poziom 2}: Wybierz zdolność, której nie wybrałeś na 1 poziomie.

\textbf{Poziom 3}: Wybierz dwie zdolności średniego poziomu. Obydwie są opcjami tej Specjalności – BG wybierze jedną z nich.

Jedna opcja powinna być nie-fantastyczną zdolnością, która zwiększa umiejętności postaci i jest utrzymana w motywie przewodnim. Dla przykładu, jeśli motywem jest zwracanie uwagi, zdolność dająca informacje może być odpowiednia.

Inną opcją powinno być coś, co albo zwiększa Skupienie postaci w odpowiedniej statystyce, albo zapewnia postaci jakiś rodzaj obrony.

\textbf{Poziom 4}: Wybierz kolejną zdolność, która zapewnia wyszkolenie lub atut w umiejętnościach stosownych do motywu przewodniego, lub któa zapewnia 5 lub 6 punktów do jednej Puli najlepiej pasującej do Specjalności. Lub wybierz dwie zdolności, które zapewniają tylko 2 lub 3 punkty plus kolejną zdolność 4 poziomu, która polepsza jedno zadanie lub umiejętność. 

Alternatywnie, zapewnij umiejętność rozszerzającą repertuar postaci sugerowaną na 5 poziomie.

W końcu, jeśli Specjalność jeszcze nie dała żadnej mocy obronnej, zdolność obronna może być teraz zapewniona.

\textbf{Poziom 5}: Wybierz zdolność nieco rozszerzającą repertuar postaci – może coś jak Umiejętność Ekspercka, która może dać Ci automatyczny sukces na zadaniu, w którym jesteś wyszkolony.

Alternatywnie, jeśli na poziomie 4 był zapewniony niestandardowy benefit, daj tutaj opcje zasugerowane na 4 poziomie.

\textbf{Poziom 6}: Wybierz dwie zdolności wysokiego poziomu. Obydwie są opcjami Specjalności – BG wybiera jedną z nich.

Jedna opcja powinna być zdolnością, która zapewnie dodatkowe 5 lub 6 punktów w Puli powiązanej z Specjalizacją lub które postać może rozdzielić zgodnie z własnym życzeniem. Alternatywnie, trening w obronie lub ataku także mógłby tutaj wystąpić.

Druga opcja na 6 poziomie powinna dać postaci zupełnie nową zdolność powiązanąz motywem przewodnim, ale taką, która nie jest fantastyczna. Dla przykładu, zdolność która pozwala postaci na wzięcie dwóch akcji zamiast jednej byłaby ok. Dawanie dodatkowych wyszkoleń, atutów lub Skupienia także jest możliwe.

\subsubsection{Manipulacja Energią}\index{Specjalizacje!Kategorie Specjalizacji!Manipulacja Energią}

Manipulacja energią oferuje zdolności które kontrolują ogień, elektryczność, siły, magnetyzm i niestandardowe formy energii takie jak zimno, kamień lub coś dziwniejszego, jak "pustka" lub "cień". Te zdolności zazwyczaj dają postaci sposoby na osiągnięcie równowagi między atakowaniem przeciwników a uzyskiwaniem dodatkowej ochrony dla siebie i sprzymierzeńców. Specjalizacja zazwyczaj zapewnia też zdolności, które umożliwiają inne użytkowanie konkretnej energii, takie jak transport, tworzenie wielkich zbiorowisk energii, które atakująwiele celów, lub tworzenie chwilowych obiektów lub barier z energii. 

\textbf{Połączenia z innymi BG}: Wybierz 4 odpowiednie połączenia z listy powyżej.

\textbf{Dodatkowy Ekwipunek}: Jeden lub więcej przedmiotów w ekwipunku, które są odporne na manipulowanie daną energią – może to być zestaw ubrań. Alternatywnie, coś powiązanego z energią, którą się generuje. Pewne Specjalizacje w tej kategorii nie oferują dodatkowego ekwipunku.

\textbf{Zdolności Energetyczne}: Jeśli typ postaci zapewnia specjalne zdolności, które normalnie korzystają z innego rodzaju energii, teraz działają one w oparciu o energię tej Specjalności. Dla przykładu, jeśli postać ma Specjalizację w manipulowaniu elektrycznością, jej Wystrzał Mocy teraz działa na bazie elektryczności. Te zmiany nie dotyczą niczego poza typem obrażeń i efektami specjalnymi (dla przykładu, elektryczność może chwilowo zaburzyć systemy elektroniczne).

\textbf{Sugestie Mniejszego Efektu}: Cel lub coś w jego pobliżu jest Utrudnione z powodu pozostałości energetycznych.

\textbf{Sugestie Większego Efektu}: Ważny przedmiot we władaniu celu jest zniszczony.

Poniższa lista to przykłady i nie jest kompletną listą wszystkich możliwych Specjalności w tej kategorii.

\begin{itemize}
\item Absorbuje Energię
\item Nosi Halo Ognia
\item Tańczy z Czarną Materią
\item Ujeżdża Błyskawicę
\item Grzmi
\item Przywdziewa Połyskliwy Lód
\end{itemize}    
    
\textbf{Porady odnośnie wyboru zdolności}

\textbf{Poziom 1}: Wybierz zdolność niskiego poziomu, która albo zadaje obrażenia, albo zapewnia ochronę korzystając z odpowiedniego rodzaju energii.
Czasami, stosowna jest zdolność niskiej mocy, w zależności od typu energii. Dla przykładu, specjalizacja, która manipuluje zimnem może dać zdolność tworzenia lodowych rzeźb. Specjalność, która manipuluje elektrycznością może dać zdolność ładowania zużytych artefaktów lub atut przy korzystaniu z systemów elektrycznych. Specjalizacja, która pochłania energię może dać zdolność jej wyzwalania jako formę bazowego ataku. I tak dalej.

\textbf{Poziom 2}: Wybierz zdolność, której nie wybrałeś na 1-szym poziomie.

\textbf{Poziom 3}: Wybierz dwie zdolności średniego poziomu. Obydwie są opcjami tej specjalizacji  BG wybiera jedną z nich.

Pierwsza opcja powinna być zdolnością, która zadaje obrażenia, korzystając z odpowiedniej energii (może tu być też dodatkowy efekt).

Druga opcja powinna ulepszać zdolność ruchu korzystając z odpowiedniej energii, dawać dodatkową obronę, lub korzystać z energii w kompletnie nowy sposób, np.: wysysając energię z maszyn (elektryczność), wiążąc ofiarę w warstwach lodu (zimno), tworzyć absolutną ciszę (dźwięk), tworzyć oślepiający pokaz świateł (światło) itp.

\textbf{Poziom 4}: Wybierz zdolność, której nie wybrano na 3 poziomie.

\textbf{Poziom 5}: Wybierz umiejętność wysokiego poziomu, która zadaje obrażenia (i może ma dodatkowy efekt) więcej niż jednemu celowi korzystając z odpowiedniej energii, lub zdolność, która korzysta z energii w sposób, który nie został wykorzystany na poziomach 3 i 6.

\textbf{Poziom 6}: Wybierz dwie zdolności wysokiego poziomu. Obydwie są opcjami tej specjalności – BG wybierze jedną z nich.

Pierwsza opcja powinna korzystać z energii, by zadać dużo obrażeń jednemu celowi lub kilku.

Druga opcja powinna korzystać z energii, by osiągnąć jakiś cel, który nie był możliwy na niższych poziomach. Przykłady to przywołanie kompana stworzonego z ognia, teleportacja na długi zasieg korzystając z błyskawicy, tworzenie przedmiotów z utwardzonej energii itp.

\subsubsection{Manipulacja Środowiskiem}\index{Specjalizacje!Kategorie Specjalizacji!Manipulacja Środowiskiem}

Specjalności, które pozwalają postaci na przemieszczanie przedmiotów, wpływanie na grawitację, tworzenie obiektów (lub iluzji obiektów) itp. należą do kategorii manipulacji środowiskiem. Zważywszy na to, że w wielu przypadków energia jest używana jako część procesu, ta kategoria i manipulacja energią do pewnego stopnia na siebie nachodzą. Manipulacja środowiskiem priorytyzuje zdolności, które niebezpośrednio wpływają na wrogów i sprzymierzeńców poprzez obiekty, moce i zmiany w otoczeniu; specjalizacje manipulacji energią są skupione na bezpośrednim uszkadzaniu obiektów wybraną energią lub siłą.

Dla przykładu, zamiast atakować wroga pulsem grawitacyjnym, który zadaje obrażenia, postać korzystająca z manipulacji środowiskiem bazującą na grawitacji będzie raczej korzystała z mocy trzymających cel w miejscu, korzystała z grawitacji, by rzucać ciężkie obiekty jako atak, lub obniżała grawitację na danym obszarze lub nawet dla konkretnego obiektu.

\textbf{Połączenia z innymi BG}: Wybierz 4 odpowiednie połączenia z powyższej listy.

\textbf{Dodatkowy Ekwipunek}: Każdy obiekt konieczny, by manipulować środowiskiem. Dla przykładu, ktoś z specjalizacją dającą możliwość tworzenia obiektów potrzebuje podstawowych narzędzi. Pewne specjalizacje w tej kategorii nie wymagają niczego do zyskania jej korzyści.

\textbf{Zdolności manipulacji środowiskiem}: Motywy specjalizacji wliczają obrazy lub widoczne energie, które wpływają na wygląd Twoich zdolności. Takie zmiany, jeśli jakiekolwiek, tylko zmieniają wygląd efektu. Jeśli manipuluje się grawitacją, może błękitna poświata towarzyszy korzystaniu ze zdolności, wliczając w to zdolności typu. Jeśli tworzy się iluzje, może okazałe efekty wizualne towarzyszą zdolnościom typu, np.: mackowata bestia trzymająca cele, jeśli isę korzysta z umiejętności Zastój. I tak dalej.

\textbf{Sugestie Mniejszego Efektu}: Cel się wywraca i jego następny atak jest utrudniony.

\textbf{Sugestia Większego Efektu}: Postać się regeneruje i odzyskuje 4 punkty w jednej Puli.

Poniższe to przykłady i nie jest to kompletna lista wszystkich możliwych specjalizacji w tej kategorii:

\begin{itemize}
\item Przebudza Sny
\item Jaśnieje Światłością
\item Oblicza Nieobliczalne
\item Kontroluje Grawitację
\item Tworzy Iluzje
\item Tworzy Unikalne Obiekty
\item Włada Magnetyzmem
\item Stawia Umysł Ponad Materią
\end{itemize}    
    
\textbf{Porady odnośnie wyboru zdolności}

\textbf{Poziom 1}: Wybierz zdolność niskiego poziomu, która daje podstawową zdolność, która zmienia środowisko (lub jest w stanie je przewidzieć) korzystając z motywu specjalności. Dla przykładu, specjalizacja wpływająca na grawitację może dać zdolność czyniącą cel lżejszym lub cięższym. Specjalizacja tworząca iluzje może dać zdolność tworzenia obrazu. Specjalność tworząca przedmioty może dać podstawową  zdolność w tworzeniu określonych przedmiotów. Specjalność przewidująca przyszłość może wyliczyć wynik i zapewnić postaci korzyść z owej wiedzy. I tak dalej.

Czasami, dodatkowa zdolność niskiego poziomu jest stosowna, w zależności od specjalizacji. Często, jest to zdolność, która zapewnia wyszkolenie w określonej dziedzinie wiedzy.

\textbf{Poziom 2}: Wybierz zdolność niskiego poziomu, która zapewnia nową zdolność obronną ;lu ataku, zgodną z motywem przewodnim specjalności. 

Alternatywnie, wybierz zdolność, która zapewnia dodatkową lub zupełnien ową zdolność manipulowania środowiskiem, co jest powiązane z motywem przewodnim specjalizacji.

\textbf{Poziom 3}: Wybierz dwie zdolności średniego poziomy. Obydwie są opcjami specjalizacji – BG wybierze jedną lub drugą.

Pierwsza opcja powinna być zdolnością średniego poziomu powiązaną ze specjalnością, która zapewnia dodatkową zdolność wpływania na środowisko lub ulepsza podstawową zdolność wpływania na środowisko, którą BG już pozyskał. Ta zdolność nie jest bezpośrednio ofensywna lub defensywna, ale zapewnia albo nową zdolność powiązaną z podstawową zdolnością, lub polepsza jej siłę, zasięg lub w inny sposób ją rozszerza. 

Druga opcja powinna byś średniego poziomu i zapewniać zdolność obronną lub ataku, powiązaną ze specyficzną formą ruchu która zapewnia specjalność, jeśli tylko to możliwe.

\textbf{Poziom 4}: Wybierz zdolność średniego poziomu, która albo zapewnia ochronne lub atakujące wykorzystanie zdolności, w zależności od tego, która opcja nie została wybrana na poprzednim poziomie. 

\textbf{Poziom 5}: Wybierz zdolność wysokiego poziomu, która jest prawie ostatecznym wpływem na środowisko. Dla przykładu, jeśli manipulacja jest iluzją, ta zdolność może prześladować cel przerażającymi obrazami. Jeśli specjalność polega na grawitacji, może odblokować lot. Jeśli jest magnetyczna, może pozwolić użytkownikowi na przekształcenia metali. Jeśli specjalność zapewnia moce telekinetyczne, ta zdolność może pozwolić postaci rzucac ciężkie przedmioty na swoich wrogów. I tak dalej.

\textbf{Poziom 6}: Wybierz dwie zdolności wysokiego poziomu. Są one opcjami tej specjalności – BG wybierze jedną z nich.

Pierwsza z tych zdolności powinna zapewnić zdolność obronną lub ataku, (odmienna kategoria niż zdolność z 4 poziomu, ale oczywiście, wysokiego, a nie średniego poziomu).

Druga opcja pwinna być czymś, co dalej rozszerza podstawową zdolność manipulacji środowiskiem. Na 5 poziomie wybierano zdolność prawie ostateczną, tutaj można wybrać zdolność ostateczną, powiązaną z oferowanym rodzajem manipulacji, lub inny sposób wykorzystania zdolności, który korzysta nieznanego wcześniej aspektu tej specjalności.

\subsubsection{Eksploracja}\index{Specjalizacje!Kategorie Specjalizacji!Eksploracja}

Specjalności, które pozwalają postaci na pozyskiwanie informacji, przetrwanie w nieznanych środowiskach i znalezienie drogi do nowych lokacji lub wyśledzenie konkretnych istot są specjalnościami eksploracji. Przetrwanie w nieznanym środowisku wymaga odpowiedniej selekcji zdolności defensywnych; jednakże zdolności, które pozwalają na znajdowanie rzeczy i naukę są na pierwszym miejscu.

Specjalności eksploracji polegają na różnorodnych metodach, lecz wyszkolenie i doświadczenie są najważniejsze. Niektóre metody wymagają specyficznych narzędzi (takich jak pojazdy) by skorzystać z zapewnianych korzyści, a inne polegają na nadnaturalnych lub naukowych zdolnościach, by uczyć się nowych rzeczy i badać dziwne miejsca z daleka.

\textbf{Połączenia z innymi BG}: Wybierz 4 odpowiednie połączenia z powyższej listy.

\textbf{Dodatkowy Ekwipunek}: Przedmiot potrzebny do eksploracji. Dla przykładu, mapy i/lub kompass, a ktoś, kto korzysta z mocy psionicznych mógłby otrzymać lustro lub kryształową kulę. W ekwipunek może się także wliczać dostęp do pojazdu wymaganego do eksploracji, jak wcześniej zanotowano. 

\textbf{Sugestia Mniejszego Efektu}: Masz atut na każdej akcji, która korzysta z Twoich zmysłów, takiej jak percepcja lub atak, do końca następnej tury.

\textbf{Sugestia Większego Efektu}: Twoje Skupienie w Intelekcie zwiększa się o 1 do końca następnej tury.

Poniższe to przykłady i nie jest to kompletna lista wszystkich możliwych specjalizacji w tej kategorii:

\begin{itemize}
\item Bada Ciemne Miejsca
\item Infiltruje
\item Działa pod Przykrywką
\item Pilotuje Statki Kosmiczne
\item Jest Jasnowidzem
\item Izoluje Umysł od Ciała
\end{itemize}

\textbf{Porady odnośnie wyboru zdolności}

\textbf{Poziom 1}: Wybierz zdolność niskiego poziomu, która zapewnia podstawową zdolność eksploracji, przetrwania lub zbierania informacji powiązaną z motywem specjalizacji.
Czasami, dodatkowa zdolność niskiej mocy jest stosowna, w zależności od specjalności. Często jest to zdolność, która zapewnia wyszkolenie w powiązanej dziedzinie wiedzy lub powiązanej umiejętności (choć za to może już odpowiadać główna zdolność). Alternatywnie, może to być prosty bonus 2 lub 3 punktów do Puli Mocy.

\textbf{Poziom 2}: Wybierz kolejną moc niskiego poziomu, która zapewnia dodatkowe możliwości powiązane  eksploracją, przetrwaniem lub zbieraniem informacji. 

Dla przykładu, specjalność poświęcona przetrwaniu w ciężkich warunkach może zaoferować zdolność (lub dwie) które czynią łatwiejszym unikanie naturalnych niebezpieczeństw, trucizn, trudnego terenu itp. Specjalność poświęcona eksploracji konkretnego terenu może dawać zdolności dostępu do niego, lub możliwość, której inni nie mają (np.: zdolność widzenia w ciemnościach).

\textbf{Poziom 3}: Wybierz dwie zdolności średniego poziomu. Obydwie są opcjami tej specjalności – BG wybierze jedną lub drugą.

Jedna opcja powinna polepszać bazową zdolność eksploracji lub dawać nową zdolność eksploracji, przetrwania lub zbierania informacji.

Druga opcja powinna być jakąś korzyścią dla postaci, atakiem lub obroną (zwłaszcza, jeśli specjalność jeszcze tego nie zapewniła) lub czymś, co poszerza zdolność postaci do eksploracji, zgodnie z motywem specjalności. 

\textbf{Poziom 4}: Wybierz zdolność średniego poziomu (atak lub obronę, w zależności od tego, czego nie zapewniono na 3 poziomie), która jest korzyścią dla postaci. Alternatywnie, jeśli zdolności ofensywne lub defensywne postaci są już dobrze rozwinięte, wybierz inną zdolność średniego poziomu, która poszerza zdolność postaci do eksploracji, przetrwania lub zbierania informacji. 

\textbf{Poziom 5}: Wybierz zdolność wysokiego poziomu, która wpływa na pewne kary związane z eksploracją, przetrwaniem lub zbieraniem informacji w normlanie niedostępnym miejscu.

\textbf{Poziom 6}: Wybierz dwie zdolności wysokiego poziomu. Obydwie są opcjami tej specjalności – BG wybierze jedną z nich.

Pierwsza opcja powinna znowu polepszać bazową zdolność eksploracji, którą już otrzymano, lub dać kompletnie nową zdolność eksploracji, przetrwania lub zbierania informacji.

Druga opcja powinna być czymś, z czego postać będzie czerpała korzyść, zdolnością ataku lub obrony, lub kolejną zdolnością, która rozszerza możliwość eksploracji.

\subsubsection{Wpływ}\index{Specjalizacje!Kategorie Specjalizacji!Wpływ}

Specjalność która stawia na pierwszym miejscu autorytet i wpływ – na ludzi lub maszyny zgodnie z wolą posiadacza, by pomagać innym, lub by posiąść jakąś inną prestiżową i ważną pozycję. Te specjalności zapewniają wpływ poprzez trening i perswazję, bezpośrednią manipulację mentalną, użycie sławy, by skupić na sobie uwagę i wpłynąć na akcje innych, lub po prostu wiedząc i zdobywając wiedzę, która wpływa na późniejsze decyzje. W ten sposób, koncept wpływu jest bardzo szeroki.

\textbf{Połączenia z innymi BG}: Wybierz 4 odpowiednie połączenia z powyższej listy.

\textbf{Dodatkowy Ekwipunek}: Każdy obiekt konieczny, by wpływać na innych. Niektóre specjalności wpływu nie wymagają niczego by korzystać ze swoich benefitów.

\textbf{Sugestie Mniejszego Efektu}: Zasięg lub czas trwania zdolności wywierającej wpływ się podwaja.

\textbf{Sugestie Większego Efektu}: Sprzymierzeniec lub wskazany cel może wykonać dodatkową akcję.

Poniższe to przykłady i nie jest to kompletna lista wszystkich możliwych specjalizacji w tej kategorii:

\begin{itemize}
\item Włada Mocami Mentalnymi
\item Tworzy Dziwną Naukę
\item Łączy Umysł i Maszynę
\item Jest Idolem Milionów
\item Rozwiązuje Tajemnice
\item Mówi z Maszynami
\item Łamie Systemy
\end{itemize}

\textbf{Porady odnośnie wyboru zdolności}

\textbf{Poziom 1}: Wybierz zdolność niskiego poziomu, która pozwala postaci na zdobycie wanej informacji (na tyle, by dokonać mądrego wyboru – alternatywnie, użyć tej wiedzy do perswazji lub zastraszania). Jak postać zdobywa tę informację zależy od specjalności. Jedna postać może wykonać eksperymenty by poznać odpowiedzi, inna może utworzyć link telepatyczny z innymi by sekretnie wymieniać dane, a inna może po prostu być wyszkolona w interakcjach społecznych. 

Czasami, dodatkowa zdolność niskiego poziomu jest stosowna, w zależności od specjalności. Często ta zdolność zapewnia trening w powiązanej dziedzinie wiedzy.

\textbf{Poziom 2}: Wybierz zdolność niskiego poziomu, która polepsza zdolność postaci do wywierania presji. Może to otworzyć nowe sposoby na skorzystanie ze specjalizacji lub po prostu ulepszyć podstawową zdolność, którą BG już posiada. Dla przykładu, ta zdolność 2 poziomu może ułatwić zadania powiązane z wpływem o parę stopni, pozwolić telepacie na czytanie umysłów innych (ujawniając sekrety, które inaczej pozostały tajne) lub zapewnić wpływ na fizycznie obiekty (by je ulepszyć lub dowiedzieć się czegoś o nich). I tak dalej.

\textbf{Poziom 3}: Wybierz dwie zdolności średniego poziomu. Obydwie są opcjami tej specjalizacji – BG wybierze jedną lub drugą.

Pierwsza opcja powinna zapewniać zdolność ataku lub obrony powiązaną z wpływem specjalności, jeśli to możliwe. Dla przykładu, wynalazca może stworzyć serum, które daje im zwiększone możliwości (których można użyć do ataku lub obrony), telepata może mieć zdolność atakowania wrogów mentalną energią, a ktoś z podstawowymi zdolnościami debaty i wpływu poprzez sławę może polegać na treningu w używaniu broni lub na swojej świcie.

Druga opcja średniego poziomu powinna zapewniać dodatkową zdolność wpływania na świat, lub rozszerzyć ;podstawowy wpływ zapewniany przez podstawową zdolność, którą postać już posiada. Ta opcja nie jest bezpośrednio ofensywna lub defensywna, ale zapewnia albo zupełnie nową zdolność powiązaną z podstawową zdolnością, albo zwiększa siłę, zasięg, lub inną właściwość poprzednio przyznanej podstawowej zdolności. Dla przykładu, telepata może mieć zdolność psionicznej sugestii.

\textbf{Poziom 4}: Wybierz zdolność średniego poziomu, która jest albo ofensywnym, albo defensywnym użyciem wpływu, w zależności od tego, której opcji nie wybrałeś na poprzednim poziomie.

Alternatywnie, ta zdolność może zapewnić dodatkowe możliwości powiązane z wpływem, który wywiera specjalizacja. 

\textbf{Poziom 5}: Wybierz prawie-ostateczną zdolność wysokiego poziomu powiązaną ze zdolnością zapewnianą na niższym poziomie. Alternatywnie, wybierz zdolność, której nie otrzymałeś na niższym poziomie, która pozwala Ci wywierać wpływ na nowy sposób. Dla przykładu, jeśli zdolnością specjalności jest telepatia, zdolność poziomu 5-tego może pozwolić postaci na spojrzenie w przyszłość by zyskać atuty podczas działań przeciwko przeciwnikom (i wspierania sprzymierzeńców).

\textbf{Poziom 6}: Wybierz dwie zdolności 6 poziomu. Obywie są opcjami tej specjalności – BG wybierze jedną lub drugą.

Jedna z tych opcji powinna zapewnić ofensywną lub defensywną zdolność, przeciweństwo tej, którą zapewniono na 4 poziomie (choć wysokiego poziomu zamiast średniego). 

Druga opcja powinna być czymś, co dalej posuwa motyw podstawowego wpływu zapewnianego przez specjalizację. Jeśli zdolność poziomu 5-tego była zdolnością prawie-ostateczną, tutaj może pojawić się jeszcze lepsze skorzystanie z wywieranego wpływu, lub odmienny sposób korzystania z tej zdolności jako wcześniej niezbadane wykorzystanie tej zdolności. 

\subsubsection{Nieregularna}\index{Specjalizacje!Kategorie Specjalizacji!Nieregularna}

Większość specjalizacji posiada bazowy motyw, pewną "historię postaci", z której logicznie wynikają powiązane zdolności. Jednakże, pewne specjalizacje mają tak szerokie tematy przewodnie, że nie wliczają się do żadnej innej kategorii z wyjątkiem nieregularnej. Nieregularne specjalizacje zapewniają zbiór różnych zdolności. Zazwyczaj dzieje się tak, gdyż nadrzędny motyw jest takim, który wymaga różnorodności i dostępu do paru różnych rodzajów zdolności. Często, te specjalizacje można znaleźć w konwencjach, które sugerują dodatkowe zmiany zasad, takie jak punkty mocy w konwencji superbohaterskiej lub rzucanie zaklęć w fantasy. Jednakże, inne specjalności nieregularne są możliwe.

\textbf{Połączenia z innymi BG}: Wybierz 4 pasujące połączenia z powyższej listy.

\textbf{Dodatkowy Ekwipunek}: Każdy przedmiot konieczny z uwzględnieniem motywu specjalizacji. Przykładowo, motyw superbohaterski może dać strój superbohatera.

\textbf{Sugestia Mniejszego Efektu}: Cel jest także oszołomiony na jedną rundę, podczas której wszystkie jego zadania są utrudnione.

\textbf{Sugestia Większego Efektu}: Cel jest wstrząśnięty i traci następną turę. 

Poniższe to przykładowa lista i nie wyczerpuje ona listy specjalności w tej kategorii:

\begin{itemize}
\item Otrzymuje Boskie Błogosławieństwa
\item Ma Szlachetną Krew
\item Wyszedł z Obelisku
\item Fruwa Szybciej niż Pocisk
\item Włada Zaklęciami
\item Mówi Głosem Ziemi
\end{itemize}

\textbf{Porady odnośnie wyboru zdolności}

\textbf{Poziom 1}: Wybierz zdolność niskiego poziomu, która zapewnie jedną z korzyści tematycznych specjalizacji, taką, którą mogłaby mieć postać 1-szego poziomu.

Czasami, dodatkowa zdolność niskiej mocy jest wskazana, w zależności od specjalizacji. Często, ta zdolność zapewnia trening w odpowiedniej dziedzinie wiedzy lub powiązanej umiejętności. Alternatywnie, może ona oferować prosty bonus od 2 do 3 punktów Puli.

\textbf{Poziom 2}: Wybierz zdolność niskiego poziomu, która jest powiązana z motywem przewodnim specjalizacji. Powinna ona nie być powiązana ze zdolnością 1-szego poziomu. Warto zaznaczyć, że jeśli nie zapewniono zdolności defensywnej na poziomie 1-zym, poziom 2 to dobre miejsce, by go przyznać.

\textbf{Poziom 3}: Wybierz dwie zdolności średniego poziomu. Obydwie są opcjami tej specjalizacji – BG wybiera jedną lub drugą.

Pierwsza opcja powinna zapewniać jedną z korzyści powiązanych ze specjalnością, która nie musi być bezpośrednio powiązana z tymi na wcześniejszych poziomach.

Druga opcja powinna zapewniać jakąś metodę ataku, jeśli wcześniej nie zapewniono żadnej. Alternatywnie, jeśli zdolności niższego poziomu nie zapewniły postaci tego, co powinna mieć, ta zdolność może ulepszyć zdolność zapewnianą na niższym poziomie.

\textbf{Poziom 4}: wybierz zdolność średniego poziomu, która zapewnia jedną z korzyści tematycznych specjalizacji, która nie musi być powiązana ze zdolnościami niższego poziomu.

\textbf{Poziom 5}: Wybierz zdolność wysokiego poziomu, która zapewnia jedną z korzyści tematycznych specjalizacji – nie musi ona być powiązana ze zdolnościami wcześniejszych poziomów.

\textbf{Poziom 6}: Wybierz 2 zdolności wysokiego poziomu. Obydwie są opcjami tej specjalizacji – BG wybierze jedną lub drugą. 

Pierwsza opcja powinna zapewniać korzyść tematyczną specjalizacji, która nie musi być powiązana z tymi zapewnianymi wcześniej. Jednakże, ta zdolność może także zapewniać ostateczną wersję zdolności niższego poziomu, jeśli wersja z średniego lub niskiego poziomu nie była wystarczająca.

Druga opcja powinna zapewniać alternatywną metodę uzupełnienia postaci, w sposób, który jest odmienny od opcji numer 1. Dla przykładu, jeśli pierwsza opcja zapewnia jakiś atak, to druga może być interakcją, zbieraniem informacji lub zdolnością leczącą, w zależności od motywu przewodniego specjalizacji. 

\subsubsection{Ruch}\index{Specjalizacje!Kategorie Specjalizacji!Ruch}

Specjalizacje, która zapewniają nowe formy ruchu - w celu przodowania w walce, uciekania od sytuacji, od których inni niem ogą uciec, poruszania się cicho w celu złodziejstwa lub ucieczki, lub poruszania się w lokacjach normalnie niedostępnych – znajdują się w tej kategorii. Te specjalizacje zazwyczaj mają metody zapewniające atak lub obronę poprzez ruch, choć mogą też zapewnić inne sposoby na osiągnięcie tych celów. 

Klasyczna specjalność ruchu jest taką, która polega na szybkości by wykonywać więcej ataków i uniknąć zranienia, choć ogólne zdolności mogą zapewnić te same korzyści. Inne specjalności w tej kategorii mogą zapewnić postaci zdolność stanie się niematerialną, zmienić jej formę w np.: wodę lub powietrze, lub natychmiast się poruszyć za pośrednictwem teleportacji.

\textbf{Połączenia z innymi BG}: Wybierz 4 stosowne połączenia z powyższej listy.

\textbf{Dodatkowy Ekwipunek}: Każdy obiekt konieczny do osiągnięcia wielkich prędkości, zmieny stanu skupienia, lub innego pozyskania korzyści tej specjalizacji powinien być zapewniony jako dodatkowy ekwipunek. Pewne specjalizacje nie wymagają niczego by pozyskać ich benefity.

\textbf{Sugestia Mniejszego Efektu}: Cel jest oszołomiony i jego następna akcja jest utrudniona.

\textbf{Sugestia Większego Efekt}u: Cel jest wstrząśnięty i traci następną akcję.

Poniższa lista to przykłady i nie jest wyczerpująca względem wszystkich specjalizacji w tej kategorii:

\begin{itemize}
\item Istnieje Częściowo Poza Fazą
\item Porusza się jak Kot
\item Porusza się jak Wiatr
\item Ucieka Precz
\item Rozdziera Ściany Świata
\item Podróżuje przez Czas
\item Pracuje w Ciemnych Uliczkach
\end{itemize}

\textbf{Porady odnośnie wyboru zdolności}

\textbf{Poziom 1}: Wybierz zdolność niskiego poziomu, która zapewnia podstawową korzyść specyficznego stylu ruchu – np.: zwiększoną szybkość, zręczność, niematerialność itp.

Czasami, dodatkowa zdolność niskiej mocy jest stosowna, w zależności od specjalizacji. Jeśli podstawowa korzyść ruchu wymaga pewnego rodzaju dodatkowego zrozumienia lub wyszkolenia, ta zdolność może ją zapewniać. Alternatywnie, jeśli zapewniony ruch wydaje się odblokowywać podstawowy atak lub obronę, (polegający na używaniu pierwszej zdolności), dodaj to.

\textbf{Poziom 2}: Wybierz zdolność niskiego poziomu, która zapewnia nową zdolność ataku lub obrony powiązaną z motywem przewodnim specjalności. 

Alternatywnie, ta zdolność moża zapewnić dodatkową możliwość związaną z formą ruchu, która zapewnia użyteczną informację, która normalnie byłaby niemożliwa do zdobycia przez kogoś bez tej specjalizacji. 

\textbf{Poziom 3}: Wybierz dwie umiejętności średniego poziomu. Obydwie są opcjami tej specjalizacji – BG wybierze jedną lub drugą.

Pierwsza opcja powinna zapewniać dodatkową możliwość ruchu lub ulepszyć podstawową możliwość ruchu w nawiązaniu do motywu przewodniego specjalizacji. Nie jest to bezpośrednio ofensywne lub defensywne, ale zapewnia postaci nowy poziom zdolności lub kompletnie nową zdolność powiązaną z jej podstawową zdolnością ruchu.

Druga zdolność powinna zapewniać ofensywną lub defensywną zdolność powiązaną ze specyficzną formą ruchu, którą zapewnia specjalizacja.

\textbf{Poziom 4}: Wybierz zdolność średniego poziomu, która ulepsza przewagi zapewniane przez paradygmat ruchu specjalności. Może to zapewnić nową lub lepszą formę obrony (bezpośrednio lub pośrednio, jeśli przemieszczamy się do lokacji lub czasu, w których niebezpieczeństwo nie istnieje) lub nową/lepszą formę ataku.

\textbf{Poziom 5}: Wybierz prawie-ostateczną zdolność wysokiego poziomu, która korzysta z ruchu. Dla przykładu, jeśli specjalizacja zapewnia przemieszczanie się w czasie, ta zdolność może zapewnić faktyczne (lecz chwilowe) przemieszczenie siew czasie. Jeśli specjalizacja ulepsza szybkość, ta zdolność może pozwolić postaci poruszać się na bardzo długi dystans w ciągu akcji. I tak dalej.

Alternatywnie, odblokuj jeszcze nie używane zdolności, które wynikają z bazowej formy ruchu zapewnianej przez tą specjalizację. 

\textbf{Poziom 6}: Wybierz dwie zdolności wysokiego poziomu. Obydwie są opcjami tej specjalności – BG wybierze jedną lub drugą.

Pierwsza z opcji powinna zapewniać ofensywną lub defensywną zdolność, odmienną od tej zapewnianej na 4 poziomie (lecz wysokiego poziomu zamiast średniego).

Druga opcja powinna dalej poszerzać możliwości ruchu podstawowej zdolności. Jeśli wybór 5-tego poziomu był zdolnością prawie-ostateczną, tutaj powinna pojawić się opcja ostateczna powiązana z ruchem.

\subsubsection{Atak w walce}\index{Specjalizacje!Kategorie Specjalizacji!Atak w walce}

Specjalności ataku w walce zajmują się zadawaniem obrażeń. Specjalności w tej kategorii oferują także defensywne zdolności, ale kładą nacisk przede wszystkim na obrażenia, których nie osiągają inne specjalizacje.

Aby osiągnąć ten cel, specjalność ataku w walce może oferować mistrzostwo w danym stylu walki, co może być treningiem w konkretnej broni lub sztuce walki, lub użyciem konkretnego narzędzia (lub nawet rodzaju energii). Styl może być czymś tak prostym jak bycie najlepszym przeciwko konkretnemu rodzajowi przeciwnika, lub czymś znacznie szerszym, jak zastosowanie szczególnie wrednego lub nieuczciwego stylu. Walczący atakiem w walce może użyć ognia, siły lub magnetyzmu jako swoją preferowaną metodę zadawania obrażeń.

\textbf{Połączenia z innymi BG}: Wybierz 4 stosowne połączenia z powyższej listy.

\textbf{Dodatkowy Ekwipunek}: Broń, narzędzie lub inny specjalny przedmiot lub substancja (jeśli jakakolwiek) wymagana, by stosować dany styl walki. Dla przykładu, porcja 5-poziomowej trucizny dla Walczy Nieczysto lub Morduje, trofeum po poprzednio pokonanym wrogu dla Walczy z Robotami, lub stylowe ubrania dla Walcząc, Porywa Tłum.

\textbf{Sugestia Mniejszego Efektu}: Cel jest tak oszołomiony przez Twoje manewry, że na jedną turę wszystkie jego zadania są utrudnione. 

\textbf{Sugestia Większego Efektu}: Wykonaj natychmiastowo dodatkowy atak, korzystając z ataku zapewnianego przez specjalizację w swojej turze.

Poniższe to przykłady i ta lista nie wyczerpuje wszystkich możliwych specjalizacji w tej kategorii:

\begin{itemize}
\item Walczy z Robotami
\item Walczy Nieczysto
\item Walcząc, Porywa Tłum 
\item Poluje
\item Posiada Licencję na Broń
\item Szuka Kłopotów
\item Mistrzowsko Posługuje się Bronią
\item Morduje
\item Nie Potrzebuje Broni
\item Jest Bardzo Silny
\item Wpada w Furię
\item Zabija Potwory
\item Rzuca ze Śmiertelną Dokładnością
\item Dzierży Dwie Bronie Naraz
\end{itemize}

\textbf{Porady odnośnie wyboru zdolności}

\textbf{Poziom 1}: Wybierz zdolność niskiego poziomu, która zadaje dodatkowe obrażenia, korzystając z stylu walki, energii lub podejścia specjalności, lub gdy jest używane przeciwko wybranemu wrogowi.

Czasami, dodatkowa zdolność niskiego poziomu jest wskazana, w zależności od specjalizacji. Dla przykładu, specjalizacja zapewniająca zdolność walki specjalną bronią może oferować trening w zadaniach tworzenia tej broni. Specjalność, która zapewnia zwiększone obrażenia przeciwko danemu typowi wrogów może zapewnić trening w rozpoznawaniu, lokalizowaniu lub po prostu ogólnej wiedzy o tym wrogu. Styl walki, który jest szczególnie nieczysty, może zapewniać trening w zastraszaniu. I tak dalej.

Jeśli specjalność dotyczy walki z specyficznym wrogiem, dodatkowe moce (więcej niż normalnie byłyby oferowane) mogą być odpowiednie. Albo zwiększają one efektywność przeciwko wybranemu wrogowi, albo oferują szersze, lecz powiązanie zdolności, które dają specjalizacji pewne funkcjonalności, nawet, gdy nie walczy się z wrogiem.

\textbf{Poziom 2}: Wybierz zdolność niskiego poziomu która zapewnia jakąś formęobrony korzystając z broni, stylu walki lub wybranej energii. Jeśli styl walki jest szczególnie dobry w walczeniu z wybranymi rodzajami przeciwników, zdolność powinna zapewniać obronę przed tym rodzajem wrogów. Alternatywnie, specjalizacja może oferować kolejną metodę zwiększania obrażeń korzystając z wybranego paradygmatu.
Czasami, dodatkowa zdolność niskiego poziomu jest stosowna na poziomie 2. Jeśli tak jest, wybierz rodzaj zdolności niskiego poziomu, której nie wybrano na poziomie 1.

\textbf{Poziom 3}: Wybierz dwie zdolności średniego poziomu. Obydwie sąopcjami tej specjalizacji: BG wybiera jedną z nich.

Jedna opcja powinna zadawać dodatkowe obrażenia korzystając z stylu walki, energii lub podejścia specjalności, lub przeciwko konkretnemu wrogowi. Może być to coś tak prostego jak dodatkowy atak danego rodzaju.

Druga opcja powinna zapewniać metodę chwilowej neutralizacji przeciwnika poprzez rozbrajanie go, oszołomienie lub wstrząsanie nim, spowalnianie go lub utrzymywanie w miejscu, lub w inny sposób ograniczanie jego możliwości poprzez korzystanie ze stylu walki, energii lub podejścia specjalności lub przeciwko wybranemu wrogowi.

\textbf{Poziom 4}: Wybierz zdolność średniego poziomu, która poszerza paradygmat specjalności.  Często chodzi o wytrenowanie w konkretnym rodzaju ataku. Alternatywnie, ta zdolność może zwiększyć przewagę, uzyskując konkretny status w walce, taki jak zaskoczenie.

\textbf{Poziom 5}: Wybierz zdolność wysokiego poziomu, która zadaje obrażenia. Alternatywnie, jeśli specjalizujesz się w walce z danym typem wroga, ta zdolność może zapewnić postaci szansę na kompletną jego neutralizację, destrukcję, oślepienie lub zabicie jednego celu do poziomu 3 (lub wyższego, jeśli walczysz z danym typem wroga).

\textbf{Poziom 6}: Wybierz dwie zdolności wysokiego poziomu. Obydwie są opcjami tej specjalizacji: BG wybierze jedną z nich.

Jedna z opcji powinna korzystać z paradygmatu specjalności, by zadawać wyjątkowo dużą ilość obrażeń. Druga opcja powinna być odmiennym sposobem na zadawanie obrażeń, korzystając z paradygmatu specjalności lub po prostu zadając dużo obrażeń w ogólności (i polegając na poprzednich zdolnościach specjalizacji, by polepszyć celowanie). To może być atak przeciwko wielu celom, jeśli pierwsza opcja dotyczyła pojedynczego celu, zdolność natychmiastowego zabicia lub zneutralizowania wroga (zaczynając od poziomu 4, lecz z możliwością korzystania z Wysiłłku, by zwiększyć poziom celu) lub zdolność wybrania kolejnego wroga, wykonania kolejnego ataku lub zdolność ucieczki, by móc walczyć w przyszłości.

\subsubsection{Wsparcie}\index{Specjalizacje!Kategorie Specjalizacji!Wsparcie}

Specjalizacje, które pozwalają na pomaganie innym w osiągnięciu sukcesu, obrony innych, leczeniu ich i tak dalej są specjalizacjami wsparcia. Oczywiście, większość innych zdolności można wykorzystać, wspomagając innych, ale specjalizacje wsparcia (takie jak Wysysa Energię) są nastawione głównie na pomaganie, leczenie i ulepszania postaci, która wybiera tę specjalność. Specjalności wsparcia polegają na rożnych metodach zapewniania pomocy, wliczając trening bitewny wykorzystywany do bronienia, zdolności nadnaturalne lub wysoko-technologiczne umożliwiające leczenie, lub po prostu pozwalanie innym, by się zapomnieli poprzez zapewnianie rozrywki.

\textbf{Połączenia z innymi BG}: Wybierz 4 odpowiednie Połączenia z listy powyżej.

\textbf{Dodatkowy Ekwipunek}: Każdy obiekt odpowiedni, by zapewniać wsparcie. Dla przykładu, ktoś ze specjalizacją, która zabawia innych mógłby otrzymać muzyczny instrument lub podobny obiekt swojej specjalności. Niektóre specjalności z tej kategorii nie wymagają niczego, by zyskać swoje korzyści.

\textbf{Sugestia Mniejszego Efektu}: Do końca następnej rundy możesz ściągnąć na siebie atak bez konieczności korzystania z akcji.

\textbf{Sugestia Większego Efektu}: Możesz wziąć dodatkową darmowąakcję, by wesprzeć swojego sprzymierzeńca. 

Poniższe specjalności to przykłady i nie są kompletną listą możliwych specjalności w tej kategorii.

\begin{itemize}
\item Chroni Słabszych
\item Zabawia
\item Pomaga Swoim Przyjaciołom
\item Zaprowadza Sprawiedliwość
\item Wspiera Społeczność
\item Wysysa Energię
\item Uzdrawia
\end{itemize}

\textbf{Porady odnośnie wyboru zdolności}

\textbf{Poziom 1}: Wybierz zdolność niskiego poziomu, która zapewnia jakąś formę obrony, wsparcia lub rozrywki, korzyść w procesie leczenia lub obronę. Ta obrona może dotyczyć BG, an ie jego sprzymierzeńca, gdyż postać nie może chronić innych, najpierw nie broniąc samej siebie (a czasami bronienie samego siebie to ta rzecz, o którą chodzi).

Czasami, dodatkowa zdolność niskiego poziomu jest wskazana, w zależności od specjalizacji. Często, jest to zdolność zapewniająca trening w powiązanej dziedzinie wiedzy lub umiejętności, ale może to być coś, co łączy się z zapewnianą zdolnością, ale samo w sobie nie robiłoby zbyt wiele.

\textbf{Poziom 2}: Wybierz zdolność niskiego poziomu która dalej wspiera styl poprzedniego poziomu. Jeśli zdolność poprzedniego poziomu zapewniała sposób na obronę, który dotyczył tylko posiadacza umiejętności, poziom 2 powinien zapewniać zdolność chronienia innych. Jeśli taka zdolność już została zapewniona, zdolność 2 poziomu powinna chronić posiadacza zdolności lub zapewniać zdolność ataku, która, jeśli to możliwe, powinna być powiązana z motywem specjalizacji.

\textbf{Poziom 3}: Wybierz dwie zdolności średniego poziomu. Obydwie są opcjami tej specjalności: BG wybierze jedną z nich.

Pierwsza opcja powinna działać zgodnie z motywem przewodnim leczenia, wspierania lub bronienia, lub w inny sposób pomagać drugiej osobie. Druga opcja powinna w jakiś sposób wspierać postać, ofensywnie lub defensywnie lub w sposób, który rozszerzy jej zdolności. Alternatywnie, może to być kolejna, zupełnie odmienna metoda pomocy komuś innemu.

\textbf{Poziom 4}: Wybierz zdolność średniego poziomu, która daje sprzymierzeńcowi bezpośredni bonus lub daje postaci możliwość pomagania innym. Może to być także coś, co rani lub obezwładnia wroga, gdyż eliminacja przeciwników z pewnością pomaga sojusznikom.

\textbf{Poziom 5}: Wybierz zdolność wysokiego poziomu, która zapewnia opcję defensywną lub ofensywną, jeśli jeszcze nie została zapewniona. Jeśli już siętym zajęto lub jest to uważane za zbędne, wybierz zdolnosć wysokiego poziomu, która zapewnia jakąś formę obrony, pomocy lub rozrywki, korzyść w leczeniu lub ochronę innej postaci. Dla przykładu, zdolność 5 poziomu może dać sprzymierzeńcowi dodatkową darmową akcję lub pomóc mu w powtórce akcji, jeśli wcześniej ją oblał.

\textbf{Poziom 6}: Wybierz dwie zdolności wysokiego poziomu. Obydwie są opcjami tej specjalności: BG wybierze jedną z nich.

Pierwsza opcja powinna zapewniać ostateczną metodą pomocy innej postaci, zgodną z motywem przewodnim specjalności. Druga opcja może zapewnić alternatywną metodą pomagania innym, wiele specjalności z tej grupy to robi. Jednakże, opcja zapewniająca wysoko-poziomowy atak lub obronę także może wystąpić. 

\subsubsection{Przyjmowanie Obrażeń}\index{Specjalizacje!Kategorie Specjalizacji!Przyjmowanie Obrażeń}

Specjalności, które skupiają się na przyjmowaniu wielu obrażeń znajdują się w tej kategorii. Te specjalizacje zapewniają także zdolności ofensywne, jak i dodatkowe zdolności powiązane z ich motywem przewodnim, ale najważniejsze są tutaj zdolności obronne.

Pewne specjalności przyjmowania obrażeń polegają na fizycznej transformacji, która zapewnia dodatkową obronę, a inne polegają na wyspecjalizowanym treningu, korzystaniu z narzędzi takich jak tarcze lub ciężki pancerz, lub zapewniają zdolność naprawdę szybkiego leczenia się. Rodzaj fizycznej transformacji, który zapewnia ta specjalizacja, jeśli występuje, jest bardzo różnorodny. Specjalność może zamienić skórę postaci w kamień, wzmocnić ciało metalem, zamienić ją w potwora, powiększyć ją tak, by było trudniej ją zranić, i tak dalej.

\textbf{Połączenia z innymi BG}: Wybierz 4 odpowiednie połączenia z listy powyżej.

\textbf{Dodatkowy Ekwipunek}: Każdy obiekt konieczny, by utrzymywać fizyczną transformację (taki jak narzędzia do naprawy robotycznych części, tarcza lub inne narzędzie defensywne do umiejętnej obrony, lub może jakiś rodzaj amuletu bądź serum)/ Niektóre specjalności przyjmowania obrażeń nie wymagająniczego, by uzyskać ich korzyści.

\textbf{Sugestia Mniejszego Efektu}: +2 do Pancerza na kilka tur.

\textbf{Sugestia Większego Efektu}: Odzyskaj 2 punkty w Puli Mocy.

Poniższe są tylko przykładami i nie są kompletną listą wszystkich możliwych specjalności w tej kategorii:

\begin{itemize}
\item Jest Stworzony z Kamienia
\item Nosi Egzotyczną Tarczę
\item Chroni Wrota
\item Łączy Ciało i Stal
\item Rośnie do Gigantycznych Rozmiarów
\item Wyje do Księżyca
\item Żyje w Dziczy
\item Jest Mistrzem Obrony
\item Nigdy się nie Poddaje
\item Jest Jednoosobowym Bastionem
\end{itemize}

\textbf{Porady odnośnie wyboru zdolności}

\textbf{Poziom 1}: Wybierz zdolność niskiego poziomu, która zapewnia obronę zogdną z motywem przewodnim specjalności. Jeśli motyw to po prostu intensywny trening lub użytkownik ma defensywne narzędzie, ta zdolność może być tak prosta jak bonus do Pancerza. Jeśli obrona pochodzi od fizycznej transformacji, ta zdolność zapewnia bazową formę obrony wraz z efektami, korzyściami i czasami problemami wynikającymi z transformacji. Nisko-poziomowa zdolność lecząca także byłaby stosowna na pierwszym poziomie.

Czasami, dodatkowa zdolność niskiego poziomu jest potrzebna, w zależności od specjalności. Jeśli postać przechodzi przemianę, ta zdolność może zapewnić dodatkowy efekt, choć w stosunku do pewnych transformacji, może to być po prostu opis jak ktoś z anormalną fizjologią w pełni się leczy. W innych przypadkach, dodatkowy efekt może być treningiem w powiązanej umiejętności, lub może odblokować zdolność korzystania z konkretnej zbroi lub tarczy bez kary.

\textbf{Poziom 2}: Jeśli motyw przewodni specjalizacji nie jest fizyczną transformacją, wybierz nisko-poziomową zdolność, która zapewnia dodatkową metodę obronną, leczenie obrażeń, lub unikanie ataków.

Jeśli motyw przewodni specjalności to fizyczna transformacja, wybierz nisko-poziomową zdolność, która odblokowuje nowe możliwości związane z formą postaci. Może to być lepsza kontrola nad transformacją, odblokowywanie robotycznego interfejsu, lub w inny sposób w pełni odblokowanie tej formy. Ta zdolność nie jest koniecznie defensywna, choć może być.

\textbf{Poziom 3}: Wybierz dwie zdolności średniego poziomu. Obydwie są opcjami tej specjalizacji: BG wybiera jedną z nich.

Pierwsza opcja powinna zapewniać dodatkową obronę związaną z motywem przewodnim specjalizacji, taką jak zdolności obronne odblokowane poprzez transformację (co może także zwiększyć zdolności ataku) lub proste fizyczne ulepszenie jeśli obrona jest zapewniana przez umiejętności lub ulepszone uzdrawianie.

Druga opcja powinna zapewnić ofensywną zdolność, zwłaszcza jeśli tworzysz specjalizację nie skupioną na transformacji, która jeszcze nie ma ofensywnych korzyści. Ta zdolność może być ulepszonym atakiem lub zapewnić jakąś inną korzyść użyteczną w walce, taką jak szybkie uniku lub (na innym końcu kontinuum) bycie nieporuszalnym.

\textbf{Poziom 4}: Wybierz zdolność średniego poziomu dodatkowo ulepsza zalety zapewniane przez paradygmat unikania obrażeń specjalizacji. Często, wlicza się w to trening w konkretnym rodzaju obrony. Alternatywnie, może to zapewniać zalety zapewnione poprzednio, niezależnie, czy to oznacza większą kontrolę nad transformacją, zyskanie dodatkowych szans by uniknąć obrażeń lub powtórzenie zadania związanego z ulepszoną determinacją, itp. Jeśli specjalizacji brakuje opcji ofensywnej, jest to dobre miejsce, by ją zyskała.

\textbf{Poziom 5}: Wybierz zdolność wysokiego poziomu, która zapewnia obronę, możliwe, że w formie odrzucenia jakiegoś trudnego stanu (wliczając śmierć). Jeśli specjalizacja oferuje fizyczną transformację, ta zdolność może odblokować dalszą dodatkową, powiązaną zdolność, ofensywną, defensywną lub coś powiązanego z eksploracją i interakcją (np.: lot jeśli forma posiada skrzydła, zastraszanie jeśli jest straszna itp.).

\textbf{Poziom 6}: Wybierz dwie zdolności wysokiego poziomu. Obydwie są opcjami tej specjalności: BG wybierze jedną z nich.

Pierwsza opcja powinna używać paradygmatu specjalności, by zwiększać obronę lub zdolność poradzenia sobie z obrażeniami.

Druga opcja powinna oferować odmienny sposób na bycie defensywnym. W pewnych wypadkach, najlpeszą obroną jest dobry atak, więc ta opcja może zapewnić wysoko-poziomową zdolność ataku zgodnąz motywem przewodnim specjalizacji, albo jako po prostu wzrost obrażeń przy atakowaniu, albo jako lepszą kontrolę niestabilnej, fizycznej transformacji.

\subsubsection{Customizacja Specjalności}\index{Specjalizacjei!Customizacja Specjalności}

Czasami nie wszystkie opcje specjalizacji pasują do konceptu postaci, lub może MG potrzebuje dodatkowych porad odnośnie tworzenia nowej specjalności. Niezależnie od tego, odpowiedź leży w patrzeniu na zdolności specjalności na najbardziej fundamentalnym poziomie.

Na każdym poziomie, gracz może wybrać jedną z poniższych zdolności zamiast zdolności zapewnianej na danym poziomie. Wiele z nich to zdolności zastępcze, zwłaszcza na wysokim poziomie, które polegają na modyfikacji ciała, integracji z wysoko-technologicznym ekwipunkiem, uczeniem się potężnych zaklęć, odkrywaniem zapomnianych sekretów lub czymś podobnym, jak dyktowane przez konwencję gry.

\textbf{Poziom 1}

\begin{itemize}
\item Zdolności Bojowe
\item Potencjał
\end{itemize}

\textbf{Poziom 2}

\begin{itemize}
\item Zdolność niższego poziomu – wybierz zdolność 1-szego poziomu, powyżej.
\item Umiejętna Obrona
\item Wyszkolony We Wszystkich Broniach
\item Umiejętny Atak
\end{itemize}

\textbf{Poziom 3}

\begin{itemize}
\item Zdolność niższego poziomu – wybierz dowolną zdolność 1 lub 2 poziomu, powyżej.
\item Ulepszone Zdrowie
\item Zbroja Fuzyjna
\end{itemize}

\textbf{Poziom 4}

\begin{itemize}
\item Zdolność niższego poziomu – wybierz dowolną zdolność 1, 2 lub 3 poziomu, powyżej.
\item Odporność na Trucizny
\item Wbudowanie Bronie
\end{itemize}

\textbf{Poziom 5}

\begin{itemize}
\item Zdolność niższego poziomu – wybierz dowolną zdolność 1, 2, 3 lub 4 poziomu, powyżej.
\item Adaptacja
\item Pole Obronne
\end{itemize}

\textbf{Poziom 6}

\begin{itemize}
\item Zdolność niższego poziomu – wybierz dowolną zdolność 1, 2, 3, 4 lub 5 poziomu, powyżej.
\item Pole Reakcyjne
 \end{itemize}
