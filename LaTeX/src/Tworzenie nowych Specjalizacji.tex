\section{Tworzenie nowych Specjalziacji}\index{SPecjalizacje!Tworzenie nowych Specjalizacji}

Ta sekcja zawiera wszystko, czego potrzebujesz, by tworzyć nowe specjalizacje. 

Każda specjalizacja ma swój własny styl, taki jak eksploracja, manipulowanie energią, lub po prostu zadawanie wielkich obrażeń w walce. Te ogólne klasyfikacje nazywamy kategoriami specjalizacji.

Każda kategoria specjalizacji ma swój motyw, razem z selekcją porad opisujących jak wybrać zdolności dla każdego poziomu z rozdziału Zdolności, od poziomu 1 do 6.

Nowo stworzona specjalizacja powinna być nazwana w formie czasownikowej, odpowiednio odmienionej, takiej jak Kontroluje Bestie lub Jest Stworzony z Kamienia. Dla przykładu, specjalizacja skupiona na ogniu jest tworzona na bazie porad odnośnie kategorie specjalizacji manipulacja energią i może być nazwana Nosi Halo Ognia (jedna z przykładowych specjalizacji w tym rozdziale). Alternatywnie, nowo stworzona specjalizacja może otrzymać nazwę Wznieca Ognie Apokalipsy lub Rozpala Ogień Samą Myślą. 

\subsection{Kategorie Specjalizacji}\index{Specjalizacje!Kategorie Specjalizacji}

 1. Korzystanie ze Sprzymierzeńców
   
2. Podstawowa
   
3. Manipulacja Energią
   
4. Eksploracja
   
5. Wpływ
   
6. Nieregularna
   
7. Ruch
   
8. Atak w walce
   
9. Wsparcie
   
10. Przyjmowanie Obrażeń

\subsubsection{Wybieranie zdolności, bazując na ich relatywnej mocy}\index{Specjalizacje!Wybieranie zdolności}

Porady odnośnie wybierania zdolności sugerują, by wybrać zdolności z jednego z trzech zbiorów – niski poziom, średni poziom i wysoki poziom. Te poziomy odpowiadają z "poziomami" danymi każdej zdolności. Te zdolności są dalej posegregowane w kategorie zdolności bazujące na tym, co robią – zdolności, które ulepszają fizyczne ataki są w kategorii umiejętności ataków, zdolności które wspierają sprzymierzeńców są w kategorii wsparcie itp. Patrz na poziomy i kategorie w sekcji Kategorie Zdolności i Relatywnej Mocy w rozdziale Zdolności.

Zdolności niskiego poziomu najlepiej się nadają na opcje na 1 i 2 poziomie. Zdolności średniego poziomu nadają się na opcje na 3 i 4 poziomie. Zdolności wysokiego poziomy pasują do poziomów 5 i 6

Powiedziawszy to – czasami uznasz za stosowne danie zdolności niskiego poziomu na poziomach 3 lub 4, lub może zdolność średniego poziomu na poziomie 1 i 2. Rób tak rzadko, ale bądź świadom takiej możliwości. To może być jedyny sposób na otrzymanie wszystkich zdolności, których chcesz, gdy tworzysz specjalizację. "Wyższe" zdolności zazwyczaj kosztują więcej punktów z Pul. Tak więc, jeśli zdolność średniego poziomu jest dostępna na poziomie 1 lub 2, lub zdolność wysokiego poziomu jest dostępna na poziomie 3 lub 4, wyższy koszt uczyni ją bardziej zbalansowaną.

\subsubsection{Balansowanie zdolności}

Porady odnośnie każdej kategorii mają za zadanie zapewnić, że zbudowane Specjalizacja będzie zbalansowana. Czasami jest stosowne dać zdolność niskiej mocy razem ze zwykłą zdolnością na danym poziomie, w zależności od potrzeb związanych ze Specjalnością. "Zdolność niskiej mocy" nie jest nigdzie zdefiniowana i pozostaje do interpretacji MG, ale ogólnie mówiąc, nie powinna być potężniejsza niż zdolność niskiego poziomu (co znaczy, poziomiu 1 lub 2).

Dla przykładu, ktoś kto ma lodowe moce może stworzyć małe rzeźby ze śniegu w dodatku do emisji promienia zimna. Ktoś, kto korzysta z elektryczności może naładować rozładowany artefakt lub mieć atut dla korzystania z elektrycznych systemów. I tak dalej.

Często, porady odnośnie Specjalizacji notują to jako możliwość. Jednakże, masz duża dowolność w zdecydowaniu, czy Specjalizacja potrzebuje dodatkowej zdolności, nawet jeśli porady odnośnie tego poziomu nie przewidują jej. Jeśli dodajesz zdolność, lub jest tam zdolność wysokiej mocy, której normalnie nie powinno być, może to oznaczać, że wybór dany na następnym poziomie lub na poprzednim nie był całkiem dobry. Zbalansowanie Specjalizacji to poniekąd sztuka. Musisz się oprzeć chęci przeładowania mocy Specjalizacji, ale także nie może ona być zbyt słaba.

\subsubsection{Porady odnośnie zdolności nie są wyryte w kamieniu}

Każda kategoria Specjalizacji zawiera porady, jaką zdolność powinno się wybrać na jakim poziomie. Ale nie patrz an te porady jak na coś, czego nie można przeskoczyć. Nie są one wyryte w kamieniu – po po prostu startowa pozycja. Możesz chcieć zmienić zdolność na danym poziomie na taką, której nie ma w poradach. Tak długo, jak wybrana zdolność znajduje się w odpowiedniej krzywej mocy dla tego poziomu, wszystko jest ok. Porady nie zostały zaprojektowane zm yślą o ograniczaniu graczy.

Dla przykładu, jeśli budujesz Specjalność skupioną na zimnie dla gry w świecie fantasy, możesz zadecydować, że zdolność, która wzywa demona, jest lepszym wyborem na danym poziomie niż zdolność, która zadaje obrażenia obszarowe, co jest poradą dla 5 poziomu manipulacji energią. Dokonanie takiej zmiany jest szczególnie wskazane, jeśli nowa Specjalność nazywa się Sięga do Dziewiątego Kręgu Piekieł.

\subsubsection{Zamienianie zdolności}

Jeśli tworzysz Specjalizację i myślisz, że powinna zawierać zbiór zdolności na pierwszym poziomie, które by ją mechanicznie przeciążyły, masz opcję dodać jedną jako zdolność "zamienianą". Łatwo to zrobić – postać może zamienić jedną ze swoich zdolności z typu na wskazaną zdolność niskiego poziomu z Specjalizacji. Tę zdolność zyskuje się zamiast jednej zwyczajowej wynikającej z typu postaci.

\subsubsection{Koncept i kategoria}

Wybór, by stworzyć Specjalizację, która korzysta z konkretnego konceptu – np.: tworzenia iluzji – nie oznacza, że musisz stworzyć focus określonej kategorii – w tym przypadku manipulacji środowiskiem. Specjalizację można stworzyć na wiele sposobów, korzystając z określonej energii, narzędzia lub pomysłu – każdy poskutkuje Specjalizacją odmiennej formy. Wszystko zależy od Ciebie. W tym przypadku, tworzenie iluzji może być wykorzystane, by zwodzić innych, co w takim wypadku powinno być kategorią wpływu.

W ten sam sposób, jeśli Specjalizacja daje postaci moc, by korzystać z jakiejś siły lub energii, nie oznacza to automatycznie, że powinna ona należeć do kategorii manipulacji energią (ale, oczywiście, może tam należeć, jeśli atakowanie i obrona są celem korzystania z tej energii). Ale może być zbudowana Specjalność, by dawała zdolności korzystające z energii lub siły, które są skupione na wytrzymałości, co sugeruje Specjalność Przyjmowanie Obrażeń (ktoś, kto może wytrzymać wiele ataków w walce); lub, jeśli główną cechą ma być maksymalizacja zadawanych obrażeń, to sugeruje to Specjalność Atak w Walce; lub tworzenie kompana stworzonego z tej energii lub siły, co sugeruje Specjalizację Korzystanie ze Sprzymierzeńców (czyli ktoś, kto korzysta z pomocnych istot, BN-ów lub nawet zduplikowanych wersji samego siebie by zyskać przewagę).

Oto inny przykład: Specjalizacja Kontroluje Grawitację może należeć do kategorii Manipulacja Środowiskiem lub Manipulacja Energią. To zależy od tego, czy ta Specjalność bardziej koncentruje się na miażdżeniu i trzymaniu przedmiotów (Manipulacja Środowiskiem) lub na uderzaniu w rzeczy i chronieniu siebie przy pomocy grawitacji (Manipulacja Energią).
Ta sama mnogość możliwych rozwiązań jest prawdziwa w innych przypadkach. Dla przykładu, jeśli ktoś może wzywać i kształtować ziemię, może on wykorzystać swoją moc, by zamienić się w istotę z kamienia (Przyjmowanie Obrażeń), by zaatakować wrogów (Atak w Walce) lub by tworzyć ściany, barykady i tarcze, by wspierać swoich sprzymierzeńców (Wsparcie).

Jeśli szukasz zdolności i nie możesz znaleźć odpowiedniej w długaśnym katalogu zdolności, możesz chcieć zmodyfikować jedną z nich, by mieć iluzję nowości (i uzyskać to, czego chcesz). Modyfikowanie oznacza skorzystanie z mechaniki zdolności, ale zmienieniu jej szczegółów w pewien sposób. Dla przykładu, może tworzysz nową Specjalność przemieszczania ziemi ale nie możesz znaleźć odpowiedniej ilości zdolności powiązanych z ziemią, by zaspokoić Twoje potrzeby. Łatwo jest zmienić inne zdolności, tak, by korzystały z ziemi zamiast z ognia, zimna lub magnetyzmu. Dla przykładu, Skrzydła Ognia mogą zostać zmienione na Skrzydła Ziemi, Lodowa Zbroja na Zbroję Ziemi itp. Te zmiany nie zmieniają niczego poza typem obrażeń i efektami odrzucenia (dla przykładu, Skrzydła Ziemi mogą generować chmury pyłu przy swoim przelocie).

\subsubsection{Zdolności, które odnoszą się do innych zdolności}

Pewne zdolności w rozdziale im poświęconym odnoszą się do innych zdolności. Jeśli wybierasz zdolność dla swojej Specjalizacji lub typu która odnosi siędo lub modyfikuje zdolnosć niskiego poziomu, umieść ową zdolność w swoim typie lub Specjalności, jako zdolność, którą gracz może wybrać na niższym poziomie.

\subsubsection{Tworzenie kompletnie nowych zdolności}

Możesz pójść dalej niż modyfikowanie i stworzyć jedną lub więcej kompletnie nowych zdolności. Kiedy to czynisz, spróbuj znaleźć coś podobnego do niej i skorzystaj z niej jak z szablonu. W każdym wypadku, zdecydowanie jak dużo powinno kosztować korzystanie z mocy (poprzez wydawanie punktów z Pul) jest jednym z ważniejszych aspektów stworzenia odpowiedniej zdolności.

Możesz zauważyć, że zdolności wyższego poziomu są droższe. Po części dzieje się tak dlatego, że czynią więcej, ale także dlatego, że postaci wyższego poziomu mają wyższe Skupienie niż postaci mniejszego poziomu, co oznacza, że płacą mniej punktów z odpowiednich Pul. Trzeciopoziomowa postać ze Skupieniem 3 w jednej z Pul nie płaci kosztu za zdolności kosztujące 3 lub mniej punktów. To świetne odnośnie zdolności mniejszego poziomu, ale zazwyczaj będziesz chciał, by gracz pomyślał trochę o tym, jak często chce skorzystać ze swoich najpotężniejszych zdolności. To oznacza, że powinny one kosztować przynajmniej 1 punkt więcej niż prawdopodobne Skupienie postaci na danym poziomie. (Bardzo często, postać będzie miała Skupienie w odpowiedniej Puli odpowiadające jej poziomowi.)

Jako ogólna zasada, typowa zdolność powinna kosztować tyle punktów, na jakim poziomie się ona znajduje. 

\subsubsection{Wybierz wtrącenia MG}

Pomysł o tym, jakie rodzaje wydarzeń mogą zaskoczyć, zaalarmować BG lub być dla niego katastrofą, gdy tworzysz nową Specjalizację, i przypisz te wtrącenia MG do niej. Zazwyczaj wydarza się to "w biegu" na sesji. Ale pomyślenie o tym zawczasu, gdy Specjalizacja jest tworzona i masz świeże pomysły w głowie, na pewno da Ci szczególnie piekielne opcje.

\subsection{Kategorie Specjalizacji}\index{Specjalizacje!Kategorie Specjalizacji}

\subsubsection{Korzystanie ze sprzymierzeńców}\index{Specjalizacje!Kategorie Specjalizacji!Korzystanie ze Sprzymierzeńców}

Te Specjalności przede wszystkim zapewniają BN-ów (kompanów). Ci kompani zapewniają pomoc BG na różne sposoby, ale zazwyczaj w formie atutu do akcji postaci.
Istnieje wiele potencjalnych motywów w tej kategorii, od zdolności, które pozwalają postaci na przywoływanie lub tworzenie sprzymierzeńców do takich, które pozwalają im na przyciąganie sprzymierzeńców poprzez sławę, magię, lub autorytet i charyzmę. 

\textbf{Połączenia z innymi BG}: Wybierz 4 odpowiednie połączenia z powyższej listy.

\textbf{Dodatkowy Ekwipunek}: Każdy przedmiot koniczny, by bohater mógł zachować przy sobie sprzymierzeńca. Dla przykładu, ktoś ze Specjalizacją, która wymaga super-nauki do tworzenia robotów-kompanów może mieć narzędzia wymagane, by budować i naprawiać owych sprzymierzeńców. Pewne Specjalności w tej kategorii nie wymagają niczego, by zyskać lub utrzymać korzyści.

\textbf{Sugestia Mniejszego Efektu}: BN-Sprzymierzeniec otrzymuje Ułatwienie w swojej następnej turze.

\textbf{Sugestia Większego Efektu}: BN-Sprzymierzeniec zyskuje natychmiastową dodatkową akcję.

Poniższe to przykłady i nie są kompletną listą wszystkich możliwych Specjalizacji w tej kategorii.

\begin{itemize}
\item  Buduje Roboty
\item  Zadaje się z Martwymi
\item  Kontroluje Bestie
\item  Istnieje w Dwóch Miejscach na Raz
\item  Przewodzi
\item  Włada Rojem
\item Mówi do Duchów
\end{itemize}    

\textbf{Porady odnośnie wyboru zdolności}

\textbf{Poziom 1}: Wybierz zdolność niskiego poziomu, która daje kompan 2 poziomu postaci, lub daje podobną korzyść zapewnioną przez BN-a. Alternatywnie, zapewnij podstawy pod pozyskiwanie takich kompanów na wyższych poziomach przez wybór zdolności, która daje postaci wpływ na innych.

Czasami dodatkowa zdolność niskiej mocy jest wskazana, w zależności od Specjalności. Często, jest to umiejętność, która przyznaje wyszkolenie w odpowiedniej dziedzinie wiedzy lub powiązanej umiejętności. Dla przykładu, wyszkolenie w umiejętności powiązanej z rodzajem kompana, który zostanie pozyskany przez BG, byłoby wskazane. 

\textbf{Poziom 2}: Wybierz zdolność niskiego poziomu która zapewnia wpływ na zbliżone rodzaje BN-ów jak kompan pozyskany na poprzednim poziomie. Jeśli żaden kompan nie był pozyskany na poprzednim poziomie, ta zdolność powinna teraz zapewnić ów benefit. Czasami drugorzędna zdolność może być wskazana w dodatku do mocy zapewnionej powyżej, może zdolność niskiego poziomu, która dodaje 2 lub 3 punkty do Puli.

\textbf{Poziom 3}: Wybierz dwie zdolności średniego poziomu. Daj je obydwie jako opcje dla Specjalności: BG wybiera jedną lub drugą.

Jedna opcja powinna dawać zdolność średniego poziomu, która ulepsza kompana, którego już ma BG (zazwyczaj awans z poziomu 2 do 3) lub daje dodatkowego kompana.

Druga opcja powinna dawać coś korzystnego postaci – może moc ofensywną lub defensywną, lub coś, co poszerzy wpływ jaki BG ma na swoich kompanów (lub potencjalnych kompanów).

\textbf{Poziom 4}: Wybierz zdolność średniego poziomu, która daje postaci moc ofensywną lub defensywną, jeśli BG jeszcze jej nie posiada, najlepiej powiązaną tematycznie ze Specjalizacją. Dla przykładu, jeśli BG zyskuje kompanów ze względu na swoją charyzmę, ta zdolność może mu pozwolić rozkazywać wrogom na krótki czas. Jeśli postać zyskuje kompanów poprzez udowanie ich lub przywoływanie, ta zdolność może im pozwolić wpływań na byty tego samego typu, które nie są jeszcze jej kompanami.

Alternatywnie, ta zdolność może dalej ulepszyć poprzednio uzyskanego kompana z poziomu 3 na poziom 4, lub pozwolić postaci pozyskać dodatkowego kompana.

\textbf{Poziom 5}: Wybierz zdolność, która ulepsza postać poprzez zapewniania obrony, zwiększenie jednej z Pul, lub inną moc natury defensywnej.

Alternatywnie, taa zdolność może umożliwić nowy sposób wpływania na i przywoływania BN-ów, w sposób odpowiedni do tematu przewodniego Specjalizacji. Dla przykładu, ktoś kto trzyma bestie-sprzymierzeńców może zyskać zdolność wezwania hordy mniejszych bestii. Ktoś, kto buduje roboty może zyskać zdolność by zbudować kilku mniejszych robotów-pomocników. I tak dalej.

W końcu, ta zdolność może ulepszyć poprzednio zyskanego kompana do 5 poziomu.

\textbf{Poziom 6}: Wybierz dwie zdolności wysokiego poziomu. Obydwie są opcjami w tej Specjalności – BG wybiera jedną z nich.

Jedna z tych zdolności powinna ulepszyć poprzednio pozyskanego kompana do 5 poziomu, jeśli nie wydarzyło się to na poziomie 5. W takim wypadku, ta zdolność powinna być zapewniona w dodatku do dwóch innych powiązanych zdolności.

Inna opcja wysokiego poziomu może zapewnić zbiór kompanów 3 poziomu dla postaci.

Ostatnia zdolność wysokiego poziomu może zapewnić nowe sposoby wpływania na i przywoływania BN-ów w sposób odpowiedni do motywu przewodniego Specjalności. Dla przykładu, ktoś, kto zyskuje kompanów poprzez wysoką charyzmę i trening może zyskać zdolność pozyskiwania informacji, które inaczej nie byłyby możliwe do poznania.