\section{Zdolności}\index{Zdolności}

Ten rozdział prezentuje obszerny katalog więcej niż 1000 zdolności, które postać może pozyskać ze swojego typu, posmaku (jeśli jakikolwiek) i specjalności. Są one uporządkowane alfabetycznie.

Typ postaci, posmak i specjalność przypisuje proponowany poziom do umiejętności. Jednakże, jeśli tworzysz kompletnie nową specjalność lub typ, zapewniamy parę dodatkowych narzędzi. 

Pierwszym jest ocena mocy danej zdolności, która mówi Ci, jak potężna jest ona w stosunku do innych zdolności. Odpowidnie poziomy to "niski poziom" dla poziomów 1 i 2, "średni poziom" dla poziomów 3 i 4 oraz "wysoki poziom" dla poziomów 5 i 6.

Te zdolności są dalej posortowane w kategorie zdolności, bazujące na tym, co one robią – zdolności, które polepszają fizyczne ataki są w kategorii umiejętność ataku, zdolności, które asystują sprzymierzeńcom sąw kategorii wsparcie, itd.

(Jeśli nie zaznaczono inaczej, nie możesz wybrać tej same zdolności dwa razy, nawet jeśli masz ją zapewnioną zarówno ze swojego typu, jak i z posmaku).

\subsubsection{Kategorie zdolności i relatywna moc}

Zdolności może podzielić na parę kategorii, w zależności do rzeczy, które one robią – polepszają ataki fizyczne, wspomagają sprzymierzeńców, zapewniają obronę, zapewniają specjalny atak itd. Pod każdą z poniższych kategorii znajdziesz listę zdolności, które do niej nalezą, posortowane w nisko-, średnio- i wysoko-poziomowe. Z tych kategorii głównie korzystają MG, gdy tworzą nowe specjalności do swojej kampanii, pozwalając im na przeszukanie krótkiej listy zdolności zamiast próbować znaleźć coś odpowiedniego, przeszukując tysiąc zdolności w tym rozdziale. Dla przykładu, MG może chcieć stworzyć dla swojej kampanii customową specjalizację zwaną "Jest Zrodzony w Bagnie" i chce defensywnej zdolności dla poziomu 5-tego, więc patrzy na wysoko-poziomowe zdolności w kategorii obrona i szybko zawęża dostępne opcje.

(Może być możliwe, że postać zyska tę samą zdolność z więcej niż jednego źródła (np.: z typu i deskryptora). Dopóki obydwie zdolności w oczywisty sposób się nie dodają (np.: dwie zdolności, z których każda zapewnia 3 punkty Puli Mocy, co razem dałoby +6 do Mocy) zduplikowana zdolność może być w pewien sposób ulepszona, np.: trwa dłużej, ma większy efekt lub automatycznie dodaje atut przy rzucie. Pewnie zdolności mają sugestie, jak to zrobić, w innym wypadku, gracz i MG powinni razem pracować nad tym, ja ulepszyć zdolność).

Kategoria zdolności nie zostały zaprojektowane jako sztywne lub wyczerpujące. Pewne zdolności mają więcej niż jedną kategorię, i można argumentować, że niektóre mogą być wliczone w więcej kategorii, niż tutaj podano.

Poniższe kategorie mają pewne powiązana  kategoriami z rozdziału o specjalizacjach. Dla przykładu, istnieje kategoria wsparcia zarówno tutaj, jak i w specjalnościach. Nie są one dokładnymi kopiami i nie oznaczają koniecznie tej samej rzeczy. Powiedziawszy to, jeśli tworzysz specjalność, która się koncentruje na wspomaganiu innych, wiele ze zdolności w kategorii wsparcie byłoby odpowiednimi wyborami.

\subsubsection{Umiejętność ataku}\index{Zdolności!Kategorie!Umiejętność ataku}

Daje CI wyszkolenie lub wyspecjalizowanie w specyficznym ataku fizycznym (takim jak miecze lub walka bez broni), kategorii fizycznego ataku (lekki ostrzami, ciężki obuchowy itp.) lub innej fizycznej akcji wykorzystywanej, by zadawać obrażenia (takiej jak niszczenie obiektów).

\textbf{Niski poziom}:

\begin{itemize}
\item Wyświetlacz w Hełmie
\item Wyszkolony w Broni Palnej
\item Wyszkolony w Średnich Broniach
\item Wyszkolony w Mieczach
\item Cel
\item Sztuki Walki
\end{itemize}

\textbf{Średni Poziom}:

\begin{itemize}
\item Gorączka krwi
\item Obliczenia Bitewne
\item Większa Umiejętność Obrony
\item Wyszkolony We Wszystkich Broniach
\item Walczący z Robotami
\item Celowanie Serv-0
\item Serv-0 Wojownik
\item Umiejętny Atak
\item Cel Snipera
\item Wyspecjalizowany w Rzucaniu
\end{itemize}

\textbf{Wysoki Poziom}:

\begin{itemize}
\item Jak Przepowiedziano
\item Pojedynek na Śmierć i Życie
\item Większa Umiejętność Ataku
\item Dążenie Łowcy
\item Mistrz Sztuk Walki
\item Mistrzostwo Ataków
\item Wyspecjalizowany Pięściarz
\end{itemize}

\subsubsection{Kompan}\index{Zdolności!Kategorie!Kompan}

Zapewnia Ci kompana, modyfikuje go lub daje Ci dodatkową korzyść gdy wchodzisz w interakcję z kompanem lub w jego pobliżu. W tę kategorię wliczają się kompani, zwierzęcy towarzysze i chwilowi kompani, tacy jak przywołane roje, duchy itd.

\textbf{Niski poziom}:

\begin{itemize}
\item Podstawowy Kompan
\item Zwierzęcy Kompan
\item Związana Magiczna Istota
\item Kontrola Roju
\item Futrzasty Kompan
\item Kopia
\item Świta
\item Wpływ na Rój
\item Nekromancja
\item Odporna Kopia
\item Robot-Asystent
\item Serv-0
\item Duch 
\end{itemize}

\textbf{Średni Poziom}:

\begin{itemize}
\item Oczy Bestii
\item Wezwanie Roju
\item Kompan-Ekspert
\item Kompan Eksplolator
\item Ognista Ręka Zguby
\item Pozyskanie Nietypowego Kompana
\item Większa Nekromancja
\item Ulepszona Więź z Przedmiotem
\item Żywa Zbroja
\item Kompan-Maszyna
\item Rumak
\item Służba
\item Mowa Statku Kosmicznego
\item Silniejsi Razem
\item Przywołanie Wielkiego Pająka
\item Ulepszona Kopia
\item Sobowtór Czasoprzestrzenny
\item Pętla Czasu
\end{itemize}    
    
\textbf{Wysoki Poziom}:

\begin{itemize}
\item Jak Jedna Istota
\item Drużyna Desperados
\item Drużyna Kompanów
\item Zew Dziczy
\item Wezwanie Ducha
\item Przysługa
\item Wezwanie Międzywymiarowego Ducha
\item Wezwanie Przez Czas
\item Przywołanie
\item Śmiercionośny Rój
\item Szczęki Smoka
\item Ognisty Sługa
\item Ulepszona Aportacja
\item Ulepszony Kompan
\item Ulepszony Kompan-Maszyna
\item Erupcja Insektów
\item Prawnik-Stażysta
\item Mistrzowska Modyfikacja Pancerza
\item Wielość
\item Mistrzostwo Więzi z Obiektem
\item Rekrutowanie Delegata
\item Armia Robotów
\item Przywołanie Demona
\item Sobowtór Czasoprzestrzenny
\item Prawdziwa Nekromancja
\end{itemize}

\subsubsection{Kontrola}\index{Zdolności!Kategorie!Kontrola}

Kontroluje lub wpływa na umysły w sposób który nie może być powtórzony przy pomocy zwykłego zastraszania i perswazji, np.: psioniczna kontrola umysłu, gaz strachu itp.

\textbf{Niski Poziom}:

\begin{itemize}
\item Uspokojenie Nieznajomego
\item Zauroczenie Maszyny
\item Zamglenie Pamięci Krótkotrwałej
\item Lokalny Aktywista
\item Wmawianie
\item Prowokacja
\item Hakowanie Niemożliwości
\item Kontrola Robotów
\item Ukojenie Dzikiego
\item Przerażająca Obecność
\end{itemize}

\textbf{Średni poziom}:

\begin{itemize}
\item Uspokojenie
\item Zachwyt lub Inspiracja
\item Zachwyt Światła Gwiazd
\item Rozkaz
\item Rozkazywanie Maszynom
\item Rozkazywanie Duchom
\item Kontrola Tłumu
\item Sen na Jawie
\item Wielkie Oszustwo
\item Przeszkodzenie
\item Kontrola Umysłu
\item Psioniczna Sugestia
\end{itemize}

\textbf{Wysoki Poziom}:

\begin{itemize}
\item Zaawansowany Rozkaz
\item Przejęcie Kontroli
\item Pranie Mózgu
\item Zmiana Paradygmatu
\item Kontrola Maszyny
\item Kontrola Dzikiej Bestii
\item Ogarnięcie Sytuacji
\item Ucieczka
\item Psioniczny Pasażer
\item Wskaż im Drogę
\item Sugestia
\item Słowo Rozkazu 
\end{itemize}

\subsubsection{Tworzenie}\index{Zdolności!Kategorie!Tworzenie}

Tworzy użyteczne przedmioty fizyczne, takie jak zwykłe narzędzia (młoty, łomy), przedmioty ograniczonego użytku (fizyczne cyphery, artefakty) lub niezależne istoty (roboty, żywiołaki, zombie). Wliczają się w to plany i efekty, które pomagają lub przyspieszają tworzenie. 

\textbf{Niski Poziom}:

\begin{itemize}
\item Stworzenie Śmiertelnej Trucizny
\item Budujący Umocnienia
\item Rzemieślnik Rupieci
\item Wydajność Maszyny
\item Modyfikacja Urządzenia
\item Naturalny Rzemieslnik
\item Szybka Robota
\item Twórca Robotów
\item Twórca Pułapek
\item Twórca Broni
\end{itemize}

\textbf{Średni Poziom}:

\begin{itemize}
\item Sen Staje się Prawdą
\item Ekspert-Rzemieślnik
\item Lodowa Kreacja
\item Warzenie Trucizn
\item Unowocześnienie Robota
\item Rzeźbienie Światłem
\end{itemize}    
    
\textbf{Wysoki Poziom}:

\begin{itemize}
\item Stworzenie
\item Budowla Ciemnej Materii
\item Ulepszone Rzeźbienie Światłem
\item Innowator
\item Naprawa na Oko
\item Zwiększenie Mocy Artefaktu
\item Przebudowa
\end{itemize}

\subsubsection{Leczenie}\index{Zdolności!Kategorie!Leczenie}

Leczy obrażenia, dodaje lub modyfikuje rzuty na odzyskanie zdrowia, lub neguje, leczy, zawiesza lub w inny sposób zapewnia odporność na niebezpieczny efekt lub stan, taki jak trucizna, zaraza, ataki mentalne, spadanie na liczniku obrażeń lub umieranie.

\textbf{Niski Poziom}:

\begin{itemize}
\item Uzdrowienie
\item Kryształowe Ciało
\item Przeznaczenie Wielkości
\item Nurek
\item Wyssanie Istoty
\item Wyssanie Maszyny
\item Wytrzymałość
\item Ucieczka
\item Dodatkowy Rzut na Odzyskanie Zdrowia
\item Negacja zagrożenia
\item Leczący Dotyk
\item Zignorowanie Bólu
\item Ulepszone Odzyskanie Zdrowia
\item Przetrwanie w Dziczy
\item Parcie Dalej
\item Szybkie Odzyskanie Zdrowia
\item Naprawa Ciała
\item Kojąca Obecność
\item Szybkie Zdrowienie
\item Przypływ Pewności Siebie
\item Wyluzowanie
\item Wodna Adaptacja
\item Legendarna Wola
\end{itemize}

\textbf{Średni Poziom}:

\begin{itemize}
\item Wodny Wojownik
\item Biomorficzne Leczenie
\item Przekaz Obrażeń
\item Wyssanie Ładunku
\item Niezłomny
\item Uzdrawiająca Fontanna
\item Leczący Puls
\item Zignorowanie Przeszkody
\item Nieruszony
\item Ulepszone Zdrowie
\item Cudowne Zdrowie
\item Odwaga Szlachcica
\item Jedność z Dziczą
\item Odporność na Trucizny
\item Wyczulone Zmysły
\item Uzdrowienie
\item Przechowanie Energii
\item Uprzednie Planowanie
\item Wyjątkowo Wytrzymały
\item Nieporuszalny
\item Wysuszająca Konsumpcja
\item Dzika Zachęta
\item Chętna Ofiara
\end{itemize}

\textbf{Wysoki Poziom}:

\begin{itemize}
\item Głębokie Rezerwy
\item Ostateczne Zaprzeczenie
\item Mistrz Ruchu
\item Wytrzymałość Gracza
\item Bóg Gier
\item Większy Leczący Dotyk
\item Niemożliwe Odzyskanie Zdrowia
\item Absorpcja Ducha
\item Inspiracja
\item Zainspirowanie Niewinnych
\item Regeneracja Umysłu
\item Większa Negacja Zagrożenia
\item Jeszcze Żywy
\item Szybkie Odzyskanie Zdrowia
\item Regeneracja
\item Przywrócenie Życia
\item Resuscytacja
\item Dzielona Moc
\item Regeneracja Sojuszników
\item Uzdolniony Kierowca
\item Czujny
\end{itemize}

\subsubsection{Środowisko}\index{Zdolności!Kategorie!Środowisko}

Manipuluje środowiskiem lub przedmiotami w nim, np: poprzez telekinezę,  kontrolę pogody, grawitacji, iluzje itp.

\textbf{Niski Poziom}:

\begin{itemize}
\item Stworzenie Wody
\item Iluzja Snów
\item Przywołanie
\item Chwytające Zielska
\item Sztuczki Magiczne
\item Szuflada-Skrytka
\item Dotyk Oświecenia
\item Iluzyjny Duplikat
\item Przyciągnięcie
\item Zwinne Dłonie
\item Kłódka
\item Mniejsza Iluzja
\item Poruszanie Metalu
\item Skrycie się w Cieniu
\item Telekineza
\item Badacz Dziczy
\end {itemize}

\textbf{Średni Poziom}:

\begin{itemize}
\item Sen na Jawie
\item Definiowanie Dołu
\item Pole Grawitacyjne
\item Bariera Pola Siłowego
\item Niszczyciel Pól Siłowych
\item Iluzyjne Ja
\item Zywa Ściana
\item Większa Iluzja
\item Tłumienie Dźwięków
\item Projekcja
\item Przywołanie Burzy
\item Światło Słońca
\end {itemize}

\textbf{Wysoki Poziom}:

\begin{itemize}
\item Adaptacja
\item Kontrola Pogody
\item Diamagnetyzm
\item Ściana Mocy
\item Generacja Pola Siłowego
\item Wielka Iluzja
\item Ściana z Granitu
\item Piekielny Szlak
\item Poruszenie Góry
\item Permanentna Iluzja
\item Relokacja
\item Przerażający Obraz
\item Ściana Błyskawic
\item Przyroda po Twojej Stronie
\end {itemize}

\subsubsection{Informacje}\index{Zdolności!Kategorie!Informacje}

Zapewnia zdolność pozyskania pewnych informacji. Może to być zależne od MG (jak Skan), być zadawaniem pytań na które MG odpowiada, lub być nauką języka.

