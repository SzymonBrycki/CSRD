\chapter{Tworzenie własnej postaci}

Ta sekcja wyjaśnia jak stworzyć postać, którą się będzie odgrywało w Cypher System. Należy podjąć parę decyzji, które wpłyną na postać, tak więc im lepiej rozumiesz postać, którą pragniesz zagrać, tym łatwiejsze będzie tworzenie postaci. W ten proces wlicza się rozumienie wartości trzech statystyk w grze i wybieranie trzech aspektów, które określają zdolności postaci.

\section{Statystyki postaci}\index{Statystyki postaci}

Każda postać gracza posiada trzy statystyki. Te statystyki to Moc, Szybkość i Intelekt. Są to ogólne kategorie które dotyczą wielu różnych, lecz powiązanych aspektów postaci.

\subsection{Moc}\index{Statystyki postaci!Moc}

Moc określa jak silna i wytrzymała jest postać. Koncepty siły, wytrzymałości, kondycji, twardości i zdolności fizycznych – wszystko to zawiera się w tej statystyce. Moc nie jest zależna od rozmiaru, zamiast tego, jest to wartość bezwzględna. Słoń ma więcej Mocy niż najsilniejszy tygrys, który ma więcej mocy niż najmocniejszy szczur, który ma więcej mocy od najmocniejszego pająka.

Rzuca się na Moc, kiedy chce się wyważyć drzwi, wytrzymać wiele dni bez jedzenia lub wyzdrowieć z choroby. To także główny sposób na określenie, jak wiele obrażeń Twoja postać może wytrzymać w niebezpiecznej sytuacji. Fizyczne postaci, twarde postaci, i postaci skupione na walce powinny zainwestować w Moc.

(O Mocy można myśleć jak o Mocy/Zdrowiu, gdyż określa ona jak potężna jest postać i jak wiele obrażeń może wytrzymać).

\subsection{Szybkość}\index{Statystyki postaci!Szybkość}

Szybkość opisuje jak szybka i dobrze fizycznie skoordynowana jest postać. Ta statystyka to szybkość, zdolność ruchu, zręczność i refleks. Rzuca się na szybkość, gdy chce się uniknąć ataku, zakraść gdzieś, lub rzucić trafnie piłką. Pomaga ona określić, czy możesz się poruszyć szybciej w swojej turze. Zręczne, szybkie lub cicho poruszające się postaci będą chciały mieć wysoką Szybkość, jak i te, które głównie atakują broniami dystansowymi. 

(O Szybkości można myśleć jak o Szybkości/Zręczności, gdyż dotyczy ogólnej szybkości i refleksu).

\subsection{Intelekt}\index{Statystyki postaci!Intelekt}

Ta statystyka określa jak bardzo bystra, dobrze wykształcona i lubiana jest postać. Wlicza się w to inteligencja, mądrość, charyzma, edukacja, myślenie krytyczne, bystrość, siła woli i urok osobisty. Rzuca się na Intelekt, gdy chce się rozwiązać łamigłówkę, zapamiętać fakty, opowiedzieć przekonujące kłamstwo i użyć mocy psionicznych. Postaci zainteresowane efektywną komunikacją, byciem uczonymi lub posiadającymi moce nadnaturalne powinny zainwestować w Intelekt.

(O Intelekcie można także myśleć jak o Intelekcie/Osobowości, ponieważ odnosi się zarówno do inteligencji, jak i do charyzmy).

\section{Pule, Skupienie i Wysiłek}\index{Pule, Skupienie i Wysiłek}

Każda z trzech statystyk ma dwie części składowe: Pulę i Skupienie. Pula reprezentuje czystą, wrodzoną zdolność, a Skupienie reprezentuje wiedzę o tym, jak z niej skorzystać. Trzeci element jest powiązany z tymi konceptami: Wysiłek. Kiedy postać naprawdę chce zakończyć rzut sukcesem, stosuje ona Wysiłek.

(Twoje pula statystyk, jak i Wysiłek i Skupienie, są zależne od typu postaci, deskryptora i specjalizacji, które sam wybierasz. Masz jednak w tym zakresie bardzo wielką dowolność.)


\subsection{Pula}\index{Pule, Skupienie i Wysiłek!Pula}

Twoja Pula to najbardziej podstawowy komponent statystyki. Porównanie Pul obydwu istot da ci ogólną informacje, która z nich jest lepsza w danej statystyce. Dla przykładu, postać z Pulą Mocy 16 jest silniejsza (ogólnie mówiąc) niż postać z Pulą Mocy 12. Większość postaci zaczyna grę z Pulą od 9 do 12 w większości statystyk – jest to wartość zwykłego, szarego człowieka. 

Kiedy Twoja postać jest zraniona, chora lub zaatakowana, tymczasowo tracisz punkty z jednej ze swoich Pul. Natura ataku określa, z której Puli odejmowane są punkty. Dla przykładu, fizyczny atak mieczem redukuje Pulę Mocy, trucizna, która odbiera zręczność redukuje Szybkość, a pioniczny atak redukuje Intelekt. Możesz także wydać punkty z jednej z Pul, by obniżyć trudność zadania (patrze: Wysiłek, poniżej). Możesz odpocząć, aby \mytext{odzyskać utracone punkty Pul}, i  pewne specjalne zdolności lub cyphery mogą zezwolić na odzyskanie utraconych punktów szybciej.


\subsection{Skupienie}\index{Pule, Skupienie i Wysiłek!Skupienie}

Choć Pule są podstawowym miernikiem statystyk, Twoja Skupienie jest także ważne. Kiedy coś wymaga, abyś zapłacił punktami z Puli, Twoje Skupienie redukuje ten koszt. Redukuje ono także koszt stosowania Wysiłku z danej Puli.

Dla przykładu, powiedzmy, że masz umiejętność psionicznego ataku, której aktywacja kosztuje 1 punkt z Puli Intelektu. Odejmij swoje Skupienie w Intelekcie od kosztu aktywacji, a wynik określa ile płacisz punktów z Puli, by wykorzystać psioniczny atak. Jeśli Twoje Skupienie zredukuje koszt do 0, możesz korzystać z tej zdolności za darmo.

Twoje Skupienie może być inne dla każdej statystyki. Dla przykładu, możesz mieć Skupienie w Mocy na 1, Skupienie w Szybkości na 1 i i Skupienie w Intelekcie na 0. Zawsze będziesz miał Skupienie przynajmniej w jednej statystyce. Twoje Skupienie w niej redukuje punkty wydawane z Puli tej statystyki, ale nie z innych Pul. Twoje Skupienie w Mocy redukuje koszty związane z Pulą Mocy, ale nie wpływa na Pule Szybkości bądź Intelektu. Kiedy Skupienie w statystyce sięga 3, możesz zawsze stosować jeden poziom Wysiłku za darmo.

Postać, która ma niską Pulę Mocy, ale wysokie Skupienie w Mocy, ma potencjał, by wykonywać akcje Mocy lepiej niż postać, która ma Skupienie w Mocy na 0. Wysokie Skupienie pozwoli jej zredukować koszt punktów z Puli, co znaczy, że mają więcej punktów na stosowanie Wysiłku.

\subsection{Wysiłek}\index{Pule, Skupienie i Wysiłek!Wysiłek}

Kiedy Twoja postać naprawdę pragnie ukończyć zadanie sukcesem, może zastosować Wysiłek. Dla początkującej postaci, stosowanie Wysiłku wymaga wydania 3 punktów z Puli statystyki stosownej do akcji. Tak więc, jeśli Twoja postać pragnie uniknąć ataku (Pula Szybkości) i chcesz zwiększyć szanse na sukces, możesz zastosować Wysiłek, płacąc 3 punktami z Puli Szybkości. Wysiłek ułatwia zadanie o jeden stopień. Inaczej, nazywa się to zastosowaniem jednego poziomu Wysiłku.

Nie musisz stosować Wysiłku, jeśli tego nie chcesz. Jeśli wybierzesz zastosowanie Wysiłku do zadania, musisz to zrobić zanim zdecydujesz się na rzut – nie możesz najpierw rzucić, a potem zadecydować o Wysiłku, jeśli uzyskałeś słaby rzut. 

Stosowanie większego Wysiłku może obniżyć trudność zadania jeszcze dalej – każdy dodatkowy poziom Wysiłku ułatwia zadanie o jeden stopień. Zastosowanie jednego poziomu Wysiłku obniża trudność o jeden stopień, dwóch poziomów Wysiłku – o 2 stopnie itp. Jednakże, każdy dodatkowy poziom Wysiłku po pierwszym kosztuje 2 punkty z Puli statystyki zamiast 3. Tak więc zastosowanie dwóch poziomów Wysiłku kosztuje 5 punktów (3 za 1-szy poziom plus 2 za 2-gi), zastosowanie trzech poziomów Wysiłku kosztuje 7 punktów z Puli (3 plus 2 plus 2) itp.

Każda postać posiada statystykę zwaną Wysiłek, która określa maksymalny poziom Wysiłku, który dana postać może zastosować. Początkująca (1-szo poziomowa) postać ma Wysiłek 1, co oznacza, że może zastosować 1 poziom Wysiłku w rzucie. Bardziej doświadczone postaci mają wyższy Wysiłek i mogą stosować więcej poziomów Wysiłku. Dla przykładu, postać, której Wysiłek wynosi 3 może zastosować 3 poziomy Wysiłku, by zredukować trudność rzutu.

Kiedy stosujesz Wysiłek, odejmij swoje Skupienie w odpowiedniej statystyce od całościowego kosztu Wysiłku w punktach z Puli. Dla przykładu, wykonujesz rzut na Szybkość. By zwiększyć szansę na sukces, decydujesz się zastosować 1 poziom Wysiłku, co ułatwi zadanie. Normalnie, kosztowałoby to 3 punkty z Puli Szybkości. Jednakże, masz Skupienie w Szybkości na 2, więc odejmujesz tą liczbę od kosztu. W efekcie, Wysiłek kosztuje Cię tylko 1 punkt z Twojej Puli Szybkości.

Co, gdybyś zastosował dwa poziomy Wysiłku do rzutu, zamiast tylko jednego? To by ułatwiło zadanie o dwa stopnie. Normalnie, kosztowałoby to 5 punktów z Puli, ale po odjęciu Skupienia w Szybkości o wartości 2, finalny koszt to tylko 3 punkty.

Kiedy Skupienia w statystyce osiąga 3, możesz stosować jeden poziom wysiłku za darmo. Dla przykładu, jeśli masz Skupienia w Szybkości na 3 i stosujesz 1 poziom Wysiłku na rzucie na Szybkość, będzie Cię to kosztowało 0 punktów z Twojej Puli Szybkości. (Normalnie, jeden poziom Wysiłku kosztuje 3 punkty, ale po odjęciu Skupienia w Szybkości od tego numeru, redukujemy je do 0.).

Umiejętności i inne przewagi także ułatwiają zadania, i można z nich skorzystać razem z Wysiłkiem. Dodatkowo, Twoja postać może mieć specjalne zdolności lub ekwipunek, które mogą pozwolić Ci wykorzystać Wysiłek do specjalnych zadań, takich jak przewrócenie przeciwnika przy pomocy ataku lub wpłynięcie na wiele celów przy pomocy mocy, która zazwyczaj dotyczy tylko jednej osoby.

(Kiedy stosujesz Wysiłek w walce wręcz, masz opcję wydania punktów albo z Puli Szybkości, albo z Puli Mocy. Kiedy wykonujesz atak dystansowy, możesz wydać punkty tylko z Puli Szybkości. Ta zasada odzwierciedla fakt, że w walce wręcz czasem korzysta się z brutalnej siły, a czasami z finezji, ale w atakach dystansowych zawsze chodzi o dobre wycelowanie.)

\subsection{Wysiłek i obrażenia}\index{Pule, Skupienie i Wysiłek!Wysiłek i obrażenia}

Zamiast stosować Wysiłek, by ułatwić atak, można go zastosować, by zwiększyć obrażenia zadawane w tym ataku. Na każdy poziom Wysiłku, który się stosuje w ten sposób, zadaje się dodatkowe 3 punkty obrażeń. To działa dla każdego rodzaju ataku, który zadaje obrażenia, niezależnie od tego, czy to miecz, kusza, psioniczny atak czy coś jeszcze innego.

Kiedy korzystasz z Wysiłku, by zwiększyć obrażenia ataku obszarowego, takiego jak eksplozji wywołanej przez zdolność Adepta \mytext{Wybuch}, zadajesz dodatkowe 2 punkty obrażeń zamiast 3. Jednakże, dodatkowe punkty są zadawane wszystkim celom w obszarze działania zdolności. Dodatkowo, nawet jeśli jeden lub więcej celów nie ponosi obrażeń w wyniku tego konkretnego ataku (ze względu na nieudany rzut na atak), dalej otrzymują oni 1 punkt obrażeń.

\subsection{Wiele użyć Wysiłku i Skupienia}\index{Pule, Wysiłek i Skupienie!Wiele użyć Wysiłku i Skupienia}

Jeśli Twój Wysiłek wynosi 2 lub więcej, możesz zastosować Wysiłek na wiele sposobów w jednej akcji. Dla przykładu, jeśli wykonujesz atak, możesz zastosować Wysiłek, by ułatwić atak i by zadać więcej obrażeń.

Totalny Wysiłek, z którego korzystasz, nie może być większy od Twojej wartości Wysiłku. Dla przykładu, jeśli Twój wysiłek to 2, możesz zastosować dwa poziomy Wysiłku. Możesz wykorzystać jeden z nich, by ułatwić atak, a drugi, by zwiększyć jego obrażenia, by ułatwić atak o dwa stopnie, ale nie zwiększać obrażeń, lub by nie ułatwiać ataku, ale zwiększyć obrażenia dwukrotnie.

Możesz wykorzystać Skupienie danej statystyki tylko jeden raz na akcję. Dla przykładu, jeśli stosujesz Wysiłek na ataku Mocy i zwiększasz obrażenia oraz ułatwiasz cios, możesz skorzystać z Skupienia w Mocy, by obniżyć koszt jednego z tych zastosować Wysiłku, ale nie dwóch. Jeśli wydasz 1 punkt Intelektu na aktywowanie ataku psionicznego i jeden poziom Wysiłku na ułatwienie ataku, możesz skorzystać z Skupienia w Intelekcie do jednej z tych rzeczy, ale nie obydwu.

\section{Przykładowe Statystyki}

Początkująca postać walczy z wielkim szczurem. BG rzuca się ze swoją włócznią na tego szczura, który jest istotą 2 poziomu i w związku z tym jego Stopień Trudności wynosi 6. Postać stoi wyżej od szczura i atakuje go z góry i MG uznaje, że to dobra taktyka i przyznaje atut który ułatwia atak o jeden stopień (trudność wynosi teraz 1). To obniża Stopień Trudności do 3. Atak włócznią bazuje na Mocy – postać ma Pulę Mocy 11 i Skupienie w Mocy 0. Przed wykonaniem rzutu, decyduje się ona zastosować poziom Wysiłku, by ułatwić atak. To kosztuje 3 punkty z Puli Mocy, redukując jej obecną wartość do 8. Ale te punkty są dobrze wydane. Obniża to trudność z 1 do 0, więc nie ma potrzeby, by wykonać rzut – atak automatycznie trafia.  

Inna postać chce przekonać strażnika, by pozwolił jej wejść do prywatnego biura w celu rozmowy z ważnym szlachcicem. MG oznajmia, że jest to akcja Intelektu. Postać jest na 3 poziomie i ma Wysiłek 3, Pulę Intelektu 13 i Skupienie w Intelekcie 1. Przed wykonaniem rzutu, gracz musi zadecydować, czy stosuje Wysiłek. Może on zastosować 1, 2 lub 3 poziomy Wysiłku, lub też nie zastosować żadnego. Ta akcja jest dla niego ważna, więc decyduje on się na zastosowanie 2 poziomów Wysiłku, ułatwiając zadanie o 2 stopnie. Dzięki Skupieniu w Intelekcie, płaci on tylko 4 punkty z Puli Intelektu (3 punkty za pierwszy poziom Wysiłku, plus 2 punkty za drugi poziom, minus 1 z Skupienia). Wydanie tych punktów redukuje jego Pulę Intelektu do 9. MG uznaje, że przekonanie strażnika jest zadaniem poziomu 3 (wymagającym) z Stopniem Trudności 9; dwa poziomy Wysiłku obniżają trudność zadania do poziomu trudności 1 (łatwe) a Stopień Trudności do 3. Gracz rzuca k20 i otrzymuje 8. Ponieważ ten wynik to co najmniej Stopień Trudności zadania, postać odnosi sukces. Jednakże, gdyby nie zastosowała ona żadnego Wysiłku, odniosłaby porażkę, ponieważ jej rzut (8) byłby mniejszy niż Stopień Trudności (9).

\section{Poziomy postaci}\index{Poziomy postaci}

Każda postać zaczyna grę na 1-szym poziomie. Poziom mierzy moc, wytrzymałość i zdolności postaci. Postaci awansują do 6 poziomu. Gdy postać osiąga wyższe poziomy, uzyskuje ona więcej zdolności, zwiększa swój Wysiłek, i może zwiększyć Skupienie lub liczbę punktów w Pulach. Ogolnie mówiąc, nawet postaci na 1-szym poziomie są całkiem nieźle uzdolnione. Można spokojnie założyć, że mająjuż jakieś doświadczenie. To nie jest sytuacja “od zera do bohatera”.ale raczej przykład kompetentnych ludzi polepszających swoje zdolności i wiedzę. Awansowanie na wyższe poziomy nie jest tak naprawdę celem postaci w Cypher System, lecz raczej reprezentuje to, jak postać zmienia siew czasie przygód.

Aby awansować na następny poziom, postaci muszą zyskać Punkty Doświadczenia przez wypełnianie celów postaci, uczestniczenie w przygodach i odkrywanie nowych rzeczy – ten system traktuje o zarówno odkryciach, jak i eksploracji, a także o osiąganiu osobistych celów. Punkty doświadczenia mają wiele zastosowań, a jedno z nich to zakupywanie korzyści dla postaci. Po zakupie czterech korzyści, postać awansuje na następny poziom. Każda korzyść kosztuje 4 PD-ki i można je kupować w dowolnej kolejności, ale trzeba zakupić każdy z nich (a następnie awansować na następny poziom) zanim można zakupić tę samą korzyść ponownie. Cztery korzyści postaci to:

\begin{itemize}
    \item Zwiększenie Zdolności: Dodaj 4 do swoich Pul. Możesz wybrać dowolną ilość z tych punktów na dowolne Pule.
    \item Zbliżanie się do Doskonałości: Zwiększ o 1 Skupienie w Mocy, Szybkości bądź Intelekcie (ty decydujesz).
    \item Dodatkowy Wysiłek: Zwiększ swoją wartość Wysiłku o 1.
    \item Umiejętności: Uzyskujesz trening w jednej umiejętności swojego wyboru, innej niż atak lub obrona. 
\end{itemize}    

Jak zapisano w Zasadach Gry, postać wytrenowana w danej umiejętności ułatwia powiązane z nią zadania o jeden stopień. Możesz wybrać dowolną umiejętność, której sobie zażyczysz, taką jak wspinaczka, skakanie, perswazja lub skradanie się. Możesz także wybrać jakąś dziedzinę wiedzy, taką jak historia lub geologia. Możesz nawet wybrać umiejętność bazującą na specjalnych zdolnościach swojej postaci. Dla przykładu, jeśli Twoja postać może uderzyć w przeciwnika mocą mentalną, możesz być wytrenowany w korzystaniu z tej zdolności, ułatwiając zadanie korzystania z niej. Jeśli wybierzesz umiejętność, w której już jesteś wytrenowany, zyskujesz w niej specjalizację, ułatwiając związane z nią rzuty o dwa stopnie zamiast jednego.

(Umiejętności to szeroka kategoria rzeczy, których postać może sienauczyć i wykonać. Patrz poniżej po przykładową listę umiejętności).

\begin{itemize}
        \item Inne Opcje: Gracze mogą także wydać 4 PD-ki na zakupienie innych opcji, zamiast zyskać nową umiejętność. Zakupienie dowolnej opcji z poniższej listy liczy się jak Umiejętność celem awansowania na następny poziom. Opcje speclajne to:
        \item Obniżenie kosztu noszenia zbroi. Ta opcja obniża koszt noszenia zbroi o 1.
        \item Dodaj 2 do swoich rzutów na odzyskiwanie zdrowia. 
        \item Wybierz nową zdolność zapewnianą przez swój typ, z obecnego poziomu lub niższego.
\end{itemize}

\section{Deskryptory, Type i Specjalizacje postaci}

By stworzyć postać, tworzysz proste zdanie, które ją opisuje. To zdanie przyjmuje następującą formę: “Jestem [umieść tutaj przymiotnik] [umieść tutaj rzeczownik] który [umieść tutaj czasownik].”

Tak więc powstaje zdanie “Jestem przymiotnik opisujący rzeczownik który czasownikuje”. Dla przykładu, możesz powiedzieć “Jestem Dzikim Wojownikiem który Kontroluje Bestie” lub “Jestem Czarującym Poszukiwaczem, który Stawia Umysł Ponad Materią”. 

W tym zdaniu, przymiotnik nazywany jest \mytext{deskryptorem}.

Rzeczownik to \mytext{typ} twojej postaci.

Czasownik jest nazywany \mytext{specjalizacją}.

Pomimo tego, że typ postaci znajduje się w środku zdania, to od niego zaczniemy. (Tak jak w zdaniu, rzeczownik jest podstawą).
Twój typ postaci to jądro twojej postaci. W niektórych grach fabularnych, można nim nazwać klasę postaci. Twój typ pozwala Ci określić miejsce Twojej postaci w świecie i jej relacje z innymi ludźmi w settingu. To jest rzeczownik w zdaniu “Jestem przymiotnik opisujący rzeczownik który czasownikuje”.

Możesz wybrać z czterech typów postaci – \mytext{Wojownika, Adepta, Poszukiwacza} lub \mytext{Mówcy}.

Twój deskryptor definiuje postać – wpływa na wszystko, co robisz. Twój deskryptor umieszcza Twoją postać w pewnej sytuacji (pierwszej przygodzie na początku kampanii) i pomaga zapewnić jej motywację. To przymiotnik w zdaniu “Jestem przymiotnik opisujący rzeczownik który czasownikuje”.
Jeśli Twój MG nie powie inaczej, możesz wybrać dowolny z deskryptorów postaci.

Specjalizacja to to, co Twoja postać robi najlepiej. Specjalizacja nadaje Twojej postaci specyficzność i zapewnia interesujące nowe zdolności, które mogą się przydać. Twoja Specjalizacja także pomaga Ci zrozumieć Twoje miejsce w grupie BG. Jest to czasownik w zdaniu “Jestem przymiotnik opisujący rzeczownik który czasownikuje”.

Istnieje wiele specjalizacji postaci. Twój wybór zależeć będzie zapewne od settingu i opowiadanej historii.

(Możesz wykorzystać Smaczki z odpowiedniego rozdziału, żeby zmodyfikować typy postaci tak, by pasowały do odmiennych sesji.)

\section{Specjalne Zdolności}\index{Zdolności!Wstęp}

Typ postaci i specjalizacja zapewniają BG specjalne zdolności na każdym nowym poziomie. Korzystanie z tych zdolności zazwyczaj kosztuje punkty z Puli statystyk; koszt podano w nawiasie po nazwie zdolności. Twoje Skupienie w odpowiedniej statystyce może obniżyć jej koszt, ale pamiętaj, że możesz stosować Skupienie tylko raz na akcję. Dla przykładu, powiedzmy, że Adept z Skupieniem w Intelekcie 2 chce skorzystać z zdolności Blast, aby stworzyć ładunek energii, co kosztuje 1 punkt Intelektu. Chce on także zwiększyć obrażenia z ataku korzystając z Wysiłku, co kosztuje 3 punkty Intelektu. Całościowy koszt tej akcji wynosi 2 punkty z Puli Intelektu (1 punkt za pocisk, plus 3 punkty za skorzystanie z Wysiłku, minus 2 punkty z Skupienia).

Czasami koszt umiejętności ma znak plusa (+) po liczbie. Dla przykładu, koszt może zostać podany jako “2+ punktów Intelektu”. To oznacza, że można wydać więcej punktów lub więcej poziomów Wysiłku, by ulepszyć zdolność, co wyjaśniono w jej opisie.

Wiele specjalnych zdolności daje postaci opcję zrobienia czegoś, czego normalnie nie mogłaby wykonać, jak np.: wytwarzanie promieni zimna lub atakowanie wielu celów naraz. Używanie takich zdolności jest akcją samą w sobie, a koniec opisu zdolności zawiera słowo “Akcja” aby o tym przypomnieć. Może on także zapewnić więcej informacji o tym kiedy lub jak wykonać ową akcję. 

Pewne specjalne zdolności pozwalają Ci wykonać znaną już akcję – akcję, którą można wykonać i bez tego – w odmienny sposób. Dla przykładu, zdolność może Ci pozwolić nosić ciężką zbroję, obniżyć trudność rzutów obronnych na Szybkość, lub dodać 2 punkty ognistych obrażeń do Twoich obrażeń od broni. Te zdolności nazywa się umożliwiaczami. Korzystanie z nich nie jest uważane za akcję. Umożliwiacze albo działają ciągle (np.: możliwość noszenia ciężkiej zbroi, co nie jest akcją) lub dzieją się jako część innej akcji (np.: dodawanie obrażeń od ognia do Twoich ataków, co jest częścią akcji ataku). Jeśli specjalna zdolność jest umożliwiaczem, na końcu jej opisu znajduje się słowo “Umożliwiacz” aby o tym przypomnieć.

Pewne zdolności określają swoją długość, ale zawsze możesz zakończyć wcześniej dowolną ze swoich zdolności, jeśli tylko sobie tego życzysz.
(Ponieważ Cypher System to uniwersalny system i dotyczy wielu gatunków, nie zawsze wszystkie deksryptory, typy i specjalizacje mogą być dostępne graczom. MG decyduje co jest dostępne w tej konkretnej grze i czy coś jest zmodyfikowane – poinformuje on o tym swoich graczy.)


\section{Umiejętności}\index{Umiejętności}

Czasami Twoja postać zyskuje trening w specyficznej umiejętności lub zadaniu. Przykładowo, Twoja specjalizacja może oznaczać, że jesteś wytrenowany w skradaniu się, wspinaczce i skakaniu, lub społecznych interakcjach. Innym razem, Twoja postać może wybrać umiejętność, w której jest wytrenowana, i wybierasz taką umiejętność, o której myślisz, że może się przydać w przyszłości.

Cypher System nie ma definitywnej, danej raz na zawsze listy umiejętności. Jednakże, poniżej jest trochę sugestii:

\begin{itemize}
     \item  Astronomia
    \item Utrzymywanie równowagi
    \item Biologia
    \item Botanika
    \item Noszenie ciężarów
    \item Wspinaczka
    \item Komputery
    \item Kłamstwo
    \item Przebrania
    \item Ucieczka
    \item Geografia
    \item Geologia
    \item Leczenie
    \item Historia
    \item Identyfikacja
    \item Inicjatywa
    \item Zastraszanie
    \item Skakanie
    \item Obrabianie skóry
    \item Otwieranie zamków
    \item Maszyny
    \item Kowalstwo
    \item Percepcja
    \item Perswazja
    \item Filozofia
    \item Fizyka
    \item Kradzież kieszonkowa
    \item Pilotowanie
    \item Naprawy
    \item Jeździectwo
    \item Niszczenie
    \item Skradanie się
    \item Pływanie
    \item Prowadzenie pojazdów
    \item Obróbka drewna
 \end{itemize}  
   
Możesz wybrać umiejętność, która obejmuje parę z tych aktywności (interakcje społeczne mogą pokrywać oszustwo, zastraszanie i perswazję) lub bardziej wąskie (ukrywanie sięjako skradanie się, gdy się nie rusza). Możesz też wymyślić umiejętności-profesje, takie jak piekarz, marynarz lub drwal. Jeśli chcesz wybrać umiejętność, której nie ma na liście, najpewniej najlepiej zapytać najpierw swojego MG, ale ogólnie, najważniejsze to taki wybór umiejętności, który pasuje do konceptu postaci.

Pamiętaj, że jeśli zyskujesz umiejętność, w której już jesteś wytrenowany, zostajesz w niej wyspecjalizowany. Ponieważ opisu umiejętności może być nieco mylny, stwierdzenie ,czy jesteś wytrenowany czy wyspecjalizowany może zająć nieco czasu. Dla przykładu, możesz być wytrenowany w kłamaniu,a poźniej dostać umiejętnośc do wszystkich społecznych interakcji, co oznacza, że Twoje kłamstwa są wyspecjalizowane, a inne aktywności społeczne są tylko wytrenowane. Bycie wytrenowanym 3 razy w umiejętności nie jest lepsze niż bycie wytrenowanym tylko 2 razy (innymi słowy, wyspecjalizowany to najlepszy możliwy poziom).

Tylko umiejętności pozyskane przez typ lub inne rzadkie przypadki pozwalają Ci być wyuczonym w ataku lub obronie.

Jeśli pozyskasz specjalną zdolność przez swój typ, specjalizację lub inny aspekt Twojej postaci, możesz wybrać ką w miejsce umiejętności i być wytrenowanym lub wyspecjalizowanym w danej specjalniech zdolności. Dla przykładu, jeśli masz atak mentalny, a nadchodzi czas na wybór umiejętności, możesz wybrać umiejętność w ataku mentalnym. Ułatwiłoby to ten atak za każdym razem, gdy jest używany. Każda zdolność, którą posiadasz, może mieć swoją własną umiejętność w tym właśnie celu. Nie możesz wybrać “wszystkich mocy mentalnych” lub “wszystkich zaklęć” jako jednej umiejętności i być w niej wytrenowany lub wyspecjalizowany, gdyż ta kategoria jest stanowczo zbyt szeroka.

W większości kampanii, biegłość w języku jest uważana za umiejętność. Wiec jeśli chcesz mówić po francusku, jest to traktowane tak samo jak bycie wytrenowanym w biologii lub pływaniu.