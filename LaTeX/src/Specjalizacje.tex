\section{Specjalizacje}\index{Specjalizacje}

Specjalizacja czyni postać wyjątkową. Specjalziacje postaci w grupie nie powinny się powtarzać. Specjalizacja daje postaci korzyści podczas tworzenia postaci i z każdym następnym poziomem. Jest to czasownik w zdaniu ``Jestem przymiotnikiem rzeczownikiem który czasownikuje''.

Ten rozdział zawiera około 100 przykładowych specjalizacji, takich jak Nosi Halo Ognia, Wolałby Czytać i Pilotuje Statki Kosmiczne. Te specjalizacje mogą być wybrane i użyte przez gracza, lub przez MG, który dodaje je do listy dostępnych specjalizacji dla swoich graczy w następnej kampanii. 

Dodatkowo, dalsza część rozdziału zapewnia narzędzia dla MG, który chciałby stworzyć swoją własną specjalizację, tak, by pasowała do wymogów danej gry lub kampanii.

\subsection{Wybieranie specjalizacji}\index{Specjalizacje!Wybieranie specjalizacji}

Nie wszystkie specjalizacje pasują do każdej konwencji. Rozdział Konwencja zapewnia porady i pomoc, ale tutaj pójdą pewne uogólnienia. Oczywiście, MG może zezwolić na każdą specjalizację w swoim settingu. Specjalizacje są ważną jego częścią, gdyż np.: obecność na liście dozwolonych specjalizacji Włada Mocami Mentalnymi oznacza, że w tym świecie istnieją moce psioniczne, Wyje do Księżyca oznacza istnienie w nim likantropów, a Pilotuje Statki Kosmiczne sprawia, że w settingu, no cóż, istnieją statki kosmiczne. 

Kiedy wybiera się specjalizację dla postaci, otrzymuje ona specjalne połączenie z jedną lub więcej innych postaci graczy, zdolność pierwszego poziomu i być może dodatkowy ekwipunek, niezbędny, by korzystać z mocy zapewnianych przez specjalizację, lub który dobrze z nią współgra. Dla przykładu, postać będąca rzemieślnikiem może potrzebować wielu narzędzi. Postać, która ciągle płonie, potrzebuje ubrań odpornych na ogień. Postać rysująca magiczne runy może potrzebować pędzelka i farm. Postać zabijająca potwory mieczem potrzebuje miecza. I tak dalej. Jednakże, wiele specjalizacji nie wymaga dodatkowego ekwipunku. Każda specjalizacja oferuje także jedne lub dwie sugestie na Wtrącenia MG z listą możliwych konsekwencji naprawdę dobrych i złych rzutów kością.  
 
Parę specjalizacji w tym rozdziale zapewnia “zamianę z typem”, która pozwala zamienić zdolność typu na zdolność specjalizacji. Gracz nie musi poczynić tej zamiany, ale ma taką możliwość. Dla przykładu, specjalizacja Kocha Pustkę zapewnia opcję zyskania zdolności Mam Kombinezon Kosmiczny, Będę Podróżnikiem zamiast zdolności typu.

W miarę, jak postać zyskuje nowe poziomy, specjalizacja daje więcej zdolności. Każdy poziom jest zazwyczaj oznaczony jako Akcja lub Umożliwienie, czyni inne akcje lepszymi lub zapewnia jakieś inne korzyści, ale nie akcję. Zdolność, która pozwala bohaterowi razić wrogów laserami jest Akcją. Zdolność, która daje więcej dodatkowych obrażeń, kiedy się wykonuje akcję, jest Umożliwieniem. Umożliwienie jest wykorzystywane w tej samej turze co inna akcja, i często jest częścią innej akcji. Korzyści każdego poziomu są niezależne i kumulują się z korzyściami z innych poziomów (chyba, że zaznaczono inaczej). Tak więc jeśli zdolność pierwszego poziomu daje +1 do Pancerza, a zdolność czwartego poziomu także daje +1 do Pancerza, to postać na czwartym poziomie ma w sumie pancerz na +2.
Na poziomach trzecim i szóstym, postać może wybrać zdolność z dwóch opcji.

Możesz także wybrać, czy chcesz rozwinąć historie będącą opisem danej specjalizacji (choć nie jest to wymagane).

\subsection{Połączenia z innymi BG}\index{Specjalizacje!Połączenia z innymi BG}

Wybierz połączenie, które pasuje do specjalności. Jeśli jesteś MG wybierającym (lub tworzącym) jeden lub kilka specjalizacji dla swoich postaci, wybierz do 4 z poniższych połączeń.

\begin{itemize}
\item Wybierz innego BG. Z nieznanych Tobie powodów, ta postać jest kompletnie odporna na Twoje zdolności specjalizacji, niezależnie od tego, czy pragniesz jej nimi pomóc, czy zagrozić.
\item Wybierz innego BG. Wiedziałeś o jego istnieniu przez lata, ale nie sądziłeś, że on Ciebie znał.
\item Wybierz innego BG. Zawsze chcesz mu zaimponować, ale sam nie wiesz czemu.
\item Wybierz innego BG. Ta postać ma nawyk, który Cię wkurza, ale poza tym jesteś pod wrażeniem jej zdolności.
\item Wybierz innego BG. Ta postać posiada potencjał w okiełznaniu Twojego stylu walki, paradygmatu lub innej zdolności zapewnianej przez Twoją specjalizację. Chciałbyś ją potrenować, ale nie jesteś dobrze przygotowany do nauczania (być może) a ona może nie być zainteresowana (znowu: być może).
\item Wybierz innego BG. Jeśli znajduje się on w bliski zasięgu kiedy walczysz, czasami zapewnia on atut, a czasami utrudnia Twój test na atak (szansa 50% na jedno lub drugie, rzucane raz na walkę).
\item Wybierz innego BG. Kiedyś ocaliłeś mu życie i teraz ma dług wdzięczności. Nie jesteś z tego jakoś bardzo zachwycony – zrobiłeś, co należało zrobić i tyle.
\item Wybierz innego BG. Ta postać ostatnio Cię wyśmiała, co naprawdę Cię zraniło. Jak zamierzasz sobie z tym poradzić (jeśli w ogóle) zależy od Ciebie.
\item Wybierz innego BG. Ta postać wie, że cierpiałeś z powodu robotów w przeszłości. To, czy nienawidzisz robotów, zależy od Ciebie, co może wpłynąć na Twoją relację z tym BG, jeśli jest on przyjacielem robotów lub posiada robotyczne protezy.
\item Wybierz innego BG. Ta postać pochodzi z tego samego miejsca co ty i znaliście się jako dzieci. 
\item Wybierz innego BG. W przeszłości, nauczył on Cię paru trików dy wykorzystania w walce.
\item Wybierz innego BG. Ta postać nie pochwala Twoich metod.
\item Wybierz innego BG. Dawno temu, byliście po przeciwnych stronach barykady w starciu. Wygrałeś, choć w jego oczach “oszukiwałeś” (ale z Twojej perspektywy wszystko jest ok). Może on chcieć ponownego starcia, choć to zależy od niego.
\item Wybierz innego BG. Zawsze próbujesz zachwycić tą postać swoimi umiejętnościami, sprytem, wyglądem lub odwagą. Może jest ona Twoim rywalem, może pragniesz jej szacunku, a może jest ona Twoim obiektem westchnień.
\item Wybierz innego BG. Boisz się, że jest on zazdrosny o Twoje zdolności i martwisz się, że może to doprowadzić do problemów. 
\item Wybierz innego BG. Przypadkowo został złapany w pułapkę, którą założyłeś i musiał się wydostać z niej o własnych siłach.
\item Wybierz innego BG. Kiedyś zostałeś zatrudniony, by wyśledzić kogoś, kto był blisko tej postaci.
\item Wybierz dwóch BG (najlepiej takich, którzy mogą znaleźć się na trajektorii Twoich ataków). Kiedy nie trafiasz atakiem i MG decyduje, że atak uderzył w kogoś innego niż w Twój cel, trafia on w jedną tych dwóch postaci.
\item Wybierz jednego BG. Nie jesteś pewien jak ani skąd, ale ta postać posiada butelki rzadkiego alkoholu i może go dla Ciebie sprowadzić za pół ceny.
\item Wybierz jednego BG. Ostatnio straciłeś posiadany przedmiot i przekonałeś samego siebie, że to ten BG go skradł. Czy tak jest w istocie zależy od niego.
\item Wybierz jednego BG. On zawsze zdaje się wiedzieć, gdzie jesteś, lub przynajmniej w którym kierunku się znajdujesz w stosunku do niego.
\item Wybierz jednego BG. Patrzenie jak używasz swoich zdolności specjalizacyjnych wydaje się budzić w nim nieprzyjemne wspomnienia. To wspomnienie leży w gestii tego BG, choć może on nie być w stanie przywołać je na poziomie świadomości.
\item Wybierz jednego BG. Coś sprawia, że jego obecność przeszkadza w Twoich zdolnościach. Kiedy stoi on obok Ciebie, Twoje zdolności specjalizacyjne kosztują 1 dodatkowy punkt więcej.
\item Wybierz jednego BG. Coś sprawia, że uzupełnia on Twoje zdolności. Kiedy stoi on obok Ciebie, pierwsza zdolność specjalizacji, z której korzystasz w ciągu danej doby, kosztuje 2 punkty mniej.
\item Wybierz jednego BG. Znasz tą postać już jakiś czas, i pomogła ona Ci zyskać kontrolę nad Twoimi zdolnościami specjalizacyjnymi.
\item Wybierz jednego BG. Kiedyś w przeszłości tej postaci, miała ona dewastujące doświadczenie, kiedy próbowała zrobić coś, co Tobie przychodzi łatwo dzięki Twojej specjalizacji. To od niej zależy, czy powiedziała Ci o tym.
\item Wybierz jednego BG. Jego okazjonalna niezdarność i głośne zachowania irytują Cię.
\item Wybierz jednego BG. W niedalekiej przeszłości, gdy trenowaliście, przypadkowo trafiłeś go swoim atakiem, poważnie go raniąc. To od niego zależy, czy żywi urazę, czy też może wybaczył tobie.
\item Wybierz jednego BG. Wisi Ci on dużą sumę pieniędzy.
\item Wybierz jednego BG. W niedalekiej przeszłości, kiedy uciekaliście od jakiegoś zagrożenia, przypadkowo zostawiłeś tą postać z tyłu, by dała sobie radę sama. Przetrwała ona, ale ledwie. Od gracza tamtego BG zależy, czy jego postać dalej jest zła, czy może Ci wybaczyła.
\item Wybierz jednego BG. Niedawno, przypadkowo (lub celowo) sprawił on, że znalazłeś się w niebezpieczeństwie. Wszystko z Tobą teraz w porządku, ale masz się na baczności w jego obecności.
\item Wybierz jednego BG. Z Twojego punktu widzenia, wydaje się on nerwowy w związku z pewną ideą, osobą lub sytuacją. Chciałbyś nauczyć go jak być bardziej wyluzowanym (jeśli tylko Ci na to pozwoli).
\item Wybierz jednego BG. Kiedyś nazwał Cię on tchórzem.
\item Wybierz jednego BG. Ta postać zawsze rozpoznaje Cię i Twoje ślady, nawet kiedy jesteś w przebraniu lub uciekłeś z danego miejsca dawno temu.
\item Wybierz jednego BG. Niechcący spowodowałeś wypadek, który sprawił, że zapadł on w sen tak głęboki, że nie obudził się przez trzy dni. To, czy Ci wybaczył, czy też nie, zależy od niego.
\item Wybierz jednego BG. Jesteś przekonany, że jesteście w jakiś sposób spokrewnieni.
\item Wybierz jednego BG. Przypadkowo dowiedziałeś się czegoś, co on próbuje utrzymać w tajemnicy. 
\item Wybierz jednego BG. Jest on szczególnie wrażliwy na co bardziej widocznie zdolności Twojej specjalizacji, i czasami doznaje szoku trwającego parę rund, co utrudnia jego akcje. 
\item Wybierz jednego BG. Najwyraźniej posiada on cenny przedmiot, który kiedyś był Twój, a który przegrałeś w grach hazardowych lata temu.
\item Wybierz jednego BG. Gdyby nie Ty, ta postać oblałaby w przeszłości test zdolności umysłowych.
\item Wybierz jednego BG. Bazując na paru komentarzach, które podsłuchałeś, podejrzewasz, że nie darzy on Twojej strefy kompetencji lub ulubionego hobby wielką estymą. 
\item Wybierz jednego BG, którego specjalizacja jest powiązana z Twoją. Połączenie wpływa na nie w pewien sposób. Dla przykładu, jeśli postać korzysta z broni, Twoja zdolność specjalizacyjna czasami ulepsza ten atak w pewien sposób.
\item Wybierz jednego BG. Panicznie boi się on wysokości. Chciałbyś nauczyć go, jak być bardziej wyluzowanym na wysokościach. To od niego zależy decyzja, czy się zgodzi zaakceptować Twoją pomoc.
\item Wybierz jednego BG. Jest on skeptyczny odnośnie Twoich twierdzeń o czymś ważnym, co się przytrafiło Tobie w przeszłości. Może on nawet chcieć cię zdyskredytować lub odkryć “tajemnicę” Twojej historii, choć to zależy od niego.
\item Wybierz jednego BG. Ma on talent do dostrzegania, gdzie Twoje plany mają słabe punkty.
\item Wybierz jednego BG. Twarz tej postaci jest tak intrygująca z powodów, których nie rozumiesz, że czasami ją szkicujesz w piasku lub innym medium, do którego masz dostęp.
\item Wybierz jednego BG. Ta postać ma dodatkowy zwykły przedmiot od Ciebie – może to być coś, co zrobiłeś lub po prostu coś, co jej ofiarowałeś. (Dany gracz wybiera przedmiot.)
\item Wybierz jednego BG. Wynajął on Cię, byś wykonał dla niego pewną robotę. Otrzymałeś zapłatę, ale jeszcze nie wykonałeś tej pracy.
\item Wybierz jednego BG. Pracowaliście razem kiedyś, i skończyło się to źle.
\item Wybierz jednego BG. Kiedy stoi on obok Ciebie i poświęca swoją akcję na skoncentrowanie się, by ci pomóc,  jedna z Twoich zdolności specjalizacji ma podwojony zasięg.
\end{itemize}

\subsection{O specjalizacji}

Specjalizacje w tej książce celowo mają opis w ledwie kilku zdaniach, by można było je zastosować w wielu konwencjach. Zdanie lub dwa podsumowuje każdy z nich. Po wyborze przez Ciebie specjalizacji, masz opcję rozszerzenia jej opisu, tak, by pasowała do settingu lub postaci.

Dla przykładu, jeśli wybierasz Działa pod Przykrywką, opis tej specjalizacji to “Udając kogoś innego, poszukujesz odpowiedzi, których potężni tego świata nie chcą wyjawić”. Jeśli wybierasz Tworzy Dziwną Naukę, opis brzmi “Twoje nadnaturalne wejrzenie i zdolności tworzą z Ciebie naukowca zdolnego tworzyć cuda”. Te opisy zapewniają czego potrzebujesz, by korzystać z Specjalności.

Jednakże, jeśli sobie życzysz (i tylko, jeśli sobie życzysz – nie ma takiego obowiązku) możesz dodać więcej do tych opisów w sposób, który pasuje do Twojej gry. Dla przykładu, jeśli wybierasz Działa pod Przykrywką i Tworzy Dziwną Naukę dla sesji współczesnej, takiej jak horror, urban fantasy, sesja szpiegowska lub coś podobnego, możesz rozbudować opisy, jak pokazano w poniższych przykładach.

\textbf{Działa pod Przykrywką}: Szpiegostwo nie jest czymś, o czym masz jakąkolwiek wiedzę. Przynajmniej chcesz, by wszyscy wokół w to wierzyli, ponieważ naprawdę, zostałeś wytrenowany jako szpieg lub tajny agent. Możesz pracować dla rządu lub dla siebie. Możesz być funkcjonariuszem policji lub przestępcą. Możesz nawet być dziennikarzem śledczym.

Niezależnie od okoliczności, pozyskujesz informacje, które inni chcieliby zachować w tajemnicy. Zbierasz szeptane plotki, historie i dowody, i wykorzystujesz tę wiedzę w swoich własnych misjach oraz, jeśli to stosowne, zapewniasz swoim mocodawcom informacje, których pożądają. Alternatywnie, możesz sprzedać wiedzę, którą pozyskałeś, tym, którzy płacą najwięcej.

Najpewniej nosisz ciemne kolory – czarny, szarości lub ciemny błękit – by pomóc Ci wmieszać się w cienie, chyba, że przebrałeś się akurat za kogoś innego.

\textbf{Tworzy Dziwną Naukę}: Możesz być szanowanym naukowcem, publikującym w naukowych czasopismach. Lub możesz być uznawany za szaleńca przez innych, podążając za dziwnymi teoriami, którzy inni uznają za niedostatecznie dowiedzione. Prawdą jest jednak, że masz szczególny dar do przesuwania granic tego, co możliwe. Możesz pozyskać nową perspektywę i odblokować dziwne zjawiska dzięki swoim eksperymentom. Tam, gdzie inni widzą masę bzdur, Ty przeczesujesz teorie spiskowe dla olśnienia. Możesz robić swoje badania jako badacz rządowy, uniwersytecki, naukowiec korporacyjny, lub z wnętrza swojego własnego garażu. Zawsze jednak przesuwasz granice tego, co możliwe. 

Najpewniej dbasz o swoją pracę bardziej niż o trywialności pokroju własnego wyglądu, miłe zachowanie, lub społeczne normy, jednakże, ekscentryk Twojego pokroju nawet tutaj może się wyłamywać stereotypom. 

Jeśli chcesz pójść dalej, możesz także określić skąd zdolności Twojej specjalizacji się biorą. W zależności od konwencji, mogą ona brać się z treningu, magicznych run, poprzez zdolności cybernetyczne, dziedzictwo genetyczne lub ponieważ masz dostęp do zaawansowanej technologii. Dla przykładu, postać może być w stanie atakować błyskawicami ponieważ dostała się pod wpływ dziwnego promieniowania lub ponieważ posiada blaster elektryczny. Z drugiej strony, może tak się dziać, ponieważ intensywny trening odblokował dla niej dostęp do magii błyskawic. Możliwości są prawie nieskończone, i od Ciebie zależy, czy je wylistujesz, czy też nie.  Niezależnie od tego, jak zdolności zostały pozyskane, wystarczy, że działają.

\subsection{Specializacje}

Pełen opis wszystkich zdolności można znaleźć w odpowiednim rozdziale, który ma opisy typów, posmaków i zdolności w jednym bogatym katalogu.

\subsubsection{Absorbuje Energię}\index{Specjalizacje!Lista!Absorbuje Energię}

Władasz energią i zamieniasz ją na inne jej rodzaje.

Poziom 1: Absorpcja Energii Kinetycznej

Poziom 1: Wyzwolenie Energii

Poziom 2: Zasilenie Przedmiotu

Poziom 3: Absorpcja Czystej Energii lub Ulepszona Absorpcja Energii Kinetycznej

Poziom 4: Przeładowanie Energii

Poziom 5: Zasilenie istoty

Poziom 6: Zasilenie Tłumu lub Przeładowanie Urządzenia

Wtrącenia MG: Energia wyładowuje się w destruktywny sposób. Pewni drapieżcy żywią się czystą energią. Przypadkowy przedmiot zostaje wyssany z energii.

\subsubsection{Bada Ciemne Miejsca}\index{Specjalizacje!Lista!Bada Ciemne Miejsca}

Jesteś archetypowym łowcą skarbów i znalazcą zgubionych rzeczy.

Poziom 1: Wspaniały Odkrywca

Poziom 2: Wspaniały Infiltrator

Poziom 3: Dostosowanie Oczu

Poziom 3: Nocne Uderzenie lub Śliski Klient

Poziom 4: Ciężko Zapracowana Odporność

Poziom 5: Eksplorator Ciemności

Poziom 6: Oślepiający Atak lub W Objęciach Mroku

Wtrącenia MG: Przedmioty wypadają Ci z kieszeni lub torby w mroku, mapy Ci się gubią, pozyskane informacje nie zawierają istotnego szczegółu.  

\subsubsection{Buduje Roboty}\index{Specjalizacje!Lista!Buduje Roboty}

Twoje robotyczne twory robią to, czego od nich zażądasz. 

(Słowo “robot” użyte w tej specializacji jest używane, nawet jeśli roboty tworzone przed Ciebie mogą być odmienne od tych tworzonych przez kogoś innego, w zależności on konwencji. Roboty steampunkowe, organiczne lub nawet magiczne golemy – do 
tego wszystkiego odnosi się tutja słowo “robot”.)

Poziom 1: Robot-Asystent

Poziom 1: Twórco Robotów

Poziom 2: Kontrola Robotów

Poziom 3: Kompan-Ekspert lub Umiejętna Obrona

Poziom 4: Unowocześnienie Robota

Poziom 5: Armia Robotów

Poziom 6: Robotyczna Ewolucja lub Unowocześnienie Robota

Wtrącenia MG: Robot zostaje zhackowany, działa randomowo lub niespodziewanie wybucha.

\subsubsection{Chroni Słabszych}\index{Specjalizacje!Lista!Chroni Słabszych}

Pomagasz słabszym, pragnącym pomocy i bezsilnym.

Poziom 1: Odwaga

Poziom 1: Tarcza Obronna

Poziom 2: Wierny Obrońca

Poziom 2: Empatia

Poziom 3: Podwójni Bronieni lub Prawdziwy Strażnik

Poziom 4: Wyzwanie Bojowe

Poziom 5: Chętna Ofiara

Poziom 6: Resuscytacja lub Prawdziwy Obrońca

Wtrącenia MG: Postać skupiona na ochronie innych może czasami wystawić samą siebie do ataku.

\subsubsection{Chroni Wrota}\index{Specjalizacje!Lista!Chroni Wrota}

Każdy chce mieć Ciebie po swojej stronie w walce, ponieważ nic Cię nie omija.

Poziom 1: Ufortyfikowana Pozycja

Poziom 1: Do Mnie!

Poziom 2: Moc i Umysł

Poziom 3: Budujący Umocnienia lub Odbicie Ataków

Poziom 4: Większa Ulepszona Moc

Poziom 5: Pole Wzmacniające

Poziom 6: Generacja Pola Siłowego lub Atak Oszałamiający

Wtrącenia MG: Strategicznie ważna struktura się zapada. Wróg atakuje z niespodziewanej strony.

\subsubsection{Dotyka Nieba}\index{Specjalizacje!Lista!Dotyka Nieba}

Kontrolujesz pogodę.

Poziom 1: Unoszenie się

Poziom 2: Zbroja Wiatru

Poziom 3: Promienie Mocy lub Przywołanie Burzy

Poziom 4: Jeździec Wiatru

Poziom 5: Emisja Zimna

Poziom 6: Kontrola Pogody lub Rydwan Wiatru

Wtrącenia MG: Sojusznik jest przypadkowo trafiony przez błyskawicę. Niespodziewane uziemienie zadaje obrażenia. Pogoda jest zmieniona w niewłaściwy sposób i burza wyrywa się spod kontroli. 

\subsubsection{Działa pod Przykrywką}\index{Specjalizacje!Lista!Działa pod Przykrywką}

Pod przebraniem kogoś innego, szukasz odpowiedzi, które potężni tego świata pragną zachować dla siebie. 

(Ktoś kto Działa pod Przykrywką może mieć zestaw do przebierania się jako dodatkowy ekwipunek).

Poziom 1: Śledztwo

Poziom 2: Przebranie

Poziom 3: Agent-Prowokator lub Bieg i Walka

Poziom 4: Niezłe Oszustwo

Poziom 5: Korzystanie z Dostępnych Opcji

Poziom 6: Zaufaj Swemu Szcześciu lub Śmiertelny Cios

Wtrącenia MG: Pech może zepsuć najlepszy plan. Przebrenie zawodzi. Sprzymierzeńcy okazują się również być agentami.

\subsubsection{Dzierży Dwie Bronie Naraz}\index{Specjalizacje!Lista!Dzierży Dwie Bronie Naraz}

Dzierżysz stal w obydwu rękach, gotowy stanąć naprzeciwko każdego wroga. 

Poziom 1: Podwójne Władanie Lekkimi Broniami 

Poziom 2: Podwójny Cios

Poziom 2: Infiltrator

Poziom 3: Podwójne Władanie Średnią Bronią lub Precyzyjne Cięcie

Poziom 4: Podwójna Obrona

Poziom 5: Podwójne Rozproszenie Uwagi

Poziom 6: Rozbrojenie lub Wielokrotny Atak

Wtrącenia MG: Ostrze łamie się w połowie lub broń wypada z dłoni swego nosiciela. 

\subsubsection{Dzierży Magiczną Broń}\index{Specjalizacje!Lista!Dzierży Magiczną Broń}

Posiadasz broń o dziwnych właściwościach i Twoja wiedza o jej mocy pozwoliła Ci stworzyć unikalny styl walki.

Poziom 1: Zaczarowana Broń

Poziom 1: Wrodzona Moc

Poziom 1: Naładowanie Broni

Poziom 2: Uderzenie Mocy

Poziom 3: Szybki Atak lub Rzut Zaczarowaną Bronią

Poziom 4: Broń Defensywna

Poziom 5: Zaczarowany Ruch

Poziom 6: Smiertelny Cios lub Wielokrotny Atak

Wtrącenia MG: Broń się psuje lub zostaje upuszczona. Postać traci połąćzenie ze swoją bronią aż do czasu, gdy wykorzysta swoj akcję, by odnowić połączenie. Energia broni rozładowuje się w niespodziewany sposób. 

\subsubsection{Fruwa Szybciej Niż Pocisk}\index{Specjalizacje!Lista!Fruwa Szybciej Niż Pocisk}

Możesz latać i jesteś supersilny, ciężki w uszkodzeniu, a także szybki. Czy jest coś, czego nie możesz zrobić?

Poziom 1: Unoszenie się

Poziom 2: Większy Ulepszony Potencjał

Poziom 3: Ukryta Siła lub Rentgen w Oczach

Poziom 4: W Mgnieniu Oka

Poziom 4: Rozpęd

Poziom 5: Jeszcze Żywy

Poziom 6: palące światło lub Zignorowanie Przeszkody

Wtrącenia MG: Nemezis Cię odnajduje. Odnaleziono dziwny materiał, któy niweluje moce postacu. 

\subsubsection{Gra w Zbyt Wiele Gier}\index{Specjalizacje!Lista!Gra w Zbyt Wiele Gier}

Lekcje, refleks i strategie, których się nauczyłeś, grając w zbyt wiele gier, mają zastosowanie w prawdziwym życiu, gdzie ludzie, którzy nie grają dostatecznie dużo muszą się szczególnie męczyć. 

Poziom 1: Lekcje z Gier

Poziom 1: Gamer

Poziom 2: Oczy Przyzwyczajone do Ciemności

Poziom 2: Odporność na Sztuczki

Poziom 3: Cel Snipera lub Ulepszone Skupienie w Szybkości

Poziom 4: Gierki Umysłowe

Poziom 4: Ulepszony Intelekt

Poziom 5: Wytrzymałość Gracza

Poziom 6: Regeneracja Umysłu lub Bóg Gier

Wtrącenia MG: Chybiony atak trafie nie ten cel. Ekwipunek siępsuje. Czasami ludzie reaguję nagatywnie na kogoś, kto przeżył większość swego życia w wyimaginowanych światach gier.

\subsubsection{Grzmi}\index{Specjalizacje!Lista!Grzmi}

Emitujesz destruktywne dźwięki i manipulujesz nimi.

Poziom 1: Promień Grzmotu

Poziom 2: Bariera Konwersji Dźwięku

Poziom 3: Tłumienie Dźwięków lub Echolokacja

Poziom 4: Okrzyk Roztrzaskania

Poziom 5: Subsoniczny Hałas

Poziom 5: Wzmocnienie Dźwięku

Poziom 6: Trzęsienie Ziemi lub Śmiertelna Wibracja

Wtrącenia MG: Głośne hałasy przyciągają uwagę.

\subsubsection{Ignoruje Fizyczny Dystans}\index{Specjalizacje!Lista!Ignoruje Fizyczny Dystans}

Możesz się teleportować w jedno miejsce z drugiego poprzez krótki pobyt w równoległym wymiarze.

Poziom 1: Wymiarowy Ścisk

Poziom 2: Oportunista

Poziom 3: Obronna Teleportacja lub Skoki Teleportacyjne

Poziom 4: Krótka Teleportacja

Poziom 5: Średnia Teleportacja

Poziom 6: Teleportacja lub Rana Teleportacyjna

Wtrącenia MG: Teleportacja kończy się źle, umieszczając postać w niebezpiecznym miejscu. Bezwładność (np.: wskutek spadania) trwa podczas teleportacji, raniąc postać. 

\subsubsection{Infiltruje}\index{Specjalizacje!Lista!Infiltruje}

Subtelność, chytrość i ukradkowość pozwalają Ci na dostęp tam, gdzie inni nie mogą.

Poziom 1: Umiejętności Złodzieja

Poziom 1: Wyczucie Pobudek

Poziom 2: Impersonacja

Poziom 2: Ucieczka, nie Walka

Poziom 3: Świadomość lub Umiejętny Atak

Poziom 4: Niewidzialność

Poziom 5: Unik

Poziom 6: Pranie Mózgu lub Odsunięcie się

Wtrącenia MG: Szpiegów traktuje się okrutnie, gdy się ich złapie. Ich sprzymierzeńcy się ich wypierają. Pewnych sekretów lepiej nigdy nie poznać.

\subsubsection{Interpretuje Prawo}\index{Specjalizacje!Lista!Interpretuje Prawo}

Jest twoją rzeczą naginanie innych do swoich poglądów.

Poziom 1: Dyplomata

Poziom 1: Wiedza Prawnicza

Poziom 2: Debata

Poziom 3: Przydatna Pomoc lub Ulepszone Skupienie w Inteligencji

Poziom 4: Przerażenie

Poziom 5: Nikt nie Wie Lepiej

Poziom 6: Większy Ulepszony Potencjał lub Prawnik-Stażysta

Wtrącenia MG: Ludzie nie lubią wszystkowiedzących. Rozproszenie lub przeszkodzenie przeszkadza w argumencie prawniczym.

\subsubsection{Istnieje Częściowo Poza Fazą}\index{Specjalizacje!Lista!Istnieje Częściowo Poza Fazą}

Częściowo przezroczysty, jesteś w części poza fazą i możesz się przemieszczać przez ciała stałe. 

Poziom 1: Przechodzenie Przez Ściany

Poziom 2: Defensywne Znikanie

Poziom 3: Atak Fazowy lub Drzwi Fazowe

Poziom 4: Duch

Poziom 5: Nietykalny

Poziom 6: Ulepszony Atak Fazowy lub Wyfazowanie Wroga

Wtrącenia MG: Postać wyfazowuje się w nieznany wymiar. Postać gubi się w dużym ciele stałym. 

\subsubsection{Istnieje w Dwóch Miejscach Naraz}\index{Specjalizacje!Lista!Istnieje w Dwóch Miejscach Naraz}

Istniejesz w dwóch miejscach w tym samym czasie.

Poziom 1: Kopia

Poziom 2: Dzielone Zmysły

Poziom 3: Ulepszona Kopia lub Odporna Kopia

Poziom 4: Przekaz Obrażeń

Poziom 5: Skoordynowany Wysiłek

Poziom 6: Wielość lub Odporna Kopia

Wtrącenia MG: Obserwacja świata z dwóch odmiennych perspektyw dezorientuje postać, powodując zawroty głowy, wymioty lub konfuzję.

\subsubsection{Izoluje Umysł od Ciała}\index{Specjalizacje!Lista!Izoluje Umysł od Ciała}

Twój umysł opuszcza Twoje ciało, by widzieć odległe miejsca i poznawać sekrety, których nie da się poznać inaczej.

Poziom 1: Trzecie Oko

Poziom 2: Otwarty Umysł

Poziom 2: Wyostrzone Zmysły

Poziom 3: Zdalne Trzecie Oko lub Odnalezienie Ukrytych

Poziom 4: Sensor

Poziom 5: Psioniczny Pasażer

Poziom 6: Projekcja Mentalna lub Ulepszony Sensor

Wtrącenia MG: Ponowne połączenie ciała i umysłu może czasami być dezorientujące i wymagać od postaci spędzenia paru minut na dostrajaniu się. 

\subsubsection{Jaśnieje Światłością}\index{Specjalizacje!Lista!Jaśnieje Światłością}

Możesz tworzyć światło, kształtować je, naginać lub gromadzić jako broń. 

Poziom 1: Oświecony

Poziom 1: Dotyk Oświecenia

Poziom 2: Oszałamiające Światło

Poziom 3: Palące Światło lub Umiejętna Obrona

Poziom 4: Światło Słońca 

Poziom 5: Niewidzialność

Poziom 6: Żywe Światło lub Pole Obronne

Wtrącenia MG: Sprzymierzeńcy są przypadkowo oszołomieni lub oślepieni. Jasne błyski przywołują strażników. 

\subsubsection{Jest Bardzo Silny}\index{Specjalizacje!Lista!Jest Bardzo Silny}

Jesteś umięśniony, możesz podnosić wielkie ciężary i przebijać się przez drzwi.

Poziom 1: Atleta

Poziom 1: Ulepszone Skupienie w Mocy

Poziom 2: Pokaz Siły

Poziom 3: Żelazne Pięści lub Rzut

Poziom 4: Większa Ulepszona Moc

Poziom 5: Brutalne Uderzenie

Poziom 6: Większa Ulepszona Moc lub Atak z Wyskoku

Wtrącenia MG: Łatwo jest zniszczyć delikatne przedmioty lub kogoś przypadkowo zranić

\subsubsection{Jest Idolem Milionów}\index{Specjalizacje!Lista!Jest Idolem Milionów}

Jesteś celebrytą i większość ludzi Cię uwielbia.

Poziom 1: Świta

Poziom 1: Talent Celebryty

Poziom 2: Zalety Sławy

Poziom 3: Ulepszone Zdrowie lub Umiejętny Atak

Poziom 4: Zachwyt Światła Gwiazd

Poziom 4: Kompan-Ekspert

Poziom 5: Czy Ty Wiesz W Ogóle Kim Jestem?

Poziom 6: Oratorska Inspiracja lub Ulepszony Kompan

Wtrącenie GM: Fani są w niebezpieczeństwie lub odnieśli obrażenia ze względu na Ciebie. Ktośw Twojej świcie Cię zdradza. Twój show, tour, kontakt lub inne wydarzenie jest odwołane. Media publikują zdjęcia z Tobą we wstydliwych sytuacjach. 

\subsubsection{Jest Jasnowidzem}\index{Specjalizacje!Lista!Jest Jasnowidzem}

Posiadasz psioniczny dar, który  pozwala Ci widzieć to, czego inni nie mogą.

Poziom 1: Postrzeganie Niewidocznego

Poziom 2: Rentgen w Oczach

Poziom 3: Odnalezienie Ukrytych lub Sensor

Poziom 4: Widzenie na Odległość

Poziom 5: Postrzeganie Czasu

Poziom 6: Projekcja Mentalna lub Całkowita Świadomość

Wtrącenia MG: Pewne sekrety są zbyt okropne, by je poznać.

\subsubsection{Jest Jednoosobowym Bastionem}\index{Specjalizacje!Lista!Jest Jednoosobowym Bastionem}

Twoja zbroja, wraz z Twoim rozmiarem, siłą, treningiem lub wszczepami bionicznymi, czyni Ciętrudnym do przemieszczenia lub zaatakowania.

(Pewne postaci, które Są Jednoosobowym Bastionem, mogą już być ekspertami w pancerzach. Mogą one wybrać inną zdolność1-szego poziomu niż Wyszkolony w Zbroi)

Poziom 1: Wyszkolony w Zbroi

Poziom 1: Doświadczony Obrońca

Poziom 2: Odporność na Żywioły

Poziom 3: Nieporuszalny

Poziom 3: Większa Ulepszona Moc lub Wyszkolony we Wszystkich Broniach

Poziom 4: Żywa Ściana

Poziom 5: Wytrzymały

Poziom 5: Mistrzowska Biegłość w Pancerzach

Poziom 6: Śmiertelne Obrażenia lub Wyszkolony w Tarczach

Wtrącenia MG: Zbroja się uszkadza. Mali wrogowie atakują Cię w sprytny sposób. 

\subsubsection{Jest Mistrzem Obrony}\index{Specjalizacje!Lista!Jest Mistrzem Obrony}

Korzystasz z odpowiedniego ekwipunku i wyszkolenia, by uniknąć zranienia w walce.

Poziom 1: Mistrz Tarcz

Poziom 2: Twardy

Poziom 2: Wyszkolony w Zbroi

Poziom 3: Unik i Odporność lub Unik i Rewanż

Poziom 4: Wieża Siły Woli

Poziom 4: Przywykły do Noszeni Zbroi

Poziom 5: Nic Tylko Obrona

Poziom 6: Mistrz Obrony lub Jak Druga Skóra

Wtrącenia MG: Tarcza pęka przy trafieniu, jak i bronie, którymi się blokuje. Paski od pancerza pękają.

\subsubsection{Jest Poszukiwany Przez Prawo}\index{Specjalizacje!Lista!Jest Poszukiwany Przez Prawo}

Plakaty "POSZUKIWANY, ŻYWY LUB MARTWY" posiadają Twoje podobieństwo. To od Ciebie zależy, czy to koszmarna pomyłka, która się wymknęła spod kontroli, czy może potrafisz kogoś zabić, bo na Ciebie krzywo spojrzał. 

Poziom 1: Ulepszona Szybkość

Poziom 1: Zmysł Niebezpieczeństwa

Poziom 2: Atak z Zaskoczenia

Poziom 3: Reputacja Spoza Prawa lub Następny Atak

Poziom 4: Szybkie Zabójstwo

Poziom 5: Drużyna Desperados

Poziom 6: Jeszcze Żywy lub Śmiertelne Obrażenia

Wtrącenia MG: Większość ludzi nie reaguje dobrze na poszukiwanego listem gończym w swoich szeregach.

\subsubsection{Jest Stworzony z Kamienia}\index{Specjalizacje!Lista!Jest Stworzony z Kamienia}

Twoje ciało jest stworzone z twardego minerału, czyniąc się twardym, ciężkim do zranienia humanoidem.

Poziom 1: Ciało Golema

Poziom 1: Uzdrawianie Golema

Poziom 2: Chwyt Golema

Poziom 3: Wyszkolony Miażdżyciel

Poziom 3: Tupnięcie Golema lub Uzbrojenie

Poziom 4: Głębokie Rezerwy

Poziom 5: Wyspecjalizowany Pięściarz

Poziom 5: Jak Posąg

Poziom 6: Ultra Wzmocnienie lub Regeneracja Umysłu

Wtrącenia MG: Istoty z kamienia czasami zapominają o własnej sile lub wadze. Chodzący posąg może przerazić zwykłych ludzi.

\subsubsection{Jeździ Jak Maniak}\index{Specjalizacje!Lista!Jeździ Jak Maniak}

Niezależnie od tego, czy balansujesz na dwóch kołach, przeskakujesz między pojazdami lub ruszasz na przód ku niebezpiezeństwu, nie myślisz zbyt dużo o własnym bezpieczeństwie, gdy jesteś za kierownicą.  

(Ktoś to Jeździ Jak Maniak potrzebuje dostępu do pojazdu.)

Poziom 1: Kierowca

Poziom 1: Atak Podczas Kierowania

Poziom 2: Surfer Aut

Poziom 2: Pojedynek Spojrzeń

Poziom 3: Doświadczony Kierowca lub Ulepszone Skupienie w Szybkości

Poziom 4: Bystrooki

Poziom 4: Ulepszona Szybkość

Poziom 5: Coś na Drodze

Poziom 6: Uzdolniony Kierowca lub Śmiertelne Obrażenia

Wtrącenia MG: Silnik odmawia posłuszeństwa. Most na końcu drogi jest wyłączony z ruchu. Przednia szyba się roztrzaskuje. Ktoś nagle wyskakuje na przód pojazdu.

\subsubsection{Kocha Pustkę}\index{Specjalizacje!Lista!Kocha Pustkę}

Kiedy jesteś tylko Ty, Twój skafander kosmiczny i panorama niekończących się gwiazd, osiągasz stan spokoju.

Opcja do podmiany: Mam Kombinezon Kosmiczny, Będę Podróżnikiem 

Poziom 1: Umiejętności Kosmiczne

Poziom 1: Przyzwyczajony do Mikrograwitacji

Poziom 2: Ulepszone Skupienie w Szybkości

Poziom 2: Ulepszona Muskulatura

Poziom 3: Walka w Kosmosie lub Zbroja Fuzyjna

Poziom 4: Cichy jak Kosmos

Poziom 4: Odepchnięcie i Rzut

Poziom 5: Uniki w Mikrograwitacji

Poziom 6: Wystrzał Mikrograwitacyjny lub Pole Reakcyjne

Wtrącenia MG: Kombinezony kosmiczne mogą się zepsuć. Wskaźniki poziomu tlenu czasami mogą być mylące. Mikrometeoryty są powszechne w kosmosie.

\subsubsection{Kontroluje Bestie}\index{Specjalizacje!Lista!Kontroluje Bestie}

Masz rzadką zdolność komunikowania się i przewodzenia bestiom.

Poziom 1: Zwierzęcy Kompan

Poziom 2: Ukojenie Dzikiego

Poziom 2: Komunikacja

Poziom 3: Rumak lub Silniejsi Razem

Poziom 4: Oczy Bestii

Poziom 4: Ulepszony Kompan

Poziom 5: Zew Dziczy

Poziom 6: Jak Jedna Istota lub Kontrola Dzikiej Bestii

Wtrącenia MG: Społeczność jest niechętna dzikim zwierzęciom. Bestia wyrwane spod kontroli stają się prawdziwym niebezpieczeństwem. 

\subsubsection{Kontroluje Grawitację}\index{Specjalizacje!Lista!Kontroluje Grawitację}

Manipulujesz siłami grawitacyjnymi.

Opcja do podmiany: Ciężki

Poziom 1: Unoszenie się

Poziom 2: Ulepszone Skupienie w Szybkości

Poziom 3: Definiowanie Dołu lub Szarpnięcie Grawitacyjne

Poziom 4: Pole Grawitacyjne

Poziom 5: Lot

Poziom 6: Ulepszone Szarpnięcie Grawitacyjne lub Ciężar Świata

Wtrącenia MG: Świadkowie reagują nierozsądnym strachem. Dziwna interakcja posyła obiekt lub sprzymierzeńca ku przestworzom.

\subsubsection{Kopiuje Supermoce}\index{Specjalizacje!Lista!Kopiuje Supermoce}

Możesz kopiować umiejętności, zdolności i supermoce innych. 

Poziom 1: Skupienie na Umiejętności

Poziom 1: Skupienie na Umiejętności

Poziom 2: Skopiuj Moc

Poziom 3: Kradzież Mocy lub Dzikie Zdolności

Poziom 4: Ulepszone Skopiowanie Mocy

Poziom 5: Pamięć Mocy

Poziom 6: Cudowne Kopiowanie lub Wielość Kopii

Wtrącenia MG: Skopiowana moc przestaje nagle działać lub wymyka się z kontroli. Skopiowana moc nie posiada drugorzędnych mocy (np.: superszybkość bez odporności na pęd powietrza lub bycie odpornym na żar własnych ognistych pocisków). 

\subsubsection{Leci na Wspaniałych Skrzydłach}\index{Specjalizacje!Lista!Leci na Wspaniałych Skrzydłach}

Wielu superbohaterów może latać, niektórzy z nich nawet mają skrzydła. Możesz korzystać ze swoich skrzydeł do poruszania się, atakowania i obrony.

Poziom 1: Unoszenie się

Poziom 1: Krótki Lot

Poziom 2: Skrzydła-Bronie

Poziom 3: Akrobatyczny Atak lub or Latający Kompan

Poziom 4: Trudny do Trafienia

Poziom 5: Rozpęd

Poziom 6: Trudny Cel lub Mistrz Obrony

Wtrącenia MG: Skrzydło może być zranione lub nie mieć dość miejsca, przez co bohater upada. Latanie wysoko czyni postać wyraźnym celem dla niespodziewanego wroga. 

\subsubsection{Łamie Systemy}\index{Specjalizacje!Lista!Łamie Systemy}

Wykorzystujesz słabości sztucznych systemów, wliczając (ale nie ograniczając się) do programów komputerowych.

Poziom 1: Hakowanie Niemożliwości

Poziom 1: Programowanie

Poziom 2: Połączenia

Poziom 3: Sprawny Oszust lub Umiejętny Atak

Poziom 4: Skonfunduj Wroga

Poziom 5: Wsparcie Przyjaciela

Poziom 6: Przysługa lub Większy Ulepszony Potencjał

Wtrącenia MG: Twoje kontakty czasami mają swoje własne motywacje. Niekiedy urządzenia mają zabezpieczenia lub nawet pułapki.

\subsubsection{Łączy Ciało i Stal}\index{Specjalizacje!Lista!Łączy Ciało i Stal}

Twoje ciało jest częściowo maszyną.

Poziom 1: Ulepszone Ciało

Poziom 2: Interfejs

Poziom 3: Pakiet Sensoryczny lub Uzbrojenie

Poziom 4: Fuzja

Poziom 5: Głębokie Reserwy

Poziom 6: Regeneracja Umysłu lub Ultra Wzmocnienie

Wtrącenia MG: Ludzie w większości społeczności boją się kogoś, kto ma w sobie mechaniczne części.

\subsubsection{Łączy Umysł i Maszynę}\index{Specjalizacje!Lista!Łączy Umysł i Maszynę}

Elektroniczne implanty w Twoim mózgu czynią Cię supermyślicielem.

Poziom 1: Ulepszony Intelekt

Poziom 1: Umiejętności Wiedzy

Poziom 2: Kwerenda

Poziom 3: Procesor Akcji lub Telepatia Maszyn

Poziom 4: Większy Ulepszony Intelekt

Poziom 4: Umiejętności Wiedzy

Poziom 5: Wizja Przyszłości

Poziom 6: Ulepszenie Maszyny lub Regeneracja Umysłu

Wtrącenia MG: Maszyny się psują. Potężne maszyny myślące mogą przejąć kontrolę nad mniejszymi maszynami. Pewni ludzie nie ufają komuś, kto nie jest w pełni organiczny. 

\subsubsection{Ma Szlachetną Krew}\index{Specjalizacje!Lista!Ma Szlachetną Krew}

Dziedzic bogactwa i mocy, masz tytuł szlachecki i zdolności przyznae przez uprzywilejowane wychowanie. 

Opcja Zamiany Typu: Służba

Poziom 1: Przywileje Szlachty

Poziom 2: Wyszkolony Dyskutant

Poziom 3: Zaawansowany Rozkaz lub Odwaga Szlachcica

Poziom 4: Kompan-Ekspert

Poziom 5: Potwierdzenie Własnego Przywileju

Poziom 6: Przydatna Pomoc lub Umysł Lidera

Wtrącenia MG: Długi rodziny szlacheckiej są problemem bohatera. Dawno zagubione rodzeństwo chce się pozbyć swego rywala. Zabójca odnajduje postać. 

\subsubsection{Ma Tysiąc Twarzy}\index{Specjalizacje!Lista!Ma Tysiąc Twarzy}

Możesz zmienić swój wygląd, by wyglądać jak zupełnie inna osoba. 

Poziom 1: Morficzna Twarz

Poziom 1: Umiejętności Międzyludzkie

Poziom 2: Zmiana Ciała

Poziom 2: Ciało Bitewne

Poziom 3: Przebranie Innej Osoby lub Odporność

Poziom 4: Nieśmiertelny

Poziom 4: Przemyślenie Problemów

Poziom 5: Pamięć w Czyn

Poziom 6: Rozdzielenie Jaźni lub Odczytanie Myśli

Wtrącenia MG: Część przebrania zawodzi. BN myśli, że przebrana postać to ktoś, kogo zna bardzo dobrze.

\subsubsection{Mistrzowsko Posługuje się Bronią}\index{Specjalizacje!Lista!Mistrzowsko Posługuje się Bronią}

Jesteś mistrzem w używaniu pewnego rodzaju broni, czy to mieczy, biczy, noży, pistoletów, czy czegoś innego.

(Ktoś, kto Mistrzowsko Posługuje się Bronią, może mieć dodatkowy ekwipunek, wliczając broń wysokiej jakości.)

Poziom 1: Mistrz Broni

Poziom 1: Twórca Broni

Poziom 2: Obrona Bronią

Poziom 3: Szybki Atak lub Cios Rozbrajający

Poziom 4: Bez Wpadek

Poziom 5: Wyjątkowe Mistrzostwo

Poziom 6: Morderca lub Śmiertelny Cios

Wtrącenia MG: Bronie siępsują. Broniem ogą zostać ukradzione. Bronie można upuścić lub zostać rozbrojonym. 

\subsubsection{Morduje}\index{Specjalizacje!Lista!Morduje}

Jesteś asasynem, z profesji, chęci lub ponieważ na tym świecie mordujesz lub zostajesz zamordowany.

(Ktoś kto Morduje może mieć dodatkowy ekwipunek, wliczywszy 3 dawki trucizny 2 poziomu która zadaje 5 punktów obrażeń). 

Poziom 1: Atak z Zaskoczenia

Poziom 1: Umiejętności Zabójcy

Poziom 2: Szybka Śmierć

Poziom 2: Infiltrator

Poziom 3: Świadomość lub Warzenie Trucizn

Poziom 4: Lepszy Atak z Zaskoczenia

Poziom 5: Dodatkowe Obrażenia

Poziom 6: Plan Ucieczki lub Morderca

Wtrącenia MG: Większość ludzi nie reaguje dobrze na profesjonalnego zabójcę.

\subsubsection{Mówi Głosem Ziemi}\index{Specjalizacje!Lista!Mówi Głosem Ziemi}

Twoje duchowe połączenie z naturą i środowiskiem daje Ci mistyczne moce.

Poziom 1: Nasiona Furii

Poziom 1: Wiedza o Dziczy

Poziom 2: Chwytające Zielska

Poziom 3: Ukojenie Dzikiego lub Komunikacja

Poziom 4: Księżycowa Zmiana Kształtu

Poziom 5: Erupcja Insektów

Poziom 6: Wezwanie Burzy lub Trzęsienie Ziemi

Wtrącenia MG: Ranna naturalna (lecz niebezpieczna) istota jest odkryta. Ktoś poluje dla skór, zostawiając zwłoki, by gniły. Drzewo upada w lesie, jedno z ostatnich tak wielkich.

\subsubsection{Mówi do Duchów}\index{Specjalizacje!Lista!Mówi do Duchów}

Niespokojne dusze, duchy natury i żywiołaki wspomagają Cię.

(W pewnych settingach, Specjalizacja Mówi do Duchów dotyczy tylko jednego rodzaju duchów, takich jak duchy martwych, duchy natury itp.)

Poziom 1: Przepytanie Ducha

Poziom 2: Duch Kompan

Poziom 3: Rozkazywanie Duchom lub Wyczulone Zmysły

Poziom 4: Płaszcz Gniewu

Poziom 5: Wezwanie Ducha

Poziom 6: Wezwanie Międzywymiarowego Ducha lub Absorpcja Ducha

Wtrącenia MG: Niektórzy nie ufają tym, którzy się zadają z duchami. Martwi czasami wcale nie chcą rozmawiać. 

\subsubsection{Mówi do Maszyn}\index{Specjalizacje!Lista!Mówi do Maszyn}

Używasz swojego organicznego mózgu jak komputera, bezprzewodowo łącząć się z dowolnym urządzeniem elektronicznym. Możesz je kontrolować i wpływać na nie w sposób, w jaki inni nie mogą. 

Poziom 1: Umiłowanie do Maszyn

Poziom 1: Interfejs Zasięgowy

Poziom 2: Ulepszenie Maszyny

Poziom 2: Zauroczenie Maszyny

Poziom 3: Inteligentny Interfejs lub Rozkazywanie Maszynom

Poziom 4: Kompan-Maszyna

Poziom 4: Walczący z Robotami

Poziom 5: Zbieranie Informacji

Poziom 6: Kontrola Maszyny lub Ulepszony Kompan-Maszyna

Wtrącenia MG: Maszyna się psuje lub działa w nieprzewidziany sposób.

\subsubsection{Nie Potrzebuje Broni}\index{Specjalizacje!Lista!Nie Potrzebuje Broni}

Potężne ciosy, kopnięcia, zamachy łokciami i kolanami oraz ruchy całego ciała są wszystkimi broniami, których potrzebujesz. 

Poziom 1: Pięści Furii

Poziom 1: Ciało z Kamienia

Poziom 2: Atak z Rozbrojeniem

Poziom 2: Sztuki Walki

Poziom 3: Gibkość Niczym Woda lub Większy Ulepszony Potencjał

Poziom 4: Odbicie Ataków

Poziom 5: Atak Oszałamiający

Poziom 6: Mistrz Sztuk Walki lub Śmiertelne Obrażenia

Wtrącenia MG: Uderzanie pewnych wrogów boli Cię tak mocno, jak ich ranisz. Wrogowie z broniami mają większy zasięg. Skomplikowane ruchy sztuk walki mogą Cię wytrącić z równowagi.

\subsubsection{Nie Robi Zbyt Dużo}\index{Specjalizacje!Lista!Nie Robi Zbyt Dużo}

Jesteś obibokiem, ale wiesz coś o wielu rzeczach. 

Poziom 1: Lekcje Życiowe

Poziom 2: Wyluzowanie

Poziom 3: Umiejętny Atak lub Improwizacja

Poziom 4: Lekcje Życiowe

Poziom 4: Większa Umiejętność Obrony

Poziom 5: Większy Ulepszony Potencjał

Poziom 6: Korzystając z Doświadczenia Życiowego lub Szybki Umysł

Wtrącenia MG: Nowe sytuacje są konfundujące i stresujące. Przeszłe akcja (lub ich brak) wracają, by gnębić postać. 

\subsubsection{Nigdy się nie Poddaje}\index{Specjalizacje!Lista!Nigdy się nie Poddaje}

Nigdy się nie poddajesz, radzisz sobie z każdą raną, i zawsze jesteś gotowy na więcej.

Poziom 1: Ulepszone Odzyskanie Zdrowia

Poziom 1: Parcie Dalej

Poziom 2: Zignorowanie Bólu

Poziom 3: Gorączka Krwi lub Ukryta Siła

Poziom 4: Determinacja lub Wytrzymalszy Niż Wróg

Poziom 5: Jeszcze Żywy

Poziom 6: Ostateczne Zaprzeczenie lub Zignorowanie Przeszkody

Wtrącenia MG: Czasami, to ekwipunek i broń się poddają. 

\subsubsection{Nosi Egzotyczną Tarczę}\index{Specjalizacje!Lista!Nosi Egzotyczną Tarczę}

Posiadasz wspaniałą tarczę czystej mocy, która zapewnia obronę i pewne zdolności ataku.

Poziom 1: Tarcza Pola Siłowego

Poziom 1: Uderzenie Mocy

Poziom 2: Ulepszona Tarcza

Poziom 3: Leczący Puls lub Rzut Tarczą Siłową

Poziom 4: Zasilona Tarcza

Poziom 5: Ściana Mocy

Poziom 6: Skacząca Tarcza lub Wybuch Tarczy

Wtrącenia MG: Tarcza jest chwilowo nieaktywna. Wróg chwilowo przejmuje kontrolę nad tarczą.

\subsubsection{Nosi Halo Ognia}\index{Specjalizacje!Lista!Nosi Halo Ognia}

Możesz pokryć swe ciało płomieniami, co chroni Ciebie i rani Twoich wrogów. 

Poziom 1: Płaszcz Ognia

Poziom 2: Macki Płomieni

Poziom 3: Skrzydła Ognia lub Ognista Ręka Zguby

Poziom 4: Ostrze Ognia

Poziom 5: Ogniste Macki

Poziom 6: Ognisty Sługa lub Piekielny Szlak

Wtrącenia MG: Ogień pali łatwopalne materiały. Płomienie wyzwalają się spod kontroli. Prymitywne istoty boją się ognia i często atakują źródło swoich lęków. 

\subsubsection{Nosi Zasilany Pancerz}\index{Specjalizacje!Lista!Nosi Zasilany Pancerz}

Poziom 1: Zasilany Pancerz

Poziom 1: Ulepszona Moc

Poziom 2: Wyświetlacz w Hełmie

Poziom 3: Zbroja Fuzyjna lub Ulepszone Zdrowie

Poziom 4: Wystrzał Mocy

Poziom 5: Pole Mocy Zasilanego Pancerza

Poziom 6: Mistrzowska Modyfikacja Pancerza (Krótki Lot) lub Mistrzowska Modyfikacja Pancerza (Pojemnik na Cypher)

Wtrącenia MG: Nie możesz zdjąć pancerza. Pancerz działa samodzielnie. Pancerz chwilowo się wyłącza. BN-i boją się pancerza. 

\subsubsection{Oblicza Nieobliczalne}\index{Specjalizacje!Lista!Oblicza Nieobliczalne}

Nadludzkie zdolności matematyczne pozwalająci Ci na modelowanie świata na bieżąco, dając Ci przewagę nad innymi. 

Poziom 1: Prorocze Równanie

Poziom 1: Wyższa Matematyka

Poziom 2: Proroczy Model

Poziom 3: Podświadoma Obrona lub Ulepszony Intelekt

Poziom 4: Obliczenia Bitewne

Poziom 5: Większy Ulepszony Intelekt

Poziom 5: Najwyższa Matematyka

Poziom 6: Wiedza o Nieznanym lub Większy Ulepszony Intelekt 

Wtrącenia MG: Zbyt wiele przewidzianych wyników przeraża lub przeciąża i oszałamia postać. Wynik wskazuje na nadchodzącą klęskę. 

\subsubsection{Otrzymuje Boskie Błogosławieństwa}\index{Specjalizacje!Lista!Otrzymuje Boskie Błogosławieństwa}

Jako oddany wyznawca boskiej istoty, posiadasz pewne moce swego bóstwa, by czynić cuda. 

Poziom 1: Błogosławieństwo Bóstw

Poziom 2: Ulepszony Intelekt

Poziom 3: Boski Blask lub Kwiat Ognia

Poziom 4: Niebiańska Gloria

Poziom 5: Boska Interwencja

Poziom 6: Boski Symbol lub Przywołanie Demona

Wtrącenia MG: Demon bada użytkowników boskiej magii. Rywalizujący kult ma problemy z naukami postaci.

\subsubsection{Pilotuje Statki Kosmiczne}\index{Specjalizacje!Lista!Pilotuje Statki Kosmiczne}

Jesteś pilotem statku kosmicznego

Poziom 1: Pilotaż

Poziom 1: Planetarna Wiedza

Poziom 2: Kryjówka w Kosmosie

Poziom 2: Umysłowa Odporność

Poziom 3: Expert-Pilot

Poziom 3: Obeznanie ze Statkiem Kosmicznym lub Kompan-Maszyna

Poziom 4: Sensory Statku Kosmicznego

Poziom 4: Ulepszona Szybkość

Poziom 5: Znam Ten Statek Jak Własną Dłoń)

Poziom 6: Wspaniały Pilot

Poziom 6: Kontrola Zdalna lub Umiejętny Atak

Wtrącenia MG: Statek się gubi, psuje, lub zostaje zaatakowany w kosmosie. Dokonujesz odkrycia obcego pasażera na gapę. 

\subsubsection{Podróżuje przez Czas}\index{Specjalizacje!Lista!Podróżuje przez Czas}

Widzisz poprzez strumienie czasu, próbujesz w nie sięgnąć i w końcu nawet przez nie podróżować. 

(Choć wszystkie wybory postaci są zależne od zgody MG, Podróżuje przez Czas jest specjalizacją, odnośnie której MG i gracz powinni odbyć długą konwersację, by gracz znał zasady gry odnośnie podróży w czasie, jeśli istnieją w settingu MG. Postać z tą 
specjalizacją może znacząco zmienić świat gry, jeśli zasady gry na to pozwalają.) 

Poziom 1: Przebłysk

Poziom 2: Historia Przedmiotu

Poziom 3: Przyspieszenie Czasoprzestrzenne lub Pętla Czasu

Poziom 4: Czasoprzestrzenne Przesunięcie

Poziom 5: Sobowtór Czasoprzestrzenny

Poziom 6: Wezwanie przez Czas lub Podróż w Czasie

Wtrącenia MG: Powstają paradoksy. Inni pamiętają przeszłe wydarzenia inaczej.

\subsubsection{Poluje}\index{Specjalizacje!Lista!Poluje}

Jesteś wytrwałym łowcą, który potrafi upolować to, co zechce.

Poziom 1: Estetyczny Atak

Poziom 1: Łowczy

Poziom 2: Cel

Poziom 2: Skradanie się

Poziom 3: Walczący z Hordą lub Bieg i Chwyt

Poziom 4: Atak z Zaskoczena

Poziom 5: Dążenie Łowcy

Poziom 6: Większa Umiejętność Ataku lub Wiele Celów

Wtrącenia MG: Ofiara dostrzega postać. Cel nie jest taki słaby, jak się wydawał.

\subsubsection{Pomaga Swoim Przyjaciołom}\index{Specjalizacje!Lista!Pomaga Swoim Przyjaciołom}

Kochasz swoich przyjaciół i pomagasz im w każdej trudności, niezależnie od wszystkiego.

Opcja do podmiany: Porada od Przyjaciela

Poziom 1: Pomoc Przyjaciela

Poziom 1: Odwaga

Poziom 2:  Ochrona Przed Zmiennym Losem

Poziom 3: Koka lub Umiejętny Atak

Poziom 4: Obrońca Przyjaciół

Poziom 4: Ulepszona Muskulatura

Poziom 5: Zainspirowanie Akcji

Poziom 6: Głębokie Przemyślenia lub Umiejętna Obrona

Wtrącenia MG: Inni czasami mają niecne motywacje. Służby porządkowe się Tobą interesują. Nawet, gdy wszystko idzie dobrze, będzie to miało swoje konsekwencje. 

\subsubsection{Porusza się jak Kot}\index{Specjalizacje!Lista!Porusza się jak Kot}

Lekki, zwinny i pełen gracji, poruszasz się szybko i łatwo, co pozwala Ci unikać niebezpieczeństw.

Poziom 1: Większa Ulepszona Szybkość

Poziom 1: Balansowanie

Poziom 2: Umiejętności Ruchu

Poziom 2: Bezpieczny Upadek

Poziom 3: Trudny to Trafienia

Poziom 3: Ulepszone Skupienie w Szybkości lub  Większa Ulepszona Szybkość

Poziom 4: Szybki Cios

Poziom 5: Śliski

Poziom 6: Perfekcyjna Szybkość or  Większa Ulepszona Szybkość

Wtrącenia MG: Nawet kot może być niezdarny. Skok nie jest tak łatwy jak wygląda. Ucieczka jest na tyle przesadzona, że umieszcza postać w bardzo niebezpiecznym położeniu.

\subsubsection{Porusza się jak Wiatr}\index{Specjalizacje!Lista!Porusza się jak Wiatr}

Możesz się poruszać tak szybko, że rozmywasz się w oczach.

Poziom 1: Większa Ulepszona Szybkość

Poziom 1: Szybkostopy

Poziom 2: Trudny do Trafienia

Poziom 3: Perfekcyjna Szybkość lub  Większa Ulepszona Szybkość

Poziom 4: W Mgnieniu Oka

Poziom 5: Rozmazany

Poziom 6: Perfekcyjna Szybkość or Niemożliwa Szybkość

Wtrącenia MG: Powierzchnie mogą być śliskie lub oferować ukryte przeszkody. Ruch innych istot może być trudny do przewidzenia, i postać może w nie wbiec.

\subsubsection{Posiada Licencję na Broń}\index{Specjalizacje!Lista!Posiada Licencję na Broń}

Posiadasz pistolet i wiesz, jak z niego skorzystać w walce. 

(Choć Posiada Licencjęna Broń zaprojektowaniu z myślą o współczesnej broni, może także dotyczyć futurystycznych blasterów lub innych broni dystansowych.)

Poziom 1: Rewolwerowiec

Poziom 1: Wyszkolony w Broni Palnej

Poziom 2: Ostrożny Strzał

Poziom 3: Wyszkolony Rewolwerowiec lub Dodatkowe Obrażenia

Poziom 4: Podwójny Wystrzał

Poziom 5: Potrójny Wystrzał

Poziom 6: Specjalny Strzał lub Śmiertelne Obrażenia

Wtrącenia MG: Chybiony strzał lub zacięcie się broni! Atak nie odnosi stutku i akcja jest stracona, plus potrzeba dodatkowej akcji, by zająć się problemem. 

\subsubsection{Posiada Magicznego Sprzymierzeńca}\index{Specjalizacje!Lista!Posiada Magicznego Sprzymierzeńca}

Sprzymierzona magiczna istota, przywiązana do przedmiotu (np.: pomniejszy dżin w lampie lub duch w fajce) to Twój przyjaciel, obrońca i broń.

Poziom 1: Związana Magiczna Istota

Poziom 2: Więź z Obiektem

Poziom 2: Szuflada-Skrytka

Poziom 3: Mniejsze Życzenie lub Rumak

Poziom 4: Ulepszona Więź z Przedmiotem

Poziom 5: Średnie Życzenie

Poziom 6: Mistrzostwo Więzi z Obiektem lub Zaufaj Swemu Szczęściu

Wtrącenia MG: Istota nagle znika w swoim przedmiocie. Związany obiekt zostaje uszkodzony. Istota nie zgadza się i nie czyni tego, o co się nią prosi. Istota twierdzi, że odchodzi, jeśli nie wykona się dla niej pewnego zadania.

\subsubsection{Pracuje w Ciemnych Uliczkach}\index{Specjalizacje!Lista!Pracuje w Ciemnych Uliczkach}

Działasz niedostrzeżenie, kradnąc od bogatych, by osiągnąć swoje cele.

Poziom 1: Umiejętności Złodzieja

Poziom 2: Kontakty w Półświatku

Poziom 3: Niezłe Oszustwo lub Trening Gildii

Poziom 4: Złodziejski Mistrz

Poziom 5: Nieczyste Zagrania

Poziom 6: Szczur Miejski lub Wysokie Skupienie

Wtrącenia MG: Złodzieje lądują w więzieniu. Dorabiasz się potężnych wrogów.

\subsubsection{Pracuje, by Żyć}\index{Specjalizacje!Lista!Pracuje, by Żyć}

Cieszysz się, gdy możesz dobrze wykonać swoją pracę, niezależnie, czy to programowanie, budowanie domów, czy górnictwo asteroidów.

Poziom 1: Zręczny Rzemieślnik

Poziom 2: Mięśnie z Żelaza

Poziom 3: Oko do Szczegółów lub Improwizacja

Poziom 4: Ulepszona Moc

Poziom 4: Stwardniały

Poziom 5: Umiejętność Eksperta

Poziom 6: Większy Ulepszony Potencjał lub  Ciężko Zapracowana Odporność

Wtrącenia MG: Naprawy czasami zawodzą. Kable mogą być trudne do odkodowania i ciągle bbyć pod napięciem. Czasami ludzie są niegrzeczni dla tych, co pracują, by żyć. 

\subsubsection{Przebudza Sny}\index{Specjalizacje!Lista!Przebudza Sny}

Możesz wyciągnąć obrazy ze snów i umieścić je w świecie jawy.

Poziom 1: Iluzja Snów

Poziom 1: Oneiro-alchemia

Poziom 2: Złodziej Snów

Poziom 3: Sen Staje się Prawdą lub Ulepszony Intelekt

Poziom 4: Sen na Jawie

Poziom 5: Koszmar

Poziom 6: Komnata Snów lub Pole Reakcyjne

Wtrącenia MG: Niespodziewany epizod lunatykowania stawia postać w niebezpiecznej sytuacji. Koszmar wyzwala się ze snu.

\subsubsection{Przeszukuje Ruiny}\index{Specjalizacje!Lista!Przeszukuje Ruiny}

Kiedy nie biegniesz lub się chowasz, przeszukujesz ruiny cywilizacji w celu znalezienia użytecznych pozostałości, co pozwala Ci przetrwać. 

Poziom 1: Ocalały z Apokalipsy

Poziom 1: Wiedza o Ruinach

Poziom 2: Rzemieślnik Rupieci

Poziom 3: Korzystanie z Okazji lub Ulepszone Zdrowie

Poziom 4: Wiesz, Gdzie Szukać

Poziom 5: Cyphery z Odzysku

Poziom 6: Artefakty z Odzysku lub Pole Reakcyjne

Wtrącenia MG: Przedmiot stworzony z zrecyklingowanych śmieci psuje się. Ktoś pojawia się i twierdzi, że użyteczne pozostałości nalezą do niego. Zrecyklingowany cypher eksploduje. 

\subsubsection{Przewodzi}\index{Specjalizacje!Lista!Przewodzi}

Twoje naturalne zdolności przywódcze pozwalają Ci na wydawanie rozkazów, wliczając w to Twoich oddanych podległych.

Poziom 1: Naturalna Charyzma

Poziom 1: Dobra Porada

Poziom 2: Potencjał

Poziom 2:  Podstawowy Kompan

Poziom 3: Zaawansowany Rozkaz or Kompan-Ekspert

Poziom 4: Zachwyt lub Inspiracja

Poziom 5: Większy Ulepszony Potencjał

Poziom 6: Drużyna Kompanów lub Umysł Lidera

Wtrącenia MG: Kompani odnoszą klęskę, zdradzają Cię, okłamują, przechodzą na złą stronę, dają się porwać lub umierają.

\subsubsection{Przyjmuje Zwierzęcy Kształt}\index{Specjalizacje!Lista!Przyjmuje Zwierzęcy Kształt}

Możesz się zmienić w zwierzę.

Poziom 1: Zwierzęcy Kształt

Poziom 2: Komunikacja

Poziom 2: Ukojenie Dzikiego

Poziom 3: Większy Zwierzęcy Kształt lub Większa Likantropia

Poziom 4: Zwierzęce Szpiegowanie

Poziom 5: Trudny do Zamordowania

Poziom 6: Rozmazana Prędkość lub  Rozszerzony Zwierzęcy Kształt

Wtrącenia MG: Postać niespodziewanie zmienia kształt. BN jest przerażony lub agresywny w stosunku do zmiennokształtnego. Transformacja zajmuje dłużej, niż się spodziewano.

Większa Likantropia stosuje się do używania Zwierzęcego Kształtu.

\subsubsection{Przywdziewa Połyskliwy Lód}\index{Specjalizacje!Lista!Przywdziewa Połyskliwy Lód}

Rozkazujesz zimowej mocy zimna i lodu.

Poziom 1: Lodowa Zbroja

Poziom 2: Lodowy Dotyk

Poziom 3: Mrozący Dotyk lub Lodowa Kreacja

Poziom 4: Twarda Lodowa Zbroja

Poziom 5: Emisja Zimna

Poziom 6: Śniegowa Zamieć lub Lodowe Rękawice

Wtrącenia MG: Lód czyni powierzchnie śliskimi. Ektremalne zimno sprawia, że przedmioty pękają i się psują. 

\subsubsection{Pływał z Piratami}\index{Specjalizacje!Lista!Pływał z Piratami}

Pływałeś razem ze straszliwymi piratami, ale zdecydowałeś się zerwać z piractwem i poświęcićsię innemu celowi. Powstaje pytanie: czy twoja przeszłość pozwoli o sobie zapomnieć?

Poziom 1: Zignorowanie Bólu

Poziom 1: Marynarz

Poziom 2: Korzystanie z Okazji

Poziom 2: Przerażająca Reputacja

Poziom 3: Umiejętny Atak lub Umiejętna Obrona

Poziom 4: Morskie Nogi

Poziom 4: Umiejętności Ruchu

Poziom 5: Zagubieni w Chaosie

Poziom 6: Pojedynek na Śmierć i Życie lub Następny Atak

Wtrącenia MG: Istnieje wiele niebezpieczeństw Siedmiu Mórz, wliczając sztormy i zarazy. Inni piraci czasem awansują poprzez zdradę. Pirat wyśledził dawnych kompanów, by odkryć ukryty skarb.

\subsubsection{Rozciąga się}\index{Specjalizacje!Lista!Rozciąga się}

Twoje ciało jest gumowe i elastyczne, zdolne rozciągać się na wielkie długości i kompresować z powrotem.

Poziom 1: Człowiek-Guma

Poziom 1: Daleki Krok

Poziom 2: Elastyczny Chwyt

Poziom 2: Bezpieczny Upadek

Poziom 3: Przeniknięcie Przez Barierę lub  Przekierowanie Ataku

Poziom 4: Odporność

Poziom 5: Mistrz Ruchu

Poziom 6: Ruch i Multiatak lub Jeszcze Żywy

Wtrącenia MG: Atak lub efekt wchodzi w interakcję z elastycznością BG. Rozciągnięta kończyna staje się przeciążona i słaba. 

\subsubsection{Rozdziera Ściany Świata}\index{Specjalizacje!Lista!Rozdziera Ściany Świata}

Szybkość i fazowanie daje Ci unikalną zdolność unikania zagrożeń i zadawania obrażeń jednocześnie. 

Poziom 1: Bieg Fazowy

Poziom 1: Przeszkadzający Dotyk

Poziom 2: Fazowe Zadrapanie

Poziom 3: Niewidzialne Fazowanie lub Przechodzenie Przez Ściany

Poziom 4: Detonacja Fazowa

Poziom 5: Bardzo Długi Bieg Fazowy

Poziom 6: Potężniejsze Fazowanie lub Nietykalny Podczas Ruchu

Wtrącenia MG: Poruszanie się tak szybko czasami prowadzi wprost na niespodziewane, egzotyczne przeszkody.

\subsubsection{Rozwiązuje Tajemnice}\index{Specjalizacje!Lista!Rozwiązuje Tajemnice}

Jesteś mistrzem dedukcji, używającym dowodów, by odnaleźć odpowiedzi.

Poziom 1: Śledczy

Poziom 1: Detektyw

Poziom 2: Z Dala od Niebezpieczeństwa

Poziom 3: Dobrze Wykształcony lub Umiejętny Atak

Poziom 4: Wyciągnięcie Wniosków

Poziom 5: Ogarnięcie Sytuacji

Poziom 6: Przejęcie Inicjatywy lub Większa Umiejętność Obrony

Wtrącenia MG: Dowody znikają, fałszywe tropy konfundują, a świadkowie kłamią. Początkowe wnioski mogą być błędne.

\subsubsection{Rośnie do Gigantycznych Rozmiarów}\index{Specjalizacje!Lista!Rośnie do Gigantycznych Rozmiarów}

Na krótkie okresy, rośniesz większy i, z odpowiednim doświadczeniem, do prawdziwie gigantycznych rozmiarów.

Poziom 1: Wzrost

Poziom 1: Przeogromny

Poziom 2: Większy

Poziom 2:  Zalety Bycia Dużym

Poziom 3: Wielki lub Rzut

Poziom 4: Chwyt

Poziom 5: Wielgachny

Poziom 6: Kolos lub Śmiertelne Obrażenia

Wtrącenia MG: Nagły wzrost przewraca meble lub sprawia, że przebijasz sufit. Powiększona postać przebija podłogę. 

\subsubsection{Rzeźbi Twardym Światłem}\index{Specjalizacje!Lista!Rzeźbi Twardym Światłem}

Tworzysz fizyczne przedmioty z twardego światła, które możesz wykorzystać do obrony lub ataku.

Poziom 1: Automatyczny Blask

Poziom 1: Chwilowe Światło

Poziom 2: Macki Mocy

Poziom 3: Twardsze Światło lub Rzeźbienie Światłem

Poziom 4: Większy Ulepszony Intelekt

Poziom 5: Ulepszone Rzeźbienie Światłem

Poziom 6: Pole Obronne lub Lot

Wtrącenia MG: Przedmiot z twardego światła przedwcześnie znika. Przedmiot z twardego światła nie może wpłynąć na daną istotę lub kolor.

\subsubsection{Rzuca ze Śmiertelną Dokładnością}\index{Specjalizacje!Lista!Rzuca ze Śmiertelną Dokładnością}

Wszystko co opuszcza Twoją dłoń idzie dokładnie gdzie sobie tego życzysz, z prędkością i w miejsce, gdzie osiągnie najlepszy efekt.

Poziom 1: Precyzja

Poziom 2: Ostrożny Rzut

Poziom 3: Szybki Rzut lub Umiejętna Obrona

Poziom 4: Wszystko Jest Bronią

Poziom 4: Wyspecjalizowany w Rzucaniu

Poziom 5: Wir Rzutek

Poziom 6: Śmiertelne Obrażenia lub Mistrzostwo Obrony

Wtrącenia MG: Chybione ataki trafiają w zły cel. Rykoszety mogą być niebezpieczne. Improwizowane bronie się psują. 

\subsubsection{Spaceruje w Dzikich Lasach}\index{Specjalizacje!Lista!Spaceruje w Dzikich Lasach}

Posługujesz się magią natury, która czerpie z potęgi drzew. 

Poziom 1: Życie w Dziczy

Poziom 1: Dodatkowy Odpoczynek

Poziom 2: Ciało z Drewna

Poziom 3: Drzewny Kompan lub Dzika Świadomość

Poziom 4: Podróż Przez Drzewa

Poziom 5: Wielkie Drzewo

Poziom 6: Straszny Las lub Odżywczy Wykwit

Wtrącenia MG: Drewniana postać zapala się. Dziki zamach konarem drzewa uderza w sprzymierzeńca. Pewne drzewa mają mroczne serca i nienawidzą wszystkich ludzi.

\subsubsection{Stawia Umysł Ponad Materią}\index{Specjalizacje!Lista!Stawia Umysł Ponad Materią}

Możesz poruszać telekinetycznie przedmioty bez fizycznego dotykania ich.

Poziom 1: Odbicie Ataków

Poziom 2: Telekineza

Poziom 3: Chmura Ochronna lub Ulepszenie Siły

Poziom 4: Aportacja

Poziom 5: Atak Psychikinetyczny

Poziom 6: Ulepszona Aportacja lub Przebudowa

Wtrącenia MG: Jeden mentalny błąd, a poruszanie obiekty upadają, a kruche obiekty niszczeją. Czasami zły przedmiot się porusza, upada lub niszczeje.

\subsubsection{Szuka Kłopotów}\index{Specjalizacje!Lista!Szuka Kłopotów}

Jesteś niebezpieczny i lubisz dobrą walkę.

Poziom 1: Pięści Furii

Poziom 1: Opatrywanie Ran

Poziom 2: Obrońca

Poziom 2: Bezpośredni

Poziom 3: Umiejętny Atak lub Większy Ulepsozny Potencjał

Poziom 4: Pozbawienie Przytomności

Poziom 5: Mistrzostwo Ataków

Poziom 6: Większa Ulepszona Moc lub Śmiertelne Obrażenia

Wtrącenia MG: Bronie psują się lub wypadają nawet z najsilniejszego uchwytu. Atakujący mogą się potknąć i upaść. Nawet pole bitwy może działać przeciwko Tobie, gdy przedmioty upadają.

\subsubsection{Szybko się Uczy}\index{Specjalizacje!Lista!Szybko się Uczy}

Radzisz sobie z trudnymi sytuacjami w miarę, jak się pojawiają, ucząc się czegoś nowego za każdym razem.

Poziom 1: Ulepszony Intelekt

Poziom 1: Oto Twój Problem

Poziom 2: Szybka Nauka

Poziom 3: Ciężki do Rozproszenia

Poziom 3: Ulepszone Skupienie w Inteligencji lub Skupienie na Umiejętności

Poziom 4: Dzielenie się Wiedzą

Poziom 5: Ulepszony Intelekt

Poziom 5: Parę Sztuczek w Zanadrzu

Poziom 6: Dwie Sprawy na Raz lub Umiejętna Obrona

Wtrącenia MG: Wypadki i pomyłki są świetnymi nauczycielami.

\subsubsection{Tańczy z Czarną Materią}\index{Specjalizacje!Lista!Tańczy z Czarną Materią}

Możesz manipulować ciemnością i "ciemną materią".

Poziom 1:  Wstęgi Mrocznej Materii

Poziom 2: Skrzydła Pustki

Poziom 3: Płaszcz Ciemnej Materii lub Cios Ciemnej Materii

Poziom 4: Powłoka Ciemnej Materii

Poziom 5: Podróżnik Niszczącego Wiatru

Poziom 6: Budowla Ciemnej Materii lub W Objęciach Nocy

Wtrącenia MG: Czarna Materia wycofuje się, zupełnie, jakby miała swoją własną wolę. 

\subsubsection{Tworzy Dziwną Naukę}\index{Specjalizacje!Lista!Tworzy Dziwną Naukę}

Twoja nadnaturalne wejrzenie w rzeczywistość czyni z Ciebie naukowca zdolnego do wielu rzeczy.

Poziom 1: Analiza Laboratoryjna

Poziom 1: Umiejętności Wiedzy

Poziom 2: Modyfikacja Urządzenia

Poziom 3: Lepsze Życie Dzięki Chemii lub Ulepszone Zdrowie

Poziom 4: Umiejętności Wiedzy

Poziom 4: Troszkę Szalony

Poziom 5: Przełom w Badaniach Dziwnej Nauki

Poziom 6: Niemożliwe Osiągnięcie Naukowe

Poziom 6: Wynalazca lub Pole Obronne

Wtrącenia MG: Twoje twory mogą się wymknąć spod kontroli. Czasami nie można przewidzieć efektów ubocznych. Dziwna nauka przeraża ludzi i przyciąga uwagę mediów. Kiedy przedmiot stworzony lub zmodyfikowany przez dziwną naukę się 
rozładowuje, wybucha. 

\subsubsection{Tworzy Iluzje}\index{Specjalizacje!Lista!Tworzy Iluzje}

Tworzysz obrazy ze światła tak perfekcyjne, że wydają się być realne. 

Poziom 1: Mniejsza Iluzja

Poziom 2:  Iluzoryczne Przebranie

Poziom 3: Rzuć Iluzję lub Większa Iluzja

Poziom 4: Iluzyjne Ja

Poziom 5: Przerażający Obraz

Poziom 6: Wielka Iluzja lub Permanentna Iluzja

Wtrącenia MG: Ciężko uwierzyć w iluzję. Iluzja zostaje przejrzana w najgorszym możliwym momencie.

\subsubsection{Tworzy Unikalne Obiekty}\index{Specjalizacje!Lista!Tworzy Unikalne Obiekty}

Jesteś wynalazcą dziwnych i użytecznych przedmiotów. 

Poziom 1: Rzemieślnik

Poziom 1: Mistrz Identyfikacji

Poziom 2: Mechanik Artefaktów

Poziom 2: Szybka Robota

Poziom 3: Mistrzowski Rzemieślnik lub Wbudowane Bronie

Poziom 4: Twórca Cypherów

Poziom 5: Innowator

Poziom 6: Wynalazca lub Zbroja Fuzyjna

Wtrącenia MG: Przedmiot zalicza awarię, niszczeje lub kończy swój byt w katastrofalny lub niespodziewany sposób.

\subsubsection{Ucieka Precz}\index{Specjalizacje!Lista!Ucieka Precz}

Twoim pierwszym instynktem jest ucieczka od niebezpieczeństwa, i jesteś w tym bardzo dobry. 

Poziom 1: W Defensywie

Poziom 2: Ulepszona Szybkość

Poziom 2: Szybka Ucieczka

Poziom 3: Niemożliwa Szybkość lub Większa Ulepszona Szybkość

Poziom 4: Determinacja

Poziom 4: Szybki Umysł

Poziom 5: Ponowne Ukrycie się

Poziom 6: Ucieczka lub Umiejętna Obrona

Wtrącenia MG: Szybkie ruchy czasami sprawiają, że upuszczasz przedmioty, poślizgujesz się lub przypadkowo kierujesz się w złą stronę

\subsubsection{Ujeżdża Błyskawicę}\index{Specjalizacje!Lista!Ujeżdża Błyskawicę}

Generujesz i wyzwalasz energię elektryczną.

Poziom 1: Szok

Poziom 1: Naładowanie

Poziom 2: Jeździec Błyskawicy

Poziom 3: Elektryczny Pancerz lub Wyssanie Ładunku

Poziom 4: Promienie Mocy

Poziom 5: Elektryczny Lot

Poziom 6: Szybkość Błyskawicy lub Ściana Błyskawic

Wtrącenia MG: Przypadkowi ludzie zostają zaatakowani prądem. Obiekty eksplodują. 

\subsubsection{Uzdrawia}\index{Specjalizacje!Lista!Uzdrawia}

Możesz leczyć innych dotykiem, wpływać na czas, by pomagać innym, i ogólnie jesteś kochany przez wszystkich.

Poziom 1: Leczący Dotyk

Poziom 2: Uzdrowienie

Poziom 3: Uzdrawiająca Fontanna lub Cudowne Zdrowie

Poziom 4: Zainspirowanie Akcji

Poziom 5: Cofnij

Poziom 6: Większy Leczący Dotyk lub Przywrócenie Życia

Wtrącenia MG: Próby uzdrowienia mogą zamiast tego spowodować krzywdę. Społeczność lub jednostka potrzebują uzdrowiciela tam bardzo, że przetrzymują go wbrew jego woli.

\subsubsection{Walczy Nieczysto}\index{Specjalizacje!Lista!Walczy Nieczysto}

Zrobisz wszystko, by wygrać walkę: będziesz gryzł, drapał, kopał, oszukiwał i czynił jeszcze gorsze rzeczy.

Poziom 1: Łowczy

Poziom 1: Stalker

Poziom 2: Skradanie się

Poziom 2: Cel

Poziom 3: Zdrada lub Atak z Zaskoczenia

Poziom 4: Gierki Umysłowe

Poziom 4: Zręczny Wojownik

Poziom 5: Korzyści z Otoczenia

Poziom 6: Obrócenie Noża lub Morderca

Wtrącenia MG: Ludzie nie cenią tych, którzy oszukują lub walczą bez honoru. Czasami brudna sztuczka uderza w Ciebie z powrotem. 

\subsubsection{Walczy z Robotami}\index{Specjalizacje!Lista!Walczy z Robotami}

Łatwo przychodzi Ci walka z robotami, automatonami i maszynami.

Poziom 1: Słabe Punkty Maszyn

Poziom 1: Umiejętności Technologiczne

Poziom 2: Ochrona Przed Robotami

Poziom 2: Polowanie na Maszyny

Poziom 3: Rozbrojeni Urządzenia lub Atak z Zaskoczenia

Poziom 4: Walczący z Robotami

Poziom 5: Wyssanie Mocy

Poziom 6: Deaktywacja Mechanizmu lub Śmiertelne Obrażenia

Wtrącenia MG: Robot eksploduje po pokonaniu. Inne roboty szukajązemsty na postaci. 

\subsubsection{Walcząc, Porywa Tłum}\index{Specjalizacje!Lista!Walcząc, Porywa Tłum}

Jesteś ryzykantem wywijającym mieczem, który posiada zachwycający styl walki, który przyjemnie jest oglądać. 

Poziom 1: Estetyczny Atak

Poziom 2: Szybki Blok

Poziom 3: Akrobatyczny Atak lub Czcze Przechwałki

Poziom 4: Chronienie Sprzymierzeńca

Poziom 4: Szybkie Zabójstwo

Poziom 5: Korzyśći z Otoczenia

Poziom 6: Szybki Umysł lub Kontratak

Wtrącenia MG: Pokaz okazuje się głupiutki, niezdarny lub nieatrakcyjny. 

\subsubsection{Wolałby Czytać}\index{Specjalizacje!Lista!Wolałby Czytać}

Książki to Twoi przyjaciele. Co jest ważniejszego od wiedzy? Nic. 

Poziom 1: Wiedza to Potęga

Poziom 2: Większy Ulepszony Intelekt

Poziom 3: Stosowanie Swojej Wiedzy lub Skupienie na Umiejętności

Poziom 4: Wiedza to Potęga

Poziom 4: Wiedza o Nieznanym

Poziom 5: Większy Ulepszony Intelekt

Poziom 6: Wiedza to Potęga

Poziom 6: Wieża Intelektu lub Czytając Znaki

Wtrącenia MG: Książki się palą, moczą lub gubią. Komputery się psują lub tracą zasilanie. Okulary się tłuką.

\subsubsection{Wpada w Furię}\index{Specjalizacje!Lista!Wpada w Furię}

Kiedy wpadasz w furię, wszyscy wpadają w popłoch.

Poziom 1: Szał

Poziom 2: Większa Ulepszona Moc

Poziom 2: Umiejętności Ruchu

Poziom 3: Potężne Uderzenie lub Wojownik Bez Zbroi

Poziom 4: Większy Szał

Poziom 5: Atak i Ponowny Atak

Poziom 6: Większy Ulepszony Potencjał lub Śmiertelne Obrażenia

Wtrącenia MG: To łatwe dla berserka, by utracić kontrolę i zaatakować zarówno przyjaciół jak i wrogów.

\subsubsection{Wspiera Społeczność}\index{Specjalizacje!Lista!Wspiera Społeczność}

Utrzymujesz miejsce gdzie żyjesz bezpieczne od wszelkich niebezpieczeństw.

Poziom 1: Wiedza o Społeczności

Poziom 1: Lokalny Aktywista

Poziom 2: Umiejętny Atak

Poziom 3: Furia Pasterza lub Umiejętna Obrona

Poziom 4: Większy Ulepszony Potencjał

Poziom 5: Unik

Poziom 6: Większa Umiejętność Ataku lub Mur Obronny

Wtrącenia MG: Ludzie w społeczności nie rozumieją motywów postaci. Rywele próbują pozby się postaci.

\subsubsection{Wyje do Księżyca}\index{Specjalizacje!Lista!Wyje do Księżyca}

Na krótki czas stajesz się przerażającą i potężną istotą, która ma problemy, żeby się kontrolować. 

Poziom 1: Likantropia

Poziom 2: Kontrolowana Przemiana

Poziom 3: Większy Likantrop lub Większa Likantropia

Poziom 4: Większa Kontrolowana Zmiana

Poziom 5: Ulepszona Likantropia

Poziom 6: Śmiertelne Obrażenia lub Perfekcyjna Kontrola

Wtrącenia MG: Przemiana przebiega w sposób niekontrolowany. Ludzie boją się potworów. 

\subsubsection{Wysysa Energię}\index{Specjalizacje!Lista!Wysysa Energię}

Wysysasz energię tak z maszyn, jak i z istot w celu wzmocnienia samego siebie.

(Roboty i inne żywe maszyny powinny być traktowane jak istoty, a nie maszyny, dla celów wysysania z nich energii.)

Poziom 1: Wyssanie Maszyny

Poziom 2: Wyssanie Istoty

Poziom 3: Wyssanie na Zasięg lub Wysuszająca Konsupcja

Poziom 4: Przechowanie Energii

Poziom 5: Dzielona Moc

Poziom 6: Wybuchowe Rozładowanie lub Pula Słoneczna

Wtrącenia MG: Wyssana moc nosi z sobą coś niechcianego – przymusy, choroby lub obce myśli. Wyssana moc może przeciążyć postać, powodując kłopoty.

\subsubsection{Wyszedł z Obelisku}\index{Specjalizacje!Lista!Wyszedł z Obelisku}

Twoje ciało, twarde jak kryształ, daje Ci unikalne zdolności, zyskane po wejściu w interakcję z lewitującym, kryształowym obeliskiem.

Poziom 1: Kryształowe Ciało

Poziom 2: Unoszenie się

Poziom 3: Zamieszkując Kryształ lub Nieruszony

Poziom 4: Kryształowe Soczewki

Poziom 5: Częstotliwość Rezonansowa

Poziom 6: Trzęsienie Rezonansowe lub Powrót do Obelisku

Wtrącenia MG: Cyphery i artefakty działają niespodziewanie w rękach postaci. 

\subsubsection{Włada Dziką Magią}\index{Specjalizacje!Lista!Włada Dziką Magią}

Jesteś użytkownikiem magii, który uczy się różnorodnych zaklęć zamiast skupiać na jednej szkole magii.

Poziom 1: Magiczne Zasoby

Poziom 1: Rzucanie Cypherów

Poziom 2: Zwiększenie Limitu Subtelnych Cypherów

Poziom 3: Przypływ Cyphera lub Szybsza Dzika Magia

Poziom 4: Zwiększenie Limitu Subtelnych Cypherów

Poziom 5: Wyuczone Zaklęcia

Poziom 6: Maksymalizacja Cyphera lub Dzikie Oświecenie

Wtrącenia MG: Zaklęcie działa losowo lub uderza w rzucającego. Coś przeszkadza w przygotowaniu zaklęć. Rzucanie zaklęć przyciąga uwagę potężnej istoty lub rywala.  Zaklęcie-cypher podczas rzucania zamienia się w przypadkowy cypher. 

\subsubsection{Włada Magnetyzmem}\index{Specjalizacje!Lista!Włada Magnetyzmem}

Rozkazujesz metalom i mocom magnetycznym.

Poziom 1: Poruszanie Metalu

Poziom 2: Odparcie Metalu

Poziom 3: Niszczenie Metalu lub Nakierowywany Pocisk

Poziom 4: Pole Magnetyczne

Poziom 5: Kontrola Metalu

Poziom 6: Diamagnetyzm lub Stalowy Cios

Wtrącenia MG: Metal się obraca, zgina i produkuje odpryski. Problem z koncentracją może sprawić, że coś Ci upadnie o złym czasie. 

\subsubsection{Włada Mocami Mentalnymi}\index{Specjalizacje!Lista!Włada Mocami Mentalnymi}

Wytrenowałeś swój umysł, by wykonywać zaskakujące psychiczne zadania. 

Poziom 1: Telepatia

Poziom 2: Czytanie Myśli

Poziom 3: Psioniczna Erupcja lub Psioniczna Sugestia

Poziom 4: Podłączony do Cudzych Zmysłów

Poziom 5: Wizja Przyszłości

Poziom 6: Kontrola Umysłu lub Sieć Telepatyczna

Wtrącenia MG: Coś podejrzanego w umyśle celu jest przerażające. Cel może odczytać myśli postaci.

\subsubsection{Włada Niewidzialną Mocą}\index{Specjalizacje!Lista!Włada Niewidzialną Mocą}

Naginasz światło i manipulujesz promieniami mocy dla ataku i obrony.

Poziom 1: Zniknięcie

Poziom 2: Macki Mocy

Poziom 2: Wyostrzone Zmysły

Poziom 3: Bariera Pola Siłowego lub Masowe Znikanie

Poziom 4: Niewidzialność

Poziom 5: Pole Obronne

Poziom 6: Wybuch lub Generacja Pola Siłowego

Wtrącenia MG: Niewidzialność częściowo zanika, ujawniając obecnośc postaci. Pole silowe jet przebite przez niecodzienny lub niespodziewany atak.

\subsubsection{Włada Rojem}\index{Specjalizacje!Lista!Włada Rojem}

Owady. Szczury. Nietoperze. Nawet ptaki. Władasz jednym rodzajem małych istot, które Tobie podlegają. 

Poziom 1: Wpływ na Rój

Poziom 2: Kontrola Roju

Poziom 3: Żywa Zbroja lub Umiejętny Atak

Poziom 4: Wezwanie Roju

Poziom 5: Pozyskanie Nietypowego Kompana

Poziom 6: Śmiercionośny Rój lub Umiejętna Obrona

Wtrącenia MG: Polecenie jest omylnie zinterpretowane. Kontrola jest chwilowa lub utracona. Ugryzienia i użądlenia nie są nietypowe dla władających rojami.

\subsubsection{Włada Zaklęciami}\index{Specjalizacje!Lista!Włada Zaklęciami}

Poprzez specjalizowanie się w zaklęciach i posiadanie księgi zaklęć, możesz szybko rzucać zaklęcie, takie jak błyskawicę, ogień, żywe cienie i przywoływanie. 

Poziom 1: Magiczny Błysk

Poziom 2: Promień Konfuzji

Poziom 3: Kwiat Ognia lub Przywołanie Wielkiego Pająka

Poziom 4: Przesłuchanie Duszy

Poziom 5: Ściana z Granitu

Poziom 6: Przywołanie Demona lub Słowo Śmierci

Wtrącenia MG: Zaklęcie działa źle. Przywołana istota rzuca się na czarownika. Mag-przeciwnik jest przyciągany przez magię użytkownika. 

\subsubsection{Zabawia}\index{Specjalizacje!Lista!Zabawia}

Występujesz, głównie dla innych ludzi.

Poziom 1: Beztroska

Poziom 2: Zainspirowanie Ułatwienia

Poziom 3: Umiejętności Wiedzy lub Większy Ulepszony Potencjał

Poziom 4: Uspokojenie

Poziom 5: Przydatna Pomoc

Poziom 6: Inspirujący Performer lub Okrutne Przedstawienie

Wtrącenia MG: Publiczność jest ziritowana lub obrażona. Muzyczne instrumenty się psują. Farby usychają w słoiczkach. Słowa wiersza lub piosenki wypadają Ci z pamięci.

\subsubsection{Zabija Potwory}\index{Specjalizacje!Lista!Zabija Potwory}

Zabijasz potwory.

(Choć noszenie miecza w settingu, w którym ludzie zazwyczaj nie noszą takich broni jest ok, możesz zmienić moce powiązane z mieczem Zabija Potwory na inną broń, taką jak pistolet ze srebrnymi pociskami.)

Poziom 1: Wyszkolony w Mieczach

Poziom 1: Sposób na Potwory

Poziom 1: Wiedza o Potworach

Poziom 2: Legendarna Wola

Poziom 3: Wyszkolony Morderca

Poziom 3: Ulepszony Sposób na Potwory lub Przekierowanie Ataku

Poziom 4: Niezłomny

Poziom 5: Większa Umiejętnosć Ataku (miecze)

Poziom 6: Morderca lub Heroiczny sposób na Potwory

Wtrącenia MG: Potwór stworzył pułapkę lub wziął Cię z zaskoczenia. Potwór ma zdolności, o których początkowo nie wiedziałeś. Matka potwora poprzysięga Ci zemstę. 

\subsubsection{Zadaje się z Martwymi}\index{Specjalizacje!Lista!Zadaje się z Martwym}

Martwi odpowiadają na Twoje pytania, a ich reanimowane ciała służą Tobie.

Poziom 1: Mówiący ze Zmarłymi

Poziom 2: Nekromancja

Poziom 3: Poznanie Lokacji lub Naprawa Ciała

Poziom 4: Większa Nekromancja

Poziom 5: Przerażające Spojrzenie

Poziom 6: Prawdziwa Nekromancja lub Słowo Śmierci

Wtrącenia MG: Reputacja postaci jako nekromanty ją wyprzedza. Zwłoki szukają zemsty za grzech zostania ożywionymi.

\subsubsection{Zaprowadza Sprawiedliwość}\index{Specjalizacje!Lista!Zaprowadza Sprawiedliwość}

Naprawiasz krzywdy, bronisz niewinnych i karasz winnych.

Poziom 1: Dokonanie Osądu

Poziom 1: Osąd

Poziom 2: Obrona Niewinnego 

Poziom 2: Ulepszony Osąd

Poziom 3: Chroń Wszystkich Niewinnych lub Ukaranie Winnego

Poziom 4: Odnalezienie Winnych

Poziom 4: Większy Osąd

Poziom 5: Ukaranie Wszystkich Winnych 

Poziom 6: Potępienie Winnych lub Zainspirowanie Niewinnych

Wtrącenia MG: Wina lub niewinność mogą być skomplikowane. Niektórzy ludzie gardzą samo-ustanowionymi sędziami. Dokonywanie osądów sprawia, iż zyskujemy sobie wrogów. 

\subsubsection{Zmniejsza się}\index{Specjalizacje!Lista!Zmniejsza się}

Możesz się zmniejszać do rozmiarów robaka, a z odpowiednią praktyką, być nawet mniejszym.

Poziom 1: Zmniejszenie się

Poziom 1: Niezauważalny

Poziom 2: Mniejszy

Poziom 2: Zalety Bycia Małym

Poziom 3: Wzrost lub Szybkie Skurczenie się

Poziom 4: Mały Lot

Poziom 5: Zmniejszenie Innych

Poziom 6: Większy lub Malutki

Wtrącenia MG: Istota myśli, że bohater to potencjalne pożywienie. Mała postać zostaje uwięziona w małej przestrzeni lub pod spadającym obiektem.

Postać, która Zmniejsza się, która wybiera zdolności takie jak Wzrost, nigdy nie będzie tak duża jak ktoś, kto Rośnie do Gigantycznych Rozmiarów, ale może cię cieszyć zaletami bycia dużym lub małym, w zależności od potrzeb.

\subsubsection{Został Przepowiedziany}\index{Specjalizacje!Lista!Został Przepowiedziany}

Jesteś "Wybrańcem" i przepowiednie, prognozy lub inne metody mówią, że dokonasz w pewnym momencie wielkich rzeczy.

Poziom 1: Umiejętności Międzyludzkie

Poziom 1: Wiedza

Poziom 2: Przeznaczenie Wielkości

Poziom 3:  Przezwyciężając Wszystkie Przeciwności lub Ciężko Zapracowana Odporność

Poziom 4: Centrum Uwagi

Poziom 5: Wskaż Im Drogę

Poziom 6: Jak Przepowiedziano or Większy Ulepszony Potencjał

Wtrącenia MG: Przeciwnik przepowiedziany w przepowiedni się pojawia. Niewierni grożą, że zrujnują plany postaci. Postać ma reputację w pewnych kręgach jako udawaniec. 

\subsubsection{Żyje w Dziczy}\index{Specjalizacje!Lista!Żyje w Dziczy}

Możesz przetrwać w dziczy, w której inni nie potrafią.

Poziom 1: Życie w Dziczy

Poziom 1: Ulepszona Moc

Poziom 2: Przetrwanie w Dziczy

Poziom 2: Badacz Dziczy

Poziom 3: Zwierzęce Zmysły lub Dzika Zachęta

Poziom 4: Dzika Świadomość

Poziom 5: Przyroda po Twojej Stronie

Poziom 6: Jedność z Dziczą lub Dziki Kamuflaż 

Wtrącenia MG: Ludzie w miastach i miasteczkach czasami są niechętni tym, którzy wyglądają (i pachną) jakby żyli w dziczy, uważając ich za ignorantów lub barbarzyńców.